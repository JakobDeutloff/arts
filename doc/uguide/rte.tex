%
% To start the document, use
%  \levela{...}
% For lover level, sections use
%  \levelb{...}
%  \levelc{...}
%
\levela{Basic radiative transfer}
 \label{sec:rte}


%
% Document history, format:
%  \starthistory
%    date1 & text .... \\
%    date2 & text .... \\
%    ....
%  \stophistory
%
\starthistory
  000307 & Created and written by Patrick Eriksson. \\
\stophistory



%
% Symbol table, format:
%  \startsymbols
%    ... & \verb|...| & text ... \\
%    ... & \verb|...| & text ... \\
%    ....
%  \stopsymbols
%
%
\startsymbols
  \mpbi     & \verb|y|       & monochromatic pencil beam intensity      \\
  $\mpbi_1$ & \verb|y_cbgr|          & cosmic background radiation      \\
  $\mpbi_1$ & \verb|y_ground|        & ground blackbody radiation       \\
  $S$       & \verb|S|               & source function                  \\
  $k$       & \verb|Abs|             & total absorption                 \\
  $T$       & \verb|t|               & temperature                      \\
  \f        & \verb|f|               & frequency                        \\
  $h$       & \verb|PLANCK_CONST|    & the Planck constant              \\ 
  $c$       & \verb|SPEED_OF_LIGHT|  & speed of light                   \\
  $k_B$     & \verb|BOLTZMAN_CONST|  & the Boltzmann constant           \\
  $\tau$    & -                      & optical thickness                \\
  $\zeta$   & \verb|Tr|              & transmission between LOS points  \\
  $\zeta^{atm} $& \verb|y|           & total atmospheric transmission   \\
  $l$       & \verb|l|               & distance along LOS               \\
  $\Delta l$ &\verb|lstep|           & step length along LOS            \\
  $e$       & \verb|e_ground|        & ground emissivity                \\
  $T_{ground} $&\verb|t_ground|      & physical temperature of the ground\\
  $T_{cbgr}$  &\verb|COSMIC_BG_TEMP| & temperature representing cosmic  \\ 
         &                &        background radiation                 \\
 \label{symtable:rte}     
\stopsymbols



%
% Introduction
%
This section presents the atmospheric radiative transfer equation
(RTE) for a non-scattering atmosphere in local thermodynamic
equilibrium.  However, the calculation scheme described here can
easily be extended to include effects of e.g. scattering. The
radiative transfer equation gives the monochromatic (infinite
frequency resolution) pencil beam (infinite spatial resolution)
spectrum. The main problem, in this context, is how to practically and
accurately estimate the (continuous) integral in the discrete forward
model.
 
The discussion treats mainly measurements of atmospheric emission. The 
forward model can also handle pure absorption measurements and such 
observations are also discussed briefly last in the section.

The equations of this section are valid for monochromatic pencil beam
spectra, that is, no effects of the sensor are considered. How to
incorporate sensor effects in the spectra and weighting functions is
discussed separately (Sec. \ref{sec:sensor}).



\levelb{Introduction}
 \label{sec:rte:intro}
 
 Atmospheric radiative transfer can be expressed generally as
 \begin{equation}
   I = I_1e^{-\int_{l_1}^{l_2}{\kappa(l)dl}} +
        \int_{l_1}^{l_2}{\kappa(l)S(l)e^{-\int_{l}^{l_2}{\kappa(l')dl'}}dl}
    \label{eq:rte:rte}
 \end{equation}
 where $I$ is the monochromatic pencil beam intensity, $l$ distance
 along the line of sight (LOS), $l_1$ the point of the considered part
 of the LOS furthest away from the sensor, $l_2$ the closest point of
 the LOS, $I_1$ the intensity at $l_1$, $\kappa$ the total absorption along
 the LOS and $S$ the source function.
  
 Equation \ref{eq:rte:rte} is of general validity if $S$ and $\kappa$
 consider the relevant effects, for example, scattering. However, below in
 this section it is assumed that there is no scattering and the
 atmosphere is in local thermodynamic equilibrium.
  
 Note that Eq. \ref{eq:rte:rte} is valid both for the case when the LOS is
 determined by geometrical calculations and when refraction is
 considered (the refraction changes however the LOS).
  
 With the assumptions of no scattering and local thermodynamic
 equilibrium, $\kappa$ is the summed gaseous absorption, and the source
 function equals the Planck function, $B$:
 \begin{equation}
    S = B(\f,T) = \frac{2h\f^3}{c^2} \frac{1}{e^{h\f/k_B T}-1}
    \label{eq:rte:planck}
 \end{equation}
 giving the blackbody radiation for a temperature $T$ and frequency
 $\f$.
  
 If $S$ is constant along the considered part of the LOS, that is, the
 temperature is constant for the case $S=B$, the RTE can be solved
 analytically to give
 \begin{equation}
   I = I_1e^{-\tau} + S\left(1-e^{-\tau}\right)
  \label{eq:rte:step}
 \end{equation}
 where $\tau$ is the optical thickness
 \begin{equation}
   \tau = \int_{l_1}^{l_2}{\kappa(l)dl}
 \end{equation}
 The transmission corresponding to $\tau$ is
 \begin{equation}
   \zeta = e^{-\tau}
 \end{equation}  



\levelb{Practical considerations}
 \label{sec:rte:practical}
 
 The LOS can be divided into parts in several ways. As absorption and
 temperature most likely are avaliable at some vertical grid, the most
 natural choice would be to define the LOS using this vertical grid.
 This solution is problematic for limb sounding as the ratio between
 the distance along LOS and the corresponding vertical distance
 becomes infinite at the tangent point. Another solution would be to
 base the division on $\tau$, but such a division does not guarantee
 that $T$ is close to constant inside the slabs as the vertical
 extension in some cases could be very large, and each combination of
 frequency and viewing angle should require a specific division.
  
 As a practical compromise, it was here decided to divide LOS into
 equal long geometrical steps. With this scheme the division is
 identical for all frequency components, but changes between the
 viewing angles, and should give relatively fast and straightforward
 calculations, maintaining a good accuracy. This approach has
 successfully been applied in the Odin sub-mm forward model 
 \citep{eriksson:97a,eriksson:00a}.
  
 The next question is when and how to calculate LOS and the associated
 variables. As the determination of weighting functions associated
 with the absorption, e.g. species WFs, needs basically the same
 quantities as RTE, it is most efficient to do this procedure only
 once and in such way that the values are suitable for both RTE and
 the weighting functions. Hence, the LOS calculations shall be a
 separate part, not included in the RTE functions. The standard use of
 the forward model should then be:
  \begin{enumerate}
    \item Calculation of absorption coefficients.
    \item Determination of LOS.
    \item Calculation of the source function and transmission along LOS.
    \item Iteration to solve RTE.
    \item Calculation of weighting functions.
    \item Saving etc.
  \end{enumerate}
 The determination of LOS is described separately in Section \ref {sec:los}. 
  

  
\levelb{Practical solution}
 \label{sec:rte:iter}
 
 The LOS is here assumed to be defined with $n$ points, and the
 transmission and a mean value for the source function for each
 intermediate layer are assumed to be known (see Fig
 \ref{fig:rte:los}). With this approach, the LOS must have at least two 
 points ($n>1$). 
 The transmission between the points is stored
 instead of optical thicknesses as evaluation of the exponential
 function is relatively computationally demanding and this calculation
 should not be duplicated. The relationship between
 the optical thicknesses and the transmissions is
 \begin{equation}
   \zeta_i = e^{-\tau_i}
 \end{equation}
 Note that
 \begin{equation}
   e^{-\left(\tau_1+\tau_2\dots\tau_n\right)}=\zeta_1\zeta_2\dots\zeta_n
 \end{equation}
 The intensity at point $n$ can be expressed as

  \begin{figure}[t]
    \includegraphics*[width=0.98\hsize]{Figs/los.eps}
    \caption{Schematic description of the variables along 
             the line of sight. All the points are separated by the distance
             $\Delta l$ (along the LOS). The transmission along the LOS
             between the points is $\zeta_i$ and the mean value of the source
             function is $S_i$.}  
    \label{fig:rte:los}  
  \end{figure}

 \begin{equation}
   I = I_1 \prod_{j=1}^{n-1}\zeta_j + 
       \sum_{i=1}^{n-1}\left[S_i(1-\zeta_i)\prod_{j=i+1}^{n-1}\zeta_j\right]
  \label{eq:rte:rteprod}
 \end{equation}
 However, an alternative approach, requiring less computer memory, is
 to follow the radiation from one slab of the atmosphere to next, and
 is the method of choice here. Following Equation \ref{eq:rte:step},
 the following iterative expression can be determined
 \citep{eriksson:97a}
 \begin{equation}
   I_{i+1} = I_i\zeta_i + S_i\left(1-\zeta_i\right)\qquad i=1,2,...,n\!\!-\!\!1
  \label{eq:rte:iteration}
 \end{equation}
 where $I_i$ is the intensity at point $i$.
 The iteration is started by setting $I_1$ to the intensity at the end
 of the atmosphere, for example, cosmic background radiation
 or blackbody radiation from the ground (see further below).

 
 \levelc{Starting the iteration} 
 For the initialization of the iteration there are three possibilities:
    
 \begin{enumerate}
      \item LOS starts in space and cosmic background radiation is not
            considered: $I_1=0$
      \item LOS starts in space and cosmic background radiation is 
            considered: $I_1(\f)=B(\f,T_{cbgr})$
      \item LOS intersects with the ground and the ground is assumed to 
            be a perfect blackbody: $I_1(\f)=B(\f,T_{ground})$
 \end{enumerate}
 where $T_{cbgr}$ is the temperature corresponding to the cosmic
 background radiation, and $T_{ground}$ is the physical temperature of
 the ground.
 
 Note that case 3 is the only time when the LOS not starts in space.  The
 iteration can be started at the ground for case 3 as what happens
 before the ground reflection is of no concern (consider Eq.
 \ref{eq:rte:ground} with $e$=1).


 \levelc{Considering ground reflection}  
  \label{sec:rte:ground}

 The effect of a ground reflection is modelled as
 \begin{equation}
   I^{after} = I^{before}(1-e) + eB(\f,T_{ground})
  \label{eq:rte:ground}
 \end{equation} 
 where $e$ is the ground emission factor and $I^{before}$ and
 $I^{after}$ is the intensity before and after the reflection,
 respectively. 


\levelb{Total atmospheric transmission}
 \label{sec:rte:trans}
  
 The atmospheric emission can be neglected if the observation is
 performed towards a sufficiently strong source, such as the Sun, and
 the measurement gives basically the total atmospheric transmission,
 $\zeta^{tot}$. This transmission is
 \begin{equation}
   \zeta^{tot} = e^{-\int_{l_1}^{l_2}{\kappa(l)dl}}
  \label{eq:rte:tottrans}
 \end{equation}
 The corresponding iterative formula used in the forward model is
 simply (cf. Eq. \ref{eq:rte:iteration})
 \begin{equation}
   \zeta_{i+1}^{tot} = \prod_{i=1}^{n-1}\zeta_{i}
 \end{equation} 
 where $\zeta_1$ is 1. It is noteworthy that the multiplication order
 is of no importance, a fact that can be used for 1D limb sounding where
 the conditions are assumed to be symmetrical around the tangent point and
 only one half of the line of sight is stored. The transmission can here be 
 calculated as
 \begin{equation}
   \zeta_{i+1}^{tot} = \prod_{i=1}^{n-1}\zeta_{i}^2
 \end{equation} 

 The ground is considered as
 \begin{equation}
   \zeta_{i+1}^{tot} = (1-e)\prod_{i=1}^{n-1}\zeta_{i}
  \label{eq:rte:tground}
 \end{equation} 
 where $e$ is the ground emission factor.
 

\levelb{Calculation of transmission and source function}
 \label{sec:rte:calc}
   
 The source function is simply obtained by interpolating linearly the
 temperature profile (or the 2D temperature field) at the points of
 LOS, and calculating the Planck function (Eq. \ref{eq:rte:planck}) for
 the mean temperatures of neighboring points, i.e.
 \begin{equation}
   S_i(\f) = B(\f,(T_i+T_{i+1})/2)
  \label{eq:rte:planck2}
 \end{equation}
 where $T_i$ is the temperature at point $i$.
  
 The transmission is calculated assuming the absorption to vary
 linear between the points of LOS:
 \begin{equation}
   \zeta_i = e^{-\Delta l \left(\kappa_i+\kappa_{i+1}\right)/2 }
  \label{eq:rte:transm}
 \end{equation}
 where $\kappa_i$ is the interpolated absorption value for point $i$, and
 $\Delta l$ is the distance along the LOS between the points. 

 The absorption is determined by !!.


%%% Local Variables: 
%%% mode: latex
%%% TeX-master: "main"
%%% End: 
