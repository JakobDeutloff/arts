\levela{Clear sky radiative transfer}
 \label{sec:rte}


\starthistory
  040427 & Started by Patrick Eriksson. \\
\stophistory


To be written \dots


\levelb{Surface properties}
 \label{sec:rte:surfprop}

 The properties of a material are reported either as the relative
 dielectric constant, $\epsilon$, or the refractive index, $n$. Both
 these quantities can be complex and are related as
 \begin{equation}
   \label{eq:rte_eps2n}
   n = \sqrt{\epsilon}.
 \end{equation}
 
 
\levelc{Sea water}
 \label{sec:rte:surfprop:water}

 What is the standard model for $\epsilon$?

 

\levelb{Surface emission and reflection}
 \label{sec:rte:surface}
 
 Basic facts regarding the treatment of surface emission and reflections
 are given in Section~\ref{sec:fm_defs:groundrefl}.
 
 We start with a simple example, to explain the usage of the
 workspace describing the surface properties. We will here assume that
 the surface has an isotropic scattering, with no absorption of the
 downwelling radiation. This assumption is made for all polarisation
 states. We assume further a 1D simulation and that the downwelling
 radiation shall be calculated for nine zenith angles. The relevant
 workspace variables should then be set as follows:
  
 \wsvindex{ground\_emission}: A matrix (of correct size) of zeros.

 \wsvindex{ground\_los}: A vector of length 9, covering the zenith
 angle range. A possible choice would be [5,15,25,\dots,85].
 
 \wsvindex{ground\_refl\_coeffs}: All values with index (:,:,0,0) are
 set to 1/9 (using Matlab notation, but zero based indexing).
 Remaining values are set to zero.  Size matching
 \artsstyle{ground\_los}, \artsstyle{f\_grid} and
 \artsstyle{stokes\_dim}


\levelc{Specular reflections}
 \label{sec:rte:surface:specular}
 
 If the surface is sufficiently smooth, radiation will be
 reflected/scattered only in the complementary angle, specular
 reflection. Required smoothness for assuming specular reflection is
 normally estimated by the Rayleigh criterion:
 \begin{equation}
   \label{eq:rte:rayleigh}
   \Delta h < \frac{\Wvl}{8\cos\theta_1}
 \end{equation}
 where $\Delta h$ is the root mean square variation of the surface
 height, \Wvl\ the wavelength and $\theta_1$ the angle between the
 surface normal and the incident direction of the radiation. The
 criterion can also be defined with the factor 8 replaced with a lower
 integer number.
 
 The reflection coefficient for the amplitude of the electromagnetic
 wave for vertical ($R_v$) and horizontal ($R_v$) polarisation is
 for a flat surface given by the Fresnel equations:
 \begin{eqnarray}
   \label{eq:rte_fresnel}
   R_v &=& \frac{n_2\cos\theta_1-n_1\cos\theta_2}
                                           {n_2\cos\theta_1+n_1\cos\theta_2} \\
   R_h &=& \frac{n_1\cos\theta_1-n_2\cos\theta_2}
                                           {n_1\cos\theta_1+n_2\cos\theta_2} \\
 \end{eqnarray}
 [* check if correct *] where $n_1$ is refractive index for
 the medium where the incoming radiation is propagating, $\theta_1$ is
 the incident angle (measured from the local surface normal) and $n_2$
 is the refractive index of the reflecting medium. The angle
 $\theta_2$ is the propagation direction for the transmited part, and
 is given by Snell's law:
 \begin{equation}
   \label{eq:rte:snell}
   \Re(n_1)\sin\theta_1 = \Re(n_2)\sin\theta_2.
 \end{equation}
 where $\Re(\cdot)$ denotes the complex real part.
 For cases where medium 1 is air, $n_1$ can (in this context) be set to 1.
 
 The power reflection coefficients are converted to an intensity
 reflection coefficient as
 \begin{equation}
   \label{eq:rte:R2r}
   r = |R|^2,
 \end{equation}
 where $|\!\cdot\!|$ denotes the absolute value. Note that $R$ can be
 complex, while $r$ is always real.
 
 By extracting vertical and horizontal polarisation from the Stokes
 vector, applying the reflection coefficients and recreating the
 Stokes vector, it can be realised that the multiplication
 \begin{equation}
   \label{eq:rte:specular_matrix}
   \left[\begin{array}{c}I\\Q\\U\\V\end{array}\right]_\mathrm{up} =
      \left[\begin{array}{cccc}
        \frac{r_v+r_h}{2}&\frac{r_v-r_h}{2}&0&0\\
        \frac{r_v-r_h}{2}&\frac{r_v+r_h}{2}&0&0\\
        0&0&\frac{r_v+r_h}{2}&0\\
        0&0&0&\frac{r_v+r_h}{2}\\
      \end{array}
      \right]
      \left[\begin{array}{c}I\\Q\\U\\V\end{array}\right]_\mathrm{down}
 \end{equation}
 [* This equation is probably not correct. Use Sec 5.4.3 of Liou to
 derive this matrix. *]
 
 The workspace variable \artsstyle{ground\_los} shall here
 of course be set to have the length 1. The specular direction is
 calculated by the internal function \funcindex{ground\_specular\_los}.

 [* Fill in what workspace methods that match this section. *]


\levelc{Surface emission}
 \label{sec:rte:surface:emission}
 
 [* Can we relate \artsstyle{ground\_refl\_coeffs} and
 \artsstyle{ground\_emission}? 

 The emission for specular reflection should be 
 \begin{equation}
    \left[\begin{array}{c}
      B\left(1-\frac{r_v+r_h}{2}\right) \\
      B\frac{r_h-r_v}{2} \\
      0\\0
    \end{array}\right]
 \end{equation}
 Correct? *]



%%% Local Variables: 
%%% mode: latex
%%% TeX-master: "main"
%%% End: 
