%
% To start the document, use
%  \levela{...}
% For lover level, sections use
%  \levelb{...}
%  \levelc{...}
%
\levela{Scattering}
 \label{sec:scattering}

%
% Document history, format:
%  \starthistory
%    date1 & text .... \\
%    date2 & text .... \\
%    ....
%  \stophistory
%
\starthistory
   100502 & Created and written by Claudia Emde.\\
\stophistory


%
% Symbol table, format:
%  \startsymbols
%    ... & \artsstyle{...} & text ... \\
%    ... & \artsstyle{...} & text ... \\
%    ....
%  \stopsymbols
%
%
%\startsymbols
%  \Ind           & -                 & vector/matrix/tensor index           \\
%  \aInd{\Lat}    & -                 & the \Ind:th latitude                 \\
%  \VctLng        & -                 & vector length or size of matrix/tensor for a dimension \\
%  \aVctLng{\Lat} & -                 & length of the latitude grid \\
%  \Prs           & \artsstyle{p}     & pressure                             \\
%  \PrsAlt        & \artsstyle{pz}    & pressure altitude                    \\
%  \Rds           & \artsstyle{r}     & radius from the geoid centre         \\
%  \Alt           & \artsstyle{z}     & geometrical altitude above the geoid \\
%  \Lat           & \artsstyle{alpha} & latitude                             \\
%  \Lon           & \artsstyle{beta}  & longitude                            \\
%  \ZntAng        & \artsstyle{psi}   & zenith angle                         \\
%  \AzmAng        & \artsstyle{omega} & azimuthal angle                      \\
% \label{symtable:ppath_defs}     
%\stopsymbols

\startsymbols
\StoVec       & \artsstyle{i\_field}       & monochromatic intensity field, Stokes Vector at
each grid point \\
              & \artsstyle{i\_field\_old}  & needed only internally
              for the iteration\\
              & \artsstyle{scat\_i\_p}   & intensity field on boundary
              of cloud box on the pressure surfaces\\
              & \artsstyle{scat\_i\_lat} & \\
              & \artsstyle{scat\_i\_lon} & \\
\ExtMat       & -                        & total extinction matrix \\
\SExMat        & \artsstyle{scat\_ext\_mat} & extinction martrix for a
single particle type\\
\AbsVec       & -                        & total absorption vector \\
\SAbVec       & \artsstyle{scat\_abs\_vec} &   absorption vector for a
single particle type\\
\SEfMat       & -                        & scattering efficiency
matrix\\
\PhaMat       & \artsstyle{scat\_pha\_mat} & phase matrix for a particle
type\\
\AmpMat       & \artsstyle{scat\_amp\_mat} & amplitude matrix\\
\PDir         & -                        & propagation direction \\
\Frq          & \artsstyle{f\_grid}       & frequency\\
\Tmp          & \artsstyle{t\_field}         & temperature\\
\PDen         & \artsstyle{scat\_part\_dens} & particle density \\
\ScaInt       & \artsstyle{scattered\_field} & scattered field,
evaluated scattering integral\\
\ScaP         & \artsstyle{scat\_p\_grid} & pressure grid inside cloud
box\\
\ScaLat       & \artsstyle{scat\_lat\_grid} & latitude grid inside cloud
box\\
\ScaLon       & \artsstyle{scat\_lon\_grid} & longitude grid inside cloud
box\\
\ScaZa        & \artsstyle{scat\_za\_grid}  & zenith angle \\
\ScaAa        & \artsstyle{scat\_aa\_grid}  & azimuth angle  \\
              & \artsstyle{scat\_f\_index}  & frequency index\\
\IPart        & \artsstyle{scat\_pt\_index} & particle type index\\
\Planck       &                             & Planck function
\label{symtable:scattering}
\stopsymbols

\newcommand{\DirFre} {(\PDir, \Frq)}



\levelb{General Radiative Transfer Equation}
%===================
\label{sec:scattering:general_rte}
 

The radiative transfer equation is \citep{mishchenko00:_light_scatt_nonsp_partic}: 
\begin{eqnarray}
     \frac{\DiffD \StoVec}{\DiffD s} =
     -\ExtMat\DirFre+\AbsVec\DirFre \Planck(\Tmp, \Frq)
     +\int_{4\pi} \DiffD \PDir' \SEfMat\DirFre \StoVec\DirFre 
     \label{eq:RTE} 
\end{eqnarray} 
Formally it is a four dimensional vector equation where \StoVec
$(I,Q,U,V)$ is the Stokes vector.
Its first component $I$ is the monochromatic flux, the
second and third components $Q$ and $U$ describe the state of linear
polarisation and the last component $V$ denotes the state of circular
polarisation of the radiation propagating in the direction \PDir.
\DiffD $s$ denotes the pathlength element along direction \PDir.
\\

The first term of equation (\ref{eq:RTE}) corresponds to the extinction of
radiation determined by the extinction coefficient matrix \ExtMat
. For microwave radiation travelling through the atmosphere we
have extinction due to gaseous
absorption, particle absorption and  particle scattering. Lambert's law
states that the extinction process is
proportional to the amount of matter. Accordingly K can be written as
a sum of two matrices, the first for particle extinction ( \aExtMat{p})
and the second for gaseous extinction (\aExtMat{g}). Then
\begin{eqnarray}
  \ExtMat\DirFre &=&
  \aExtMat{p}\DirFre+\aExtMat{g}\DirFre
\end{eqnarray}

The particle extinction term can be written as a sum, each term
corresponding to one particle type:
\begin{eqnarray}
  \aExtMat{p}\DirFre = \sum_i \PDen_i \aSExMat{p} \DirFre
\end{eqnarray}

Here \PDen$_i$ is the particle density of the
{\sl i}th gaseous species and  \aSExMat{p}$_i$\DirFre  the particle
extinction cross section matrix of the
{\sl i}th gaseous species.

The second term in equation (\ref{eq:RTE})  describes the thermal
emission. Therefore \Planck  is the Planck
function at temperature T and \AbsVec  is the absorption
coefficient vector which can be written as
 \begin{eqnarray}
  \AbsVec \DirFre  &=& \aAbsVec{p} \DirFre + \aAbsVec{g} \DirFre 
  \end{eqnarray}
with \aAbsVec{p} and \aAbsVec{g} as the particle
absorption 
cross section
and the gaseous absorption cross section, respectively. 

The particle absorption term can also be written as a sum over all
particle types:
\begin{eqnarray}
  \aAbsVec{p}\DirFre = \sum_i \PDen_i \aSAbVec{p} \DirFre
\end{eqnarray}

The last term in Equation (\ref{eq:RTE}) is the scattering source
term. It is the 
amount of radiation which is scattered from the direction \PDir'   
into the line of sight.  The scattering efficiency matrix
\SEfMat is the 
product of the particle density \PDen  and the phase matrix \PhaMat
\begin{eqnarray}
\SEfMat(\PDir, \PDir', \Frq) = \PDen\PhaMat(\PDir, \PDir', \Frq).
\end{eqnarray}

Molecular scattering is neglected as it is not important in the 
considered frequency range from 10 to 1000 GHz. 


\levelb{Iteration Scheme for Solving the RTE}
%===============================
\label{sec:scattering:solution_rte}

It is not possible to find an analytic solution to the general
radiative transfer equation. We use an iterative method to find a
numerical solution.

\levelc{Scattering Integral}
%===================================
\label{sec:scattering:scat_int}

The first step is solving the scattering integral
\begin{eqnarray}
  \ScaInt_0 \DirFre  = \int_{4\pi} \DiffD \PDir' \SEfMat\DirFre \StoVec_0\DirFre 
\end{eqnarray}
using a first guess initial field for \StoVec$_0$\DirFre. The integration
has to be performed over all incident directions \PDir' for each
propagation direction \PDir. The result is the first guess scattered field \ScaInt$_0$\DirFre
which has the same dimension as the intensity field \StoVec$_0$\DirFre. 

\levelc{Radiative Transfer with fixed Scattered Field}
%======================================
\label{sec:scattering:RTE}
With a fixed scattered field \ScaInt\DirFre the radiative transfer
equation (\ref{eq:RTE}) can be written as
\begin{eqnarray}
     \frac{\DiffD \StoVec}{\DiffD s} =
     -\ExtMat\DirFre+\AbsVec\DirFre \Planck(\Tmp, \Frq)
     +\ScaInt_0\DirFre
\end{eqnarray} 
This linear differential equation can be solved analytically leading
to the result
\begin{eqnarray}
  \label{eq:scattering:RTE_sol}
  \StoVec_1 = e^{-\ExtMat s}\cdot\StoVec_0 + (\IdnMat - e^{-\ExtMat
    s})\Inv{\ExtMat} (\AbsVec \Planck + \ScaInt_0)
\end{eqnarray}
where \StoVec$_0$ is given by the initial condition
\begin{eqnarray}
  \StoVec_0 =  \StoVec(s = 0).
\end{eqnarray}
Here we take for \StoVec$_0$ the initial field and we find the first
iteration field \StoVec$_1$.

\levelc{Iteration and Convergence}
%===============================
\label{sec:scattering:conv}

The procedure above is repeated, i.e. \StoVec$_1$ is taken to evaluate
the scattering integral ledaing to the first iterated scattered field
\ScaInt$_1$. Then (\ref{eq:scattering:RTE_sol}) is again used to compute the
second iteration field \StoVec$_2$. \\
After each iteration the convergence is checked. If 
\begin{eqnarray}
|\StoVec_m \DirFre -  \StoVec_{m-1} \DirFre| < \epsilon
\end{eqnarray}
a solution to the vector radiative transfer equation (\ref{eq:RTE})
has been
found.

\levelb{Database for the Single Scattering Properties}
%============================================================
\label{sec:scat_database}

The required quantities for the radiative transfer calculations are
the  extinction cross section matrix, the
absorption coefficient vector and the phase
matrix. All these quantities can be calculated for spherical,
cylindrical and spheroidal particles using the T-matrix
method. Applying Mishchenko's FORTRAN code the amplitude matrix can be calculated. The
amplitude matrix ${\bf S}$ is a 2x2 complex matrix, which linearly transforms 
the electric field vector components of the incident wave into the
electric field vector components of the scattered wave.  
\begin{eqnarray}
  \label{eq:ampl_matrix}
  \left[{E^{sca}_{\theta}\atop E^{sca}_{phi}}\right] =
  \frac{exp(ikR)}{R}{\bf S}({\bf n};{\bf
      n'};\epsilon_1,\epsilon_2,\epsilon_3)\left[{E^{inc}_{\theta
          0}\atop E^{inc}_{\phi 0}}\right] 
\end{eqnarray}
The amplitude matrix depends on the directions of incidence  $(\theta,
\phi)$  and
scattering $(\theta', \phi')$ as  well as on size, morphology and composition of the
scattering material, also on the orientation of the particles which
is specified by the Euler angles of rotation $\epsilon _i$. 
The equation above is valid for a monochromatic wave, in general, the
amplitude matrix is also frequency dependent.

The amplitude matrix contains all the information about the optical
properties of the scattering medium. Extinction cross section
matrix, absorption cross section vector and phase matrix can be
obtained by simple additions or multiplications 
of components of the amplitude matrix. 
For this reason, it is sufficient to store only the amplitude matrices
in the scattering data base. 

For each considered frequency there is one file for each particle
type in the database. The particle
type depends on the particle size, the particle shape and the
orientation. We consider only ice crystals, so the material of the
particles is always the same. If there is a cloud consisting of
different particles, the scattering properties for that particles
distribution can be derived from the single scattering properties.

Each file in the database contains for each propagation direction (\ScaZa,
\ScaAa) and each scattered direction (\ScaZa', \ScaAa') the real
components of the amplitude matrix ($S^{real}_{11}$, $S^{real}_{12}$,
$S^{real}_{21}$, $S^{real}_{22}$) and the imaginary components
($S^{imag}_{11}$, $S^{imag}_{12}$,$S^{imag}_{21}$, $S^{imag}_{22}$).



\levelc{Generate Extinction Matrix, Absorption
  Vector and Phase Matrix} 
%==================================================================
\label{sec:scattering:gen_ext_abs}

There are three methods to generate extinction cross section
matrix, phase matrix and absorptien cross section vector, all needed for the
radiative transfer calculation. 
\begin{itemize}
\item  \funcindex{calc\_ext\_matrix} - The extinction cross section
  matrices \SExMat are
  calculated for all patricle types chosen in the control
  file. The function reads the
  required amplitude matrices \AmpMat from the 
  data base and calculates the coefficients using the following
  formulas:
  \begin{eqnarray}
    \label{eq:gen_ext_matrix}
    L^p_{jj}({\bf n}) &=& \frac{2\pi \nu}{c} Im[S_{11}({\bf n},{\bf
      n})+S_{22}({\bf n},{\bf n})], \quad j=1,...,4 \\
    L^p_{12}({\bf n}) = L^p_{21}({\bf n}) &=& \frac{2\pi \nu}{c} Im[S_{11}({\bf n},{\bf
      n})-S_{22}({\bf n},{\bf n})]\\
    L^p_{13}({\bf n}) = L^p_{31}({\bf n}) &=& -\frac{2\pi \nu}{c} Im[S_{12}({\bf n},{\bf
      n})+S_{21}({\bf n},{\bf n})]\\
    L^p_{14}({\bf n}) = L^p_{41}({\bf n}) &=& \frac{2\pi \nu}{c} Re[S_{21}({\bf n},{\bf
      n})-S_{12}({\bf n},{\bf n})]\\
    L^p_{23}({\bf n}) = -L^p_{32}({\bf n}) &=& \frac{2\pi \nu}{c} Im[S_{21}({\bf n},{\bf
      n})-S_{12}({\bf n},{\bf n})]\\
    L^p_{24}({\bf n}) = -L^p_{42}({\bf n}) &=& -\frac{2\pi \nu}{c} Re[S_{12}({\bf n},{\bf
      n})+S_{21}({\bf n},{\bf n})]\\
    L^p_{34}({\bf n}) = L^p_{43}({\bf n}) &=& \frac{2\pi \nu}{c} Re[S_{22}({\bf n},{\bf
      n})-S_{11}({\bf n},{\bf n})]
  \end{eqnarray}
  The coefficients are stored in the variable
  \begin{center}
 \wsvindex{scat\_ext\_mat} $\in$ (\IPart, \ScaZa, \ScaAa, \ScaZa', \ScaAa',\aInd{K}).   
  \end{center}

\item \funcindex{calc\_phase\_matrix} - This method generates the
  phase matrix \PhaMat from the amplitude matrix \AmpMat applying the following
  equations:
\begin{eqnarray}
 \label{eq:gen_phase_matrix}
   Z_{11} &=& \frac{1}{2}(|S_{11}|^2+|S_{12}|^2+|S_{21}|^2+|S_{22}|^2)\\
   Z_{21} &=& \frac{1}{2}(|S_{11}|^2+|S_{12}|^2-|S_{21}|^2-|S_{22}|^2)\\
   Z_{31} &=& -Re(S_{11}S_{21}^*+S_{22}S_{12}^*)\\
   Z_{41} &=& -Im(S_{21}S_{11}^*+S_{22}S_{12}^*)\\
\cdots ... {\rm to be completed}
\end{eqnarray}
The phase matrix is stored in the
variable\\ 
\begin{center}
  \wsvindex{phase\_matrix} $\in$ (\IPart, \ScaZa, \ScaAa,
  \ScaZa',\ScaAa',\aInd{Z}).
\end{center}

 
\item \funcindex{calc\_abs\_vector} - This method calculates the
  absorption coefficient vectors \SAbVec for all particle distributions using
  the following formulas:
  \begin{eqnarray}
    \label{eq:gen_abs_vector}
b^p_1({\bf n}) &=& n^p L^p_{11}({\bf n}) - n^p \int_{4\pi} d{\bf n'}
Z_{11}({\bf n, n'})\\
b^p_2({\bf n}) &=& n^p L^p_{21}({\bf n}) - n^p \int_{4\pi} d{\bf n'}
Z_{21}({\bf n, n'})\\
b^p_3({\bf n}) &=& n^p L^p_{31}({\bf n}) - n^p \int_{4\pi} d{\bf n'}
Z_{31}({\bf n, n'})\\
b^p_4({\bf n}) &=& n^p L^p_{41}({\bf n}) - n^p \int_{4\pi} d{\bf n'}
Z_{41}({\bf n, n'})
  \end{eqnarray}
This equations show, that the first column of extinction and phase
matrix are needed. So it is obligatory to call the methods
\funcindex{calc\_ext\_matrix} and 
\funcindex{calc\_phase\_matrix} before calling
\funcindex{calc\_abs\_vector}. 

As only the first column of the phase matrix  is required here it is
extracted from \wsvindex{phase\_matrix} and  stored in \\
\wsvindex{phase\_column} $\in$ (\IPart, \ScaZa, \ScaAa, \ScaZa', \ScaAa',\IAmp).

The finally calculated absorption coefficent vector is stored in the
variable
\begin{center}
\wsvindex{absorption\_vector} $\in$ (\IPart, \ScaZa, \ScaAa, \ScaZa',
\ScaAa',\aInd{b}).  
\end{center}
\end{itemize}
Note that all the variables described above are
independent of frequency. 

\levelb{Methods and Agendas for the RT inside the Cloud Box}
%=================================================================
\label{sec:scattering:scat_meth_rt}

The scattering radiative transfer calculation is performed
independently from the clear sky radiative transfer calculation. It is
limited only to a small part of the atmosphere defined by the cloud box.

\levelc{Define the cloud box}
%==========================================
\label{sec:scattering:cloudbox}

The concept of a cloud box is explained in
\ref{sec:fm_defs:cloudbox}. It is activated by setting the variable
\wsvindex{cloudbox\_on} to 1. The
limits of the cloud box are stored in \wsvindex{cloudbox\_limits}
which in an array of indices, containing tho upper and the lower
pressure index, the upper and lower latitude index and the upper and
lower longitude index. If the dimensionality is 1D, the latitude and longitude
indices are missing.

As inside a cloud the optical depth is much larger than in the clear
atmosphere it is useful to define higher solution grids inside the
cloud box. These are the pressure grid
\wsvindex{scat\_p\_grid}, the latitude grid  \wsvindex{scat\_lat\_grid}
and the longitude grid \wsvindex{scat\_lon\_grid}. The end points of
these grids must correspond to the corner points of the scattering
box. Furthermore angle grids have to be defined as the radiation
field depends on the direction. For this purpose we have defined the
variables \wsvindex{scat\_za\_grid} and \wsvindex{scat\_aa\_grid}.

The interface between the clear sky RT calculation and the cloudbox RT
calculation is the radiation field on the boundary of the scattering
box. The radiation field \StoVec inside the scattering box is stored
in the variables \wsvindex{scat\_i\_p}, \wsvindex{scat\_i\_lat} and
\wsvindex{scat\_i\_lon}. The dimension of  \wsvindex{scat\_i\_p} is for
example 
\begin{center}
  \artsstyle{scat\_i\_p} $\in$ (2(two surfaces), \ScaLat, \ScaLon, \ScaZa,
\ScaAa, \Frq).
\end{center}
The dimensions of the other variables can be found in the same manner.

\levelc{Scatterin Main Function}
%=====================================
\label{sec:scattering:main_function}

The scattering main function \wsfindex{ScatCalc} requires the clear sky radiation 
field on the cloud box boundary as input.
It loops over the frequencies and in each loop it executes the agenda 
\wsvindex{scat\_mono\_agenda} (see section \ref{sec:scattering:scat_mono_ag}).\\
Output of \wsfindex{ScatCalc} is the scattered radiation field on the
boundary also stored in the variables \wsvindex{scat\_i\_p}, \wsvindex{scat\_i\_lat} and
\wsvindex{scat\_i\_lon}.

\levelc{Agenda \wsfindex{scat\_mono\_agenda}}
%===================================
\label{sec:scattering:scat_mono_ag}

The iteration scheme outlined in \ref{sec:scattering:solution_rte} is
performed by \wsfindex{scat\_mono\_agenda} for one frequency specified
by the frequency index \artsstyle{scat\_f\_index}. 
The agenda consists of a number of methods which have to be executed in
the correct order.

\leveld{Generate extinction matrix and absorption vector}

Executing \wsfindex{gen\_scat\_lookup\_agenda}, which needs the
\artsstyle{scat\_f\_index} and \artsstyle{scat\_part\_types}, a vector containig the indices for the
particle types, as input, produces 
the internal lookup tables for the
single scattering properties of the cloud. It uses the functions   
\funcindex{calc\_ext\_matrix}, \funcindex{calc\_phase\_matrix}
and  \funcindex{calc\_abs\_vector}
described in section \ref{sec:scattering:gen_ext_abs}. 
Output variables of \wsfindex{gen\_scat\_lookup\_agenda} are the
extinction matrix
\wsvindex{scat\_ext\_mat}, the phase matrix \wsvindex{scat\_pha\_mat}
and the absorption vector \wsvindex{absorption\_vector}.

\leveld{Initial field}

Input to the agenda is the radiation field on the cloud
box boundary.
Using a linear 3D interpolation scheme, we obtain a
radiation field on all grid points inside the cloud box.
This can be taken as a first guess for the iteration. The method
\funcindex{i\_fieldInterpolateFromBoundary} performs this
interpolation. It picks only the monochromatic radiation field
corresponding to the actual frequency out of the variables
\wsvindex{scat\_i\_p}, \wsvindex{scat\_i\_lat} and
\wsvindex{scat\_i\_lon}. Output of the method is the initial field 
which is stored in the internal variable
\begin{center}
  artsstyle{i\_field} $\in$ (\ScaP, \ScaLat, \ScaLon, \ScaZa,
\ScaAa). 
\end{center}

\leveld{Iteration}

The method \funcindex{i\_fieldCalc} performes the following steps 
until convergence is obtained. As input the initial field stored in
\artsstyle{i\_field} is required. Furthermore the method needs two
agendas, the \wsfindex{scattered\_field\_agenda} and the
\wsfindex{i\_field\_update\_agenda}.

\begin{itemize}
\item \artsstyle{i\_field} is copied to \artsstyle{i\_field\_old}.
\item The scattered field is calculated using
  \wsfindex{scattered\_field\_agenda}
    (see sections \ref{sec:scattering:scat_field_ag} and
    \ref{sec:scattering:scat_int}).
\item Calculate the new radiation field using
  \wsfindex{i\_field\_update\_agenda}
 (see sections \ref{sec:scattering:RT_ag} and
  \ref{sec:scattering:RTE}).
\item Do the convergence test (see sections
  \ref{sec:scattering:conv_method} and \ref{sec:scattering:conv})
\end{itemize}
The solution of the radiative transfer equation is returned as output
using the variable \artsstyle{i\_field}. 

\leveld{Generate full radiation field inside cloud box}
The variable \artsstyle{i\_field} contains the radiation field only
for one frequency. So there has to be a method which puts
\artsstyle{i\_field} into the variables \wsvindex{scat\_i\_p},
\wsvindex{scat\_i\_lat} and \wsvindex{scat\_i\_lon}. The method
\funcindex{scat\_iPutIn} is doing this task.
Of course it has to use the frequency index to know where
to put \artsstyle{i\_field}.
 

\levelc{Agenda to compute the Scattered Field}
%=========================================
\label{sec:scattering:scat_field_ag}

\levelc{Agenda for RT with fixed Scattering Integral}
%=====================================
\label{sec:scattering:RT_ag}
The \wsfindex{i\_field\_update\_agenda} needs the variables
\artsstyle{i\_field\_old}, \artsstyle{scattered\_field}, \wsvindex{scat\_ext\_mat}
and  \wsvindex{absorption\_vector} as input.\\ 
According to equation(\ref{eq:scattering:RTE_sol}) it computes the new
radiation field. The agenda starts with a loop over all grid points
inside the scattering box. Then for each grid point the new intensity
vector
 \StoVec  is obtained for all directions by calculating the
radiative transfer through the adjacent grid cells. We consider paths
starting at the grid point into all directions and ending at the next
intersection points with a grid cell boundary. The paths are calculated
using the method \ref{sec:ppaths_scat}.\\
 The intensity vector at
the intersection points \StoVec{0} is obtained by linear interpolation
between the grid points using the interpolation routine
\ref{sec:interpolation}. \\
Computing the matrix emponential is done using the method
\wsfindex{matrix\_exp} explained in section
\ref{sec:lin_alg:mat_exp}.\\
 The term
\ExtMat$^{-1}$[\AbsVec\Planck+\ScaInt] can be computed numerically  
performing a LU decomposition. This is implemented in the workspace
methods \wsfindex{ludcmp} and \wsfindex{lubacksub} which are described
in the sections \ref{sec:lin_alg:backsub} and
\ref{sec:lin_alg:lu_decomp}. \\
Output of the \wsfindex{i\_field\_update\_agenda} is the variable 
\artsstyle{i\_field} which contains the new radiation field at all
grid points in the cloudbox for all directions defined by 
\wsvindex{scat\_za\_grid} and \wsvindex{scat\_aa\_grid}.

\levelc{Convergence test method}
%====================================
\label{sec:scattering:conv_method}





 


%%% Local Variables: 
%%% mode: latex
%%% TeX-master: "uguide"
%%% End: 
