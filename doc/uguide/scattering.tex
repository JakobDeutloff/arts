%
% To start the document, use
%  \levela{...}
% For lover level, sections use
%  \levelb{...}
%  \levelc{...}
%
\levela{Scattering}
 \label{sec:scattering}

%
% Document history, format:
%  \starthistory
%    date1 & text .... \\
%    date2 & text .... \\
%    ....
%  \stophistory
%
\starthistory
   100502 & Created and written by Claudia Emde.\\
\stophistory


%
% Symbol table, format:
%  \startsymbols
%    ... & \artsstyle{...} & text ... \\
%    ... & \artsstyle{...} & text ... \\
%    ....
%  \stopsymbols
%
\startsymbols
\StoVec       & \artsstyle{i\_field}       & monochromatic intensity field\/ Stokes Vector\\
              & \artsstyle{i\_field\_old}  & needed for the iteration\\
i$_I$         & -                          & Stokes component \\
\ExtMat       & -                        & total extinction matrix \\
\SExMat        & \artsstyle{ext\_mat} & extinction matrix for a
single particle type\\
\AbsVec       & -                        & total absorption vector \\
\SAbVec       & \artsstyle{abs\_vec} &   absorption vector for a
single particle type\\
\SEfMat       & -                        & scattering efficiency
matrix\\
\PhaMat       & \artsstyle{pha\_mat} & phase matrix for a particle
type\\
\AmpMat       & \artsstyle{amp\_mat} & amplitude matrix\\
\PDir         & -                        & propagation direction \\
\Frq          & \artsstyle{f\_grid}       & frequency grid\\
\Tmp          & \artsstyle{t\_field}         & temperature field\\
\PDen         & \artsstyle{pnd\_field} & particle density field \\
\ScaInt       & \artsstyle{sca\_vec} & scattered vector,
evaluated scattering integral\\
\Prs         & \artsstyle{p\_grid} & pressure grid inside cloud
box\\
\Lat       & \artsstyle{lat\_grid} & latitude grid inside cloud
box\\
\Lon       & \artsstyle{lon\_grid} & longitude grid inside cloud
box\\
\ScaZa        & \artsstyle{scat\_za\_grid}  & zenith angle \\
\ScaAa        & \artsstyle{scat\_aa\_grid}  & azimuth angle  \\
              & \artsstyle{scat\_f\_index}  & frequency index\\
\Planck       & \artsstyle{a\_planck\_value} & Planck function
\label{symtable:scattering}
\stopsymbols
%=====================================================================


%=====================================================================
% Definition of new commands:
% ====================================================================
\newcommand{\DirFre} {(\PDir, \Frq, \Tmp)}
\newcommand{\DirFrePr} {\ensuremath{(\PDir, \PDir^\prime, \Frq, \Tmp)}}


\levelb{Vector radiative transfer equation (VRTE)}
%=====================================================================
\label{sec:scattering:general_rte}
 
The radiative transfer equation for a medium with thermal emission
comprising sparsely and randomly distributed, spherical or arbitrarily
oriented non-spherical particles is according to \cite{sree02}:
\begin{eqnarray}
\label{eq:scattering:RTE} 
     \frac{\DiffD \StoVec}{\DiffD s}(\PDir, \Frq) =
     -\ExtMat\DirFre\StoVec(\PDir, \Frq)+\AbsVec\DirFre
     \Planck(\Frq) \\ \nonumber
     +\int_{4\pi} \DiffD \PDir^\prime \SEfMat\DirFrePr
     \StoVec(\PDir^\prime, \Frq) 
\end{eqnarray} 
Formally it is an inhomogeneous four dimensional vector differential
equation for the Stokes vector \StoVec $(I,Q,U,V)$. To describe the
full state of an electromagnetic wave one could use the
electromagnetic field vector, which consists of two complex numbers.
This is the common notation in Maxwell's theory. An alternative
notation is the Stokes vector which is more convenient for
experimental purposes, as it consists of real numbers which can be
associated more directly with the measurements.  The first component
of the Stokes vector $I$ signifies the monochromatic flux or the
intensity of the radiation. The other components describe the
polarization state: the second and third components $Q$ and $U$
characterize the state of linear polarization and the last component $V$
characterizes the state of circular polarization.\\
In Equation (\ref{eq:scattering:RTE}) \PDir\ denotes the propagation
direction and \DiffD $s$ the path-length element along this direction.
The radiation field depends on position, propagation direction {\bf
  n}, and frequency \Frq.  \vspace{1ex} The first term of Equation
(\ref{eq:scattering:RTE}) corresponds to the extinction of radiation
determined by the extinction coefficient matrix \ExtMat . For
microwave radiation traveling through the atmosphere there is
extinction due to gaseous absorption, particle absorption and particle
scattering. Lambert's law states that the extinction process is
proportional to the amount of matter. Therefore \ExtMat\ can be
written as a sum of two matrices, the first for particle extinction (
\aExtMat{p}) and the second for gaseous extinction (\aExtMat{g}):
\begin{eqnarray}
  \ExtMat\DirFre &=&
  \aExtMat{p}\DirFre+\aExtMat{g}\DirFre
\end{eqnarray}

The particle extinction term can be written as a sum, each term
corresponding to one particle type:
\begin{eqnarray}
  \aExtMat{p}\DirFre = \sum_i \PDen_i \aSExMat{p}_i \DirFre
\end{eqnarray}

Here $\PDen_i$ is the particle number density of the {\sl i}th
particle type and $\aSExMat{p}_i\DirFre$ the particle extinction cross
section matrix of the {\sl i}th particle type.


The gaseous ectinction matrix is derived from the scalar gas absorption. As extiction by gas absorption does not cause any polarization effects, the off-diagonal elements of the gaseous extinction matrix have to be zero. Gaseous extinction is determined only by absorption, so the coefficients on the diagonal correspond to the gaseous absorption coeffient ({\bf FIXME: Why are the coefficients equal, may be only K11 = a, and all others 0 ?}):

\[
  \aExtMat{g}_{i,j}(\Prs, \Frq, \Tmp) = \left\{\begin{array}{r@{\quad:\quad}l}
       \alpha(\Prs, \Frq, \Tmp)  & i = j \\ 0 & i \not= j \end{array} \right. 
\]

Here \Prs and \Tmp are pressure and temperature. $\alpha$ is the scalar gaseous absroption coefficient.

The second term in Equation (\ref{eq:scattering:RTE}) describes the
thermal emission. \Planck\ is the Planck function and \AbsVec\ is the
absorption coefficient vector which can be written as
\begin{eqnarray}
  \AbsVec \DirFre  &=& \aAbsVec{p} \DirFre + \aAbsVec{g} \DirFre 
\end{eqnarray}
with \aAbsVec{p} and \aAbsVec{g} as the particle absorption
coefficient and the gaseous absorption coefficient, respectively.

The particle absorption term can also be written as a sum over all
particle types:
\begin{eqnarray}
  \aAbsVec{p}\DirFre = \sum_i \PDen_i \aSAbVec{p}_i \DirFre
\end{eqnarray}
Here $\aAbsVec{p}_i$ is the particle absorption cross section vector
of the {\sl i}th particle type.
The gaseous absorption vector can be written as follows:

\begin{eqnarray}
  \aAbsVec{g}(\Prs, \Frq, \Tmp)  = (\alpha(\Prs, \Frq, \Tmp), 0, 0, 0) 
\end{eqnarray}


The last term in Equation (\ref{eq:scattering:RTE}) is the scattering
source term. It is the amount of radiation which is scattered from all
directions \PDir$^\prime$ into the propagation direction \PDir.  The
scattering efficiency matrix \SEfMat\ is the sum of products of the
particle densities \PDen\ and the phase matrices \PhaMat
\begin{eqnarray}
\SEfMat(\PDir, \PDir', \Frq) = \sum_i{\PDen_i\PhaMat_i(\PDir,
  \PDir', \Frq)}.
\end{eqnarray}

Molecular scattering is neglected as it is not important in the
considered frequency range from 10 to 1000 GHz.


\levelb{Scalar radiative transfer equation}
%======================================================
\label{sec:scattering:scalar_rte}

If we know that the radiation field is unpolarized we may use the
scalar radiative transfer equation, which is easier to handle than the
vector equation. It can directly be derived from the general equation
provided that there are no non-spherical particles in the scattering
medium. If there are non-spherical particles the polarization state of
the radiation will be changed, in this case the medium is called
dichroistic.  For spherical particles the extinction matrix has only
elements on the diagonal and all of them have equal values. Only the
first component of the absorption vector will be nonzero.

Instead of taking the full phase matrix, commonly only its first
element, which is usually called the phase function, is used, for
aspherical particles as well as for spherical particles. This concept has no physical justification as
there are non-diagonal terms in the phase matrix.

Thus to write down the scalar radiative transfer equation the
extinction matrix in (\ref{eq:scattering:RTE}) is replaced by the
first element of the extinction matrix $K_{11}$, the scalar absorption
coefficient is replaced by the first element of the absorption vector
$a_1$ and the scalar scattering coefficient by the first element of
the scattering efficiency matrix $Y_{11}$:
\begin{eqnarray}
  \label{eq:scattering:scalar_rte}
\frac{\DiffD I}{\DiffD s} (\PDir, \Frq) = -K_{11}(\PDir, \Frq)I
(\PDir, \Frq) + a_1(\PDir, \Frq)\Planck(\Frq) + \\ \nonumber
 +\int_{4\pi} \DiffD \PDir^\prime Y_{11}\DirFrePr I(\PDir^\prime,
  \Frq)
\end{eqnarray}

\levelb{Iteration scheme for solving the vector RTE}
%===============================
\label{sec:scattering:solution_rte}

There exists no analytical solution for differential equations as the general
radiative transfer equation. A numerical solution can be obtained using for example an iterative method. The method implemented in ARTS is described in this section. 

\levelc{Basic definitions}
\begin{itemize}
\item{\bf Cloudbox}

It is not necessary to solve the VRTE in the whole atmosphere. In ARTS it is possible to define a region called cloudbox, in which cloud particles may exist. Only in this particular region the VRTE is solved, this saves computation time and memory. 
The cloudbox is defined by its corner points which have to correspond to atmospheric grid points (cp. Figure \ref{fig:scattering:cloudbox}). For a 3D calulation the user has to define lower and upper pressure limit (\aPrs{1} and \aPrs{2}), two latitude limits (\aLat{1} and \aLat{2}) and two longitude limits (\aLon{1} and \aLon{2}), the cloudbox concept is described mor detailed in section \ref{sec:fm_defs:cloudbox}.

\begin{figure}[htbp]
 \begin{center}
  \begin{minipage}[c]{0.65\textwidth}
   \begin{center}
    \includegraphics*[width=0.9\hsize]{Figs/scattering/cloudbox}
   \end{center}
  \end{minipage}%
  \begin{minipage}[c]{0.35\textwidth}
   \caption{2D cloudbox defined by lower and upper pressure limit (\aPrs{1} and \aPrs{2}) and two latitude limits (\aLat{1} and \aLat{2}).}
   \label{fig:scattering:cloudbox}
  \end{minipage}
 \end{center}
\end{figure}   

The aim is to calculate the radiation at all grid points inside the cloudbox for all viewing angles, this quantity we call the radiation field. 

\item{\bf Radiation field}

To describe the radiation field inside the cloudbox completely we need the Stokes vectors at all grid points. The Stokes vector depends not only on the position in the cloudbox but also on the propagation direction of the radiation (cp. equation \ref{eq:scattering:RTE}). Thus the radiation field is defined to be a set including the Stokes vectors at all grid points in the cloudbox, which are elements of the atmospheric grids, and for all propagation directions defined by special angular grids for the scattering calculation: 

\begin{eqnarray}
{\mathcal I} = \left\{ \StoVec \left( p_i, \alpha_i, \beta_i, \theta_i, \phi_i\right)\right\}   \quad
\forall \quad p_i &\in& p\_grid,\\
 \alpha_i &\in& lat\_grid, \\
 \beta_i &\in& lon\_grid, \\
 \theta_i &\in& scat\_za\_grid, \\
 \phi_i &\in& scat\_aa\_grid 
\end{eqnarray}

Of course the Stokes vector also depends on the frequency of the radiation. But in order to save computation memory, the scattering calculation is only done for one frequency at a time.

\end{itemize}


\levelc{First guess field}
%===========================================================================

The starting point of the iteration method is the first guess field  \aIFld{0}. This is partly determined by the boundary condition given by the radiation coming from the clear
sky part of the atmosphere traveling into the cloudbox. To take into
account this condition, the first guess field on the cloudbox boundary
has to be set to the clear sky field on the cloudbox boundary. 

Inside
the cloudbox any field can be chosen as a first guess. In order to minimize the number of iterations it should be close to the iterative solution. Our test calculations have shown, that taking the clearsky field as first guess is reasonable.


\levelc{First scattering integral field}
%============================================================================

  The next step is solving the scattering
  integral
\begin{eqnarray}
  \ScaInt^{(0)} = \int_{4\pi} \DiffD \PDir^\prime
  \SEfMat \StoVec^{(0)}
\label{eq:scat_int}
\end{eqnarray}
using the first guess initial field. For the integration we use again the angular grid defined for scattering calculations.

The integration has to be performed over all incident directions
$\PDir^\prime$ for each propagation direction \PDir{}. The evaluation
of the scattering integral is done for all grid points inside the
cloudbox. The result is the first guess scattered field \aSFld{0} which is a set of scatterd field vectors at all positions in the cloudbox for all directions:

\begin{eqnarray}
{\mathcal S}^{(0)} = \left\{ \bf{S}^{(0)} \left( p_i, \alpha_i, \beta_i, \theta_i, \phi_i\right)\right\}  \qquad  
&\forall& {\rm positions } (p_i,\alpha_i, \beta_i)\\
&\forall& {\rm directions } (\theta_i,\phi_i)
\end{eqnarray}


\levelc{Averaging of the coefficients of the VRTE}
%============================================================================

  In the following we are looking at one grid point inside the
  cloudbox into one specified direction.  We can substitute
  $\ScaInt^{(0)}$, the scattering integral vector calculated in the
  previous step at this point for the specified direction, into
  Equation (\ref{eq:scattering:RTE}), in order to get the VRTE for
  this point and direction:
\begin{eqnarray}
     \frac{\DiffD \StoVec^{(1)}}{\DiffD s} =
     -\ExtMat \StoVec^{(1)} + \AbsVec \Planck
     +\ScaInt^{(0)}
\label{eq:scattering:vrte_fs}
\end{eqnarray} 
Here $\StoVec^{(1)}$ is the Stokes vector at the considered grid point
for the specified direction.

\begin{figure}[htbp]
 \begin{center}
  \begin{minipage}[c]{0.65\textwidth}
   \begin{center}
    \includegraphics*[width=0.9\hsize]{Figs/scattering/average}
   \end{center}
  \end{minipage}%
  \begin{minipage}[c]{0.35\textwidth}
   \caption{Path from one grid point (black) to the intersection point with the next grid cell boundary. Multilinear intepolation is done from the grid points on the intersection point. The coefficients are averaged between the black grid point and the intersection point.}
   \label{fig:scattering:averaging}
  \end{minipage}
 \end{center}
\end{figure}   

As mentioned before, all optical properties are assumed to vary
linearly between the grid points. Equation (\ref{eq:scattering:vrte_fs}) is
formally a linear differential equation, which could be solved
analytically if the coefficients were constant.  To find a constant
approximation for the coefficients, we follow a propagation path from
the considered point into the specified direction until the path
intersects a grid cell boundary (cp. Figure \ref{fig:scattering:averaging}). By multi-linear interpolation we
obtain all coefficients and the temperature at the intersection point.
Also the scattering integral vector is obtained for the intersection
point by multi-linear interpolation of the first guess scattered
field. Now we approximate the coefficients (extinction matrix and
absorption vector) in Equation (\ref{eq:scattering:vrte_fs}) by taking the
average of the coefficients at the intersection point and the
coefficients at the considered point element-vise for each
matrix-/vector-element. The same averaging is done for the scattering
integral vector and the temperature. The average value of the
temperature is used to get the averaged Planck function.

Then Equation (\ref{eq:scattering:vrte_fs}) can be written as follows:
 \begin{eqnarray}
     \frac{\DiffD \StoVec^{(1)}}{\DiffD s} =
     -\bar{\ExtMat} \StoVec^{(1)} + \bar{\AbsVec} \bar{\Planck}
     +\bar{\ScaInt^{(0)}}
\label{eq:scattering:vrte_fs_av}
\end{eqnarray} 
Here $\bar{\ExtMat}$, $\bar{\AbsVec}$, $\bar{\Planck}$ and
$\bar{\ScaInt^{(0)}}$ are the averaged quantities.


\levelc{Radiative transfer with fixed scattered field and averaged coefficients}
%===================================================================

Equation (\ref{eq:scattering:vrte_fs_av}) can be solved analytically using the following matrix exponential approach:
 \begin{eqnarray}
 \label{eq:scattering:ansatz}
&   \StoVec^{(1)} = e^{-\bar{\ExtMat}s}{\bf C_1} + {\bf C_2}
\end{eqnarray}
Here ${\bf C_1}$ and ${\bf C_2}$ are constants which have to be determined. Substituting (\ref{eq:scattering:ansatz}) into (\ref{eq:scattering:vrte_fs_av}) gives the constant  ${\bf C_2}$:
\begin{eqnarray}
 \Rightarrow& -\bar{\ExtMat}e^{-\bar{\ExtMat}s}{\bf C_1} = -\bar{\ExtMat}e^{-\bar{\ExtMat}s}{\bf C_1} - \bar{\ExtMat} {\bf C_2}+ \bar{\AbsVec} \bar{\Planck} +\bar{\ScaInt}^{(0)}\\
\Rightarrow&  \bar{\ExtMat} {\bf C_2} = \bar{\AbsVec} \bar{\Planck} +\bar{\ScaInt}^{(0)}\\
\Rightarrow& {\bf C_2} =  \bar{\ExtMat}\Inv\left(\bar{\AbsVec} \bar{\Planck} +\bar{\ScaInt}^{(0)}\right)
 \label{eq:scattering:c1}
\end{eqnarray}
${\bf C_1}$ can be determined using the initial condition, which is the radiation at the intersection point ($s=0$) travelling towards the considered gridpoint:
\begin{eqnarray}
   \StoVec^{(1)}(s=0) &=& \StoVec^{(0)} {\rm(at\ intersection\ point)}\\
\rm{From\ Ansatz:} &&\\
\StoVec^{(0)} &=& {\bf C_1} +  \bar{\ExtMat}\Inv\left(\bar{\AbsVec} \bar{\Planck} +\bar{\ScaInt^{(0)}}\right)\\
\Rightarrow   {\bf C_1} &=& \bar{\StoVec^{(0)}} -  \bar{\ExtMat}\Inv \left(\bar{\AbsVec} \bar{\Planck} +\bar{\ScaInt^{(0)}}\right)
\label{eq:scattering:c2}
\end{eqnarray}
Substituting  (\ref{eq:scattering:c1}) and (\ref{eq:scattering:c2}) into Equation (\ref{eq:scattering:ansatz}) leads to the solution: 
\begin{eqnarray}
  \StoVec^{(1)} =  e^{-\bar{\ExtMat}s}\cdot\left(\StoVec^{(0)} -\bar{\ExtMat}\Inv\left(\bar{\AbsVec} \bar{\Planck} +\bar{\ScaInt}^{(0)}\right)\right) + 
 \bar{\ExtMat}\Inv\left(\bar{\AbsVec} \bar{\Planck} +\bar{\ScaInt}^{(0)}\right)
\end{eqnarray}
The can be resorted to the following form:
\begin{eqnarray}
\label{eq:scattering:vrte_sol}
 \StoVec^{(1)} =  e^{-\bar{\ExtMat}s} \StoVec^{(0)} + \left( {\bf E} -  e^{-\bar{\ExtMat}s}\right)  \bar{\ExtMat}\Inv\left(\bar{\AbsVec} \bar{\Planck} +\bar{\ScaInt}^{(0)}\right)
\end{eqnarray}

Here \IdnMat\ denotes the identity matrix.  $\StoVec^{(0)}$ is given
by the initial condition. It is the Stokes vector at the intersection
point which is obtained by interpolation of the first guess field.

This radiative transfer step calculation is done for all points inside
the cloudbox in all directions. The resulting Stokes Vectors (
$\StoVec^{(1)}$ for all points in all directions) form the first
iteration field \aIFld{1}:

\begin{eqnarray}
{\mathcal I}^{(1)} = \left\{ \bf{I}^{(1)} \left( p_i, \alpha_i, \beta_i, \theta_i, \phi_i\right)\right\}  \qquad  
&\forall& {\rm positions } (p_i,\alpha_i, \beta_i)\\
&\forall& {\rm directions } (\theta_i,\phi_i)
\end{eqnarray}


\levelc{Iterations}

  The first iteration field is used to evaluate the scattered field
  \ScaInt\ again at all grid points inside the cloudbox:
\begin{eqnarray}
  \ScaInt^{(1)} = \int_{4\pi} \DiffD \PDir^\prime
  \SEfMat \StoVec^{(1)} 
\end{eqnarray}
Now $\ScaInt^{(1)}$ is used as fixed scattering term in the radiative
transfer equation for the second iteration Stkes vectors $\StoVec^{(2)}$
\begin{eqnarray}
     \frac{\DiffD \StoVec^{(2)}}{\DiffD s} =
     -\bar{\ExtMat} \StoVec^{(2)} + \bar{\AbsVec} \bar{\Planck}
     +\bar{\ScaInt}^{(1)}.
\end{eqnarray} 
This equation contains already the averaged values and is valid for
specified position and direction.
 
The solution is given by:
\begin{eqnarray}
   \StoVec^{(2)} = e^{-\bar{\ExtMat} s}\cdot\StoVec^{(1)} + (\IdnMat - e^{-\bar{\ExtMat}
    s}) \bar{\ExtMat}\Inv (\bar{\AbsVec} \bar{\Planck} + \bar{\ScaInt}^{(1)})
\end{eqnarray}
Thus the Stokes vectors ( $\StoVec^{(2)}$ for all points in all
directions) form the second iteration field.

After that the scattering integral and higher order iteration fields
are calculated alternately. We can formulate a differential equation
for the $n$-th order iteration field. The fixed scattering integral
term in this equation is
 \begin{eqnarray}
  \ScaInt^{(n-1)} = \int_{4\pi} \DiffD \PDir^\prime
  \SEfMat \StoVec^{(n-1)}
\end{eqnarray}
and the differential equation for specified position and direction is
given by
\begin{eqnarray}
     \frac{\DiffD \StoVec^{(n)}}{\DiffD s} =
     -\bar{\ExtMat} \StoVec^{(n)} + \bar{\AbsVec} \bar{\Planck}
     +\bar{\ScaInt}^{(n-1)}.
\end{eqnarray} 
Thus the $n$-th order iteration is given by:
\begin{eqnarray}
   \StoVec^{(n)} = e^{-\bar{\ExtMat} s}\cdot\StoVec^{(n-1)} + (\IdnMat - e^{-\bar{\ExtMat}
    s}) \bar{\ExtMat}\Inv (\bar{\AbsVec} \bar{\Planck} + \bar{\ScaInt}^{(n-1)})
\end{eqnarray}


\levelc{Convergence test}

  After each iteration the convergence is checked. If
\begin{eqnarray}
|\StoVec^{(m)} \DirFre -  \StoVec^{(m-1)} \DirFre| < {\bf \epsilon}
\end{eqnarray}
for all points inside the cloudbox and all directions a solution to
the vector radiative transfer Equation (\ref{eq:scattering:RTE}) has
been found:
\begin{eqnarray}
{\mathcal I}^{(m)} = \left\{ \bf{I}^{(m)} \left( p_i, \alpha_i, \beta_i, \theta_i, \phi_i\right)\right\}  \qquad  
&\forall& {\rm positions } (p_i,\alpha_i, \beta_i)\\
&\forall& {\rm directions } (\theta_i,\phi_i)
\end{eqnarray}



\levelb{Iteration scheme for solving the scalar RTE}
%===============================
\label{sec:scattering:solution_rte_scalar}
\begin{itemize}
\item {\bf Scattering integral}
  
  In analogy to the scattering integral vector field the scalar
  scattering integral field is obtained:
\begin{eqnarray}
  S^{(0)}  = \int_{4\pi} \DiffD \PDir^\prime Y_{11} I^{(0)} 
\end{eqnarray}
As first guess field only the first Stokes component of the initial
field is needed.

\item{\bf Radiative transfer with fixed scattered field}
  
  The scalar radiative transfer equation
  (\ref{eq:scattering:scalar_rte}) with fixed scattering integral is
\begin{eqnarray}
  \label{eq:scattering:scalar_rte_scatint}
\frac{\DiffD I^{(1)}}{\DiffD s} = -K_{11} I^{(1)}
 + a_1 \Planck + S^{(0)}.
\end{eqnarray} 
Assuming constant coefficients this equation is solved analytically
after averaging extinction coefficient, absorption coefficient,
scattered vector and the temperature. The averaging procedure is done
analogously to the procedure described for solving the VRTE (see
Section \ref{sec:scattering:solution_rte}).  The solution of the
averaged differential equation is
\begin{eqnarray}
   \label{eq:scattering:scalar_rte_sol}
I^{(1)} = I^{(0)} e^{-\bar{K_{11}}s} + \frac{\bar{a_1}
  \bar{B} + \bar{S^{(0)}}}{\bar{K_{11}}}\left(1-e^{-\bar{K_{11}}s}\right)
\end{eqnarray}
where $I^{(0)}$ again obtained by interpolating the initial field.
$\bar{K_{11}}$, $\bar{a_1}$, $\bar{B}$ and $\bar{S^{(0)}}$ are the
averaged values for extinction coefficient, absorption coefficient,
Planck function and the scattered integral respectively.

Applying this equation leads to first iteration scalar intensity
field, consisting of the intensities $I^{(1)}$ at all points in the
cloudbox for all directions.

\item{\bf Iteration and Convergence}
  
  As the solution to the vector radiative transfer equation the
  solution to the scalar radiative transfer equation is found
  numerically by iterations.
  
  The convergence test for the scalar equation compares the values of
  the calculated intensities of two successive radiation fields.

\end{itemize}


\levelb{Scattering databases}
%============================================================
\label{sec:scattering:database}

Two databases which are required for scattering calculations
inside the cloudbox. One contains the optical properties of different
particle types and the other contains the corresponding particle
number density fields.

\levelc{Database for optical properties}
%============================================================
\label{sec:scattering:amp_mat_data}

The required quantities for the radiative transfer calculations are
the  extinction cross section matrix, the
absorption coefficient vector and the phase
matrix. All these quantities can be calculated for spherical,
cylindrical and spheroidal particles using the T-matrix
method. Applying Mishchenko's FORTRAN code, which is explained in
\citet{Mishchenko:98} and \citet{Mishchenko:00},   
the amplitude matrix can be calculated. The
amplitude matrix ${\bf S}$ is a 2x2 complex matrix, which linearly transforms 
the electric field vector components of the incident wave into the
electric field vector components of the scattered wave:  
\begin{eqnarray}
  \label{eq:ampl_matrix}
  \left[{E^{\rm sca}_{\theta}\atop E^{\rm sca}_{\phi}}\right] =
  \frac{\exp({\rm i} kR)}{R}{\bf S}({\bf n};{\bf
      n'};\epsilon_1,\epsilon_2,\epsilon_3)\left[{E^{\rm inc}_{\theta
          0}\atop E^{\rm inc}_{\phi 0}}\right] 
\end{eqnarray}
It depends on the directions of incidence  $(\theta,
\phi)$  and
scattering $(\theta', \phi')$ as  well as on size, morphology and composition of the
scattering material, furthermore on the orientation of the particles which
is specified by the Euler angles of rotation $\epsilon _i$. 
The equation above is valid for a monochromatic wave, generally the
amplitude matrix is also frequency dependent.

All information about the optical
properties of the scattering medium is contained in the amplitude
matrix. Extinction cross section
matrix, absorption cross section vector and phase matrix can be
obtained by simple additions or multiplications of amplitude matrix
components (see Sections
\ref{sec:scattering:ext_mat_spt}, \ref{sec:scattering:pha_mat_spt}  and
\ref{sec:scattering:abs_vec_spt}).
For this reason, it is sufficient to store only the amplitude matrices
in the scattering data base instead of storing all coefficients. 

For each particle type determined by the particle size, the particle
shape and the
orientation there exists the corresponding file in the database  containing the
amplitude matrix for this type.
So far we have only considered ice particles as we are mainly
interested in cirrus clouds.\\
The methods to read from the database allow to construct homogeneous
clouds as well as inhomogeneous clouds, which may consist of various
particle types. 
The data files are stored in XML-format. The
datatype of each file is a 6D gridded field (cp. Section FIXME
%\ref{sec:gridded_fields}
). 
The following example shows the datafile
for spherical particles having 200 $\mu$m equal volume sphere radius:

\begin{verbatim}
<?xml version="1.0"?>
<arts format="ascii" version="1">
<comment>
  ------------------------------------------
  Describtion of the particle:
  =========================================
  ice particle                                
  radius : 200 micrometer 
  shape: spherical          
  aspect ratio: 1
 ------------------------------------------- 
</comment>
<Array type="Tensor6" nelem="7">
<Tensor6 nvitrines="1" nshelves="1" nbooks="1" npages="1"
                                nrows="1" ncols="1">
  3.25E+11    // frequency grid, this file contains only one
</Tensor6>
<Tensor6 nvitrines="1" nshelves="19" nbooks="1" npages="1" 
                                 nrows="1" ncols="1">
  0.          
  10.
  20.         // incoming zenith angle grid
   .
   .
   .
 </Tensor6>
 <Tensor6 nvitrines="1" nshelves="1" nbooks="37" npages="1"
                                 nrows="1" ncols="1">
   0.
  10.
   .          // incoming azimuth angle grid   
   .
   .
 </Tensor6>
 <Tensor6 nvitrines="1" nshelves="1" nbooks="1" npages="19"
                                 nrows="1" ncols="1">
  0.
  10.
   .          // scattered zenith angle grid
   .
   .
 </Tensor6>
 <Tensor6 nvitrines="1" nshelves="1" nbooks="1" npages="1"
                                 nrows="37" ncols="1">
  0.
  10.
   .           // scattered azimuth angle grid
   .
   .
  </Tensor6>
 <Tensor6 nvitrines="1" nshelves="1" nbooks="1" npages="1"
                                 nrows="1" ncols="8"> 
  1.
  1.           // dummy grid for the amplitue matrix 
  .            // components
  .
  .
  </Tensor6>
 <Tensor6 nvitrines="1" nshelves="19" nbooks="37" 
                        npages="19" nrows="37" ncols="8">
   0.204E+03   0.102E+03  -0.382E-37   0.186E-37   0.000E+00  
                    0.000E+00   0.204E+03   0.102E+03
   0.201E+03   0.100E+03  -0.354E+02  -0.177E+02   0.354E+02 
                    0.177E+02   0.201E+03   0.100E+03
   .
   .          // the amplitude matrix data 
   .    
   .          // the order of the elements is:
   .          // S11_real, S11_imag, S12_real, S12_imag, 
   .          // S21_real, S21_imag, S22_real, S22_imag    
   .
   0.201E+03   0.100E+03  -0.348E+02  -0.174E+02   0.348E+02  
                    0.174E+02   0.201E+03   0.100E+03
   0.204E+03   0.102E+03  -0.234E-22  -0.446E-22   0.234E-22   
                    0.446E-22   0.204E+03   0.102E+03
</Tensor6>
</Array>
</arts>
\end{verbatim}

This example file shows that each file in the database contains
for each frequency, for each propagation
direction (\ScaZa, \ScaAa) and each scattered direction
 (\ScaZa$^\prime$, \ScaAa$^\prime$) the real
components of the amplitude matrix ($S^{real}_{11}$, $S^{real}_{12}$,
$S^{real}_{21}$, $S^{real}_{22}$) and the imaginary components
($S^{imag}_{11}$, $S^{imag}_{12}$,$S^{imag}_{21}$, $S^{imag}_{22}$).

\levelc{Database for particle number density fields}
%==========================================================
\label{sec:scattering:pnd_data}

A second database contains the particle number density fields. 
The datatype contained in each data file is a
3D gridded field. It contains three elements corresponding to 
pressure, latitude and longitude grid. The forth element contains the
data for the particle number density. Each chosen particle type 
has to be defined together with a
particle number density field.

\levelc{Reading the databases}
%===========================================================
\label{sec:scattering:read_data}

Two workspace methods are implemented for reading the database. These
functions are used to define the clouds, the horizontal and vertical
extension and the particle distribution inside the cloud. They
are called \wsmindex{ParticleTypeInit} and \wsmindex{ParticleTypeAdd}. 

The workspace variable containing the amplitude matrix data is
\wsvindex{amp\_mat\_raw} which is an array of gridded fields. The
array contains for each particle type one gridded field having the
same dimensions as the files in the database. Therefore the first six
elements of a gridded field contain the
frequency grid, the angular grids and a dummy grid corresponding
to the number of
elements of the amplitude matrix (i.e. 8) and the seventh element 
contains the data itself.

The analogous workspace variable for the particle number density
field is called \wsvindex{pnd\_field\_raw} which as well is an array of
gridded fields containing one element for each particle type. The
first three elements of each gridded field contain the pressure, the
latitude and the longitude grids and the fourth element the particle
number density data. 

The workspace method \wsmindex{ParticleTypeInit} is used for initializing
\wsvindex{amp\_mat\_raw} and \wsvindex{pnd\_field\_raw}. 
The method \wsmindex{ParticleTypeAdd} puts the data for one particle
type into \wsvindex{amp\_mat\_raw} and \wsvindex{pnd\_field\_raw}. The
filenames are passed into the methods by keywords which are specified
in the control file.

The following example shows a part of a control file where
\wsvindex{amp\_mat\_raw} and \wsvindex{pnd\_field\_raw}  are created
for two particle types:

\begin{verbatim}

# Initialize variables
# --------------------------------------------
ParticleTypeInit{}

# Add spherical particles, size 200 mirons
# --------------------------------------------
ParticleTypeAdd{
        filename_amp_mat = "sph_200_ampmat.xml" 
        filename_pnd_field = "sph_200_pnd.xml"
        }       

# Add cylindrical particles, size 100 mirons
#---------------------------------------------
ParticleTypeAdd{
        filename_amp_mat = "cyl_100_ampmat.xml" 
        filename_pnd_field = "cyl_100_pnd.xml"
        }   

\end{verbatim}
 
There is no convention for the filenames. All needed information about the particles is included in the data files.


\levelb{Radiative transfer implementation in the cloudbox}
%=================================================================
\label{sec:scattering:rt_cloudbox}


\levelc{Scattering main function}
%=====================================
\label{sec:scattering:main_function}
For a scattering calculation only the workspace method \wsmindex{ScatteringMain} has to be executed in the control file. 
But before executing the main fuction, agendas required for a scattering calculation must be defined. 
The following example shows a part of a controlfile where a scattering calculation is performed:
\begin{verbatim}

#===========================================================
# Agendas for calculating the sigle scattering properties
# from the amplitude matrix data:
#===========================================================

  AgendaSet(spt_calc_agenda){
      pha_mat_sptCalc{}
      ext_mat_sptCalc{}
      abs_vec_sptCalc{}
  }

  AgendaSet( abs_vec_agenda ) {
      abs_vec_partCalc{}
      abs_vec_gasExample{}
      abs_vecCalc{}
  }

  AgendaSet( ext_mat_agenda ){
        ext_mat_partCalc{}
        ext_mat_gasExample{}
        ext_matCalc{}
  }

#===========================================================
# Define the method for the radiative transfer calculation:
#===========================================================

  AgendaSet( scat_rte_agenda ) { stokes_vecGeneral{}}

#===========================================================
# Define the method for the convergence test:
#===========================================================
  AgendaSet( convergence_test_agenda) {
        convergence_flagAbs{
        epsilon = [1e-17, 1e-18, 1e-18, 1e-18] 
        # epsilon = [1e-16]
   }
  
#==========================================================
# Methods for monochromatic scattering calculation:
#==========================================================
  AgendaSet(scat_mono_agenda){
  i_fieldSetClearsky{}
  WriteXML(i_field){"initial_field.xml"}        
  i_fieldIterate{}
  scat_iPut{}
 }
                
#==========================================================
# Execute the scattering main function:
#==========================================================       
        
  ScatteringMain{}

\end{verbatim}

All listed agendas are described in detail in the next sections.\\
\vspace*{1ex}

The first step in the scattering main function is the calculation of the clearsky field on the cloudbox boundary using the method \wsmindex{CloudboxGetIncoming}. 

Then it  executes the agenda 
\wsvindex{scat\_mono\_agenda}(see Section \ref{sec:scattering:scat_mono_ag})
 inside a loop over all frequencies defined in \wsvindex{f\_grid}.

\levelc{Numerical grids and fields}
%====================================================================
\label{sec:scattering:grids}
The calculations inside the cloudbox are performed on the common 
atmospheric grids: the pressure grid
\wsvindex{p\_grid}, the latitude grid  \wsvindex{lat\_grid}
and the longitude grid \wsvindex{lon\_grid}.
For calculation the gaseous absorption and  also \wsvindex{t\_field} and \wsvindex{vmr\_field} are required. More atmospheric grids and fields (cp. section  \ref{sec:fm_defs:grids}) are needed for the ray tracing.
Furthermore anglular grids are required for the scattering calculations  
as the radiation field, the scattering efficiency matrix etc. depend
on the propagation and incident direction of the radiation. For this purpose  the workspace  
variables \wsvindex{scat\_za\_grid} and \wsvindex{scat\_aa\_grid}
have been defined in the control file. They hold a zenith angle grid and a azimuthal angle grid. These angles are defined in the same manner as the angles which describe the viewing direction of the sensor (cp. section \ref{sec:fm_defs:los}).
The user has to find a compromise. The grids should not be too coarse for accuracy reasons but they also can not be very fine as there is only limited random acces memory available.
The grids do not neccesarily have to be equidistant. It is reasonble to take a finer resolution around zenith angle equal 90� because here a high variability of the radiation field is found. The zenith angles up to an angle a bit higher than 90� correspond to uplooking angles and the angles above correspond to downlooking angles. Of course the intensity of the radiation coming from below is much higher than the intensity of the radiation coming from above as in the lower atmosphere the density is much higher.\\
\vspace{1ex}
The radiation field in the cloudbox is stored in the workspace variable 
 \wsvindex{i\_field} which has the dimension:
\begin{center}
 \wsvindex{i\_field} = \wsvindex{i\_field} (\Prs, \Lat, \Lon, \ScaZa,
\ScaAa, i$_I$). 
\end{center}
It has to be noted that \wsvindex{i\_field} is defined only in the cloudbox which means that its size is   
\[ \begin{array}{rl}
 N(\artsstyle{i\_field}) =& [\artsstyle{cloudbox\_limits}[1] - \artsstyle{cloudbox\_limits}[0] + 1,\\
              &\artsstyle{cloudbox\_limits}[3] -\artsstyle{cloudbox\_limits}[2] + 1, \\
              &\artsstyle{cloudbox\_limits}[5] -\artsstyle{cloudbox\_limits}[4] + 1,  \\
              & N_{za}, N_{aa}, N_{I} ] 
\end{array}\]
The first three numbers give the sizes of the pressure grid, latitude grid
and longitude grid respectively inside the cloud box. $N_{za}$ and
$N_{aa}$ are the size of the zenith and azimuth angle grids and
$N_{I}$ is the Stokes dimension.  


\levelc{Interface between cloudbox and clearsky}
%================================================================
\label{sec:scattering:interface}
The interface between the clearsky part and the cloudbox part of the radiative transfer calulation is the cloudbox boundary. The scattering calculation requires the incloming clearsky radiation field on the boundary as initial condition and the clearsky part needs the outcoming radiation field on the boundary, which is propagated from there to the sensor.
The interface variables are  \wsvindex{scat\_i\_p}, \wsvindex{scat\_i\_lat} and
\wsvindex{scat\_i\_lon}. They have the dimensions 
\begin{center}
  \artsstyle{scat\_i\_p} = \artsstyle{scat\_i\_p} (\Frq, 2(two surfaces), \Lat, \Lon, \ScaZa, \ScaAa, i$_I$)\\
 \artsstyle{scat\_i\_lat} = \artsstyle{scat\_i\_lat} (\Frq, \Prs, 2(two surfaces), \Lon, \ScaZa,
\ScaAa, i$_I$ )\\
 \artsstyle{scat\_i\_lon} = \artsstyle{scat\_i\_lon} (\Frq, \Prs, \Lat, 2(two surfaces), \ScaZa,
\ScaAa, i$_I$).
\end{center}
where \Frq\ is the frequency, \Prs\ the pressure, \Lat\ the latitude,
\Lon\ the longitude, \ScaZa\ and \ScaAa\  the zenith of the propagation
the azimuthal angles of the propagation direction respectively and
i$_I$ is the Stokes component. 
In 3D geometry the variable  \wsvindex{scat\_i\_p} for example has the
size:
\begin{center}
  N(\artsstyle{scat\_i\_p}) = $[N_\Frq, 2, N_\Lat, N_\Lon, N_\ScaZa,
  N_\ScaAa, N_I]$
\end{center} 
 
There are no special workspace variables for 1D calculations. The
variables which are not needed, e.g. \wsvindex{scat\_i\_lat} and
\wsvindex{scat\_i\_lon}, are still in the workspace but they are
empty. For the interface only \wsvindex{scat\_i\_p} is needed and its
size is in the 1D case
\begin{center}
N( \artsstyle{scat\_i\_p} ) = $[N_\Frq, 2, 1, 1,  N_\ScaZa, 1,  N_I]$
\end{center}  
Inside the scattering box we need to specify only  
\wsvindex{scat\_za\_grid}. 

As interface between the clear sky calculation and the cloud box calculation only 
  \begin{center}
  \artsstyle{scat\_i\_p} = \artsstyle{scat\_i\_p} (\Frq, 2(two surfaces), 1 , 1, \ScaZa,
1, i$_I$)
\end{center}
is required. The other interface variables are empty. 

\leveld{Clearsky field on cloudbox boundary}
The method \wsmindex{CloudboxGetIncoming} calculates
 for each grid point on the cloudbox boundary in each propagation direction
   the radiation field using the method \wsmindex{RteCalc}, which 
   performs a clearsky radiative transfer calculation.
Output of this method are the interface variables listed above.

\leveld{Scattered field on cloudbox boundary} 
The scattered field on the cloudbox boundary for given position and direction is obtained by using the method \wsmindex{CloudboxGetOutgoing}. The method performs an interpolation from the interface variables on the requested position and direction.


\levelc{Solution of the monochromatic VRTE}
%========================================================================
\label{sec:scattering:scat_mono_ag}

The method  \wsmindex{ScatteringMain} executes \wsvindex{scat\_mono\_agenda}
for each frequency defined in  \wsvindex{f\_grid}. The frequency index \wsvindex{scat\_f\_index} is used in the agenda as the frequncy for which the calculations are done has to be known. 
The agenda is a list of methods which solve the monochromatic VRTE and it must be defined in the control file before executing \wsmindex{ScatteringMain} for examle in the following way:

\begin{verbatim}
AgendaSet(scat_mono_agenda) {
  i_fieldSetConst{}
  i_fieldIterate1D{}
  scat_iPut{}
}
\end{verbatim}

\noindent
The methods used by this agenda are described below.

 
\leveld{First guess field}
%=======================================================================

The workspace method \wsmindex{i\_fieldSetClearsky} 
uses a linear 3D interpolation scheme to obtain the 
radiation field on all grid points inside the cloud box from the clear
sky field on the cloudbox boundary.
This can be taken as a first guess for the iterative solution method
of the RTE.  The method picks only the monochromatic radiation field
corresponding to the actual frequency out of the variables
\wsvindex{scat\_i\_p}, \wsvindex{scat\_i\_lat} and
\wsvindex{scat\_i\_lon}. Output of the method is the initial field 
stored in the workspace variable \wsvindex{i\_field}. 
The workspace method \wsmindex{i\_fieldSetClearsky} is implemented in such
a way that the method should be able to adapt itself for 1D or 3D
atmosphere.  Only the 1D atmosphere case is in proper shape now. In
this case the main input is the interface field \artsstyle{scat\_i\_p},
the pressure grid \artsstyle{p\_grid}, the frequency grid
\artsstyle{f\_grid} and frequency index \artsstyle{scat\_f\_index} and
of course the \artsstyle{cloudbox\_limits}.  An interpolation of
\artsstyle{scat\_i\_p} which is defined only at the
\artsstyle{cloudbox\_limits} to all points inside the cloud box is
done to get the initial field, \artsstyle{i\_field}. \\

\vspace*{1ex}
Another method to set the first guess field is \wsmindex{i\_fieldSetConst}. Using this method the user can assign a constant value for each Stokes component of the first guess field. On the boundary of course the clearsky field has to be taken, as this is the boundary condition for the scattering calculation.
Setting the first guess field to a constant unrealistic value is useful for testing the solution. A calculation can be done with different first guess fields. If the radiation field converges each time towards the same solution field, it can be assumed that the solution field is the correct one.


\leveld{Iteration}
%=====================================================================
The method \wsmindex{i\_fieldIterate} solves the VRTE using the
iterative method.  The function has
included switches to adapt automatically to the atmospheric
dimensionality specified
in the workspace variable \wsvindex{atmosphere\_dim}. Note that only
1D or 3D scattering calculations are possible. 
The following steps are performed in each iteration until the solution
of the radiation field converges.

\begin{itemize}
\item \artsstyle{i\_field} is copied to \artsstyle{i\_field\_old}.
\item The scattered field is calculated using the method
  \wsmindex{scat\_fieldCalc}
    (see Sections \ref{sec:scattering:sca_fieldCalc} and
    \ref{sec:scattering:solution_rte}).
\item Calculate the new radiation field using\\
  \wsmindex{i\_fieldUpdate1D} or \wsmindex{i\_fieldUpdate3D}.
 (see Sections \ref{sec:scattering:RT_methods} and
  \ref{sec:scattering:solution_rte}).
\item Do the convergence test (see Sections
  \ref{sec:scattering:conv_method} and \ref{sec:scattering:solution_rte}).
\end{itemize}
The solution of the radiative transfer equation is returned as output
using the variable \artsstyle{i\_field}. 


\levelc{Computation of the scattered field}
%=========================================
\label{sec:scattering:sca_fieldCalc}
The method \wsmindex{scat\_integralCalc} does exactly what is
described in Section 
\ref{sec:scattering:solution_rte}.  The output of this method is
\wsvindex{scat\_field} which has the same dimension as
\artsstyle{i\_field}.  This is generated by integrating the product of
intensity field \artsstyle{i\_field} and phase matrix
\artsstyle{pha\_mat} over all incident zenith and azimuth angles.  In
addition to \artsstyle{i\_field}, the phase matrix for single 
particle \artsstyle{pha\_mat\_spt}, the particle number density field
\artsstyle{pnd\_field} are inputs to this method.  They are used
to calculate the phase matrix \artsstyle{pha\_mat} by calling the
method \artsstyle{pha\_mat\_partCalc}\ (see Section
\ref{sec:scattering:pha_mat} ). \\ 
The integration method used is the trapezoidal integration implemented
by the function \artsstyle{AngIntegrate\_trapezoid} in the file
\artsstyle{math\_funcs.cc}.  This explains why we need this
computation of scattered field to be
put in as an agenda since there can be different methods by which we
can perform the integral. 
The output \artsstyle{scat\_field} is used in the
methods \artsstyle{i\_fieldUpdate1D} and \artsstyle{i\_fieldIterate}.


\levelc{Solving the VRTE with fixed scattering integral}
%==================================================================
\label{sec:scattering:RT_methods}

\leveld{1D atmosphere}
%===================================================================
For a 1D atmosphere the method  \wsmindex{i\_fieldUpdate1D} is used to
evaluate the RT Equation (\ref{eq:scattering:vrte_fs}) and update the
radiation field \wsvindex{i\_field} for each iteration. 
The function loops over pressures given in \wsvindex{p\_grid} and
propagation directions given in \wsvindex{scat\_za\_grid}.  
The lower and upper indices for the loop over pressures are passed
into the function by the workspace variable
\wsvindex{cloudbox\_limits}. Inside the loops the following steps are
performed:
\begin{itemize}
\item Calculate the single scattering properties:
  \begin{itemize}
  \item Extinction coefficient matrix: Execute \wsvindex{ext\_mat\_agenda} (see
    Section \ref{sec:scattering:ext_mat_agenda}).
  \item Scattering efficiency matrix: Execute \wsvindex{sca\_matCalc} (see
    Section \ref{sec:scattering:pha_mat}).
  \item Absorption coefficient vector:  Execute \wsvindex{abs\_vec\_agenda} (see
    Section \ref{sec:scattering:abs_vec_agenda}).
  \end{itemize}
\item A  propagation path starting at the current point specified by
  the internal variable \artsstyle{p\_index} is initialized. Its
  direction is specified by \artsstyle{scat\_za\_index}. Using the
  \wsvindex{ppath\_step\_agenda} (cf. Section
  \ref{sec:ppath:stepcalc}) 
  the intersection point with the next
  layer is determined as well as the pathlength d{\bf s} from the starting
  point to the intersection point.  
\item The coefficients are calculated on the layer below and above the considered grid point.
\item The coefficients are averaged for the traversed layer. On the cloudbox boundary a special treatment is required: The intensity coming from the clearsky atmosphere is known and does not change. That means that no radiative transfer step calculation is done for propagation directions from outside the cloudbox.
\item Solve VRTE (\ref{eq:scattering:vrte_fs_av})  with averaged coefficients using methods to be defined in 
  \wsvindex{scat\_rte\_agenda} (see Section \ref{sec:scattering:rte_step} ).
\end{itemize}



\levelc{Single scattering properties}
%==================================================================
\label{sec:scattering:single_scat_prop}

\leveld{Method which generates \wsvindex{amp\_mat} from
  \artsstyle{amp\_mat\_raw}}
%==================================================================
\label{sec:scattering:gen_ampmat}

As we know from the previous section \ref{sec:scattering:read_data} 
the workspace variable containing the amplitude matrix data is 
\artsstyle{amp\_mat\_raw} which is an array of gridded
fields. The amplitude matrix data given by the seventh element of the
workspace variable \artsstyle{amp\_mat\_raw} is calculated for
frequencies given by the first element  and angles given by the
second, third,  fourth and fifth elements.  The method
\wsmindex{amp\_matCalc} takes \artsstyle{amp\_mat\_raw} as input and
interpolates the amplitude matrix data onto the respective frequency
and angular grids necessary for the calculation.  The output of this
method is \artsstyle{amp\_mat}.  The size of \artsstyle{amp\_mat} is  
[$N_{pt}, N_{za}, N_{aa}, N_{za}, N_{aa}, 8$]. Here $N_{pt}$ gives
the number of particle types considered for the calculation which is
specified from the input \artsstyle{amp\_mat\_raw}. $N_{za}$ gives
the number of zenith angles considered for scattering calculation 
which is specified by the input \wsvindex{scat\_za\_grid}, and $N_{aa}$
gives the number of azimuth angle considered for scattering
calculation which is specified by the input \wsvindex{scat\_aa\_grid}.
The workspace variable \artsstyle{amp\_mat} is calculated for one
frequency specified by the input variables \artsstyle{f\_grid} and
\artsstyle{scat\_za\_grid}. This is in compliance with our radiative
transfer scheme which has frequency as its outermost loop. 

Chapter \ref{sec:interpolation} has a more
detailed documentation of the different interpolation schemes
implemented in ARTS. A couple of examples are also discussed there.
In this case we interpolate the seventh element of
\artsstyle{amp\_mat\_raw} which is a Tensor6 having size
[$N_{f}, N_{za}, N_{aa}, N_{za}, N_{aa}, 8$] from one gridded field to
another.  This is a green type interpolation of all columns of this
Tensor6.  The interpolation is done simultaneously for the 5 dimensions.
The first step is to set up the grid position arrays by calling the
function \artsstyle{gridpos}.  The output of this function is the grid
positions for the new grid stored in ArrayOfGridPos and the inputs are
the original grid and the new grid where we want to have the
interpolated values. Interpolation weight tensors can be computed by
the function called \artsstyle{interpweights}.  This computes the
interpolation weights simultaneously for all the five dimensions
simultaneously. For this step also we do not need the actual fields,
just the grid positions. Now, the last step is to do a green
interpolation for all columns of the Tensor6.  This is by calling the
function \artsstyle{interps} which takes in as input the
gridpositions, interpolation weights and the source field and output
is the new field.  


\leveld{Agenda for calculating the extinction matrix}
%==================================================================
\label{sec:scattering:ext_mat_agenda}

The agenda \wsvindex{ext\_mat\_agenda} calculates the total extinction
matrix \ExtMat{}
for each grid point. This is the physical extinction matrix, that means
that it includes gaseous extinction and particle
extinction.\\
Output is the variable \wsvindex{ext\_mat} which is given
as a $N_I$x $N_I$ Matrix, i.e. one dimension for each Stokes component. If the
Stokes dimension is only one, only the element $\ExtMat_{11}$ will be
nonzero, it corresponds to the scalar extinction cross section. \\
As input for calculating the particle extinction the following
variables are required:
\wsvindex{pnd\_field}, i.e. the particle number density, and 
\wsvindex{ext\_mat\_spt}, which contains the extinction cross section
matrices for each chosen particle type. How this variable is generated
is explained in Section \ref{sec:scattering:ext_mat_spt}. For calculating the
gaseous extinction the volume mixing ratio (\wsvindex{vmrs}), the
pressure (\wsvindex{p\_grid}), the temperature  (\wsvindex{t\_field}) and the
propagation direction given by zenith angle (\wsvindex{scat\_za\_grid}) and
azimuth angle (\wsvindex{scat\_aa\_grid}) are needed.\\
The agenda can for example be defined in the following manner:

\vspace{1ex}
\begin{minipage}{0.9\hsize}
\begin{verbatim}
AgendaDefine(ext_mat_agenda)
{
ext_mat_partCalc{} // Function to calculate particle
                   // extinction.
ext_mat_gasCalc{}  // Function to calculate gaseous
                   // extinction. 
ext_matCalc{}      // Function which sums up
                   // particle extinction and 
                   // gaseous extinction.
}
\end{verbatim}
\end{minipage}

\vspace{2ex}
\noindent
The workspace methods \wsmindex{ext\_mat\_partCalc} and
\wsmindex{ext\_mat\_gasCalc} are explained in Sections
\ref{sec:scattering:ext_mat_part} and \ref{sec:scattering:ext_mat_gas}
respectively.

\levele{Generate particle extinction matrix:}
%==================================================================0
\label{sec:scattering:ext_mat_part}

The workspace method \wsmindex{ext\_mat\_partCalc} gives as output the
particle extinction matrix \aExtMat{p} stored in the variable
\wsvindex{ext\_mat\_part} which has the chosen Stokes dimension. So in
the most general case it is a 4x4 matrix.\\
First input is the workspace variable \wsvindex{ext\_mat\_spt} (see Section
\ref{sec:scattering:ext_mat_spt}). Second input is the
local particle number density field for all particle types
\artsstyle{pnd\_field}, which
 is a Tensor4 with dimension 
[$N_{pt}, N_{p}, N_{lat}, N_{lon}$]. Furthermore the  workspace
variables
\wsvindex{atmosphere\_dim}, \wsvindex{scat\_p\_index},
\wsvindex{scat\_lat\_index} and \wsvindex{scat\_lon\_index} are
required as input. \\
The method sums up the extinction matrices for all particle types
weighted with the particle number density:
\begin{eqnarray}
  \aExtMat{p} = \sum_i \PDen_i \aSExMat{p}_i 
\end{eqnarray}


\levele{Generate gaseous extinction matrix:}
%===============================================================
\label{sec:scattering:ext_mat_gas}

\levele{Generate extinction matrix for single particle types:}
%==============================================================
\label{sec:scattering:ext_mat_spt}

From the database for single scattering properties (see Section 
\ref{sec:scattering:database}) the extinction matrices for each
particle type can be calculated using the workspace method
\wsmindex{ext\_mat\_sptCalc}.\\
Output is a tensor including local extinction matrices for each
particle type, so it is of the dimension [$N_{pt}, N_{I}, N_{I}$],
where $N_{I}$ denotes the Stokes dimension.\\
As input variables it needs the amplitude matrix (\wsvindex{amp\_mat})
and the propagation direction specified by the angular grid indices
\wsvindex{scat\_za\_grid} and \wsvindex{scat\_aa\_grid}. Furthermore the
frequency is required. It is passed into the function by the workspace
variables \wsvindex{scat\_f\_index} and \wsvindex{f\_grid}.
\wsvindex{amp\_mat} is
obtained using the reading routine \wsmindex{get\_amp\_fromDb} which is
explained in Section \ref{sec:scattering:database}.


The extinction cross section matrices \SExMat{} are
calculated for all particle types chosen in the control
file. The following formulas are used:
\begin{eqnarray}
  \label{eq:gen_ext_matrix}
  L^p_{jj}({\bf n}) &=& \frac{c}{\nu} \Im[S_{11}({\bf n},{\bf
    n})+S_{22}({\bf n},{\bf n})], \quad j=1,...,4 \\
  L^p_{12}({\bf n}) = L^p_{21}({\bf n}) &=& \frac{c}{\nu} \Im[S_{11}({\bf n},{\bf
    n})-S_{22}({\bf n},{\bf n})]\\
  L^p_{13}({\bf n}) = L^p_{31}({\bf n}) &=& -\frac{c}{\nu} \Im[S_{12}({\bf n},{\bf
    n})+S_{21}({\bf n},{\bf n})]\\
  L^p_{14}({\bf n}) = L^p_{41}({\bf n}) &=& \frac{c}{\nu} \Re[S_{21}({\bf n},{\bf
    n})-S_{12}({\bf n},{\bf n})]\\
  L^p_{23}({\bf n}) = -L^p_{32}({\bf n}) &=& \frac{c}{\nu} \Im[S_{21}({\bf n},{\bf
    n})-S_{12}({\bf n},{\bf n})]\\
  L^p_{24}({\bf n}) = -L^p_{42}({\bf n}) &=& -\frac{c}{\nu} \Re[S_{12}({\bf n},{\bf
    n})+S_{21}({\bf n},{\bf n})]\\
  L^p_{34}({\bf n}) = -L^p_{43}({\bf n}) &=& \frac{c}{\nu} \Re[S_{22}({\bf n},{\bf
    n})-S_{11}({\bf n},{\bf n})]
\end{eqnarray}


\leveld{Generate scattering efficiency matrix}
%========================================================================
\label{sec:scattering:tot_pha_mat}

\levele{Calculate the total scattering efficiency matrix:}
%=======================================================================
\label{sec:scattering:pha_mat}

Output of the workspace method \wsmindex{pha\_matCalc} is the phase
matrix stored in the variable \wsvindex{pha\_mat}  with dimension
[$N_{\ScaZa}, N_{\ScaAa}, N_{I}, N_{I}$], where $N_{I}$ denotes the
chosen Stokes dimension.  $N_{\ScaZa}$ and 
$N_{\ScaAa}$ are the number of angles in the angular grids defined for
the scattering calculations.\\
Input is first the workspace variable \wsvindex{pha\_mat\_spt} (see Section
\ref{sec:scattering:pha_mat_spt}). The second input variable is the
local particle number density vector \artsstyle{pnd\_field} for all
particle types chosen in the control file.\\
The function sums up the phase matrices for all particle types
weighted with the particle number density.
\begin{eqnarray}
\SEfMat = \sum_i{\PDen\PhaMat}.
\end{eqnarray}

\levele{Generate phase matrix for single particle types:}
%==============================================================
\label{sec:scattering:pha_mat_spt}

From the amplitude matrix the phase matrix for each
particle type is calculated using the workspace method
\wsmindex{pha\_mat\_sptCalc}. The phase matrix depends on the
direction of the incoming radiation, so it is stored for all incoming
directions specified by \wsvindex{scat\_za\_grid} and
\wsvindex{scat\_aa\_grid}.\\
Output of the method is a tensor5 including phase matrices for all particle
types. Therefore it is  of  dimension
[$N_{pt}, N_{\ScaZa}, N_{\ScaAa}, N_{I}, N_{I}$], where $N_{pt}$ is
the number of particle
types  and $N_{I}$ denotes the Stokes dimension. $N_{\ScaZa}$ and
$N_{\ScaAa}$ are the number of angles in the angular grids defined for
the scattering calculations.\\
As input variables it needs the amplitude matrix (\wsvindex{amp\_mat})
and the propagation direction given by the local zenith and azimuth
angle specified by the indices \wsvindex{scat\_za\_index} and
\artsstyle{scat\_aa\_index}.  
The phase matrices \PhaMat{} are
calculated for all particle types chosen in the control
file using the following equations: 
\begin{eqnarray}
  Z_{11} &=& \frac{1}{2}(|S_{11}|^2+|S_{12}|^2+|S_{21}|^2+|S_{22}|^2)\\
  Z_{12} &=& \frac{1}{2}(|S_{11}|^2-|S_{12}|^2+|S_{21}|^2-|S_{22}|^2)\\
  Z_{13} &=& -\Re(S_{11}S_{12}^*+S_{22}S_{21}^*)\\
  Z_{14} &=& -\Im(S_{11}S_{12}^*-S_{22}S_{21}^*)\\
  Z_{21} &=& \frac{1}{2}(|S_{11}|^2+|S_{12}|^2-|S_{21}|^2-|S_{22}|^2)\\
  Z_{22} &=& \frac{1}{2}(|S_{11}|^2-|S_{12}|^2-|S_{21}|^2+|S_{22}|^2)\\
  Z_{23} &=& -\Re(S_{11}S_{12}^*-S_{22}S_{21}^*)\\
  Z_{24} &=& -\Im(S_{11}S_{12}^*+S_{22}S_{21}^*)\\
  Z_{31} &=& -\Re(S_{11}S_{21}^*+S_{22}S_{12}^*)\\
  Z_{32} &=& -\Re(S_{11}S_{21}^*-S_{22}S_{12}^*)\\
  Z_{33} &=& \Re(S_{11}S_{22}^*+S_{12}S_{21}^*)\\
  Z_{34} &=& \Im(S_{11}S_{22}^*+S_{21}S_{12}^*)\\
  Z_{41} &=& -\Im(S_{21}S_{11}^*+S_{22}S_{12}^*)\\
  Z_{42} &=& -\Im(S_{21}S_{11}^*-S_{22}S_{12}^*)\\
  Z_{43} &=& \Im(S_{22}S_{11}^*-S_{12}S_{21}^*)\\
  Z_{44} &=& \Re(S_{22}S_{11}^*-S_{12}S_{21}^*)
\end{eqnarray}






\leveld{Generate absorption coefficient vector:}
%==========================================================================
\label{sec:scattering:abs_vec}

Agendas and methods to generate the absorption coefficient vector are
defined in analogy to those generating the extinction coefficient matrix.

\levele{Agenda for calculating the absorption vector:}
%==================================================================
\label{sec:scattering:abs_vec_agenda}

The agenda \wsvindex{abs\_vec\_agenda} calculates the total absorption
matrix \AbsVec{}
for each grid point. It is the physical absorption vector including
particle absorption and gaseous absorption.\\
Output is the variable \wsvindex{abs\_vec} which is given
as a $N_I$ component vector, i.e. one dimension for each Stokes component. If the
Stokes dimension is only one, it will only have one component
corresponding to the scalar absorption coefficient. \\
Input for this agenda are the following variables:
the particle number density \wsvindex{pnd\_field} and 
\wsvindex{abs\_vec\_spt}, which contains the absorption vectors  for
each chosen particle type. How this variable is obtained
is described in Section \ref{sec:scattering:abs_vec_spt}. For calculating 
gaseous absorption the volume mixing ratio (\wsvindex{vmr}), the
pressure  (\wsvindex{p\_grid}), the temperature  (\wsvindex{t\_field}) and the
propagation direction given by zenith angle (\wsvindex{scat\_za\_grid}) and
azimuth angle (\wsvindex{scat\_aa\_grid}) are needed.\\
The definition of the agenda is listed below:

\vspace{2ex} 
\begin{minipage}[h]{.9\hsize}
\begin{verbatim}
AgendaDefine(abs_vec_agenda)
{
abs_vec_partCalc{}  // Function to calculate particle
                    // absorption.
abs_vec_gasCalc{}   // Function to calculate gaseous
                    // absorption.
abs_vecCalc{}       // Function which sums up 
                    // particle absorption and
                    // gaseous absorption.
}
\end{verbatim}
\end{minipage}
\vspace{2ex} 

\noindent
The workspace methods \wsmindex{abs\_vec\_partCalc} and
\wsmindex{abs\_vec\_gasCalc} are described in the sections
\ref{sec:scattering:abs_vec_part} and
\ref{sec:scattering:abs_vec_gas}. 


\levele{Calculate particle absorption coefficient vector:}
%=======================================================================
\label{sec:scattering:abs_vec_part}

The workspace method \wsmindex{abs\_vec\_partCalc} gives as output the
particle absorption vector \aAbsVec{p} stored in the variable
\wsvindex{abs\_vec\_part} which has the chosen Stokes dimension, so in
general a four component vector.\\
First input is the workspace variable \wsvindex{abs\_vec\_spt} (see Section
\ref{sec:scattering:abs_vec_spt}). The second input variable is the
local particle number density field \artsstyle{pnd\_field} for all particle types,
that means \artsstyle{pnd\_field} is a Tensor4 with dimension 
[$N_{pt}, N_{p}, N_{lat}, N_{lon}$]. The other workspace variable
inputs required are \wsvindex{atmosphere\_dim}, \wsvindex{scat\_p\_index},
\wsvindex{scat\_lat\_index}, and \wsvindex{scat\_lon\_index}. \\  
The function sums up the absorption vectors for all particle types
weighted with the particle number density.
\begin{eqnarray}
  \aAbsVec{p} = \sum_i \PDen_i \aSAbVec{p}
\end{eqnarray}


\levele{Calculate gaseous absorption coefficient vector:}
%=======================================================================
\label{sec:scattering:abs_vec_gas}


\levele{Generate absorption vector for single particle types:}
%==============================================================
\label{sec:scattering:abs_vec_spt}

From the amplitude matrix the absorption vector for each
particle type is calculated using the workspace method
\wsmindex{abs\_vec\_sptCalc}.\\
Output is the absorption vector tensor, which is  of  dimension
[$N_{pt}, N_{I}$], where $N_{pt}$ is the number of particle
types  and $N_{I}$ denotes the Stokes dimension.\\
As input it uses the extinction
matrix (\wsvindex{ext\_mat}) and the phase matrix
(\wsvindex{pha\_mat}).\\
The absorption cross section vectors \SAbVec{} are
calculated for all particle types chosen in the control
file. The following formulas are used:
\begin{eqnarray}
  \label{eq:gen_abs_vector}
  b^p_1({\bf n}) &=&  L^p_{11}({\bf n}) - \int_{4\pi} d{\bf n'}
  Z_{11}({\bf n, n'})\\
  b^p_2({\bf n}) &=&  L^p_{21}({\bf n}) - \int_{4\pi} d{\bf n'}
  Z_{21}({\bf n, n'})\\
  b^p_3({\bf n}) &=&  L^p_{31}({\bf n}) - \int_{4\pi} d{\bf n'}
  Z_{31}({\bf n, n'})\\
  b^p_4({\bf n}) &=&  L^p_{41}({\bf n}) - \int_{4\pi} d{\bf n'}
  Z_{41}({\bf n, n'}) \\
\end{eqnarray}
These equations show that the first column of extinction and phase
matrix are needed. So it is obligatory to execute the agenda
\funcindex{ext\_mat\_agenda} and the method
\funcindex{pha\_matCalc} before executing
\funcindex{abs\_vec\_agenda}. 


\levelc{Radiative transfer step calculation with averaged coefficients}
%=================================================================
\label{sec:scattering:rte_step}
In the agenda \wsvindex{scat\_rte\_agenda} the method for doing a
radiative step calculation through one grid cell for a 3D atmosphere
or one layer for a 1D atmosphere. The VRTE with constant coefficients and fixed scattering integral vector is solved.

\begin{itemize}
\item {\bf Method \artsstyle{stokes\_vecGeneral}:}\\
This method is the most general method. It can be applied for all
Stokes dimensions but it should not be used for the scalar equation as
for this case the method is numerically inefficient. 

There are two terms in the radiative transfer equation
\ref{eq:scattering:vrte_fs_av} which are
computed separately using different functions. The first term 
\begin{equation}
 e^{-\ExtMat s}\cdot\StoVec_0
\end{equation}
includes the matrix exponential function. This is solved numerically
using the Pad\'e approximation as implemented in the function 
\artsstyle{matrix\_exp} (see Section \ref{sec:lin_alg:mat_exp}).
The second term 
\begin{equation}
(\IdnMat - e^{-\ExtMat
    s}) \ExtMat\Inv (\AbsVec \Planck + \ScaInt_0)
\end{equation}
includes the inverse of \ExtMat\ multiplied with a vector. This is
computed numerically by a LU decomposition as described in the
Sections \ref{sec:lin_alg:backsub} and
\ref{sec:lin_alg:lu_decomp}. The functions used here are
\artsstyle{ludcmp} and \artsstyle{lubacksub}. The matrix exponential is
computed using again \artsstyle{matrix\_exp}.\\
Finally the method sums up the two terms to get the updated Stokes Vector. 
 
\item {\bf Method \artsstyle{stokes\_vecScalar}:}\\
For the scalar case it is straightforward to compute a radiative
transfer step through one grid cell/layer according to equation
(\ref{eq:scattering:scalar_rte_sol}). We only
have to compute the scalar exponential function and do not need the
matrix inverse, such that the standard C math library is sufficient.
\end{itemize}


\levelc{Convergence test method}
%====================================
\label{sec:scattering:conv_method}

Different methods to do the convergence test are implemented. The user
has to chose one by setting the
\wsvindex{convergence\_test\_agenda}. The following methods can be
used:

\begin{itemize}
\item {\bf Method \artsstyle{convergence\_flagAbs}:}\\
The function calculates the absolute differences for two successive
iteration fields. It picks out the maximum values for each Stokes 
component separately. The convergence test is fulfilled under the
following conditions:
\begin{eqnarray}
|I_{m+1} - I_m| < \epsilon_1    \\
|Q_{m+1} - Q_m| < \epsilon_2    \\
|U_{m+1} - U_m| < \epsilon_3    \\
|V_{m+1} - V_m| < \epsilon_4     
\end{eqnarray}

The limits for convergence have to be set in the control file by 
setting the vector \artsstyle{epsilon} as a keyword 
to appropriate values.
The conditions have to be valid for all positions in the cloudbox 
and for all directions.
Then the workspace variable \wsvindex{convergence\_flag} is set to 1.

\item {\bf Method \artsstyle{convergence\_flagLsq}:}\\
This method performs a least square test.
\end{itemize}



 


%%% Local Variables: 
%%% mode: latex
%%% TeX-master: "uguide"
%%% End: 
% LocalWords:  Emde ext matrix abs vec pha pnd sca lat lon za aa pt FIXME Eq
% LocalWords:  Eqs mishchenko scatt nonsp partic RTE
