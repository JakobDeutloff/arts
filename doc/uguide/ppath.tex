%
% To start the document, use
%  \levela{...}
% For lover level, sections use
%  \levelb{...}
%  \levelc{...}
%
\levela{Propagation paths}
 \label{sec:ppath}

%
% Document history, format:
%  \starthistory
%    date1 & text .... \\
%    date2 & text .... \\
%    ....
%  \stophistory
%
\starthistory
  02xxxx & xxx.\\
\stophistory


%
% Symbol table, format:
%  \startsymbols
%    ... & \verb|...| & text ... \\
%    ... & \verb|...| & text ... \\
%    ....
%  \stopsymbols
%
%
%\startsymbols
%  \Ind           & -                 & vector/matrix/tensor index           \\
%  \aInd{\Lat}    & -                 & the \Ind:th latitude                 \\
%  \VctLng        & -                 & vector length or size of matrix/tensor for a dimension \\
%  \aVctLng{\Lat} & -                 & length of the latitude grid \\
%  \Prs           & \verb|p|          & pressure                             \\
%  \PrsAlt        & \verb|pz|         & pressure altitude                    \\
%  \Rds           & \verb|r|          & radius from the geoid centre         \\
%  \Alt           & \verb|z|          & geometrical altitude above the geoid \\
%  \Lat           & \verb|alpha|      & latitude                             \\
%  \Lon           & \verb|beta|       & longitude                            \\
%  \ZntAng        & \verb|psi|        & zenith angle                         \\
%  \AzmAng        & \verb|omega|      & azimuthal angle                      \\
% \label{symtable:fm_defs}     
%\stopsymbols



xxx



\levelb{Geoid ellipsoids and geodetic datums}
%===================
\label{sec:ppath:geoids}


\levelc{Geoid ellipsoids}
%===================
\label{sec:ppath:geoid}



\levelc{Geocentric and geodetic latitudes}
%===================
\label{sec:ppath:geolat}

A geocentric latitude is not changed if the shape of the geoid is
changed.

Define \qindex{zenith} and \qindex{nadir}.

\begin{figure}[!t]
 \begin{center}
  \begin{minipage}[c]{0.58\textwidth}
   \begin{picture}(200,165)
    \put(30,10){\arc{300}{-1.1}{-0.5}}
    \put(40,147){{\small geoid ellipsoid}}
    \put(0,10){\vector(1,0){70}}
    \put(70,4){$x$}
    \put(10,0){\vector(0,1){70}}
    \put(4,70){$y$}
    \dottedline(30,10)(170,150)
    \put(30,10){\arc{20}{-.76}{0}}
    \put(40,13){\Lat$^*$}
    \put(136.07,116.07){\line(1,-1){25}}
    \put(136.07,116.07){\line(-1,1){25}}
    \put(162,98){{\small local}}
    \put(162,90){{\small tangent}}
    \drawline(10,10)(170,144.4)
    \put(10,10){\arc{20}{-.73}{0}}
    \put(160,130){{\small zenith}}
    \put(20,13){\Lat}
   \end{picture}
  \end{minipage}%
  \begin{minipage}[c]{0.42\textwidth}
   \caption{Definition of geocentric latitude, \Lat, and geodetic latitude, 
     \Lat$^*$. The dotted line is the normal to the local tangent of
     the geoid ellipsoid. The zenith and nadir directions, and
     geometrical altitudes, are here defined to go along the solid
     line.}
   \label{fig:ppath:lats}
  \end{minipage}
 \end{center}
\end{figure}           


%\levelc{Geodetic datums}
%===================
%\label{sec:ppath:geodatums}



\levelb{The propagation path data structure}
%===================
\label{sec:ppath:Ppath}



\levelb{Some basic geometrical relationships}
%===================
\label{sec:ppath:basicgeom}



\levelb{1D propagations paths}
%===================
\label{sec:ppath:1D}


\levelc{Without refraction}
%===================
\label{sec:ppath:1Dwithout}


\levelc{With refraction}
%===================
\label{sec:ppath:1Dwith}



\levelb{2D propagations paths}
%===================
\label{sec:ppath:2D}


\levelc{Without refraction}
%===================
\label{sec:ppath:2Dwithout}


\levelc{With refraction}
%===================
\label{sec:ppath:2Dwith}



\levelb{3D propagations paths}
%===================
\label{sec:ppath:3D}


\levelc{Without refraction}
%===================
\label{sec:ppath:3Dwithout}


\levelc{With refraction}
%===================
\label{sec:ppath:3Dwith}


%%% Local Variables: 
%%% mode: latex
%%% TeX-master: "uguide"
%%% End: 

