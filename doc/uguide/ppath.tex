%
% To start the document, use
%  \levela{...}
% For lover level, sections use
%  \levelb{...}
%  \levelc{...}
%
\levela{Calculation of propagation paths}
 \label{sec:ppath}

%
% Document history, format:
%  \starthistory
%    date1 & text .... \\
%    date2 & text .... \\
%    ....
%  \stophistory
%
\starthistory
  02xxxx & xxx.\\
\stophistory


%
% Symbol table, format:
%  \startsymbols
%    ... & \verb|...| & text ... \\
%    ... & \verb|...| & text ... \\
%    ....
%  \stopsymbols
%
%
%\startsymbols
%  \Ind           & -                 & vector/matrix/tensor index           \\
%  \aInd{\Lat}    & -                 & the \Ind:th latitude                 \\
%  \VctLng        & -                 & vector length or size of matrix/tensor for a dimension \\
%  \aVctLng{\Lat} & -                 & length of the latitude grid \\
%  \Prs           & \verb|p|          & pressure                             \\
%  \PrsAlt        & \verb|pz|         & pressure altitude                    \\
%  \Rds           & \verb|r|          & radius from the geoid centre         \\
%  \Alt           & \verb|z|          & geometrical altitude above the geoid \\
%  \Lat           & \verb|alpha|      & latitude                             \\
%  \Lon           & \verb|beta|       & longitude                            \\
%  \ZntAng        & \verb|psi|        & zenith angle                         \\
%  \AzmAng        & \verb|omega|      & azimuthal angle                      \\
% \label{symtable:fm_defs}     
%\stopsymbols



xxx



\levelb{Geoid ellipsoids and geodetic datums}
%===================
\label{sec:ppath:geoid}

A geocentric latitude is not changed if the shape of the geoid is
changed.

Define \qindex{zenith} and \qindex{nadir}.




%%% Local Variables: 
%%% mode: latex
%%% TeX-master: "uguide"
%%% End: 
