%
% To start the document, use
%  \levela{...}
% For lover level, sections use
%  \levelb{...}
%  \levelc{...}
%
\levela{Propagation paths}
 \label{sec:ppath}

%
% Document history, format:
%  \starthistory
%    date1 & text .... \\
%    date2 & text .... \\
%    ....
%  \stophistory
%
\starthistory
  02xxxx & xxx.\\
\stophistory


%
% Symbol table, format:
%  \startsymbols
%    ... & \artsstyle{...} & text ... \\
%    ... & \artsstyle{...} & text ... \\
%    ....
%  \stopsymbols
%
%
%\startsymbols
%  \Ind           & -                 & vector/matrix/tensor index           \\
%  \aInd{\Lat}    & -                 & the \Ind:th latitude                 \\
%  \VctLng        & -                 & vector length or size of matrix/tensor for a dimension \\
%  \aVctLng{\Lat} & -                 & length of the latitude grid \\
%  \Prs           & \artsstyle{p}     & pressure                             \\
%  \PrsAlt        & \artsstyle{pz}    & pressure altitude                    \\
%  \Rds           & \artsstyle{r}     & radius from the geoid centre         \\
%  \Alt           & \artsstyle{z}     & geometrical altitude above the geoid \\
%  \Lat           & \artsstyle{alpha} & latitude                             \\
%  \Lon           & \artsstyle{beta}  & longitude                            \\
%  \ZntAng        & \artsstyle{psi}   & zenith angle                         \\
%  \AzmAng        & \artsstyle{omega} & azimuthal angle                      \\
% \label{symtable:ppath_defs}     
%\stopsymbols




\levelb{Geoid ellipsoids and geodetic datums}
%===================
\label{sec:ppath:geoids}

This section defines the geoid ellipsoid and discusses releated
issues. The geoid is introduced in Section~\ref{sec:fm_defs:geoid}.
The workspace variable representing the geoid is \wsvindex{r\_geoid}.


\levelc{Geoid ellipsoids}
%===================
\label{sec:ppath:geoid}

All geodetic datums are based on a reference ellipsoid\index{geoid
  ellipsoid}. The ellipsoid is rotationally symmetric around the
north-south axis. That is, the ellipsoid radius has no longitude
variation, it is only a function of latitude. The ellipsoid is
described by a equatorial radius, \aRds{e}, and a polar radius,
\aRds{p}. These radii are indicated in Figure~\ref{fig:ppath:lats}.
The radius of the ellipsoid for a given latitude is
\begin{equation}
 \aRds{\odot}(\Lat) = \sqrt{\frac{\aRds{e}^2\aRds{p}^2}
                    {\aRds{e}^2\sin^2\Lat+\aRds{p}^2\cos^2\Lat}}
 \label{eq:ppath:ellipsradius} 
\end{equation}
The radius given by Equation~\ref{eq:ppath:ellipsradius} can be
directly applied for 2D and 3D cases. On the other hand, for 1D cases
the reference geoid is by definition a sphere and the radius of this
sphere shall be selected in such way that it represents the local
shape of a reference ellipsoid. This is achieved by setting
\aRds{\odot} to the radius of curvature of the ellipsoid. The
curvature radius differs from the local radius except at the equator
and an east-west direction. For example, at the equator and a
north-south direction, the curvature radius is smaller then the local
radius, while at the poles (for all directions) it is greater
(see further Figure~\ref{fig:ppath:wgs84radii}). 

The \textindex{curvature radius}, \aRds{c}, of an ellipsoid is 
\citep{rodgers:00}
\begin{equation}
 \aRds{c} = \frac{1}{\aRds{ns}^{-1}\cos^2 \Lat + \aRds{ew}^{-1}\sin^2 \Lat}
 \label{eq:ppath:curvradius} 
\end{equation}
where \aRds{ns} and \aRds{ew} are the north-south and east-west curvature radius, respectively,
\begin{eqnarray}
 \aRds{ns} &=& \aRds{e}^2\aRds{p}^2 (
           \aRds{e}^2\cos^2\AzmAng+\aRds{p}^2\sin^2\AzmAng )^{-\frac{3}{2}} \\
 \aRds{ew} &=& \aRds{e}^2 (
           \aRds{e}^2\cos^2\AzmAng+\aRds{p}^2\sin^2\AzmAng )^{-\frac{1}{2}} 
 \label{eq:ppath:rew} 
\end{eqnarray}
The azimuth angle, \AzmAng, is defined in
Section~\ref{sec:fm_defs:los}. The latitude and azimuthal angle to
apply in Equations \ref{eq:ppath:curvradius}~-~\ref{eq:ppath:rew}
shall rather be valid for a middle point of the propagation paths
(such as some tangent point), instead of the sensor position. 

\begin{figure}[!p]
 \begin{center}
  \begin{minipage}[c]{0.65\textwidth}
   \begin{center}
   \begin{picture}(200,165)(0,0)
    \put(30,10){\arc{301}{-1.05}{-0.55}}
    \put(43,138){{\small geoid ellipsoid}}
    \put(0,10){\vector(1,0){70}}
    \put(70,4){$x$}
    \put(10,0){\vector(0,1){70}}
    \put(4,70){$y$}
    \dottedline(30,10)(170,150)
    \put(30,10){\arc{20}{-.76}{0}}
    \put(40,13){\Lat$^*$}
    \put(136.07,116.07){\line(1,-1){20}}
    \put(136.07,116.07){\line(-1,1){20}}
    \put(158,94){\small\shortstack[l]{local\\tangent}}
    \drawline(10,10)(170,144.4)
    \put(10,10){\arc{20}{-.73}{0}}
    \put(160,130){{\small zenith}}
    \put(20,13){\Lat}
    %
    \put(30,10){\arc{310}{-.15}{0.05}}
    \put(150,10){\vector(1,0){35}}
    \put(170,14){\aRds{e}}
    \put(10,10){\arc{290}{-1.6708}{-1.4708}}
    \put(10,120){\vector(0,1){35}}
    \put(13,140){\aRds{p}}
   \end{picture}
   \end{center}
  \end{minipage}%
  \begin{minipage}[c]{0.35\textwidth}
   \caption{Definition of the ellipsoid radii, \aRds{e} and \aRds{p}, 
     geocentric latitude, \Lat, and geodetic latitude, \Lat$^*$. The
     dotted line is the normal to the local tangent of the geoid
     ellipsoid. The zenith and nadir directions, and geometrical
     altitudes, are here defined to follow the solid line.}
   \label{fig:ppath:lats}
  \end{minipage}
 \end{center}
\end{figure}   

\begin{figure}[!p]
 \begin{minipage}[c]{0.65\textwidth}
 \includegraphics*[width=0.96\textwidth]{ppath/wgs84_radii}
 \end{minipage}%
 \begin{minipage}[c]{0.35\textwidth}
  \caption{The ellipsoid radius (\aRds{\odot}) and curvature radius (\aRds{c})
    for the
    WGS-84 reference ellipsoid. The curvature radii are valid for the
    north-south direction.}
  \label{fig:ppath:wgs84radii}
 \end{minipage}%
\end{figure}   
        
\begin{figure}[!p]
 \begin{minipage}[c]{0.65\textwidth}
 \includegraphics*[width=0.96\textwidth]{ppath/wgs84_latdiff}
 \end{minipage}%
 \begin{minipage}[c]{0.35\textwidth}
  \caption{The change of the WGS-84 ellipsoid radius for  1\degree\ 
            latitude differences.}
  \label{fig:ppath:latdiff}
 \end{minipage}%
\end{figure}   



\levelc{Geocentric and geodetic latitudes}
%===================
\label{sec:ppath:geolat}

The fact that the geoid is an ellipsoid, instead of a sphere, opens up
for the two different definitions of the latitude. The
\textindex{geocentric latitude}, which is the the one used here, is the
angle between the equatorial plane and the vector from the coordinate
system centre to the position of concern. The \textindex{geodetic
  latitude} is also defined with respect to the equatorial plane, but
the angle to the normal to the reference ellipsoid is considered here, as
shown in Figure~\ref{fig:ppath:lats}. It could be mentioned that a
geocentric latitude does not depend on the geoid ellipsoid used, while
the geodetic latitudes change if another reference ellipsoid is
selected. An approximative relationship between the geodetic
($\Lat^*$) and geocentric (\Lat) latitudes is \citep{montenbruck:00}
\begin{equation}
 \Lat^* = \Lat + f\,\sin(2\Lat)  
 \label{eq:ppath:lats}
\end{equation}
where $f$ is the flattening of the ellipse:
\begin{equation}
 f = \frac{\aRds{e}-\aRds{p}}{\aRds{e}}
 \label{eq:ppath:flattening}
\end{equation}
The value of $f$ for the Earth is about 1/298.26. This means that the
largest differences between \Lat\ and $\Lat^*$ are found at
mid-latitudes and the maximum value is about 12 arc-minutes.

The \textindex{zenith} and \textindex{nadir} directions shall normally be
defined to follow the normal to the reference ellipsoid, but, if
nothing else is mentioned, these directions are here treated to go
along the vector the centre of the coordinate system, as indicated in
Figure~\ref{fig:ppath:lats}. This latter definition is preferred
as it results in that a propagation path in the zenith/nadir direction
can be described by a single latitude and longitude value. The
difference in geometrical altitude when using these two possible
definitions on the zenith direction is proportional to the deviation
between geocentric and geodetic latitude (Equation~\ref{eq:ppath:lats}).
For an altitude of 100~km around $\Lat=45\degree$, the difference is
about 350~m.


\levelc{Geodetic datums}
%===================
\label{sec:ppath:geodatums}

Table~\ref{tab:ppath:geodatums} gives the equatorial and polar radii
of the reference ellipsoid for the geodetic datums handled by ARTS.

\begin{table}[!h]
  \begin{center}
    \begin{tabular}{c c c c l}
     Datum & \aRds{e} & \aRds{p} & $1/f$ & Reference \vspace*{1mm} \\ 
     \hline 
     WGS-84 & 6378.137 km & \emph{6356.752 km} & 298.257223563 & {\small \citet{montenbruck:00}}  \rule{0mm}{5mm} \vspace*{1mm} \\
     \hline
    \end{tabular}
    \caption{Equatorial and polar radius of reference ellipsoids. Values 
      given as \emph{italic} are 
      derived by the other two values and Equation~\ref{eq:ppath:flattening}.}
    \label{tab:ppath:geodatums}
  \end{center}
\end{table}



\levelb{The propagation path data structure}
%===================
\label{sec:ppath:Ppath}

The ARTS definition of a propagation path is given in
Section~\ref{sec:fm_defs:ppaths}.  A propagation path is represented
by a structure of type \structindex{Ppath}. This structure holds also
auxiliary variables to facilitate the radiative transfer calculations
and to speed up the interpolation (if performed). The fields of
\artsstyle{Ppath} where the data type is given inside square brackets,
are:
\begin{description}
  \item[dim] [Index] The atmospheric dimensionality. This field shall always 
     be equal to the workspace variable \artsstyle{atmosphere\_dim}.
  \item[np] [Index] Number of positions to define the propagation path. The 
     number of rows, or the length, of $\mathbf{pos}$, $\mathbf{z}$, 
     $\mathbf{indexpos}$, $\mathbf{l\_step}$ and $\mathbf{los}$ shall be
     equal to $\mathbf{np}$. The number of radiative transfer steps along the 
     path is $\mathbf{np}-1$. If $\mathbf{np}\leq 1$, the observed spectrum 
     equals the radiative background. When the path is totally outside the 
     atmosphere $\mathbf{np}=0$. If the sensor is placed on the ground and 
     looks down, the sensor is inside the cloud box, or the sensor is at the 
     cloud box boundary but looks into the box, $\mathbf{np}=1$ where the 
     stored values correspond to the sensor position.
  \item[i\_start] [Index] The index of the position corresponding to the 
     starting point of the path. For 2D and 3D, the paths are stored in
     reversed order and the value of $\mathbf{i\_start}$ is always 
     $\mathbf{np}-1$. The primary reason to why the fields $\mathbf{i\_start}$
     and $\mathbf{i\_stop}$ were introduced is to make use of the symmetry
     in 1D downward looking observations. For 1D, the positions are stored 
     starting with the lowest point of the path and $\mathbf{i\_start}$ can
     have any value between 0 and $\mathbf{np}-1$. If the sensor zenith angle
     is $\geq 90\degree$, then $\mathbf{i\_start} = \mathbf{np}-1$. For
     downward observations with an intersection of the path by a blackbody 
     ground $\mathbf{i\_start}$ equals 0. For other downward looking cases,
     $\mathbf{i\_start}$ corresponds to the sensor if it is placed inside
     the model atmosphere, or $\mathbf{i\_start} = \mathbf{np}-1$ if the
     sensor is above the top of the atmosphere.
  \item[i\_stop] [Index] The index of the position corresponding to the 
     end point of the path (inside the model atmosphere), which is the 
     sensor position if inside the atmosphere, or a point at the highest
     pressure surface if the sensor is outside the model atmosphere. 
     For 2D and 3D, $\mathbf{i\_stop}$ equals always 0, while for 1D it 
     can be any value from 0 to $\mathbf{np}-1$ depending on the position 
     of the sensor. A practical example: for 1D limb sounding without an
     blackbody ground intersection, the propagation path is followed by
     looping from $\mathbf{np}-1$ down to 0 (which corresponds to the tangent
     point) and then back up to $\mathbf{np}-1$. See further [**].
  \item[pos] [Matrix] The propagation path positions. This matrix has 
     $\mathbf{np}$ rows and up to 3 columns. Each row holds a position
     where column 1 is the pressure, column 2 the latitude and column 3 the
     longitude (cf. Section~\ref{sec:fm_defs:sensorpos}). The number of 
     columns for 1D and 2D is 2, while for 3D it is 3. The latitudes are 
     stored for 1D cases as these can be of interest for some applications 
     and are useful if the propagation path shall be plotted.
  \item[z] [Vector] The geometrical altitude for each path position. The
     length of this vector is accordingly $\mathbf{np}$. This is a help
     variable for plotting, and similar purposes, and shall not be used to
     interpolate the atmospheric fields, as pressure is the main altitude
     coordinate.
  \item[l\_step] [Vector] The length along the propagation path between
     the positions in $\mathbf{pos}$. The length of the vector is
     $\mathbf{np}-1$. 
  \item[gridindex] [Matrix] The fractional grid indexes of the positions.
     The intention of this field is to speed up interpolation of the 
     atmospheric fields. The matrix has $\mathbf{np}$ rows and $\mathbf{dim}$ 
     columns, where column 1 refers to the pressure grid, column 2 to the 
     latitude grid and column 3 to the longitude grid. Note that this matrix,
     in contrast to $\mathbf{pos}$, has only one column for 1D. This is the 
     case as the latitude position is of no interest for 1D when interpolating
     an atmospheric field. 

     The grid index is best described by some examples. A grid index of
     2 means that the position equals the grid value with index 2, and
     a grid index of 2.5 signifies that the position is half the way between
     grid points 2 and 3. A more formal definition is (using latitudes):
     if $i$ is the highest index such as $\aLat{i}\leq\Lat$, then the grid
     index is $i+(\Lat-\aLat{i})/(\aLat{i+1}-\aLat{i})$, where \Lat\ is the 
     latitude of the position and \aLat{i}\ is grid point $i$. The grid index 
     for a position at the upper end of the grid is $\VctLng-1$ where
     \VctLng\ is the length of the grid. The vertical box position is given 
     with respect to pressure altitudes (Equation~\ref{eq:fm_defs:prsalt}).
  \item[los] [Matrix] The line-of-sight of the propagation path at each
     point. The number of rows of the matrix is $\mathbf{np}$. For 1D
     and 2D, the matrix has a single column holding the zenith angle. For 3D
     there is an additional column giving the azimuthal angle. The zenith and
     azimuthal angles are defined in Section~\ref{sec:fm_defs:los}.
     If the radiative background is the cloud box, the line-of-sight
     with index $\mathbf{i\_start}$
     gives the direction for the relevant direction of the cloud box intensity
     field.     
  \item[background] [String] The radiative background for the propagation path,
     that is, [**] in Equation~[**]. The possible options for this field
     are 'Cosmic background radiation', 'Blackbody ground', 'Inside cloud box'
     and 'Surface of cloud box', where the source of radiation should be clear
     the content of the strings.
  \item[ground] [Index] A boolean to indicate that there is an intersection
     of the path by the ground. If the ground is treated to be a blackbody,
     $\mathbf{ground}$ can only have the value of 0.
  \item[i\_ground] [Index] The position index of the ground. Undefined if 
     $\mathbf{ground}=0$. For 1D, $\mathbf{i\_ground}$ can only have the 
     value of 0. Row $\mathbf{i\_ground}$ of $\mathbf{pos}$ gives the 
     position of the ground reflection.
  \item[tan\_pos] [Vector] The position of the tangent point, if such exists.
     The vector has otherwise length 0. If a tangent point exists, this point
     is included in $\mathbf{pos}$ but is also stored here for simplicity
     reasons.
\end{description}



\levelb{Some basic 1D and 2D geometrical relationships}
%===================
\label{sec:ppath:basicgeom}

This section gives some expressions to determine positions along a
propagation path for 1D and 2D atmospheres when refection is
neglected. The quantities used are defined in
Figure~\ref{fig:ppath:1d2dgeom}. The ARTS function for making the
calculation of concern is given inside paranthesis above each
equation, if not stated explicitely.

\begin{figure}[!t]
 \begin{center}
  \begin{minipage}[c]{0.65\textwidth}
   \begin{center}
   \begin{picture}(200,170)(0,0)
     \put(0,0){\vector(1,4){25}}
     \dottedline(25,100)(35,140)
     \put(25,100){\arc{30}{-1.31}{-0.23}}
     \put(50,0){\vector(1,2){60}}
     \dottedline(110,120)(130,160)
     \put(110,120){\arc{30}{-1.1}{-0.23}}
     \drawline(25,100)(110,120)
     \dottedline(0,94.12)(25,100)
     \dottedline(110,120)(145,128.24)
     \put(145,128.24){\vector(4,1){25}}
     \put(0,-80){\arc{200}{-1.52}{-1.02}}
     \put(7,70){$\aRds{1}$}
     \put(99,90){$\aRds{2}$}
     \put(25,17){$\Delta\Lat$}
     \put(35,113){$\aZntAng{1}$}
     \put(122,131){$\aZntAng{2}$}
     \put(60,113){$\Delta \PpathLng$}
     \put(150,110){\small\shortstack[l]{propagation\\direction}}
   \end{picture}
   \end{center}
  \end{minipage}%
  \begin{minipage}[c]{0.35\textwidth}
   \caption{The radius (\Rds) and zenith angle (\ZntAng) for two points along
     the propagation path, and the distance along the path ($\Delta\PpathLng$)
     and the latitude difference ($\Delta\Lat$) between these points.}
   \label{fig:ppath:1d2dgeom}
  \end{minipage}
 \end{center}
\end{figure}   

The law of sines that the product $\Rds\sin(\ZntAng)$ is constant
along the propagation path (cf. Equation~[**]):
\begin{equation}
  p_c = \Rds\sin(\ZntAng)
  \label{eq:ppath:geomconst}
\end{equation}
The propagation path constant $p_c$ is determined by the position and
line-of-sight of the sensor, a calculation done by the function
\wsfindex{ppath\_constant}. The constant equals also the radius of the path
tangent point of the path, that is found along an imaginary
prolongation of the path behind the sensor if the viewing direction is
upwards. The expressions below are based on $p_c$ as the usage of a
global constant for the path should decrease the sensitivity to
numerical inaccuracies. If the calculations are based solely on the
values for the neighbouring point, a numerical inaccuracy can
accumulate when going from one point to next.

The relationship between the distance along the path for an
infinitesimal change in radius is here denoted as the
\textindex{geometrical factor}, $g$. If refraction is neglected, three
expressions for the geometrical factor are
\begin{equation}
  g = \frac{1}{\cos(\ZntAng)} = \frac{1}{\sqrt{1-\sin^2(\ZntAng)}}
                                            = \frac{\Rds}{\sqrt{\Rds^2-c^2}}
  \label{eq:ppath:geomfac}
\end{equation}
The primitive function [* correct term? *] of the geometrical factor is
\begin{equation}
  \int{g(\Rds)}\,\DiffD r = \sqrt{\Rds^2-c^2}
  \label{eq:ppath:primofg}
\end{equation}


%The distance
%along the path to point 2 is ()
\begin{equation}
  \Delta\PpathLng = 
  \label{eq:ppath:}
\end{equation}



\levelb{1D propagations paths}
%===================
\label{sec:ppath:1D}


\levelc{Without refraction}
%===================
\label{sec:ppath:1Dwithout}


\levelc{With refraction}
%===================
\label{sec:ppath:1Dwith}



\levelb{2D propagations paths}
%===================
\label{sec:ppath:2D}


\levelc{Without refraction}
%===================
\label{sec:ppath:2Dwithout}


\levelc{With refraction}
%===================
\label{sec:ppath:2Dwith}



\levelb{3D propagations paths}
%===================
\label{sec:ppath:3D}


\levelc{Without refraction}
%===================
\label{sec:ppath:3Dwithout}


\levelc{With refraction}
%===================
\label{sec:ppath:3Dwith}



\levelb{Control file examples}
%===================
\label{sec:ppath:cfile}




%%% Local Variables: 
%%% mode: latex
%%% TeX-master: "uguide"
%%% End: 

