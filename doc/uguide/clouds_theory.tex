%\graphicspath{{Figs/scattering/}}

\chapter[Scattering: Particle ensembles]{Scattering: Representation of particle ensembles}
 \label{sec:cloudtheory}

\starthistory
 161107 & Created by Jana Mendrok. Moved parts from AUG clouds chapter here.\\
\stophistory

Effects of single small particles interacting with radiation as well as the
optical properties used to describe these effects have been discussed in
Section~\ref{sec:rtetheory:theory_single_part}. Realistic media typically
contain ensembles of particles of different material, sizes, shapes, etc.
The particle size has an important impact on the scattering and
absorption properties of particle ensembles, e.g. clouds, as shown for instance
in \citep{emde04:_doit_jgr}.  Particle ensembles typically contain a whole range
of different particle sizes, which can be described by a size
distribution giving the number of particles per unit volume per unit
radius interval as a function of radius.  It is most convenient to
parameterize the size distribution by analytical functions, because in
this case optical properties can be calculated much faster than for
arbitrary size distributions. The T-matrix code for randomly oriented
particles includes several types of analytical size distributions,
e.g., the gamma distribution or the log-normal distribution.  This
section presents some examples of size distribution parameterizations that can
be used to describe particle ensembles. Some of them are applicable to general
particle ensembles, others have been developed for specific ensembles types
(e.g., cirrus clouds) and atmospheric conditions.

\subsection{Mono-disperse particle distribution}

The most simple assumption is, that all particles in an ensemble have
the same size.  In order to study scattering effects like polarization
or the influence of particle shape, it makes sense to use this most
simple assumption, because one can exclude effects resulting from the
particle size distribution itself.  

Along with the single scattering properties we need the particle
number density field, which specifies the number of particles per
unit volume at each grid point, for ARTS scattering simulations.  For
a given mass content (\mc) and mono-disperse particles the particle number density
$n^p$ is simply
\begin{equation}
\label{eq:pnd_mono}
  n^p (\mc, r) =\frac{\mc}{m} = \frac{\mc}{V\rho}
    =\frac{3}{4\pi}\frac{\mc}{\rho r^3},  
\end{equation}
where $m$ is the mass of a particle, $r$ is its equal volume sphere
radius, $\rho$ is its density, and $V$ is its volume.
     
\subsection{Gamma size distribution}
\label{sec:clouds:gamma_distr}

A commonly used distribution for radiative transfer modeling in a range of
media, e.g. cirrus clouds, is the \emph{gamma distribution}
\begin{equation}
  n(r) = a  r^\alpha \exp(-br).
\label{eq:gamma_distr}
\end{equation}
The dimensionless parameter $\alpha$ describes the width of the
distribution. The other two parameters can be linked to the effective
radius \Reff\ and the mass content \mc\ as follows:
\begin{eqnarray}
  b &= \frac{\alpha+3}{\Reff},\\
  a &= \frac{\mc}{4/3\pi\rho b^{-(\alpha+4)}\Gamma[\alpha+4]},
\label{eq:gamma_coeff}
\end{eqnarray}
where $\rho$ is the density of the scattering medium and $\Gamma$ is
the gamma function. For cirrus clouds $\rho$ corresponds to the bulk
density of ice, which is approximately 917 kg/m$^3$.

Generally, the effective radius \Reff\ is defined as the average
radius weighted by the particles' cross-sectional area
\begin{equation}
  \label{eq:Reff}
  \Reff= \frac{1}{\EnsAvr{A}} \int_{r_{min}}^{r_{max}} {A(r) r n(r) \DiffD r},
\end{equation}
where $A$ is the area of the geometric projection of a particle. The
minimal and maximal particle sizes in the distribution are given by
$r_{min}$ and $r_{max}$ respectively. In the case of spherical
particles $A = \pi r^2$. The average area of the geometric projection
per particle \EnsAvr{A} is given by
\begin{equation}
  \EnsAvr{A} = \frac{\int_{r_{min}}^{r_{max}} {A(r) n(r) \DiffD r}}{\int_{r_{min}}^{r_{max}} {n(r) \DiffD r}}.
\end{equation}

The particle number density for size distributions is obtained by
integration of the distribution function over all sizes:
\begin{eqnarray}
  n^p(\mc, \Reff) &= \int_0^\infty {n(r)\DiffD r} \\
 &= \int_0^\infty {a  r^\alpha \exp(-br)
    \DiffD r} = a \frac{\Gamma(\alpha+1)}{b^{\alpha+1}}.
\end{eqnarray}

After setting $\alpha$ = 1, inserting Equation \ref{eq:gamma_coeff} and
some simple algebra we obtain
\begin{equation}
  \label{eq:pnd_gamma}
  n^p(\mc, \Reff) = \frac{2}{\pi} \frac{\mc}{\rho \Reff^3}.
\end{equation}
Comparing Equation \ref{eq:pnd_mono} and \ref{eq:pnd_gamma}, we see
that the particle number density for mono-disperse particles with a
particle size of $R$ is smaller than the particle number density for
gamma distributed particles with $\Reff=R$. The reason is that in the
gamma distribution most particles are smaller than~\Reff.

The question is how well a gamma distribution can represent the true
particle size distribution in radiative transfer calculations. For cirrus
clouds, this question has been investigated by \citet{evans:98}. The authors
come to the
conclusion that a gamma distribution represents the distribution of
realistic clouds quite well, provided that the parameters \Reff, \imc\ 
and $\alpha$ are chosen correctly. They show that setting $\alpha = 1$
and calculating only \Reff\ gave an agreement within 15\% in 90\% of
the considered measurements obtained during the First ISCCP Regional
Experiment (FIRE).  Therefore, for all calculations including gamma
size distributions for ice particles, $\alpha = 1$ was assumed.  

\subsection[McFarquhar and Heymsfield parametrization]
{Ice particle size parameterization by McFarquhar and Heymsfield}
\label{sec:McFHey_distr}

A more realistic parameterization of tropical cirrus ice crystal size
distributions was derived by
\citet{mcfarquar97:_param_tropic_cirrus_ice_cryst}, who derived the
size distribution as a function of temperature and \imc. The
parameterization was made based on observations during the Central
Equatorial Pacific Experiment (CEPEX). Smaller ice crystals with an
equal volume sphere radius of less than 50\,\mum\ are parametrized
as a sum of first-order gamma functions:
\begin{equation}
  \label{eq:MH_gamma}
  n(r) = \frac{12\cdot \imc_{<50}\alpha^5_{<50} r}{\pi \rho
    \Gamma(5)} \exp(-2 \alpha_{<50} r), 
\end{equation}
where $\alpha_{<50}$ is a parameter of the distribution, and
$\imc_{<50}$ is the mass of all crystals smaller than 50\,\mum\ in
the observed size distribution.  Large ice crystals are represented
better by a log-normal function
\begin{eqnarray}
  \label{eq:MH_lognorm}
  n(r) = \frac{3\cdot\imc_{>50}}{\pi^{3/2}\rho \sqrt{2}
    \exp(3\mu_{>50}+(9/2)\sigma^2_{>50}) r \sigma_{>50} r_0^3}
  \nonumber \\
  \cdot \exp\left[-\frac{1}{2}\left(\frac{\log\frac{2r}{r_0} -
          \mu_{>50}}{\sigma_{>50}}\right)^2\right], 
\end{eqnarray}
where $\imc_{>50}$ is the mass of all ice crystals greater than
50\,\mum\ in the observed size distribution, $r_0$ = 1\,\mum\ is a
parameter used to ensure that the equation does not depend on the
choice of unit for r, $\sigma_{>50}$ is the geometric standard
deviation of the distribution, and $\mu_{>50}$ is the location of the
mode of the log-normal distribution.  The fitted parameters of the
distribution can be looked up in the article by
\citet{mcfarquar97:_param_tropic_cirrus_ice_cryst}.  The particle
number density field is obtained by numerical integration over a
discrete set of size bins. Implementations of this particle size distribution
parameterization exist in the ATMLAB and PyARTS packages (NOTE: PyARTS is
currently not maintained and functionality is not guaranteed!).



