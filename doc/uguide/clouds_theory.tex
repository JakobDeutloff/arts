%\graphicspath{{Figs/scattering/}}

\chapter{Particle size distributions}
 \label{sec:cloudtheory}

\starthistory
 171101 & Restarted the chapter, to instead deal with PSDs (Patrick).\\
 161107 & Created by Jana Mendrok. Moved parts from AUG clouds chapter here.\\
\stophistory

% Effects of single small particles interacting with radiation as well as the
% optical properties used to describe these effects have been discussed in
% Section~\ref{sec:rtetheory:theory_single_part}. Realistic media typically
% contain ensembles of particles of different material, sizes, shapes, etc.
% The particle size has an important impact on the scattering and
% absorption properties of particle ensembles, e.g. clouds, as shown for instance
% in \citep{emde04:_doit_jgr}.  Particle ensembles typically contain a whole range
% of different particle sizes, which can be described by a size
% distribution giving the number of particles per unit volume per unit
% radius interval as a function of radius.  It is most convenient to
% parameterize the size distribution by analytical functions, because in
% this case optical properties can be calculated much faster than for
% arbitrary size distributions. The T-matrix code for randomly oriented
% particles includes several types of analytical size distributions,
% e.g., the gamma distribution or the log-normal distribution.  This
% section presents some examples of size distribution parameterizations that can
% be used to describe particle ensembles. Some of them are applicable to general
% particle ensembles, others have been developed for specific ensembles types
% (e.g., cirrus clouds) and atmospheric conditions.

\def\no{N_0}
\def\ntot{N_{\mathrm{tot}}}
\def\la{\Lambda}
\def\ga{\gamma}
\def\dg{D_g}
\def\xmed{x_\mathrm{med}}
\def\xmea{x_\mathrm{mean}}
\def\mmea{m_\mathrm{mean}}
\def\eterm{e}
\def\rterm{r}


\section{Handling of different size descriptors}
\label{sec:psd:size}
%
Various quantities can be selected as size ($x$), such as mass, maximum
diameter and volume equivalent diameter. If size is selected to be some
geometrical diameter, $\dg$, the mass of particles is normally assumed to
follow a power-law (e.g.\ \citet{petty2011modified}, Eq.~29):
\begin{displaymath}
  m = a \dg^b.  
\end{displaymath}
However, this approach can in fact also be applied on all other standard size
descriptors (including ones based on area). That is, the following expressions is
applied generally:
\begin{equation}
  m = a x^b.  
\end{equation}
For example, if $x$ represents mass, both $a$ and $b$ are simply 1. If $x$ is
volume equivalent diameter (and the density is constant with size), $b=3$. By
this approach it can be avoided to have different equations for different size
descriptors. For example, $a$ and $b$ are included generally in the expressions
below for modified gamma distributions, and one set of equations suffices.

The workspace variables corresponding to $a$ and $b$ are
\wsvindex{scat\_species\_a} and \wsvindex{scat\_species\_b}, respectively.
These variables can be calculated with the \wsmindex{ScatSpeciesSizeMassInfo}
workspace method.


\section{Modified gamma particle size distributions}
%
This section lists the expressions needed to handle particle size distributions
(PSDs) of type modified gamma distribution (MGD). Expressions are needed for
various derivatives, as well as for mapping from bulk properties to the native
parameters of MGD. The general properties of MGD PSDs are discussed in detail
in \citet{petty2011modified}, below shortened to P\&H11. However, derivatives
are not discussed by P\&H11 and to document those expressions is the main
purpose of the section.

\subsection{Native form}
\label{sec:mgd_native}
%
The MGD function has four parameters ($\no$, $\mu$, $\la$ and $\ga$) and is
here written as (P\&H, Eq.~6):
\begin{equation}
  n(\no,\mu,\la,\ga) = \no x^\mu exp(-\la x^\ga),
\end{equation}
where $x$ is ``size'' (Sec.~\ref{sec:psd:size}). The derivatives of $n$ with respect to
the basic parameters are:
\begin{eqnarray}
  \frac{dn}{d\no} &=& x^\mu exp(-\la x^\ga) \label{eq:dndno} \\
  \frac{dn}{d\mu} &=& \no \ln(x) x^\mu exp(-\la x^\ga) = \ln(x)n \label{eq:dndmu}\\
  \frac{dn}{d\la} &=& \no x^\mu (-x^\ga) exp(-\la x^\ga) = -x^\ga n \label{eq:dndla}\\
  \frac{dn}{d\ga} &=& \no x^\mu (-\la)\ln(x) x^\ga exp(-\la x^\ga) =
                      -\la\ln(x) x^\ga n \label{eq:dndga}
\end{eqnarray}
The workspace method for the native form of MGD is \wsmindex{psdModifiedGamma}.


\subsection{Moments and gamma function}
%
It is common to express various properties of PSDs as ``moments''. The k-th
moment, $M_k$ is defined as
\begin{equation}
  M_k = \int_0^\infty x^k n(x)\, dx.
\end{equation}
For MGD, the moments can be expressed analytically (P\&H, Eq.~17):
\begin{equation}
  M_k = \frac{\no}{\ga}\frac{\Gamma(\frac{\mu+k+1}{\ga})}{\la^{(\mu+k+1)/\ga}},
  \label{eq:mdg:mk}
\end{equation}
where $\Gamma$ is the gamma function. One property of the gamma function used
below is that for $n\geq0$:
\begin{equation}
  \frac{\Gamma(n+1)}{\Gamma(n)} = n.
\end{equation}




\subsection{Mass content}
%
The mass content, $w$ (i.e.\ kg/m$^3$), for given $\no$, $\mu$, $\la$, $\ga$,
$a$ and $b$, is the size-integrated mass (P\&H, Eq.~22), that is proportional to
the b-th moment:
\begin{equation}
  w = \int_0^\infty m(x) n(x)\, dx = \int_0^\infty ax^b n(x)\, dx = aM_b.
\end{equation}
If mass content is taken as an input to the PSD, one of the four MGD parameters
must be selected as the dependent one. If $\no$ is selected as the dependent
variable, i.e. $n(w,\mu,\la,\ga)$, the implied value of $\no$ is:
\begin{equation}
  \no = \frac{w\ga\la^{(\mu+b+1)/\ga}}{a\Gamma(\frac{\mu+b+1}{\ga})}.
\end{equation}
To simplify the expressions below, we define $\eterm$ as
\begin{equation}
  \eterm = \frac{\mu+b+1}{\ga}.
  \label{eq:eterm}
\end{equation}
That is
\begin{equation}
  w = \frac{a\no}{\ga}\frac{\Gamma(\eterm)}{\la^\eterm},
  \label{eq:wfrommomb}
\end{equation}
and
\begin{equation}
  \no = \frac{w\ga\la^\eterm}{a\Gamma(\eterm)}.  
  \label{eq:n0fromw}
\end{equation}
The derivative of the PSD with respect to mass content is most easily handled
by the chain rule:
\begin{equation}
  \frac{dn}{dw} = \frac{dn}{d\no}\frac{d\no}{dw} = \frac{dn}{d\no} 
   \frac{\ga\la^\eterm}{a\Gamma(\eterm)}.
  \label{eq:dn0dw}
\end{equation}
The code in ARTS is organised in such way that the derivatives to the native
parameters are calculated as soon as they can be used (such as $dn/d\no$ in the
equation just above). Accordingly, further simplifications of expressions
involving terms from the chain-rule are not of practical interest. 

Regarding the derivatives for remaining three native parameters, they are
unchanged from the expressions given on Section~\ref{sec:mgd_native}. For
example, Eqs.~\ref{eq:dndmu} and \ref{eq:dndga} give the derivative with
respect to $\mu$ and $\ga$ independently of if $\no$ or $\la$ is selected as
dependent variable. The same is valid below, as long as $\mu$ and $\ga$ are not
dependent variables.

If instead $\la$ is the dependent parameter, $n(w,\no,\la,\ga)$, these expressions apply:
\begin{equation}
  \la = \left( \frac{w\ga}{a\no\Gamma(\eterm)} \right)^{-1/\eterm}
  \label{eq:mass:la}
\end{equation}
and
\begin{equation}
  \frac{dn}{dw} = \frac{dn}{d\la}\frac{d\la}{dw} = \frac{dn}{d\la} 
  \frac{(-1)}{\eterm}w^{-(1/\eterm+1)}\left( \frac{\ga}{a\no\Gamma(\eterm)} \right)^{-1/\eterm}.
\end{equation}
The choices of having $\mu$ and $\ga$ as dependent variables are not yet
handled. 

The workspace method for this form of MGD is \wsmindex{psdModifiedGammaMass}.



\subsection{Mass content and mean size}
%
This section deals with the case when mass content and mean size replaces two
of the native MGD parameters. There are six possible combinations of 
dependent parameters, but so far just a single one is handled, that $\no$ and
$\la$ are the dependent ones. 

Many definitions of mean and effective size exist. The definition of mean size,
$\xmea$, selected here is:
\begin{equation}
  \xmea =\frac{\int_0^\infty x m(x) n(x)\,dx}{\int_0^\infty m(x) n(x)\,dx}.
\end{equation}
That is, a mass-weighted mean size is considered. This definition is fully
consistent with e.g.\ Eq. 3 of \citet{delanoe2014normalized}, just expressed
in a more general manner. The equation above can rewritten in several ways:
\begin{equation}
  \xmea =\frac{\int_0^\infty x m(x) n(x)\,dx}{w}=\frac{M_{b+1}}{M_b}=
  \frac{\mu+b+1}{\ga}\la^{-1/\ga}=\eterm\la^{-1/\ga}.
  \label{eq:xmean2}
\end{equation}
As mentioned, so far only $\no$ and $\la$ are allowed to be the pair of
dependent variables: $n(w,\xmea,\mu,\ga)$. The implied value of $\la$ can be
obtained by Eq.~\ref{eq:xmean2}:
\begin{equation}
  \la = \left( \frac{\xmea}{e}\right)^{-\ga}.
\end{equation}
With $\la$ at hand, $\no$ can be determined as for the mass-only case
(Eq.~\ref{eq:n0fromw}).

As mass content only affects $\no$, Eq.~\ref{eq:dn0dw} is also valid here. The
partial derivative with respect to $\xmea$ is more complicated as both $\no$
and $\la$ are affected:
\begin{eqnarray}
  \frac{dn}{d\xmea} &=& \left( \frac{dn}{d\no}\frac{d\no}{d\la} + 
                        \frac{dn}{d\la} \right) \frac{d\la}{d\xmea} \nonumber\\
 &=& \left( \frac{dn}{d\no} \frac{w\ga\eterm\la^{\eterm-1}}{a\Gamma(\eterm)} + 
                        \frac{dn}{d\la} \right) (-\ga e^\ga \xmea^{-(\ga+1)}).
\end{eqnarray}
The workspace method for this form of MGD is \wsmindex{psdModifiedGammaMassXmean}.



\subsection{Mass content and median size}
%
The median size, $\xmed$, is defined as (P\&H, Eq.~21)
\begin{equation}
  \int_0^{\xmed} m(x) n(x)\, dx= \int_{\xmed}^\infty m(x) n(x)\, dx= \frac{W}{2}.
\end{equation}
There is no exact analytic expression for $\xmed$ but an approximate one is (Eq.~33)
\begin{equation}
  \xmed = \left( \frac{\mu+1+b-0.327\ga}{\la\ga} \right)^{1/\ga}
  \label{eq:xmed}
\end{equation}
As for mean size, so far only $\no$ and $\la$ are allowed to be the pair of
dependent variables: $n(w,\xmed,\mu,\ga)$. The implied value of $\la$ can be
obtained by Eq.~\ref{eq:xmed}:
\begin{equation}
  \la = \frac{\mu+1+b-0.327\ga}{\ga}\xmed^{-\ga}. 
\end{equation}
With $\la$ at hand, $\no$ can be determined as for the mass-only case
(Eq.~\ref{eq:n0fromw}).

As mass content only affects $\no$, Eq.~\ref{eq:dn0dw} is also valid here. The
partial derivative with respect to $\xmed$ is more complicated as both $\no$
and $\la$ are affected:
\begin{eqnarray}
  \frac{dn}{d\xmed} &=& \left( \frac{dn}{d\no}\frac{d\no}{d\la} + 
                        \frac{dn}{d\la} \right) \frac{d\la}{d\xmed} \nonumber\\
 &=& \left( \frac{dn}{d\no} \frac{w\ga\eterm\la^{\eterm-1}}{a\Gamma(\eterm)} + 
                        \frac{dn}{d\la} \right) (-1)(\mu+1+b-0.327\ga)\xmed^{-(\ga+1)} 
\end{eqnarray}
The workspace method for this form of MGD is \wsmindex{psdModifiedGammaMassXmedian}.



\subsection{Mass content and mean particle mass}
%
The mean particle mass is defined as
\begin{equation}
  \mmea = \frac{w}{\ntot}
  \label{eq:mmean}
\end{equation}
where $\ntot$ is the total number density, that equals the 0:th moment:
\begin{equation}
  \ntot = M_0.
  \label{eq:ntot}
\end{equation}
That is
\begin{equation}
  \mmea = \frac{aM_b}{M_0} = \frac{a}{\la^{b/\ga}}\frac{\Gamma(\eterm)}{\Gamma(\rterm)},
  \label{eq:mmean}
\end{equation}
where $\rterm$ is defined as
\begin{equation}
  \rterm = \frac{\mu+1}{\ga}.
  \label{eq:rterm}
\end{equation}
As for mean and median size, so far only $\no$ and $\la$ are allowed to be the
pair of dependent variables: $n(w,\mmea,\mu,\ga)$. From
Eq.~\ref{eq:mmean} we get
\begin{equation}
  \la = \left( \frac{a}{\mmea}
    \frac{\Gamma(\eterm)}{\Gamma(\rterm)}\right)^{\ga/b}.
  \label{eq:la:wandmmean}
\end{equation}
With $\la$ at hand, $\no$ can be determined as for the mass-only case
(Eq.~\ref{eq:n0fromw}).

As mass content only affects $\no$, Eq.~\ref{eq:dn0dw} is also valid here.
The derivative with respect to mean particle mass is 
\begin{eqnarray}
  \frac{dn}{d\mmea} &=& \left( \frac{dn}{d\no}\frac{d\no}{d\la} + 
                        \frac{dn}{d\la} \right) \frac{d\la}{d\mmea} \nonumber\\
 &=& \left( \frac{dn}{d\no} \frac{w\ga\eterm\la^{\eterm-1}}{a\Gamma(\eterm)} + 
                        \frac{dn}{d\la} \right) 
   \left(a\frac{\Gamma(\eterm)}{\Gamma(\rterm)}\right)^{\ga/b} \frac{(-\ga)}{b}\mmea^{-(\ga/b+1)}.
\end{eqnarray}
The workspace method for this form of MGD is \wsmindex{psdModifiedGammaMassMeanParticleMass}.


\subsection{Mass content and total number density}
%
The total number density is defined above by Eq.~\ref{eq:ntot}. As for the
combinations of mass and size above, so far only $\no$ and $\la$ are allowed to
be the pair of dependent variables: $n(w,\ntot,\mu,\ga)$. From
Eqs.~\ref{eq:mmean} and \ref{eq:la:wandmmean} we get
\begin{equation}
  \la = \left( \frac{a\ntot}{w}
    \frac{\Gamma(\eterm)}{\Gamma(\rterm)}\right)^{\ga/b}.
  \label{eq:la:wandntot}
\end{equation}
With $\la$ at hand, $\no$ can be determined as for the mass-only case
(Eq.~\ref{eq:n0fromw}).

As mass content is here affecting both $\no$ and $\la$,
Eq.~\ref{eq:dn0dw} is not valid here. We have instead:
\begin{eqnarray}
  \frac{dn}{dw} &=& \frac{dn}{d\no}\left(\frac{d\no}{dw} + 
     \frac{d\no}{d\la}\frac{d\la}{dw}\right) +
                    \frac{dn}{d\la} \frac{d\la}{dw} \nonumber\\
 &=& \frac{dn}{d\no} \left( \frac{\ga\la^\eterm}{a\Gamma(\eterm)} +
     \frac{w\ga\eterm\la^{\eterm-1}}{a\Gamma(\eterm)} \frac{d\la}{dw}\right) +
     \frac{dn}{d\la} \frac{d\la}{dw},
\end{eqnarray}
where
\begin{equation}
  \frac{d\la}{dw} = \left(a\ntot\frac{\Gamma(\eterm)}{\Gamma(\rterm)}\right)^{\ga/b}
     \frac{(-\ga)}{b}w^{-(\ga/b+1)}.
\end{equation}
The derivative with respect to $\ntot$ is:
\begin{eqnarray}
  \frac{dn}{d\ntot} &=& \left( \frac{dn}{d\no}\frac{d\no}{d\la} + 
                        \frac{dn}{d\la} \right) \frac{d\la}{d\ntot} \nonumber\\
 &=& \left( \frac{dn}{d\no} \frac{w\ga\eterm\la^{\eterm-1}}{a\Gamma(\eterm)} + 
                        \frac{dn}{d\la} \right) 
   \left(\frac{a}{w}\frac{\Gamma(\eterm)}{\Gamma(\rterm)}\right)^{\ga/b} \frac{\ga}{b}\ntot^{\ga/b-1}.
\end{eqnarray}
The workspace method for this form of MGD is \wsmindex{psdModifiedGammaMassNtot}.



\subsection{Avoiding numerical problems}
%
For simplicity, all MGD methods demand that both $\la$ and $\ga$ are $>0$.
First of all, this ensures that $\la x^\ga > 0$ as long as $x>0$. It further
avoids a number of possible ``division with zero'' cases, such as Eq.~\ref{eq:mdg:mk}.
It also demanded that $a$ and $b$ are $>0$. 

The gamma function is infinite at 0, and it also further demanded that argument
given to the gamma function is $>0$. When Eq.~\ref{eq:eterm} is of concern,
then $\mu+b+1>0$ is demanded. If Eq.~\ref{eq:rterm} is of concern, the demand
is harder: $\mu+1>0$.

Negative mass in unphysical but it could still be beneficial to allow such
values, to simplify performing. For this reason, negative mass contents are
allowed as far as possible. It is possible as long as mass content sets $\no$
(then giving negtive $\no$), but not when the mass content sets $\la$
(Eq.~\ref{eq:mass:la}). On the other hand, all input mean or median sizes are
required to be $>0$, both for physcial and numerical reasons.


% \section[McFarquhar and Heymsfield parametrization]
% {Ice particle size parameterization by McFarquhar and Heymsfield}
% \label{sec:McFHey_distr}

% A parameterization of tropical cirrus ice crystal size distributions was
% derived by \citet{mcfarquar97:_param_tropic_cirrus_ice_cryst}. The
% size distribution takes temperature and \imc\ as input. The parameterization
% was made based on observations during the Central Equatorial Pacific Experiment
% (CEPEX). Smaller ice crystals with an equal volume sphere radius of less than
% 50\,\mum\ are parametrized as a sum of first-order gamma functions:
% \begin{equation}
%   \label{eq:MH_gamma}
%   n(r) = \frac{12\cdot \imc_{<50}\alpha^5_{<50} r}{\pi \rho
%     \Gamma(5)} \exp(-2 \alpha_{<50} r), 
% \end{equation}
% where $\alpha_{<50}$ is a parameter of the distribution, and
% $\imc_{<50}$ is the mass of all crystals smaller than 50\,\mum\ in
% the observed size distribution.  Large ice crystals are represented
% better by a log-normal function
% \begin{eqnarray}
%   \label{eq:MH_lognorm}
%   n(r) = \frac{3\cdot\imc_{>50}}{\pi^{3/2}\rho \sqrt{2}
%     \exp(3\mu_{>50}+(9/2)\sigma^2_{>50}) r \sigma_{>50} r_0^3}
%   \nonumber \\
%   \cdot \exp\left[-\frac{1}{2}\left(\frac{\log\frac{2r}{r_0} -
%           \mu_{>50}}{\sigma_{>50}}\right)^2\right], 
% \end{eqnarray}
% where $\imc_{>50}$ is the mass of all ice crystals greater than
% 50\,\mum\ in the observed size distribution, $r_0$ = 1\,\mum\ is a
% parameter used to ensure that the equation does not depend on the
% choice of unit for r, $\sigma_{>50}$ is the geometric standard
% deviation of the distribution, and $\mu_{>50}$ is the location of the
% mode of the log-normal distribution.  The fitted parameters of the
% distribution can be looked up in the article by
% \citet{mcfarquar97:_param_tropic_cirrus_ice_cryst}.  The particle
% number density field is obtained by numerical integration over a
% discrete set of size bins. 



