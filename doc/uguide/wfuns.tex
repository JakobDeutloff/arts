\chapter{Clear-sky Jacobians}
 \label{sec:wfuns}

 \starthistory
 110826 & First complete version by Patrick Eriksson.\\
 \stophistory

\graphicspath{{Figs/wfuns/}}


Inversions of both OEM and Tikhonov type require that the Jacobian can be
provided by the forward model \citep[see e.g.][]{eriksson:analy:00}. A
retrieval characterisation following \citet{rodgers:90,rodgers:00} raises the
same demand. A column of the Jacobian, \aWfnMtr{\SttVct}, is defined as
\begin{equation}
  \frac{\partial\MsrVct}{\partial\aSttElm{p}},
  \label{eq:wfuns:ki}
\end{equation}
where \MsrVct\ is the vector of measurement data and \aSttElm{p}\ is one forward
model (scalar) variable. See further Section~\ref{T-sec:formalism:wfuns} of
\theory. The nomenclature of that section is also used here.

The quantity in Eq.~\ref{eq:wfuns:ki} is in the atmospheric sounding
community frequently denoted as a ``weighting function'', and accordingly
\aWfnMtr{\SttVct} is called the weighting function matrix. In the documentation
of ARTS both terms (Jacobian and weighting functions) are used. These names
refer normally to \aWfnMtr{\SttVct}, the partial derivatives with respect to
the variables to be retrieved, forming the state vector \SttVct. However, in
the context of retrieval characterisation, the same matrix for the remaining
model parameters is of equally high interest, denoted as \aWfnMtr{\FrwMdlVct}.
In the same manner, the terms inversion and retrieval are used interchangeably.

The main task of the user is to select which quantities that shall be
retrieved, and to define the associated retrieval grids. These aspects must be
considered for successful inversions, but are out of scope for this document.
Beside for the most simple retrievals, it is further important to understand
how the Jacobian is calculated. A practical point is the calculation speed,
primarily determined if perturbations or analytical expressions are used
(Sec.~\ref{sec:wfuns:intro}). The derivation of the different Jacobians
involves some approximations due to theoretical and practical considerations.
Such approximations can be accepted, if of low or moderate size, but will
result in a slower convergence (the inversion will require more iterations).
Due to these later aspects, and to meet the needs of more experienced users,
this section is relatively detailed and contains a (high?) number of equations.

This section is restricted to Jacobians for clear-sky conditions, i.e.\ to be
applied outside the cloudbox. So far none of the scattering methods provide
Jacobians. Sections \ref{sec:wfuns:intro}\,-\,\ref{sec:wfuns:basis} contain
information of general character, while the available quantities are discussed
in the remaining sections (Section~\ref{sec:wfuns:absspecies} and forward).



\section{Introduction}
%==============================================================================
\label{sec:wfuns:intro}
%
There are two main approaches for calculating Jacobians, by analytical
expressions and by perturbations. We start with the conceptually simplest one,
but also the more inefficient approach.



\subsection{Perturbations}
%==============================================================================
\label{sec:wfuns:pert}
%
The most straightforward method to determine the Jacobian is by perturbing the
model parameter of concern. For example, the Jacobian corresponding to state
variable $p$ can always be calculated as
\begin{equation}
  \aWfnMtr{\SttVct}^p = \frac{\FrwMdl(\SttVct+\Delta\SttElm\VctStl{e}^p,
                      \FrwMdlVct)-\FrwMdl(\SttVct,\FrwMdlVct)} {\Delta\SttElm}
 \label{eq:wfuns:perturb}
\end{equation}
where $(\SttVct,\FrwMdlVct)$ is the linearization state, $\VctStl{e}^p$ is a
vector of zeros except for the $p$:th component that is unity, and
$\Delta\SttElm$ is a small disturbance (but sufficiently large to avoid
numerical instabilities).

However, it is normally not needed to make a recalculation using the total
forward model as the variables are either part of the atmospheric or the sensor
state, but not both. In addition, in many cases it is possibly to find
short-cuts. For example, the perturbed state can be approximated by an
interpolation of existing data (such as for a perturbed zenith angle). Such
short-cuts are discussed separately for each retrieval quantity.


\subsection{Analytical expressions}
%==============================================================================
\label{sec:wfuns:anal}
%
For most atmospheric variables, such as species abundance, it is possible to
derive an analytical expression for the Jacobians. This is advantageous because
they result in faster and more accurate calculations. Such expressions are
derived below. Some of the terms involved are calculated as a perturbation.
This is partly a consequence of the flexibility of ARTS. The high-level
radiative transfer methods (e.g.\ \builtindoc{iyEmissionStandard}) do not know
how the low-level methods are defining all quantities, and a fixed analytical
expression can not be used (see e.g.\ Sec.~\ref{sec:wfuns:atmtemp}). So, in
practise, the calculations are ``semi-analytical''. 
The analytical calculations are introduced in Sec.~\ref{sec:wfuns:atmvars}. 


\subsection{Workspace variables and methods}
%==============================================================================
\label{sec:wfuns:wsm}
%
As a workspace variable, the complete Jacobian is denoted as
\wsvindex{jacobian}. Auxiliary information is provided by
\wsvindex{jacobian\_quantities} and \wsvindex{jacobian\_indices}. The actual
calculations are made as part of \builtindoc{yCalc}.

The retrieval quantities are defined separately, before calling
\builtindoc{yCalc}. This process is started by calling \wsmindex{jacobianInit}.
The retrieval quantities are then introduced through workspace methods named as
jacobianAdd{\it Something}. For example, for atmospheric temperature the method
is \builtindoc{jacobianAddTemperature}. It does not matter in which order these
``add methods'' are called.

The definition of retrieval quantities is finalised by calling
\wsmindex{jacobianClose}. To disable the calculation of the Jacobian, skip all
above, and just use \wsmindex{jacobianOff}. The methods named jacobianCalc{\it
  Something} shall never be used directly. Neither needs the user to consider
\wsaindex{jacobian\_agenda}.

The input to the ``add methods'' differs. In some cases you can select between
the analytical and perturbation options. For all perturbation calculations you
must specify the size of the perturbation. For atmospheric gases you can use
different units. For atmospheric fields, and some other quantities, you must
define the retrieval grid(s) to use.



 
\section{Basis functions}
%==============================================================================
\label{sec:wfuns:basis}

A forward model must use a discrete representation: it describes each quantity
with one or several variables. This is unproblematic for quantities that are of
discrete nature (including scalar variables). However, for atmospheric fields
and other continuous model quantities, the discrete representation inside the
forward model requires consideration. To avoid inconsistencies between model
input and output it is important that the mapping from the discrete variables
to the ``continuous view'' of the quantity is well defined, and applied
consistently through the forward model . This mapping is given by the basis
functions\index{basis function}. Similar arguments and nomenclature are found
in \citet{read:thecl:06}.

The basis functions are discussed explicitly in few places in this user guide,
but it shall be noted that all interpolations imply an underlying set of basis
functions. On the other hand, an understanding of both the derivation and the
obtained Jacobians require direct consideration of the basis functions. ARTS
operates with two types of basis functions.




\subsection{Basis functions for piece-wise linear quantities}
\label{sec:wfuns:basis1}
%
To treat an one-dimensional quantity to be piece-wise linear, or to say that a
linear interpolation shall be applied, are identical definitions. The basis
functions matching this definition have triangular shape, sometimes denoted as
``tenth functions''. Such functions are exemplified in
Fig.~\ref{fig:wfuns:zbasis}, see also \citet{buehler:artst:05}.

\begin{figure}[t]
 \begin{center}
  \includegraphics*[width=0.7\hsize]{fig_absbasis_z}
  \caption{Examples on 1d basis functions for a vertical grid with a 1 km
           spacing: \lsolid~30~km, \ldashed~31~km and \ldashdot~32~km.}
  \label{fig:wfuns:zbasis}  
 \end{center}
\end{figure}

In most cases, the quantity is considered to be undefined outside the end
points of the grid. Hence, the basis function for a grid end point is then just
``half a tenth''. The exception to this rule is retrieval grids of piece-wise
linear variables. To avoid that retrieval grids must cover the complete
atmosphere, end point values are assumed to be valid to the end of the
atmosphere (or data range of concern). That is, the basis functions for end
points of retrieval grids follow the tenth shape inside the grid range, and
have a constant value of 1 outside. In terms of interpolation, this matches to
allow extrapolation, the applying a ``nearest'' interpolation for positions
outside the covered range (the end values are valid all the way to
$\pm$infinity).
The basis functions are defined likewise for higher dimensions, but the
tenth functions are then 2D or 3D ``tenths''.




\subsection{Polynomial basis functions}
\label{sec:wfuns:basis2}
%
Some retrieval quantities are expressed using a polynomial basis. Sensor zenith
angle pointing off-set is one such quantity. The off-set is then treated to
have a polynomial variation as a function of time. If the offset is assumed to
be constant in time, a zero order polynomial shall be selected. If there is
also a linear drift with time, use a first order polynomial, etc.

For these basis functions, the explanatory variable (time in the example above)
is normalised to cover the range [-1,1], here denoted as $z$, and the
continuous representation ($f$) of the variable of concern can be written as
\begin{equation}
  f(z) = \aSttElm{0} + \aSttElm{1}(z-b_1) + \aSttElm{2}(z^2-b_2) + 
                     \aSttElm{3}(z^3-b_3) + \aSttElm{4}(z^3-b_4) + \dots  
\end{equation}
where $\aSttElm{0}, \aSttElm{1}, \dots$ are the coefficients to be retrieved
(elements of \SttVct). The interpretation of a retrieval is simplified if the
average of $f$ equals \aSttElm{p0}, and the scalars $b_1, b_2, \dots$ are
selected, schematically, as
\begin{equation}
  0 = \int_{-1}^1 \!\left(z^n-b_n\right) \, \DiffD z, \quad n>0.
\end{equation}
According to this expression, $b_n$ is zero for odd $n$. However, $z$ is in
practise a discrete variable ($z_i$, not necessarily symmetric around 0), and
$b_n$ is taken as the average of $z_i^n$: all $b_n$ can be non-zero. The
normalisation of $z$ is not only made for interpretation reasons, it can be
required for pure numerical reasons, such as when $z$ represent frequency (in
Hz).

In practise, the basis functions are vectors, denoted below as $\VctStl{z}_i$.
Element $j$ of $\VctStl{z}_i$ is
\begin{equation}
  \VctStl{z}_i(j) = z(j)^i - b_i.
\end{equation}




\section{Atmospheric variables, common expressions}
%==============================================================================
\label{sec:wfuns:atmvars}
%
The analytically-oriented calculation procedure to obtain the Jacobian for
atmospheric quantities is here outlined. The expressions are based on the
chain rule, and can be applied for absorption constituents, atmospheric
temperatures and winds. It is important to notice that only local effects are
considered, and the expressions have limitations, as discussed below in
Sec.~\ref{sec:wfuns:atmvars:limit}.


\subsection{Matrix derivatives}
\label{sec:wfuns:atmvars:matder}
%
Some of the expressions of this section treat derivatives involving vectors
and matrices. Such derivatives must be handled with some care, the standard
calculation rules do not always apply. The basic relationships used here
are\footnote{See e.g.\ \url{en.wikipedia.org/wiki/Matrix_calculus}.}
\begin{eqnarray}
  \frac{\PartD\VctStl{a}}{\PartD\VctStl{a}} &=& 
    \IdnMtr, \\
  \frac{\PartD(\MtrStl{A}\VctStl{b})}{\PartD\VctStl{a}} &=& 
    \MtrStl{A}\frac{\PartD\VctStl{b}}{\PartD\VctStl{a}}, \quad\quad
    \mathrm{when}\ \MtrStl{A}\ \mathrm{independent\ of}\ \VctStl{a},\\
  \frac{\PartD(\VctStl{b}+\VctStl{c})}{\PartD\VctStl{a}} &=& 
    \frac{\PartD\VctStl{b}}{\PartD\VctStl{a}}+
    \frac{\PartD\VctStl{c}}{\PartD\VctStl{a}},
\end{eqnarray}
where \VctStl{a}, \VctStl{b}\ and \VctStl{c}\ are vectors, and \MtrStl{A}\ is a
matrix.

Matrix exponential derivatives can not be expressed analytically in a general
manner, but some cases can be handled. If two matrices \MtrStl{A}\ and
\MtrStl{B}\ are commutative, i.e. $\MtrStl{A}\MtrStl{B}=\MtrStl{B}\MtrStl{A}$,
then the following is valid \citep{Dattorro2011}\footnote{Available at
  \url{https://ccrma.stanford.edu/~dattorro/mybook.html}.}
\begin{equation}
  \label{eq:expmder}
  \frac{\PartD}{\PartD t} e^{t\MtrStl{A}+\MtrStl{B}} = 
       \MtrStl{A}e^{t\MtrStl{A}+\MtrStl{B}},
\end{equation}
where $t$ is a scalar variable.
The cases where \MtrStl{A}\ and \MtrStl{B}\ are commutative includes the
situation where one of them is a scalar matrix (can be written as $a\IdnMtr$,
i.e.\ diagonal with all non-zero elements equal).




\subsection{Separation of terms}
\label{sec:wfuns:atmvars:sepe}
%
The overall task is to calculate (Eq.~\ref{eq:wfuns:ki} with subscript $p$ left
out)
\begin{displaymath}
  \frac{\PartD\MsrVct}{\PartD\SttElm}  
\end{displaymath}
where \SttElm\ is the element of $\SttVct$ for which we want to obtain the
weighting function. This is a column of the complete weighting function matrix.

A first step is to identify how sensor characteristics can be incorporated? To
make the nomenclature simpler, we assume here that the simulations cover only a
single measurement block and we have (cf.\ Eq.~\ref{eq:fm_defs:measseq})
\begin{equation}
  \MsrVct = \SnsMtr\aMpiVct{b}.
  \label{eq:wfuns:Hi}
\end{equation}
We can then apply the chain-rule for a first time to obtain 
\begin{equation}
  \frac{\PartD\MsrVct}{\PartD\SttElm} = 
  \frac{\PartD\MsrVct}{\PartD\aMpiVct{b}} 
  \frac{\PartD\aMpiVct{b}}{\PartD\SttElm} = 
  \SnsMtr \frac{\PartD\aMpiVct{b}}{\PartD\SttElm} = \SnsMtr\VctStl{k}_\MpiVct.
  \label{eq:wfuns:Hk}
\end{equation}
Hence, sensor characteristics can be handled by calculating the weighting
function column matching all monochromatic pencil beam calculation of the
measurement block ($\VctStl{k}_\MpiVct$) and perform a multiplication with
\SnsMtr, a parallel procedure compared to how \aMpiVct{b}\ is compiled to obtain
\MsrVct. The calculation procedure expands to allow that the complete
weighting function matrix (for quantities covered by the analytical calculation
procedure) is calculated as
\begin{equation}
  \frac{\PartD\MsrVct}{\PartD\SttVct} = \SnsMtr\aWfnMtr{\MpiVct},
  \label{eq:wfuns:HK}
\end{equation}
where $\aWfnMtr{\MpiVct} = \PartD\aMpiVct{b}/\PartD\SttVct$. One column of this
matrix is $\VctStl{k}_\MpiVct$.

The vector \aMpiVct{b}\ consists of a number of Stokes vectors appended,
$\aMpiVct{b}=[\aStoVec{1}^T, \aStoVec{2}^T,\dots,\aStoVec{n}^T]^T$, and the
calculation of $\VctStl{k}_\MpiVct$ can schematically be written as
\begin{equation}
  \VctStl{k}_\MpiVct = \sum_{j=1}^n \frac{\PartD\aMpiVct{b}}{\PartD\aStoVec{j}}
    \frac{\PartD\aStoVec{j}}{\PartD\SttElm}
  \label{eq:wfuns:kisum}
\end{equation}
The terms $\PartD\MpiVct/\PartD\aStoVec{j}$ are formally matrices. However,
these matrices are not calculated explicitly as they only contain information
on where \aStoVec{i}\ is stored inside \MpiVct. That is, these matrices are of
bookkeeping character, consisting only of zeros and ones. In practice, results
matching $\PartD\aStoVec{j}/\PartD\SttElm$ are simply inserted in correct place
of $\VctStl{k}_\MpiVct$, mimicking how is \aStoVec{i}\ put into \MpiVct.

Accordingly, the core task is to calculate $\PartD\aStoVec{j}/\PartD\SttElm$,
where for simplicity the subscript $j$ is dropped below. This calculation is
expanded as
\begin{equation}
  \frac{\PartD\StoVec}{\PartD\SttElm} = \sum_{i=1}^n
    \frac{\PartD\StoVec}{\PartD\aSttElm{i}}
    \frac{\PartD\aSttElm{i}}{\PartD\SttElm},
\end{equation}
where \aSttElm{i}\ is the value of the quantity of concern, at each point of
the propagation path. That is, $i$ indexes the path points. The actual
radiative transfer enters by the terms $\PartD\StoVec/\PartD\aSttElm{i}$, and
they are discussed separately below.

The term $\PartD\aSttElm{i}/\PartD\SttElm$ appears due to the fact that
\aSttElm{i}\ and \SttElm\ are placed at different positions, and the
representation of the atmospheric fields must be considered here. In practise,
the term is calculated as the value of the basis function for \SttElm\ at the
location of \aSttElm{i}\ (further discussed in \citet{buehler:artst:05}). This
is a slight approximation with respect to the goal of fully incorporating the
piece-wise linear representation in the weighting functions
\citep{buehler:artst:05}. A low value of $\Delta\aPpathLng{i}$ decreases the
degree of approximation.

It can be noted that $\PartD\aSttElm{i} / \PartD\SttElm$ is normally non-zero
for more than one element of \SttVct. The exception is if the positions of
\aSttElm{i}\ and \SttElm\ are identical. Reversely, the weighting function for
element \SttElm\ can have contributions from several propagation path points
(\aSttElm{i}), as well as from several radiance spectra. Accordingly, the
practical calculations are done by first determine all
$\PartD\StoVec/\PartD\aSttElm{i}$, of the given propagation path. These data
are then used to determine $\PartD\StoVec/\PartD\SttVct$, for the retrieval
quantity of concern. That is, each $\PartD\StoVec/\PartD\aSttElm{i}$ is
combined with $\PartD\aSttElm{i}/\PartD\SttElm$ for all elements of \SttVct.
Most of these combinations yield a zero result. The terms
$\PartD\aSttElm{i}/\PartD\SttElm$ are determined with help of ARTS' internal
interpolation grid position routines.



\subsection{$\PartD\StoVec/\PartD\aSttElm{i}$, general 
case}
\label{sec:wfuns:atmvars:gene}
%
The term $\PartD\StoVec/\PartD\aSttElm{i}$ is here outlined for the general case
of vector radiative transfer. In this case, the Stokes elements can not be
treated separately, and matrix-vector notation is required. The final Stokes
vector obtained through Eq.~\ref{eq:fm_defs:vrte_step} can be expressed as
\begin{equation}
  \label{eq:s4J}
  \StoVec = \StoVec' + \TrnMat' \left[ \aTrnMat{i} \aStoVec{i} +
         (\IdnMtr-\aTrnMat{i})\bar{\EmsVec}_i \right],
\end{equation}
where $\StoVec'$ is the Stokes vector for all emission generated between the
sensor and point $i+1$, $\TrnMat'$ is the transmission Mueller matrix for the
same part of the propagation path and \aTrnMat{i} is defined in
Eq.~\ref{eq:rte:transmat}. The quantities $\StoVec'$, $\TrnMat'$ and
\aStoVec{i}\ are not function of \aSttElm{i}. This gives
\begin{equation}
  \frac{\PartD\StoVec}{\PartD\aSttElm{i}} = \TrnMat'
  \left[
    \frac{\PartD\aTrnMat{i}}{\PartD\aSttElm{i}}
      \left( \aStoVec{i}-\bar{\EmsVec}_i \right) +
    \left( \IdnMtr-\aTrnMat{i} \right) 
      \frac{\PartD\bar{\EmsVec}_i}{\PartD\aSttElm{i}}
  \right].
  \label{eq:wfuns:dsdxi1}
\end{equation}
The terms $\PartD\aTrnMat{i}/\PartD\aSttElm{i}$ and
$\PartD\bar{\EmsVec}_i/\PartD\aSttElm{i}$ are both calculated in a pure
numerical manner, for different reasons discussed below.


\subsection{Including the surface}
%
For scattered down-welling radiation the effective transmission matrix is
\begin{equation}
  \TrnMat' = \aTrnMat{2}\MtrStl{R}_i\aTrnMat{1},
  \label{eq:wfuns:surface}
\end{equation}
where \aTrnMat{2}\ is the transmission between the surface and the sensor,
$\MtrStl{R}_i$ is defined in Eq.~\ref{eq:fm_defs:surfacerefl} and
\aTrnMat{1}\ is the transmission between the point $i+1$ and the surface.



\subsection{$\PartD\StoVec/\PartD\aSttElm{i}$, locally unpolarised absorption}
\label{sec:wfuns:atmvars:unpol}
%
Eq.~\ref{eq:wfuns:dsdxi1} can be simplified for some conditions. A first case
is scalar radiative transfer, i.e. only the first element of the Stokes vector
is considered and all terms of Eq.~\ref{eq:wfuns:dsdxi1} are scalar quantities.
For all other cases, in principle vector radiative transfer must be performed,
as $\TrnMat'$ can always have off-diagonal elements. Even if atmospheric
absorption is totally unpolarised, $\TrnMat'$ can be non-diagonal due to the
surface (Eq.~\ref{eq:wfuns:surface}). However, if the absorption locally lacks
polarisation, the calculations can be handled by analytical expressions in a
higher degree.

To focus on the analytical expressions dealing with local radiative transfer
effects, let us write Eq.~\ref{eq:s4J} as
\begin{equation}
  \StoVec = \StoVec' + \TrnMat'\aStoVec{s}.
\end{equation}
In the nomenclature of Eq.~\ref{eq:fm_defs:vrte_step},
$\aStoVec{s}=\aStoVec{i+1}$ (but would be confusing to here use \aStoVec{i+1}).
Hence
\begin{displaymath}
  \frac{\PartD\StoVec}{\PartD\aSttElm{i}} = \TrnMat'
    \frac{\PartD\aStoVec{s}}{\PartD\aSttElm{i}}.
\end{displaymath}
The first element of \aStoVec{s}, \StoI, can be written as (cf.\
Eq.~\ref{eq:fm_defs:rte_step})
\begin{equation}
  \label{eq:I4J}
  \StoI = e^{-\aOth{i}}\aStoI{i} + (1-e^{-\aOth{i}})\bar{\Planck}_i.
\end{equation}
By again using the chain rule, the derivative of \StoI\ with respect to
\aSttElm{i}\ can be written as
\begin{equation}
  \frac{\PartD\StoI}{\PartD\aSttElm{i}} =
  \frac{\PartD\StoI}{\PartD\aOth{i}}\frac{\PartD\aOth{i}}{\PartD\aSttElm{i}} + 
  \frac{\PartD\StoI}{\PartD\bar{\Planck}_i}
  \frac{\PartD\bar{\Planck}_i}{\PartD\aSttElm{i}} = 
   e^{-\aOth{i}}\frac{\PartD\aOth{i}}{\PartD\aSttElm{i}}
      (\bar{\Planck}_i-\aStoI{i}) +
  (1-e^{-\aOth{i}})\frac{\PartD\bar{\Planck}_i}{\PartD\aSttElm{i}}.
  \label{eq:wfuns:dsdxi2}
\end{equation}
The two remaining derivatives $\PartD\aOth{i}/\PartD\aSttElm{i}$ and
$\PartD\bar{\Planck}_i/\PartD\aSttElm{i}$
depend on the quantity considered, and are discussed further below.

For higher Stokes components of \aStoVec{s}, or in general if emission is
totally ignored, Eq.~\ref{eq:I4J} is simplified to (here exemplified for the
second Stokes element, \StoQ)
\begin{equation}
  \label{eq:Q4J}
  \StoQ = e^{-\aOth{i}}\aStoQ{i},
\end{equation}
and the chain rule expression is correspondingly shorter:
\begin{equation}
  \frac{\PartD\StoQ}{\PartD\aSttElm{i}} =
   -e^{-\aOth{i}}
   \frac{\PartD\aOth{i}}{\PartD\aSttElm{i}}\aStoQ{i}.
  \label{eq:wfuns:dsdxi3}
\end{equation}


\subsection{Limitations}
\label{sec:wfuns:atmvars:limit}
%
A constraint for the analytical expressions above is that the effect of the
variable must only be local. Main examples on non-local effects should occur
through hydrostatic equilibrium and refraction. Significant impact of a gas
through these mechanisms should only be found for water vapour in the lower
troposphere, but is a general concern for temperature as discussed in
Sec.~\ref{sec:wfuns:atmtemp}.





\section{Absorption species}
%==============================================================================
\label{sec:wfuns:absspecies}

\subsection{Common practicalities}
%
To obtain the Jacobian for absorption species, use
\wsmindex{jacobianAddAbsSpecies}. The method handles one species at the time.
The calculations can either be done in in ``analytical'' or ``perturbation''
manner. For gases, weighting functions (WFs) can be provided for several units
of the gas abundance:
\begin{description}
\item[vmr] Volume mixing ratio (a value between 0 and 1). The WF divided by
  $10^6$ corresponds to that 1 ppm of the gas is added to the atmospheric
  volume of concern.
\item[nd] Number density. The WF corresponds here to that one molecule is added.
\item[rel] Relative/fractional change. In a perfectly linear case, the WF
  corresponds here to that the gas amount is doubled.
\item[logrel] This option returns the same WFs as ``rel'', but is included to
  flag that the natural logarithm of the ``rel'' unit is retrieved.
\end{description}
For the ``rel'' and ``logrel'' options it is important to note that ARTS
calculate the WFs with respect to the given state, ARTS does not know anything
about the actual reference state for which the ``rel'' unit is valid (where
normally the a priori state is selected). For iterative inversions, a rescaling
of the WFs provided by ARTS is likely needed, to make the WFs valid with
respect to the (original) reference state. For the assumption made inside ARTS,
the WFs for ``rel'' and ``logrel'' are identical.

A second main consideration is to select the retrieval grids. For analytical
calculations there are no other selections to be made. 


\subsection{Perturbation calculations}
%
For pure numerical calculations, also the size of the perturbation must be
specified ($\Delta\SttElm$ in Eq.~\ref{eq:wfuns:perturb}). The perturbation
shall given following the unit selected. The same value is applied for all WFs
(which can cause practical problems \dots).


\subsection{Analytical expressions}
%
If not made clear above, the only term that differs between the Jacobian
quantities is $\PartD\StoVec/\PartD\aSttElm{i}$.

\subsubsection{Unpolarised absorption}
%

The term $\PartD\bar{\Planck}_i/\PartD\aSttElm{i}$ of Eq.~\ref{eq:wfuns:dsdxi2}
is zero and the chain rule expression reduces to
\begin{equation}
  \frac{\PartD\StoI}{\PartD\aSttElm{i}} =
   e^{-\aOth{i}}\frac{\PartD\aOth{i}}{\PartD\aSttElm{i}}
      (\bar{\Planck}_i-\aStoI{i})  
\end{equation}
The absorption coefficient at point $i$, \aAbsCoef{i}, is here written as
\begin{equation}
  \label{eq:wfuns:crossec1}
  \aAbsCoef{i} = \aSttElm{i}\sigma_i + \aAbsCoef{i}^o,
\end{equation}
where $\sigma_i$ is the absorption cross-section for the species (at point $i$
and in a unit matching the unit of \SttElm), and $\aAbsCoef{i}^o$ covers the
summed absorption of all other atmospheric constituents. Considering
Eq.~\ref{eq:taustep}, the derivative of \aOth{i}\ with respect to \aSttElm{i}\
is then 
\begin{equation}
  \frac{\PartD\aOth{i}}{\PartD\aSttElm{i}} = 
  \frac{\PartD\aOth{i}}{\PartD\aAbsCoef{i}} 
  \frac{\PartD\aAbsCoef{i}}{\PartD\aSttElm{i}} = 
    \frac{\Delta\aPpathLng{i}}{2}\sigma_i.
\end{equation}

%For clarity and efficiency reasons, the following nomenclature is used in the
%source code:
%\begin{eqnarray}
%  X &=& \Trn' \aTrn{i}\frac{\Delta\aPpathLng{i}}{2}, \\
%  Y &=& \frac{\PartD\StoI}{\PartD\aTrn{i}}
%        \frac{\PartD\aTrn{i}}{\PartD\aOth{i}}
%       \frac{\PartD\aOth{i}}{\PartD\aAbsCoef{i}}
%    \qquad \left( = -X(\aStoI{i}-\bar{\Planck}_i) \right).
%\end{eqnarray}
%The expressions above are implemented in \wsmindex{iyEmissionStandard}.
%
%For higher Stokes elements, Eq.~\ref{eq:chainrule1q} is followed. Calculation
%time can be saved by reusing the $X$-term derived for the first Stokes element:
%\begin{equation}
%  \frac{\PartD\StoQ}{\PartD\aTrn{i}}
% \frac{\PartD\aTrn{i}}{\PartD\aOth{i}}
%  \frac{\PartD\aOth{i}}{\PartD\aAbsCoef{i}}
%    = -X \aStoI{i}.
%\end{equation}
%This expression is used for Stokes element 2 to 4 in
%\wsmindex{iyEmissionStandard} and for all components in
%\wsmindex{iyTransmissionStandard}.


\subsubsection{General case}
%
So far, the term $\PartD\aTrnMat{i}/\PartD\aSttElm{i}$ is only estimated by
performing a small perturbation, of the absorption associated with the species
to be retrieved.

However, there exists a potential to use an analytical expression, as discussed
below, but this is left for future consideration. The relationship between
\aSttElm{i}\ and the propagation matrix of the path step can be written as
\begin{equation}
  \bar{\ExtMat}_i = \frac{\aSttElm{i}\MtrStl{C}_i+
                      \aSttElm{i+1}\MtrStl{C}_{i+1}}{2} + \bar{\ExtMat}_i^o,
\end{equation}
where \MtrStl{C}\ can be seen as a matrix absorption cross-section
(corresponding to $\sigma$ in Eq.~\ref{eq:wfuns:crossec1}) and
$\bar{\ExtMat}_i^o$ is the propagation matrix summed for all other atmospheric
constituents. 
On the condition that Eq.~\ref{eq:expmder} is valid, we have that
\begin{equation}
  \frac{\PartD\aTrnMat{i}}{\PartD\aSttElm{i}} = 
  \frac{\PartD}{\PartD\aSttElm{i}} 
  e^{-\Delta\aPpathLng{i}\left(\frac{\aSttElm{i}\MtrStl{C}_i+
            \aSttElm{i+1}\MtrStl{C}_{i+1}}{2} + \bar{\ExtMat}_i^o \right)} =
  \frac{-\Delta\aPpathLng{i}}{2} \MtrStl{C}_i\aTrnMat{i}.
  \label{eq:wfuns:vec:dsdxi}
\end{equation}
The question is when Eq.~\ref{eq:wfuns:vec:dsdxi} can be applied? We have three
main cases:
\begin{itemize}
\item The equation is valid as long as the species to be retrieved has
  unpolarised absorption. This as $\MtrStl{C}_i$ and $\MtrStl{C}_{i+1}$ can
  then both be written as $\sigma_i\IdnMtr$, and
  $(\aSttElm{i}\MtrStl{C}_i+\aSttElm{i+1}\MtrStl{C}_{i+1})/2$ is commutative
  with any $\bar{\ExtMat}_i^o$.
\item Eq.~\ref{eq:wfuns:vec:dsdxi} is not valid when all three matrices
  involved ($\bar{\ExtMat}_i^o$ and the two \MtrStl{C}-matrices) have
  off-diagonal elements at the same time. However, this is is unlikely case.
  The main causes to polarised absorption are free electrons, Zeeman splitting
  and particles, and these mechanisms are mainly found in the ionosphere,
  mesosphere and troposphere, respectively.
\item The remaining case is when $\bar{\ExtMat}_i^o$ is diagonal but the
  species to be retrieved has polarised absorption. The likely situation is
  that both $\MtrStl{C}_i$ and $\MtrStl{C}_{i+1}$ are non-diagonal on the same
  time, and the condition of Eq.~\ref{eq:expmder} is not fulfilled. However, as
  argued below, Eq.~\ref{eq:wfuns:vec:dsdxi} should still be approximately
  valid, and this is used as ``working hypothesis'' in ARTS.
\end{itemize}
An alternative way to express the propagation matrix is
\begin{displaymath}
  \bar{\ExtMat}_i = \frac{\aSttElm{i}+\aSttElm{i+1}}{2}\bar{\MtrStl{C}}_i + 
   \bar{\ExtMat}_i^o,   
\end{displaymath}
with
\begin{displaymath}
  \bar{\MtrStl{C}}_i = \frac{\MtrStl{C}_i+\MtrStl{C}_{i+1}}{2}.   
\end{displaymath}
With this formulation even the third case above would be handled, as we now
only have two matrices to consider (and $\bar{\MtrStl{C}}_i$ is commutative
with a diagonal $\bar{\ExtMat}_i^o$) . We would then have
\begin{displaymath}
  \frac{\PartD\aTrnMat{i}}{\PartD\aSttElm{i}} =  
  \frac{-\Delta\aPpathLng{i}}{2} \bar{\MtrStl{C}}_i \aTrnMat{i}.    
\end{displaymath}
As $\MtrStl{C}_i$ and $\MtrStl{C}_{i+1}$ are (very) similar matrices (and then
also close to $\bar{\MtrStl{C}}_i)$, the practical difference between this
equation and Eq.~\ref{eq:wfuns:vec:dsdxi} should be small. Or reversely,
Eq.~\ref{eq:wfuns:vec:dsdxi} should be approximately correct as $\MtrStl{C}_i$
and $\MtrStl{C}_{i+1}$ are close to identical matrices.



\section{Winds}
%==============================================================================
\label{sec:wfuns:winds}
%
Calculation of wind weighting functions are triggered by
\wsmindex{jacobianAddWind}. Each wind component (see Sec.~\ref{sec:winds:defs})
is treated as an individual retrieval variable. That is, if you want to
retrieve all three wind components, \builtindoc{jacobianAddWind} must be called
three times. Only the analytically inclined calculation approach is at hand
for winds.

Theoretically, the Doppler shift induced by winds affects the emission source
term ($\bar{\Planck}_i$, in the scalar case), but this impact is extremely small
and the related terms are ignored. This gives a case basically identical to the
one above for absorption species. The only difference is that also the term
$\PartD\aAbsCoef{i}/\PartD\aSttElm{i}$ is obtained in a pure numerical manner,
by recalculating the absorption coefficient with the wind perturbed slightly.





\section{Atmospheric temperatures}
%==============================================================================
\label{sec:wfuns:atmtemp}

\subsection{Common practicalities}
%
To obtain WFs for absorption species, use \wsmindex{jacobianAddTemperature}.
The calculations can either be done in ``analytical'' or ``perturbation''
manner. Retrieval grids must be specified.

A special consideration for temperature is hydrostatic equilibrium. If effects
originating in hydrostatic equilibrium shall be included in the WFs, or
not, is selected by an argument denoted as \verb|hse|. A full account of
hydrostatic equilibrium is possible for perturbation calculations, while the
analytical approach only treats the local effect (see further below).


\subsection{Perturbation calculations}
%
The size of the perturbation must be selected (in K).
Complete radiative transfer calculations are done after perturbing the
temperature field. Hence, all possible effects are included, such as changed
propagation paths through the impact of temperature on the refractive index.
Please, note that hydrostatic equilibrium comes in during the perturbation. If
\verb|hse| is set to ``on'', also \builtindoc{z\_field} is recalculated as part
of the temperature perturbation (Section~\ref{sec:fm_defs:hse}). If set to
``off'', there is no change of \builtindoc{z\_field}. That is, you must make an
active choice regarding hydrostatic equilibrium, while others effects are
included automatically.


\subsection{Analytical expressions}
%
\subsubsection{Unpolarised absorption}
%
Compared to atmospheric species, the expressions become here more complex as
temperature also affects the propagation path length ($\Delta \aPpathLng{i}$)
and the emission source term ($\bar{\Planck}_i$). Accordingly, all terms of
Eq.~\ref{eq:wfuns:dsdxi2} are relevant, and the expansion of
$\PartD\aOth{i}/\PartD\aSttElm{i}$ generates additional terms
\begin{equation}
  \frac{\PartD\StoI}{\PartD\aSttElm{i}} =
   e^{-\aOth{i}}\left[
      \frac{\PartD\aOth{i}}{\PartD\aAbsCoef{i}}
      \frac{\PartD\aAbsCoef{i}}{\PartD\aSttElm{i}} + 
      \frac{\PartD\aOth{i}}{\PartD\Delta\aPpathLng{i}}
     \frac{\PartD\Delta\aPpathLng{i}}{\PartD\aSttElm{i}}
  \right](\bar{\Planck}_i-\aStoI{i})+
  (1-e^{-\aOth{i}})\frac{\PartD\bar{\Planck}_i}{\PartD\aSttElm{i}}.  
\end{equation}
Terms part of expressions found above are not discussed separately here. The
term, $\PartD\Delta\aPpathLng{i}/\PartD\aSttElm{i}$, originates in the
constrain of hydrostatic equilibrium, and is set to zero when \verb|hse| 
is set to ``off''. Otherwise it is set as derived below.
The term $\PartD\aAbsCoef{i}/\PartD\aSttElm{i}$ is calculated in a pure
numerical manner, by perturbing the temperature. Eq.~\ref{eq:taustep} gives
\begin{equation}
  \frac{\PartD \aOth{i}}{\PartD \Delta \aPpathLng{i}} = 
  \frac{\aAbsCoef{i}+\aAbsCoef{i+1}}{2}. 
\end{equation}
The path length ($\Delta \aPpathLng{i}$), for a given pressure, is linearly
proportional to the temperature, and if $\bar{\Tmp}$ is the average temperature
along the path step:
\begin{equation}
  \frac{\PartD \Delta \aPpathLng{i}}{\PartD \bar{\Tmp}} =   
                                    \frac{\Delta \aPpathLng{i}}{\bar{\Tmp}}.
  \label{eq:wfuns:ddldtbar}
\end{equation}
Following the other variables, we set $\bar{\Tmp}=(\aTmp{i}+\aTmp{i+1})/2$, and
\begin{equation}
  \frac{\PartD \Delta \aPpathLng{i}}{\PartD\aSttElm{i}} = 
                                    \frac{\Delta \aPpathLng{i}}{2\aTmp{i}}.
  \label{eq:wfuns:ddldti}
\end{equation}
In summary (assuming \verb|hse| set to ``on''):
\begin{equation}
  \frac{\PartD\StoI}{\PartD\aSttElm{i}} =
   e^{-\aOth{i}}\frac{\Delta \aPpathLng{i}}{2}\left[
      \frac{\PartD\aAbsCoef{i}}{\PartD\aSttElm{i}} + 
  \frac{\aAbsCoef{i}+\aAbsCoef{i+1}}{2\aTmp{i}}
  \right](\bar{\Planck}_i-\aStoI{i})+
  (1-e^{-\aOth{i}})\frac{\PartD\bar{\Planck}_i}{\PartD\aSttElm{i}}.  
\end{equation}
The derivative of the Planck function ($\PartD\Planck/\PartD\aSttElm{i}$) can
be expressed analytically \citep{eriksson:studi:02}, but as \Planck\ is
provided by an agenda related terms must anyhow be determined in a pure
numerical manner. For example, if the Rayleigh-Jeans approximation is for some
reason applied ($\Planck=\Tmp$) the term is just 1.

The expression for higher Stokes elements, or if emission is totally ignored,
is obtained by setting $\bar{\Planck}_i$
(and $\PartD\bar{\Planck}_i/\PartD\aSttElm{i}$) to zero.


\subsubsection{General case}
%
The term $\PartD\aTrnMat{i}/\PartD\aSttElm{i}$ is calculated in a pure
numerical manner. This is done in two steps. In the first step, the
transmission matrix is recalculated for a slightly perturbed temperature. If
hydrostatic equilibrium is considered, the transmission is also recalculated
with a disturbed $\Delta \aPpathLng{i}$. To avoid making two identical disturbances of
the transmission matrix, these calculations are performed as
\begin{displaymath}
  \frac{\PartD\aTrnMat{i}}{\PartD\aSttElm{i}} = 
  \frac{\PartD\aTrnMat{i}}{\PartD\bar{\Tmp}}\frac{\bar{\Tmp}}{2\aTmp{i}},
\end{displaymath}
a relationship that follows from Eqs.~\ref{eq:wfuns:ddldtbar} and \ref{eq:wfuns:ddldti}.



\subsubsection{Hydrostatic equilibrium and limitations}
%
A changed temperature has non-local effects, originating from refraction and
hydrostatic equilibrium. The expressions above ignore totally refraction
effects. 

As  mentioned,  if \verb|hse|  is  set  to  ``off'',  the term  $\PartD  \Delta
\aPpathLng{i}/\PartD\aSttElm{i}$  is set  to  zero. That  is,  the path  length
through the layer is not affected  by a temperature change. With \verb|hse| set
to ``on'', the complete expressions above are used.

If this treatment of hydrostatic equilibrium is sufficient or not depends on
the observation geometry. It should be insufficient for limb sounding,
where changes even at altitudes below the tangent point can have an influence
as the geometrical altitudes of all higher layers is changed through
hydrostatic equilibrium.  However, this effect vanishes for
ground-based observations at zenith and satellite measurements at nadir, giving
a full account of hydrostatic equilibrium even with the analytical expressions.
In practise it should be possible to apply the expressions outside zenith and
nadir, as long as the observations are of ``up'' or ``down-ward'' type. The
same applies to measurements from inside the atmosphere, (e.g.\ aircraft ones),
if the reference pressure for hydrostatic equilibrium (\builtindoc{p\_hse}) is
matched to the pressure of the observation point.





\section{Sensor pointing}
%==============================================================================
\label{sec:wfuns:sensorpointing}

The term ``sensor pointing'' refers to deviations between nominal and
actual viewing direction of the sensor. So far, only deviations in zenith angle
can be considered. The workspace method to initiate such Jacobians is
\wsmindex{jacobianAddPointingZa}.

The pointing deviation is treated as a time varying variable, then having a
polynomial variation. hence, the basis functions described in
Section~\ref{sec:wfuns:basis2} are applied. The time is taken from
\wsvindex{sensor\_time}. If the pointing error is assumed to be constant with
time, the polynomial order to select is 0, and so on. As a special case, the
polynomial order -1 signifies here that the pointing off-set is so highly
varying that an off-set must be assigned to each spectrum
(sometime called ``pointing jitter'').

The Jacobian can be calculated in two manners:
\begin{description}
\item[recalc] If this option is selected, radiative transfer calculations are
  performed for a shift of \builtindoc{sensor\_los} (perturbation size selected
  by \verb|dza|). Only a ``one-sided'' perturbation is applied.
\item[interp] The Jacobian is derived from existing data, by an interpolation
  of existing data. This achieved by interpolating pencil beam data to a
  shifted zenith angle grid. This will involve some extrapolation of the data,
  and this aspect should be considered when selecting the zenith angles in
  \builtindoc{mblock\_dlos\_grid}. The average of a positive and negative shift
  is determined. The shift to apply (\verb|dza|) should be smaller than the
  minimum spacing of the zenith angles in\builtindoc{mblock\_dlos\_grid} for
  accurate results. As interpolation is a relative fast operation and a
  ``double-sided'' disturbance is used, this option should in general be
  preferred.
\end{description}




\section{Sensor frequencies}
%==============================================================================
\label{sec:wfuns:sensorfreq}

This class of Jacobians treats deviations between nominal and actually recorded
frequencies. Such differences can originate in several ways, but the exact
origin can normally be ignored and the effect can be modelled as the backend
(spectrometer, filter bank \dots) channels are shifted from their nominal
position. The workspace methods to define such Jacobians are
\wsmindex{jacobianAddFreqShift} and \wsmindex{jacobianAddFreqStretch}. These
Jacobians can so far only be determined by applying an interpolation of existing
monochromatic data, then shifted \verb|df| from the nominal
values.

The methods treat either the ``shift'' or ``stretch'' effects. This follows
standard nomenclature. A ``shift'' is an off-set that is of the same size for
all backend channels. That is, if only a shift is assumed, the nominal
distances between the channels are assumed to be valid. The ``stretch'' term
considers the distance between the channels. For a backend with all channels
equally spaced, a stretch signifies that the spacing deviates from the nominal
value (but all channels still equally spaced). More generally, a stretch means
that the deviation from the nominal channel position increases linearly from
the middle point of the backend. In terms of the basis functions
Section~\ref{sec:wfuns:basis2}, shift and stretch correspond to polynomial
order 0 and 1.

Both frequency shift and stretch can be assumed to be time varying, where
exactly the same polynomial approach as for pointing is applied. This
including the case of setting the order to -1.




\section{Polynomial baseline fit}
%==============================================================================
\label{sec:wfuns:polyfit}

A ``baseline'' is microwave jargon for a disturbance of the spectrum that is
not covered by the common sensor characteristics. The most common case is that
the local oscillator signal leaks into the measurement by reflections occurring
inside the sensor, causing a pattern in the spectrum of standing-wave type.
Such effects are difficult to model in a physical manner, and a more general
fitting procedure must be applied. A common option is then to model the
baseline as a polynomial, of a specified order. That is, assuming a measurement
giving a single spectrum, the measured spectrum \MsrVct\ is modelled as
\begin{equation}
  \label{eq:polyfit}
  \MsrVct = \MsrVct' + \sum_{i=0}^n \aSttElm{i}\VctStl{z}_i,
\end{equation}
where $\MsrVct'$ is the ``baseline free'' spectrum, and \aSttElm{i}\ and
$\VctStl{z}_i$ are introduced in Section~\ref{sec:wfuns:basis2}.

The Jacobian for such baseline models are obtained by
\wsmindex{jacobianAddPolyfit}. For single spectra measurements, the only
consideration is the polynomial order to use. For measurements where several
spectra are appended to form the measurement vector, the default option is that
baseline can vary between all spectra. In some cases, it could be assumed that
the baseline is common between data for different polarisations, viewing
directions or measurement blocks, and flags can be set to mimic such
assumptions.

For a given set of retrieved \aSttElm{i}, the simplest way to determine the
estimated baseline is to perform a multiplication between the relevant parts of
the Jacobian and the state vector:
\begin{equation}
  \aWfnMtr{p}\aSttVct{p},
  \label{eq:bline:recreate}
\end{equation}
where $p$ indicates the $n+1$ columns/elements corresponding to \aSttElm{i}.


\section{Sinusoidal baseline fit}
%==============================================================================
\label{sec:wfuns:sinefit}

If the baseline has components of sinusoidal character there is also a second
option, provided the method \wsmindex{jacobianAddSinefit}. The baseline is
in this method modelled as \citep{kuntz:97}
\begin{equation}
  \label{eq:sinefit}
  \MsrVct = \MsrVct' + \sum_{i=1}^n 
       a_i\sin\left(\frac{2\pi(\Frq-\aFrq{1})}{l_i}\right)+
       b_i\cos\left(\frac{2\pi(\Frq-\aFrq{1})}{l_i}\right)
\end{equation}
where $a_i$ and $b_i$ are the coefficients to be retrieved, \Frq\ is frequency,
\aFrq{i} is a reference frequency and $l_i$ is period length. The reference
frequency (\aFrq{i}) is in practice taken as the first frequency of the
spectrometer. 

The period lengths ($l_i$) are user input. For each given period length, the
corresponding sine and cosine terms are included in the Jacobian. As for the
polynomial fit, there exist options to set the baseline to be common between
different polarisations, viewing directions or measurement blocks.
Equation~\ref{eq:bline:recreate} is also applicable for sinusoidal baseline
fits.

An alternative way to express the expression above is
\begin{equation}
  \label{eq:sinefit2}
  \MsrVct = \MsrVct' + \sum_{i=1}^n A_i
             \sin\left(\frac{2\pi(\Frq-\aFrq{1})}{l_i} + \phi_i\right)
\end{equation}
where
\begin{equation}
  A_i = \sqrt{a_i^2+b_i^2}
\end{equation}
and
\begin{equation}
  \tan{\phi_i} = \frac{a_i}{b_i}.
\end{equation}
Equation~\ref{eq:sinefit} is used to define the weighting functions as this
gives a linear retrieval problem, in contrast to if Equation~\ref{eq:sinefit2}
would be used, that would require an iterative process to determine $A_i$
and $\phi_i$.



