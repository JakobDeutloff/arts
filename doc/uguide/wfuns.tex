\chapter{Weighting functions}
 \label{sec:wfuns}

 \starthistory
 11xxyy & First complete version by Patrick Eriksson.\\
 \stophistory

\graphicspath{{Figs/wfuns/}}


Inversions of both OEM and Tikhonov type require that the ``weighting
functions'' can be provided by the forward model \citep[see
e.g.][]{eriksson:analy:00}. A retrieval characterisation following
\citet{rodgers:90,rodgers:00} raises the same demand. A weighting function is
defined as
\begin{equation}
  \frac{\partial \MsrVct}{\partial x_p}
  \label{eq:wfuns:ki}
\end{equation}
where \MsrVct\ is the vector of measurement data and $x_p$ is one forward model
(scalar) variable. See further Sec.~\ref{T-sec:formalism:wfuns} of \theory.
The nomenclature of that section is also used here.

A weighting function forms a column of the complete weighting function matrix,
\aWfnMtr{\SttVct}. The usage of the name weighting function is quite restricted
and the Jacobian is a more commonly encountered name for \aWfnMtr{\SttVct}. In
the documentation of ARTS you find both terms. Also the term ``the Jacobians''
is used, which shall be interpreted as ``the weighting functions''. These names
refer normally to \aWfnMtr{\SttVct}, the partial derivatives with respect to
the variables to be retrieved, forming the state vector \SttVct. However, in
the context of retrieval characterisation, the same matrix for the remaining
model parameters is of equally high interest, denoted as \aWfnMtr{\FrwMdlVct}.
In the same manner, the terms inversion and retrieval are used interchangeably.
``Weighting function'' is below shortened as WF.

The main task of the user is to select which quantities that shall be
retrieved, and to define the associated retrieval grids. These aspects must be
considered for successful inversions, but are out of scope for this document.
Beside for the most simple retrievals, it is further important to understand
how the different WFs are calculated. A practical point is the calculation
speed, primarily determined if perturbations or analytical expressions are used
(Sec.~\ref{sec:wfuns:intro}). The derivation of the WFs involves some
approximations due to theoretical and practical considerations. Such
approximations can be accepted, if of low or moderate size, but will result in
a slower convergence (the inversion will require more iterations). Due to these
later aspects, and to meet the needs of more experienced users, this section is
relatively detailed and contains a (high?) number of equations.



\section{Introduction}
%==============================================================================
\label{sec:wfuns:intro}
%
There are two main approaches for calculating the WFs, by analytical
expressions and by perturbations. We start with the conceptually simplest one,
but also the more inefficient approach.



\subsection{Perturbations}
%==============================================================================
\label{sec:wfuns:pert}
%
The most straightforward method to determine a WF is by perturbing
the model parameter. For example, the WF for the state variable
$p$ can always be calculated as 
\begin{equation}
  \aWfnMtr{\SttVct}^p = \frac{\FrwMdl(\SttVct+\Delta x\VctStl{e}^p,\FrwMdlVct)-
                              \FrwMdl(\SttVct,\FrwMdlVct)}{\Delta x}
 \label{eq:wfuns:perturb}
\end{equation}
where $(\SttVct,\FrwMdlVct)$ is the linearization state, $\VctStl{e}^p$ is a
vector of zeros except for the $p$:th component that is unity, and
$\Delta x$ is a small disturbance (but sufficiently large to avoid
numerical instabilities).

However, it is normally not needed to make a recalculation using the total
forward model as the variables are either part of the atmospheric or the sensor
state, but not both. In addition, in many cases it is possibly to find
short-cuts. For example, the perturbed state can be approximated by an
interpolation of existing data (such as for a perturbed zenith angle). Such
short-cuts are discussed separately for each retrieval quantity.


\subsection{Analytical expressions}
%==============================================================================
\label{sec:wfuns:anal}
%
For most atmospheric variables, such as species abundance, it is possible to
derive an analytical expression for the WFs. This is advantageous because it
results in faster and more accurate calculations. Such expressions are derived
below. Some of the terms involved are calculated as a perturbation. This is
partly a consequence of the flexibility of ARTS. The high-level radiative
transfer methods (e.g.\ \builtindoc{iyEmissionStandardClearsky}) do not know
how the low-level methods are defining all quantities, and a fixed analytical
expression can not be used (see e.g.\ Sec.~FIXME). So, in practise the
calculations are ``semi-analytical''.

To understand the analytical expressions, it is important to remember that ARTS
covers the sensor by a response matrix:
\begin{equation}
  \MsrVct = \SnsMtr\MpiVct.
\end{equation}
This equation covers a single measurement block (cf.\
Eq.~\ref{eq:fm_defs:measseq}).


\subsection{Workspace variables and methods}
%==============================================================================
\label{sec:wfuns:wsm}
%
As a workspace variable, the WF matrix is denoted as \wsvindex{jacobian}.
Auxiliary information is provided by \wsvindex{jacobian\_quantities} and
\wsvindex{jacobian\_indices}. The actual calculations are made as part of
\builtindoc{yCalc}. 

The retrieval quantities are defined separately, before calling
\builtindoc{yCalc}. This process is started by calling \wsmindex{jacobianInit}.
The retrieval quantities are then introduced through dedicated by specific
workspace methods, named as jacobianAdd{\it Something}. For example, for
atmospheric temperature the method is \builtindoc{jacobianAddTemperature}. See
Sec.~\ref{sec:wfuns:aval} for other retrieval quantities. The order in which
these ``add methods'' is called does not matter.

Finalise the definition of retrieval quantities by calling
\wsmindex{jacobianClose}. To disable the calculation of WFs, skip all above,
and just use \wsmindex{jacobianOff}.

The input to the ``add methods'' differs. In some cases you can select between
analytical and perturbation. For all perturbation calculations you must specify
the size of the perturbation. For atmospheric gases you can use different
units. For atmospheric fields, and some other quantities, you must define the
retrieval grid(s) to use.



 
\section{Basis functions}
%==============================================================================
\label{sec:wfuns:basis}

A forward model must use a discrete representation: it describes each quantity
with one or several variables. This is unproblematic for quantities that are of
discrete nature (including scalar variables). However, for atmospheric fields
and other continuous model quantities, the discrete representation inside the
forward model requires consideration. To avoid inconsistencies between model
input and output it is important that the mapping from the discrete variables
to the ``continuous view'' of the quantity is well defined, and applied
consistently through the forward model . This mapping is given by the basis
functions\index{basis function}. Similar arguments and nomenclature are found
in \citet{read:thecl:06}.

The basis functions are discussed explicitly in few places in this user
guide, but it shall be noted that all interpolations imply an underlying set
of basis functions. On the other hand, an understanding of both the derivation
and the obtained WFs require direct consideration of the basis functions. ARTS
operates with two types of basis functions.




\subsection{Basis functions for piece-wise linear quantities}
\label{sec:wfuns:basis1}
%
To treat an one-dimensional quantity to be piece-wise linear, or to say that a
linear interpolation shall be applied, are identical definitions. The basis
functions matching this definition have triangular shape, here denoted as
``tenth functions''. Such functions are exemplified in
Fig.~\ref{fig:wfuns:zbasis}, see also \citet{buehler:artst:05}. 

\begin{figure}[t]
 \begin{center}
  \includegraphics*[width=0.7\hsize]{fig_absbasis_z}
  \caption{Examples on 1d basis functions for a vertical grid with a 1 km
           spacing: \lsolid~30~km, \ldashed~31~km and \ldashdot~32~km.}
  \label{fig:wfuns:zbasis}  
 \end{center}
\end{figure}

In most cases, the quantity is considered to be undefined outside the end
points of the grid. Hence, the basis function for a grid end point is then just
``half a tenth''. The exception to this rule is retrieval grids of piece-wise
linear variables. To avoid that retrieval grids must cover the complete
atmosphere, end point values are assumed to be valid to the end of the
atmosphere. That is, the basis functions for end points of retrieval grids
follow the tenth shape inside the grid range, and have a constant value of 1
outside (to the end of the atmosphere). In terms of interpolation, this matches
to allow extrapolation, the applying a ``nearest'' interpolation for positions
outside the covered range (the end values are valid all the way to
$\pm$infinity).

The basis functions are defined likewise for higher dimensions, but the
tenth functions are then 2D or 3D ``tenths''.




\subsection{Polynomial basis functions}
\label{sec:wfuns:basis2}
%
Some retrieval quantities are expressed using a polynomial basis. Zenith angle
pointing off-set is one such quantity. The off-set is then treated to have a
polynomial variation as a function of time. If the offset is assumed to be
constant in time, a zero order polynomial shall be selected. If there is also a
linear drift with time, use a first order polynomial, etc.

For these basis functions, the explanatory variable (time in the example above)
is normalised to cover the range [-1,1], here denoted as $z$, and the
continuous representation ($f$) of the variable of concern can be written as
\begin{equation}
  f(z) = x_0 + x_1(z-b_1) + x_2(z^2-b_2) + x_3(z^3-b_3) + x_4(z^3-b_4) + \dots  
\end{equation}
where $x_0, x_1, \dots$ are the coefficients to be retrieved (elements of
\SttVct). The interpretation of a retrieval is simplified if the average of $f$
equals $x_0$, and the scalars $b_1, b_2, \dots$ are selected, schematically, as
\begin{equation}
  0 = \int_{-1}^1 \!\left(z^n-b_n\right) \, \DiffD z, \quad n>0.
\end{equation}
According to this expression, $b_n$ is zero for odd $n$. However, $z$ is in
practise a discrete variable ($z_i$, not necessarily symmetric around 0), and
$b_n$ is taken as the average of $z_i^n$: all $b_n$ can be non-zero. The
normalisation of $z$ is not only made for interpretation reasons, it can be
required for pure numerical reasons, such as when $z$ represent frequency (in
Hz).



\section{Available weighting functions}
%==============================================================================
\label{sec:wfuns:aval}

Only clear-sky.


\subsection{Absorption species}
%==============================================================================
\label{sec:wfuns:absspecies}

\subsubsection{Analytical}

The final radiance obtained through Eq.~\ref{eq:fm_defs:rte_step} can be
expressed as
\begin{equation}
  \label{eq:I4J}
  I = I'+ e^{-\Oth'}\left[ \bar{B}_i(1-e^{-\aOth{i}})+I_ie^{-\aOth{i}}\right]
\end{equation}
where $\Oth'$ is the optical thickness between the sensor and point $i+1$
and $I'$ is all emission generated along the same distance.

As example, let $x$ be one element of $\SttVct$, representing the amount of
some gas at some point in the atmosphere. The contribution to the weighting
function for $x$, originating from the radiative transfer step covered by
Eq.~\ref{eq:CSRT:I}, is calculated by applying the chain rule:
\begin{equation}
  \label{eq:chainrule}
  \frac{\PartD \MsrVct}{\PartD x} =  
  \frac{\PartD \MsrVct}{\PartD \MpiVct}
  \frac{\PartD \MpiVct}{\PartD I} \frac{\PartD I}{\PartD \aOth{i}}
  \left[\frac{\PartD \aOth{i}}{\PartD k_i}\frac{\PartD k_i}{\PartD x_i} 
        \frac{\PartD x_i}{\PartD x} +
        \frac{\PartD \aOth{i}}{\PartD k_{i+1}}\frac{\PartD k_{i+1}}{\PartD x_{i+1}}
        \frac{\PartD x_{i+1}}{\PartD x} \right],
\end{equation}
where $x_i$ and $x_{i+1}$ is the amount of the gas at point $i$ and $i+1$,
respectively. Some of these terms can be expressed explicitly:
\begin{eqnarray}
  \frac{\PartD \MsrVct}{\PartD \MpiVct} &=& \SnsMtr, \\
  \frac{\PartD I}{\PartD \aOth{i}} 
      &=& e^{-\Oth'}\left( \bar{B}_i-I_i\right)e^{-\aOth{i}}, \\
  \frac{\PartD \aOth{i}}{\PartD k_i}=\frac{\PartD \aOth{i}}{\PartD k_{i+1}} 
      &=& \Delta s_i/2, \\
  \frac{\PartD k_i}{\PartD x_i}
      &=& \sigma_i, \\
\end{eqnarray}
where $\sigma_i$ is the absorption cross-section for the species (at point $i$
and in a unit matching the unit of $x$). The term $\PartD \MpiVct/\PartD I$
gives the position in \MpiVct\ where the $I$ of concern is stored.

The term $\PartD x_i/\PartD x$ appears due to the fact that $x_i$ and $x$ are
placed at different positions, and the representation of the atmospheric fields
must be considered here. The fields are treated to vary linearly between grid
points, giving triangular shaped `basis functions' for 1D, pyramid shaped ones
for 2D etc. In practise, the term is calculated as the value of the basis
function for $x$ at the location of $x_i$, see \cite{buehler:artst:05}. This is
a slight approximation with respect to the goal of fully incorporating the
piece-wise linear representation in the weighting functions
\cite{buehler:artst:05}. A low value of $\Delta s_i$ decreases the degree of
approximation.

It can be noted that $\PartD x_i/\PartD x$ is normally non-zero for more than
one element of \SttVct. The exception is if the positions of $x_i$ and $x$ are
identical. Reversely, the weighting function for element $x$ can have
contributions from several propagation path points ($x_i$), as well as from
several radiance spectra.

In practise, all Stokes components are treated in parallel, to simply
incorporate surface scattering (and in the future possibly also cloud
scattering). Transmission $e^{-\Oth'}$ must then be expressed as a matrix,
$\MtrStl{T}'$. For scattered down-welling radiation ($\StoVec^d_i$) the
effective transmission matrix is
\begin{equation}
  \MtrStl{T}' = \MtrStl{T}_2\MtrStl{R}_i\MtrStl{T}_1,
\end{equation}
where $\MtrStl{T}_2$ is the transmission between the surface and the sensor,
$\MtrStl{R}_i$ is defined in Eq.~\ref{eq:surface} and $\MtrStl{T}_1$ is the
transmission between the point $i+1$ and the surface.

The case of gas species retrieval is used above as example. This chain rule
approach (Eq. \ref{eq:chainrule}) can be used for any variable affecting
atmospheric absorption and emission, or surface reflectance and emission.
Expressions for scalar radiative transfer are found in \cite{buehler:artst:05}.

Some terms are best calculated in a numerical manner. For atmospheric
temperature the terms $\PartD k_i/\PartD T$ and $\PartD B/\PartD T$ are found.
The first term can in principle be expressed analytically, but this would
require hard links to the absorption module. Such links are avoided in ARTS
and the term is instead calculated with a second call of
\texttt{abs\_scalar\_gas\_agenda} (Sec.~\ref{sec:agendas}), with a slightly
perturbed temperature. The same applies to the second term, even though an
analytical derivation here is straightforward \cite{eriksson:studi:02}. This, like the emission source term, is handled by an agenda, and the user can choose to set $B=T$ (can be a valid approximation for low frequency
observations), instead of $B$ being the Planck function.

A constraint for the analytical expressions above is that the effect of the
variable must only be local. Atmospheric temperature is again the most important
example, that has non-local impacts if hydrostatic equilibrium applies and if
refraction affects the propagation path significantly.


\subsubsection{Perturbation}




\subsection{Atmospheric temperatures}
%==============================================================================
\label{sec:wfuns:atmtemp}

\subsubsection{Analytical}

A changed temperature has several non-local effects, originating from
refraction and hydrostatic equilibrium. These non-local effects make the
calculations more difficult for, at least, limb sounding. For example, even
changes at altitudes below the tangent point will have an influence as the
geometrical altitudes of all higher layers is changed through hydrostatic
equilibrium. However, the non-local effects can be neglected for observations
around zenith and nadir and the temperature weighting function can
then be written as:
\begin{equation}
  \frac{\PartD \MsrVct}{\PartD T} =  
  \frac{\PartD \MsrVct}{\PartD \MpiVct} \frac{\PartD \MpiVct}{\PartD I} 
  \left[
  \frac{\PartD I}{\PartD \aOth{i}}\frac{\PartD \aOth{i}}{\PartD T} + 
  \frac{\PartD I}{\PartD \bar{B}_i}\frac{\PartD \bar{B}_i}{\PartD T}
  \right].
\end{equation}
The partial derivative of $\aOth{i}$ must now also consider the term $\Delta
s_i$ due to the constrain of hydrostatic equilibrium:
\begin{equation}
  \frac{\PartD \aOth{i}}{\PartD T} =
  \frac{\PartD \aOth{i}}{\PartD k_i}\frac{\PartD k_i}{\PartD T_i} 
  \frac{\PartD T_i}{\PartD T} +
  \frac{\PartD \aOth{i}}{\PartD k_{i+1}}\frac{\PartD k_{i+1}}{\PartD T_{i+1}}
  \frac{\PartD T_{i+1}}{\PartD T} +
  \frac{\PartD \aOth{i}}{\PartD \Delta s_i}
  \left[
    \frac{\PartD \Delta s_i}{\PartD T_i} \frac{\PartD T_i}{\PartD T} +
    \frac{\PartD \Delta s_i}{\PartD T_{i+1}} \frac{\PartD T_{i+1}}{\PartD T} +
  \right].
\end{equation}
The temperature derivative of the absorption ($\PartD k_i/\PartD T_i$) must be
calculated numercially. This derivative depends on if gas species
concentrations are given as volume mixing ratio (VMR) or number density. As
ARTS uses VMR, the term above will be calculated with the constrain of fixed
VMR. A consequence of this is that it is not possible to mix retrieval of
temperature and number densities.

Eq.~\ref{eq:taustep} gives
\begin{equation}
  \frac{\PartD \aOth{i}}{\PartD \Delta s_i} = \frac{k_i+k_{i+1}}{2}. 
\end{equation}
The path length ($\Delta s_i$), for a given pressure, is linearly proportional
to the temperature, and if $\bar{T}$ is the average temperature along the path
step:
\begin{equation}
  \frac{\PartD \Delta s_i}{\PartD \bar{T}} =   \frac{\Delta s_i}{\bar{T}} 
\end{equation}
Following the other variables, we set $\bar{T}=(T_i+T_{i+1})/2$, and
\begin{equation}
  \frac{\PartD \Delta s_i}{\PartD T_i} = \frac{\Delta s_i}{2T_i}.
\end{equation}
The terms involving $\bar{B}_i$ are
\begin{eqnarray}
   \frac{\PartD I}{\PartD \bar{B}_i} &=&
   e^{-\Oth'}\left(1-e^{-\aOth{i}}\right), \\
   \frac{\PartD \bar{B}_i}{\PartD T_i} &=& \frac{1}{2}
   \frac{\PartD B(T)}{\PartD T}. 
\end{eqnarray}
The term $\PartD B(T)/\PartD T$ can be expressed analytically, but as $B$ is
provided by an agenda this term must anyhow be determined in a pure numerical
manner; it is likely the derivative of the Planck function, but if the
Rayleigh-Jeans approximation is for some reason selected ($B=T$) the term 
is just 1.

Determine expression for BL and higher Stokes, and put into ARTS!

\subsubsection{Perturbation}
