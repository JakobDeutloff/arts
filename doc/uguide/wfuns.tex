\chapter{Weighting functions}
 \label{sec:wfuns}

 \starthistory
 11xxyy & First complete version by Patrick Eriksson.\\
 \stophistory

\graphicspath{{Figs/wfuns/}}


Inversions of both OEM and Tikhonov type require that the ``weighting
functions'' can be provided by the forward model \citep[see
e.g.][]{eriksson:analy:00}. A retrieval characterisation following
\citet{rodgers:90} raises the same demand. A weighting function is defined as 
\begin{equation}
  \frac{\partial \MsrVct}{\partial x_p}
  \label{eq:wfuns:ki}
\end{equation}
where \MsrVct\ is the vector of measurement data and $x_p$ is one forward model
(scalar) variable. See further Sec.~\ref{T-sec:formalism:wfuns} of \theory,
which nomenclature is also used here.

A weighting function forms a column of the complete weighting function matrix,
\aWfnMtr{\SttVct}. A probably more common name for \aWfnMtr{\SttVct}\ is the
Jacobian, and in the documentation of ARTS you find both terms. Also the term
``the Jacobians'' is used, which shall be interpreted as ``the weighting
functions''. These names refer normally to the partial derivates with respect
to the variables to be retrieved (i.e.\ \aWfnMtr{\SttVct}), the state vector
\SttVct. However, in the context of retrieval characterisation, the same matrix
for the remaining model paramaters is of equally high interest, denoted as
\aWfnMtr{\FrwMdlVct}. In the same manner, the terms inversion and retrieval are
used interchangeably. ``Weighting function'' is below shortened as WF.

The main task of the user is to select which quantities that shall be
retrieved, and to define the associated retrieval grids. These aspects must be
considered for successful inversions, but are out of scope for this document.
Beside for the most simple retrievals, it is further important to understand
how the different WFs are calculated. A practical point is the
calculation speed, primarily determined if perturbations or analytical
expressions are used (Sec.~\ref{sec:wfuns:overw}). The derivation of the
WFs involves also some approximations (due to theoretical or
practical considerations). Such approximations can be accepted, but
will result in a slower convergence (the inversion will require more
iterations). Due to these later aspects, and to meet the needs of more
experienced users, this section is relatively detailed and contains a (high?)
number of equations.



\section{Overwiev}
%==============================================================================
\label{sec:wfuns:overw}
%
There are two main approaches for calculation the WFs, by analytical
expressions and by perturbations. We start with the conceptually simplest one,
but also the more inefficient approach.



\subsection{Perturbations}
%==============================================================================
\label{sec:wfuns:pert}
%
The most straightforward method to determine the WFs is by perturbing
one parameter at a time. For example, the WF for the state variable
$p$ can always be calculated as 
\begin{equation}
  \aWfnMtr{\SttVct}^p = \frac{\FrwMdl(\SttVct+\Delta x\VctStl{e}^p,\FrwMdlVct)-
                              \FrwMdl(\SttVct,\FrwMdlVct)}{\Delta x}
 \label{eq:wfuns:perturb}
\end{equation}
where $(\SttVct,\FrwMdlVct)$ is the linearization state, $\VctStl{e}^p$ is a
vector of zeros except for the component $p$ that is unity, and
$\Delta\SttVct^p$ is a small disturbance (but sufficiently large to avoid
numerical instabilities).

However, it is normally not needed to make a recalculation using the total
forward model as the variables are in general either part of the atmospheric or
the sensor state, but not both. In addition, in many cases there it is possibly
to find short-cuts. For example, the perturbed state can be approximated by an
interpolation of existing data (such as for a perturbed zenith angle). Such
short-cuts are discussed seperately for each retrieval quantity.


\subsection{Analytical expressions}
%==============================================================================
\label{sec:wfuns:anal}
%
For most atmospheric variables, such as species abundance, it is possible to
derive an analytical expression for the WFs. This is advantageous because it
results in faster and more accurate calculations. Such expressions are derived
below. Some of the terms involved are calculated as a perturbation. This is
partly a consequence of the flexibility of ARTS. The high-level radiative
transfer methods (e.g.\ \builtindoc{iyEmissionStandardClearsky}) do not know
how the more level methods are defining the quantities, and a fixed analytical
expression can not be used (see e.g.\ Sec.~FIXME). So, in practice the
calculations are ``semi-analytical''.

To understand the analytical expressions, it is important to remember that ARTS
covers the sensor by a response matrix:
\begin{equation}
  \MsrVct = \SnsMtr\MpiVct.
\end{equation}
This equation covers a single measurement block (cf.\
Eq.~\ref{eq:fm_defs:measseq}).


\subsection{Workspace variables and methods}
%==============================================================================
\label{sec:wfuns:wsm}
%
As a workspace variable, the WF matrix is denoted as \wsvindex{jacobian}.
Auxilary information is provided by \wsvindex{jacobian\_quantities} and
\wsvindex{jacobian\_indices}. The actual calculations are made as part of
\builtindoc{yCalc}. 

The retrieval quantities are defined seperately, before calling
\builtindoc{yCalc}. This process is started by calling \wsmindex{jacobianInit}.
The retrieval quantities are then introduced through dedicated by specific
workspace methods, named as ``jacobianAddSomething''. For atmospheric
temperature the method is \builtindoc{jacobianAddTemperature}. See
Sec.~\ref{sec:wfuns:aval} for other retrieval quantities. The order in
which these ``add methods'' are called does not matter. 

Finalise the definition of retrieval quantities by calling
\wsmindex{jacobianClose}. To disable the calculation of WFs, skip all above,
and just use \wsmindex{jacobianOff}.

The input to the ``add methods'' differs. In some cases you can select between
analytical and perturbation. For all perturbation calculations you must specify
the size of the perturbation. For atmospheric gases you can use different
units. For atmospheroic fields, and some other quantities, you must define the
retrieval grid(s) to use.



 
\section{Basis functions}
%==============================================================================
\label{sec:wfuns:basis}

x



\section{Avaliable weighting functions}
%==============================================================================
\label{sec:wfuns:aval}

\subsection{Absorption species}
%==============================================================================
\label{sec:wfuns:absspecies}

\subsubsection{Analytical}


\subsubsection{Perturbation}
