%
% To start the document, use
%  \levela{...}
% For lover level, sections use
%  \levelb{...}
%  \levelc{...}
%
\levela{Introduction}
 \label{sec:intro}

%
% Document history, format:
%  \starthistory
%    date1 & text .... \\
%    date2 & text .... \\
%    ....
%  \stophistory
%
\starthistory
  02xxxx & xxx.\\
\stophistory


Some nice welcome text ...

\levelb{Temporary internal notes}

I have started to prepare the user guide for the new arts version.
First of all, I have commented out most of the chapters (done in main.tex). 
The idea is to enforce that the user guide only contains stuff that is 
consistent with the new version. So before a chapter is re-introduced, all
its content shall be checked and the text shall be changed where needed.
Please, remember to also check the figure folder and to remove figures
not used any longer. 

In addition, I have re-arranged the introductory part. Before we had
only one chapter, and it described mainly the programme ARTS. As the
scope of ARTS was the same as most forward models, this was OK.
However, now when ARTS will be extended, I think we need to describe
and define the concepts and terms we use in the forward model. That
is, things that are not directly connected with the practical use of
the programme, but needed to understand this user guide and to set-up
a control file. So, I suggest these introductory chapters:

{\bf 1. Introduction} History, scope etc. of ARTS.

{\bf 2. ARTS: concept and the programme} A description of how to use ARTS
as a computer programme. Basically the old introduction chapter.

{\bf 3. Forward model concepts and definitions} Stuff needed to
understand ARTS as a forward model. For example, to describe what we
mean with 1D, 2D and 3D.
 

Patrick 2002-03-09


\levelb{Documentation guide}
%====================
\label{sec:intro:guide}

Describe where different type of information can be found. For
example, refer to full control file examples in \verb|doc/examples|.





\levelb{Background}
%====================
\label{sec:intro:background}

The number of satellite sensors in the millimeter and sub-millimeter
spectral range is rapidly growing. They use various frequency
bands and observation geometries. Two important groups of
sensors are for example the nadir viewing millimeter wave
sensors like AMSU\footnote{The \textbf{A}dvanced
  \textbf{M}icrowave \textbf{S}ounding \textbf{U}nit is a
  sensor on board the polar orbiting satellites of the
  US-American National Aeronautics and Space Administration.}
and the limb viewing sub-millimeter wave sensors like the
planned SMILES\footnote{The \textbf{S}uperconducting
  Sub-\textbf{Mi}llimeter Wave \textbf{L}imb \textbf{E}mission
  \textbf{S}ounder is a Japanese Sensor which will be flown
  for the first time on the International Space Station.}.

For the data analysis all such sensors require accurate and
fast forward models, which can simulate measurements for a
given atmospheric (and maybe ground) state. Depending on the
objective of the sensor, the measurement will depend for
example on the distribution of atmospheric temperature, water
vapor, ozone, and many other trace gases.

So far, a lot of effort has been wasted in developing dedicated
forward models for different sensors, although all these models have
many features in common. Moreover, existing models were not easily
modifiable and extendable. Hence, it was decided to develop a new
model which emphasizes modularity, extensibility, and generality.

[* Describe how ARTS was initiated and started. Release of version 1. *]


\levelb{What is ARTS}
%====================
\label{sec:intro:whatis}

[* ??? *]


\levelb{The scope of ARTS}
%====================
\label{sec:intro:scope}

[* Update old text and add new stuff. *]

%The present version of ARTS is limited to cases where scattering can
%be neglected and local thermodynamic equilibrium applies. ARTS has
%been developed having passive emission measurements in mind, put pure
%transmission (that is, the emission is neglected) observations are
%also handled. The forward model can be used to simulate measurements
%for all (normal?)  observation geometries: ground-based, nadir
%looking, limb sounding and balloon/aircraft measurements. It can be
%noted that ARTS handles measurements from a point inside the
%atmosphere, such as an aircraft or a balloon, in a downward direction.
%ARTS covers so far only monochromatic pencil beam calculations, that
%is, no sensor characteristics can be included. This part is presently
%covered by the AMI (ARTS Matlab interface) set of Matlab functions 
%(see below). Sensor characteristics will be included in ARTS.

%Beside providing set of spectra, ARTS calculates weighting functions
%for a number of variables. Analytical expressions for the weighting
%functions are used for species, continuum absorption and ground
%emission, and for temperature if hydrostatic equilibrium is \emph{not}
%assumed. Weighting functions are also provided for pointing off-sets,
%calibration and temperature (with hydrostatic equilibrium).



\levelb{Additional tools}
%====================
\label{sec:intro:tools}

[* Update old text and add new stuff. *]

%For Matlab users there are two accompanying packages called AMI and
%Qpack\footnote{AMI is distrubuted by ARTS, while Qpack is a separate
%  package} which extends the usage of ARTS considerably. First of all,
%AMI has functions to include sensor characteristics in the
%calculations. AMI has further functions to read and write ARTS data
%file, and various functions that are of general usage. Qpack is an
%Matlab environment to perform OEM inversions and producing set of
%spectra to test the inversions, where ARTS is used as calculating
%engine.



%%% Local Variables: 
%%% mode: latex
%%% TeX-master: "uguide"
%%% End: 
