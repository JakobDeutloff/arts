%
% To start the document, use
%  \levela{...}
% For lover level, sections use
%  \levelb{...}
%  \levelc{...}
%
\levela{Gas Absorption}
 \label{sec:absorption}


%
% Document history, format:
%  \starthistory
%    date1 & text .... \\
%    date2 & text .... \\
%    ....
%  \stophistory
%
\starthistory
  2001-07-05 & Template created by Stefan Buehler.\\
  2001-07-19 & Continuum absorption part written, Thomas Kuhn.\\
\stophistory
%
% Symbol table, format:
%  \startsymbols
%    ... & \verb|...| & text ... \\
%    ... & \verb|...| & text ... \\
%    ....
%  \stopsymbols
%
%
%\startsymbols
%  -- & -- & -- \\
% \label{symtable:wfuns}     
%\stopsymbols

In general there three types of absorption/emission spectra:
- sharp lines of finite widt
- aggregation (series) of lines called bands
- continua extending over a broad range of wavelengths, with
  no single lines in it.
This section is therefore firstly devised to give a short theoretical
overview of the spectra quantities which are used in ARTS. The presentation
of the calculating methods within ARTS of these quantities is the
second goal.
There are two major subsections which deal separately with the respective types
of spectra regarded by ARTS: 
- lines 
- continua.
The second one extends beyond the scope of mere treatment of the
continua and thus gives an integrated view of continua and spectral lines.

%
%
% ============================================================================
% The following sections are written by Thomas Kuhn, iup, 18-07-2001
% ============================================================================
%
% general definitions for the following sections
%
\def\imn{{N''}}
\def\ime{\epsilon''}
\def\ree{\epsilon'}
%
\def\bek{\sf b_{\sf 1,k}}
\def\bzk{\sf b_{\sf 2,k}}
\def\bdk{\sf b_{\sf 3,k}}
\def\bvk{\sf b_{\sf 4,k}}
\def\bfk{\sf b_{\sf 5,k}}
\def\bsk{\sf b_{\sf 6,k}}
%
\def\bekp{\sf \widehat{b}_{\sf 1,k}}
\def\bzkp{\sf \widehat{b}_{\sf 2,k}}
\def\bdkp{\sf \widehat{b}_{\sf 3,k}}
\def\bvkp{\sf \widehat{b}_{\sf 4,k}}
\def\bfkp{\sf \widehat{b}_{\sf 5,k}}
\def\bskp{\sf \widehat{b}_{\sf 6,k}}
%
\def\beks{\sf b^*_{\sf 1}}
\def\bzks{\sf b^*_{\sf 2}}
\def\bdks{\sf b^*_{\sf 3}}
\def\bvks{\sf b^*_{\sf 4}}
\def\bfks{\sf b^*_{\sf 5}}
\def\bsks{\sf b^*_{\sf 6}}
%
\def\bekps{\sf \widehat{b^*}_{\sf 1,k}}
\def\bzkps{\sf \widehat{b^*}_{\sf 2,k}}
\def\bdkps{\sf \widehat{b^*}_{\sf 3,k}}
\def\bvkps{\sf \widehat{b^*}_{\sf 4,k}}
\def\bfkps{\sf \widehat{b^*}_{\sf 5,k}}
\def\bskps{\sf \widehat{b^*}_{\sf 6,k}}
%
\def\air{\mbox{air}}
\def\hzo{\mbox{H}_2\mbox{O}}
\def\nzo{\mbox{N}_2\mbox{O}}
\def\coz{\mbox{C}\mbox{O}_2}
\def\nz{\mbox{N}_2}
\def\oz{\mbox{O}_2}
\def\coz{\mbox{CO}_2}
%
\def\ptot{P_{\mbox{\tiny \sf tot}}}
\def\phzo{P_{\mbox{\tiny \sf H}_{2}\mbox{\tiny \sf O}}}
\def\pda{P_{\mbox{\tiny \sf d}}}
\def\pdair{P_{\mbox{\tiny \sf air}}}
\def\pan{P_{\mbox{\tiny \sf air},\mbox{\tiny \sf N}_{2}}}
\def\pcoz{P_{\mbox{\tiny \sf CO}_{2}}}
%
\def\alphampmotot{\alpha^{\mbox{\rm \tiny MPM87}}_{\mbox{\tiny tot}}} 
\def\alphampmmtot{\alpha^{\mbox{\rm \tiny MPM89}}_{\mbox{\tiny tot}}} 
\def\alphampmntot{\alpha^{\mbox{\rm \tiny MPM93}}_{\mbox{\tiny tot}}} 
\def\alphapwrtot{\alpha^{\mbox{\tiny \rm R98}}_{\mbox{\tiny tot}}} 
\def\alphacptot{\alpha^{\mbox{\tiny \rm CP98}}_{\mbox{\tiny tot}}} 
%
\def\alphampmol{\alpha^{\mbox{\rm \tiny MPM87}}_{\mbox{\tiny $\ell$}}} 
\def\alphampmml{\alpha^{\mbox{\rm \tiny MPM89}}_{\mbox{\tiny $\ell$}}} 
\def\alphampmnl{\alpha^{\mbox{\rm \tiny MPM93}}_{\mbox{\tiny $\ell$}}} 
\def\alphampml{\alpha^{\mbox{\rm \tiny MPM}}_{\mbox{\tiny $\ell$}}} 
\def\alphapwrl{\alpha^{\mbox{\tiny \rm R98}}_{\mbox{\tiny $\ell$}}} 
\def\alphacpl{\alpha^{\mbox{\tiny \rm CP98}}_{\mbox{\tiny $\ell$}}} 
%
\def\alphampmoc{\alpha^{\mbox{\rm \tiny MPM87}}_{\mbox{\tiny c}}} 
\def\alphampmmc{\alpha^{\mbox{\rm \tiny MPM89}}_{\mbox{\tiny c}}} 
\def\alphampmnc{\alpha^{\mbox{\rm \tiny MPM93}}_{\mbox{\tiny c}}} 
\def\alphapwrc{\alpha^{\mbox{\tiny \rm R98}}_{\mbox{\tiny c}}} 
\def\alphacpc{\alpha^{\mbox{\tiny \rm CP98}}_{\mbox{\tiny c}}} 
%
\def\alphatot{\alpha_{\mbox{\tiny \sf tot}}} 
\def\alphal{\alpha_{\mbox{\tiny $\ell$}}} 
\def\alphac{\alpha_{\mbox{\tiny \sf c}}} 
\def\alphacs{\alpha_{\mbox{\tiny \sf c,s}}} 
\def\alphacf{\alpha_{\mbox{\tiny \sf c,f}}} 
%
\def\gamk{\gamma_{\sf k}}
\def\gamc{\gamma_{\sf c}}
%
\def\ws{w_{\sf s,k}}
\def\xs{x_{\sf s,k}}
\def\wf{w_{\sf f,k}}
\def\xf{x_{\sf f,k}}
%
\def\wn{\bar{\nu}}
\def\nucc{\nu_{\sf c}}
\def\nucut{\nu_{\sf cutoff}}
\def\nuo{\nu_{\mbox{\sf \tiny 0}}}
\def\nuk{\nu_{\sf k}}
%
\def\shape{F(\nu,\nuk)}
\def\shapec{F_{c}(\nu,\nuk)}
\def\shapefp{f_{c}(\nu,+\nuk)}
\def\shapefm{f_{c}(\nu,-\nuk)}
\def\shapefpm{f_{c}(\nu,\pm\nuk)}
\def\inten{S_{\sf k}(T)}
\def\inteno{S_{\sf k}(300\,K)}
%
\def\fcs{\mbox{C}^{\mbox{\tiny \rm fit}}_{\mbox{\tiny \rm s}}} 
\def\fcf{\mbox{C}^{\mbox{\tiny \rm fit}}_{\mbox{\tiny \rm f}}}
\def\fcsf{\mbox{C}^{\mbox{\tiny \rm fit}}_{\mbox{\tiny \rm s,f}}} 
\def\tfcs{\mbox{\tiny C}^{\mbox{\tiny \rm fit}}_{\mbox{\tiny \rm s}}} 
\def\tfcf{\mbox{\tiny C}^{\mbox{\tiny \rm fit}}_{\mbox{\tiny \rm f}}} 
\def\tfcsf{\mbox{C}^{\mbox{\tiny \rm fit}}_{\mbox{\tiny \rm s,f}}} 
\def\tflncsf{\mbox{\tiny ln\,C}^{\mbox{\tiny \rm fit}}_{\mbox{\tiny \rm s,f}}} 
%
\def\cx{C_{\mbox{\tiny \sf x}}}
\def\cs{C_{\mbox{\tiny \sf H}_{2}\mbox{\tiny \sf O}}} 
\def\cf{C_{\mbox{\tiny \sf N}_{2}}} 
\def\cxo{C^{\mbox{\tiny \sf o}}_{\mbox{\tiny \sf X}}} 
\def\cso{C^{\mbox{\tiny \sf o}}_{\mbox{\tiny \sf H}_{2}\mbox{\tiny \sf O}}} 
\def\cfo{C^{\mbox{\tiny \sf o}}_{\mbox{\tiny \sf N}_{2}}} 
\def\cao{C^{\mbox{\tiny \sf o}}_{\mbox{\tiny \sf air}}}
\def\cdo{C^{\mbox{\tiny \sf o}}_{\mbox{\tiny \sf d}}}
\def\xx{{\sf n}_{\mbox{\tiny \sf x}}} 
\def\xs{{\sf n}_{\mbox{\tiny \sf s}}} 
\def\xf{{\sf n}_{\mbox{\tiny \sf f}}} 
\def\xd{{\sf n}_{\mbox{\tiny \sf d}}}
%

\levelb{Line Absorption}
%-----------------------
\label{sec:line_absorption}

 
Staying close to the notation used by Goody and Young in ``Atmosperic
Radiation'', 1989, we'll introduce here the main concepts concerning
line absortion. The approach, however, does not aim  at the
derivation, but confines itself within the consise and structured way
of presenting the following expressions.

Any line is presented by the corresponding profile of the absorption/emission
coefficient as dependency on the frequency, in our case that of the
absorption - given by the quantity $\alpha(\nu)$. The latter dependency
is related to other quantities, which express the three
characteristics of a single line uniqely describing it - position,
strength and shape. 
The strength, or better the line intensity, is given by the quantity
$S(T)$. The shape and the position are expressed through the
line-shape function $F(\nu)$. Thus we have the relation 
\begin{equation}
  \alpha(\nu)=nS(T)F(\nu)
\label{}
\end{equation} 
where $n$ is the numbre density pf the absorber. The line-shape
function is normalized as follows

\begin{equation}
  \int F(\nu)d\nu=1
\label{}
\end{equation}
The values of $S(T)$ at reference temperature $T_0$ are contaied in spectroscopic databases. The conversion to different
temperatures is done by
\begin{equation}
S(T)=S(T_0)~\frac{Q(T_0)}{Q(T)}~\frac{e^{-E_f/kT} - e^{-E_i/kT}}{e^{-E_f/kT_0} - e^{-E_i/kT_0}}
\label{}
\end{equation}
given the energies $E_f$ and $E_i$ of the two levels between which the
transition occurs as well as the partition function~~$Q(T)$. The
databases contain the the lower energy $E_l$ tabulated along with the
$S$ and the transition frequency $\nu$, so that the upper state can be
computed by $E_u$=$E_l$+h$\nu$. Partition functions for all molecular species are also
available along with spectroscopic databases, either in the form of tabulated values for a set
temperatures, or in the form of FORTRAN routines. One can obtain the total absorption
coefficient by adding the absorption of all spectral lines of all
molecular species.

The problem of determining the explicit kind of the line-shape
functions is treated in a following subsection and a similar one is
dedicated to the partition functions.


\levelc{Line Shape Functions}
Up to date there is no way to calculate the line-shape fuctions
analytically just through the theoretical expressions of quantum dynamics,
statistical physics and theoretical mechnanics. Certain approximations
should be done to account for different physical phenomena related to
the absorption/emission process or to suit better the calculation of
different parts of the spectral line.

There are three phenomena which contribute to the line-shape. These
are, in increasing orede of importance, the finite lifetime of an
excited state in an isolated molecule, the thermal movement of the gas
molecules and their colliisons with each other. They result in
correpsonding effects to the line-shape: natural broadenig, Doppler
and pressure broadening. Of these, the first one is completely
negligible compared to the other two. That's why we don't pay a special
attention to it but we'll present some expressions in order to convey
better understanding of the other two.

The spectral line shape can derived in this case from basic physical
considerations and the well-known Fouirier transform theorem, stating
that  multiplication in the time domain is equal to convoluting the
frequency one.

If we regard the spontaneous decay of the excited state, then the
amlitude of the oscillations decreases by $\gamma/2$ and the
respective decrease in the energy is
\begin{equation}
  E(t)=E(0)e^{-\gamma t}
\label{}
\end{equation}
By the afore mentioned theorem, multiplicating in the time domain by
$e^{-\gamma t}$ is equavelent to convolving in the frequency domain
with a function $1/[\nu^2 - (\gamma/4\pi)^2]$. Thus the normalised
line-shape function of the spectral line at frequency $\nu_0$ is 
\begin{equation}
  F(\nu)=\frac{1}{\pi}\frac{\gamma/4\pi}{(\nu - \nu_0)^2 + (\gamma/4\pi)^2}
\label{}
\end{equation}

This gives a bell-shaped profile and the function itself is called
Lorentzian. The dependance on the position of the line is apparent
through $\nu_0$, that's why some authors prefer to notate the function
by $F(\nu,\nu_0)$.
The result is important because of the two reasons. First, without the
natural broadening the line wouold be~$\delta (\nu - \nu_o)$. So the
spontaneous decay of the excited state is responsible for the finite
width and the certain shape of the line-shape function. Second, the
Lorentzian type of function comes significantly into into play when
explaining the some of the other broadening effects or the complte
picture of the broadened line.

The second effect, Doppler broadenig, is important for the upper
stratosphere and mesosphere for microwave freqeuncy. The line-shape
follows the velocity distribution of the particles. Under the conditions
of thermodynamic equilibrium, we have  a probablity distribution for
the relative velocity $u$ between the gas molecule and the oserver 
of Maxwell type 
\begin{equation}
  p(u)=\sqrt{\frac{m}{2\pi kT}}~~exp~\left[-\frac{mu^2}{2kT}\right]
\label{}
\end{equation}
where $m$ is the mass of the molecule. Using then the formula for the
Doppler shift for the non-relativistic region~~  $\nu$- $\nu_0$ =
$\nu_0$$u$ / $c$ , one can easily derive the line-shape function 
\begin{equation}
 F_D(\nu)=\frac{1}{\gamma_D\sqrt{\pi}}~~exp~\left[-\left(\frac{\nu - \nu_0}{\gamma_D}\right)^2\right]
\label{}
\end{equation}
where the quantity $\gamma_D$ is called Doppler line width and equals
\begin{equation}
 \gamma_D=\frac{\nu}{c}\sqrt{\frac{2kT}{m}}
\label{}
\end{equation}
In contrast to the afore mentioned line-shape function for the natural
broadening the Doppler broadening is expressed by a Gaussian
line-shape fuction $F(\nu)$. The Doppler line width $\gamma_D$ is so
defined that it is qual to the falf width at half of the maximum
(HWWM) of the line-shape function. The same way of notating is used
for all other width parameters $\gamma_xy$ below.

It can be said without any exaggeration that the pressure broadening
effect is the most comlpicated one among the others and still
represents a complex theoretical task to be tackled, and is in the
same time of major experimental importance. So far, there is no way to
calculate analytically from the basics the profile of a pressure, or
collisional, line through a single approach near the line center as well
as ii the far wing region. The various approximations, which are
therefore used, are immanenetly limited to the certain line regions
they deal with.
The most popular among these approximations is $impact~approximation$
which main statement is that the duration of the colllisions of the
gas particles is very small compared to the average time between the
collisions. Lorentz was the first to achieve a result exploiting this
approach, the Lorentz line-shape function:
\begin{equation}
 F_L(\nu)=\frac{\gamma_L}{\pi}\frac{1}{(\nu-\nu_0)^2+\gamma_L^2}
\label{}
\end{equation}
where $\gamma_L$ is the Lorentz line width. As one can see, the result
(3.9) is pretty similar to (3.5) but the specific line parameters
$\gamma$ and $\gamma_L$ make them differ significantly in the
corresponding frequency regions of interest. For atmospheric pressures
$\gamma_L$ is much greater and that's why of experimental significance
in contrast to $\gamma$.\\
Elaborating the model of Lorentz, van Vleck and Weisskopf made a correction to it, particularly
for the microwave region:
\begin{equation}
 F_{VVW} (\nu)=\left(\frac{\nu}{\nu_0}\right)^2\frac{\gamma_L}{\pi}\left[\frac{1}{(\nu-\nu_0)^2+\gamma_L^2}+\frac{1}{(\nu+\nu_0)^2+\gamma_L^2}\right]
\label{}
\end{equation}
which can be reduced to Lorentzian for $(\nu-\nu_0) << \nu_0$ . Except
for the additional factor $(\nu/\nu_0)^2$ ,  $F_{VVW}$ can be regarded
as the sum of two $F_L$, or respectively lines, one with its center
freqquency at  $\nu_0$, the other at $-\nu_o$. Other authors,
e.g. Rosenkrantz,  make use of an linear factor $(\nu/\nu_0)$ in their
approach too.

The combined picture of a simultaneously Doppler and pressure
broadened line is the next step of the approximations development. The
line-shape function in this case given by the Voigt line-shape
function 
\begin{equation}
 F_{Voigt}(\nu,\nu_0)= \int F_L(\nu,\nu')~F_D(\nu',\nu_0)~d\nu'
\label{}
\end{equation}
though there's no strict justfication for its use - the two processes
are assumed to act independently, which in reality is not the
fact. Regardless of this flaw, it's the only way up to date to model
the combination of the broadening processes. The integral in (3.11)
can not computed analytically, so certain approximations algorithms
must be used.


Another possibility would be the combination the last two equations
(3.10) and (3.11). The respective result then will be 
\begin{equation}
 F_S=\left(\frac{\nu}{\nu_0}\right)^2~[F_{Voigt}(\nu,\nu_0)+F_{Voigt}(\nu,-\nu_0)]
\label{}
\end{equation}
The advantage of such a model is that it behaves like a van
Vleck-Weisskopf line-shape function in the high pressure limit and
like a Voigt one in the low pressure limit. There is one important
caveat to the equation (3.12): it has to be made sure that the
algorithm that is used to compute the Voigt function really produces a
Lorentz line in the high pressure limit. Another point of signifacance
is the demand that model yields meanigful rersuts far from the line
center, since the line center from the ``mirror'' line at -$\nu_0$ is
situated approximately 2$\nu_0$ away from tthe frequency $\nu_0$ of
computation. The algorithm of Drayson [$Drayson$, 1976; $Oliveiro~and~
Longbothum$, 1977] was explicitly checked that satisfies both
requirements, while this not true for some other algorithms, commonlly
used for Voight-shape computation. In particular, the it's not true
for the Hui-Armstrong-Wray Formula, as defined in $Hui~et~al.$
[1978] and in Equation 2.60 of $Rosenkrantz$ [1993]. So, provided the
stated above is fullfilled, the $F_S$ line shape gives a smooth
transition from high tropospheric pressures to low stratospheric ones,
and to be valid near the line centers throughout the microwave region.
\levelc{Partition Functions}

The treatment of the partition functions is directly related to the molecular
energy states and their statistical distribution during the
radiation process. 

In any case of spectroscopic interest the free molecules of a gas are
not ptically thick at all freqeuncies, so the radiation energy is not
represented by blackbody radiation. The most common assumption made,
which is sufficient in the case of tropospheric and low stratospheric
research, is the {\textit{local thermodynamic equilibrium\nocorr} or $LTE$. According to it,
it's possible to find a common temperature, which may vary from place
to place, that fits the Boltzman energy population distribution and
the Maxwell velocities distribution. This practically means, that
under $LTE$ the collisional processes must be of greater importance
than radiative ones. In other words, an excited state must have a
hihger probability of de-excitation by collision than by spontaneous
radiation. This is the important factor which make natural broadening
differ quantatively so much from the pressure (collisional), though
both are described qualitatively almost identical by  Lorentzian line-
shape functions.

Accordind  to the Maxwell-Boltzman distribution law,in $LTE$ the total number
of gas particles $N_n$  in a state $E_n$ is given by 
\begin{equation}
 N_n=N_0\frac{g_n}{g_0}e^{-E_n/kT}
\label{}
\end{equation}
where $N_0$ is particle number in the ground state, and $g_n$, $g_o$
are the statistical weights (degeneracies) of the $n-$state and the
ground state. Thus the total particle number $N$ is given by
\begin{equation}
 N=\frac{N_0}{g_0}\sum_{n=0}^\infty e^{-E_n/kT}=\frac{N_0}{g_0}~Q(T)
\label{}
\end{equation}
The quantity $Q(T)$ is the {\it{partition function}\nocorr} of the gas. At
one hand all thermodynamic quantities can be expressed through it, and at
another it comes into play by the calculation line intensities.\\
Generally speaking, the partition fuction for a perfect gas can be
represented by the product of the {\it{translational}\nocorr} and the {\it{internal}\nocorr} partition functions
\begin{equation}
 Q  =  Q_{tr}~Q_{int}
\label{}
\end{equation}
bearing in mind that the respective energies are independent of each
other. The quantitity, however, which we are interested in in (3.13)
is the {\it{internal}\nocorr} partition function (or the  {\it{total internal
    partition function}\nocorr}). It describes, in general terms, the energy
distribution and redistribution during the radiation process. 

The internal partition function for free gaseous molecules is  a
function of the electronic, the vibrational, the rotational, and the
nuclear spin states. An approximation is used in order to display the
individual contribution explicitly
\begin{equation}
 Q=Q_e~Q_v~Q_r~Q_n
\label{}
\end{equation}
and thus the intraciton betweeen these various states is neglected. For
practically all polyatomic molecules the excited electronic staates
are entirely negligable to those of the ground states,
i.e. $Q_e=1$ . Only for the very few polyatomic molecules with a
multiplet ground state ($NO_2$ , $ClO_2$ , and free radicals) has the
electronic contibution to be considered.\\
If we neglect the anharmonicities, the vibratioanal partition
function, with vibrational levels measured with respect to the ground
state, the term $h\omega_n/2$ is omitted in the energy expression for
the {\it{harmonic oscilator}\nocorr}, is 
\begin{equation}
 Q_v=\left(\sum_{\nu_1}e^{-\nu_1 h\omega_1/kT}\right)\left(\sum_{\nu_2}e^{-\nu_2 h\omega_2/kT}\right)...
\label{}
\end{equation}
where $\nu_1$, $\nu_2$,..., the vibrational quantum numbers, can each
have the valies 0,1,2,... and $\omega_1$, $\omega_2$,..are the
frequencies of the fundamental modes of vibration. The summation is
taken over all values of $\nu_1$, $\nu_2$,..., and each fundamental
mode is counted separately. This result is valid for non-degenerate
vibrations. If we use the simple exprassion for geometric progression
\begin{equation}
 \sum_{\nu_i}e^{-\nu_i h\omega_i/kT}=\frac{1}{1-e^{h\omega_i/kT}}
\label{}
\end{equation}
and the degeneracies $d_1$, $d_2$,... of the fundamental modes, we ge
finally for vibrational partition function
\begin{equation}
Q_v=\left(1-e^{h\omega_1/kT}\right)^{-d_1}\left(1-e^{h\omega_2/kT}\right)^{-d_2}...
\label{}
\end{equation}

The rotational partition function looks differently for the different
symmetry types of molecules.
For diatomic and linear  polyatomic molecules with no center of
symmetry the corresponding expression is 
\begin{eqnarray}
Q_r & = & \sum_{J=0}^\infty (2J+1)e^{-hBJ(J+1)/kT}\nonumber\\
   & = & \frac{kT}{hB}+\frac{1}{3}+\frac{1}{15}\frac{hB}{kT}+\frac{4}{315}\left(\frac{hB}{kT}\right)^2+...\nonumber\\
   & \cong & \frac{kT}{hB}
\label{}
\end{eqnarray}
By {\it{symmetric-}}, {\it{asymmetric-}}, and {\it{spherical}} top molecules there are also
other factors to be taken into consideration, such as the
spatial structure of the molecules, nuclear spin, inversion and
internal rotation. The general expression in tis case us
\begin{equation}
Q_r  =  \frac{1}{\sigma}\sum_{J=0}^\infty \sum_{K=-J}^{J}(2J+1)~e^{-h[BJ(J+1)+(A-B)K^2]/kT}
\label{}
\end{equation}
where $\sigma$ is ameasure of the degree of symmetry. For the usual
symmetric top has $C_3$ or $C_{3\nu}$ symmetry, $\sigma$ = 3. To a good
aprroximation, the summation above can expressed as
\begin{equation}
Q_r  = 
\frac{1}{\sigma}\left[\left(\frac{\pi}{B^2A}\right)\left(\frac{kT}{h}\right)^3\right]^{1/2}=
\frac{5.34\times 10^6}{\sigma}\left(\frac{T^3}{B^{2}A}\right)^{1/2}
\label{}
\end{equation}
For asymmetric top the formula would then be 
\begin{equation}
Q_r = \frac{5.34\times 10^6}{\sigma}\left(\frac{T^3}{ABC}\right)^{1/2}
\label{}
\end{equation}
and for sprherical top
\begin{equation}
Q_r = \frac{5.34\times 10^6}{\sigma}\left(\frac{T^3}{A^3}\right)^{1/2}
\label{}
\end{equation}
















\levelc{Line Catalogues} 

Mostly describe the ARTS internal format, but also briefly list the
other catalogues that can be read by ARTS

\levelc{Species specific data} 

Molecular mass, etc. 


\levelc{ARTS Workspace Variables and Methods}
%
%
% ============================================================================
%
%
\levelb{Complete Absorption Models}
\label{levelb:CompAbsMod}
% =======================
%
The MPM absorption model of Liebe and coworkers consists of modules for 
$\hzo$, dry air ($\oz, \nz$), liquid water droplets, and ice particles 
absorption. The Rosenkranz (R998) and Cruz-Pol et al. (CP98) absorption 
models include absorption due to water vapor and dry air. Additionally 
the CP98 model has a strongly reduced parameter set for the $\hzo$-line 
absorption since it is especially intended for the range around the 
22\,GHz water line. The MPM and R98 are valid from the microwave 
up to the submillimeter frequency range (1-1000\,GHz).

Implemented in ARTS are the following modules of the above stated models:
%
\begin{center}
\begin{tabular}{ll}
\hline
model & implemented modules \\
\hline
MPM87 & $\hzo$ \\
MPM89 & $\hzo$ \\
MPM93 & $\hzo$, $\oz$, $\nz$, liquid water droplets, ice particles \\
R98   & $\hzo$, $\oz$, $\nz$ \\
CP98  & $\hzo$, $\oz$ \\
\hline
\end{tabular}
\end{center}
%
What follows in the sections below is a description of each module and how it is 
called within ARTS.

One remark has to be done before the description starts: becasue different 
models use different units for their physical quantities,
table \ref{table:si_units} lists some common units and their conversion into 
ARTS units which are in most caes the SI units.
\begin{table}[!ht]
\begin{center}
\begin{tabular}{r@{~}lcr@{~}lll}
\hline
$x$ & g/cm$^3$ & = & $y$ & kg/m$^3$ & $\Longrightarrow$ & $y = x \times 1.00\cdot 10^{3}$ \\
$x$ & g/m$^3$  & = & $y$ & kg/m$^3$ & $\Longrightarrow$ & $y = x \times 1.00\cdot 10^{-3}$ \\
$x$ & GHz      & = & $y$ & Hz       & $\Longrightarrow$ & $y = x \times 1.00\cdot 10^{9}$ \\
$x$ & 1/GHz    & = & $y$ & 1/Hz     & $\Longrightarrow$ & $y = x \times 1.00\cdot 10^{-9}$ \\
$x$ & hPa      & = & $y$ & Pa       & $\Longrightarrow$ & $y = x \times 1.00\cdot 10^{2}$ \\
$x$ & 1/hPa    & = & $y$ & 1/Pa     & $\Longrightarrow$ & $y = x \times 1.00\cdot 10^{-2}$ \\
$x$ & 1/cm     & = & $y$ & 1/m      & $\Longrightarrow$ & $y = x \times 1.00\cdot 10^{2}$ \\
$x$ & 1/km     & = & $y$ & 1/m      & $\Longrightarrow$ & $y = x \times 1.00\cdot 10^{-3}$ \\
$x$ & dB       & = & $y$ & Np       & $\Longrightarrow$ & $y = x~/~[10.0 \times \log_{10}{(e)}]$ \\
$x$ & dB/km    & = & $y$ & 1/m      & $\Longrightarrow$ & $y = x \times 1.00\cdot 10^{-3}~/~[10.0 \times \log_{10}{(e)}]$\\
$x$ & Np/km    & = & $y$ & 1/m      & $\Longrightarrow$ & $y = x \times 1.00\cdot 10^{-3}$ \\
\hline
\end{tabular}
\caption{SI units are meter, kilogram, second, and Kelvin. 
  This table gives a short conversion scheme to oter units
  commonly used in atmospheric absorption calculations.}
\label{table:si_units}
\end{center}
\end{table}
%
%
% ============================================================================
%
%
\levelc{Complete Water Vapor Models}
\label{levelc:CompWatVapMod}
% ==========================
%
In ARTS several complete water vapor absorption models are implemented and 
can easily be used.\\
Implemented models are the versions MPM87 \cite{liebeandlayton:87}, MPM89 
\cite{liebe:89}, and MPM93 \cite{liebeetal:93} of the Liebe absorption 
model and additionally the models of Cruz-Pol et al. (CP98) \cite{cruzpol:98} 
and P.~W. Rosenkranz (R98) \cite{pwr:98}. 
MPM and R98 are especially desigend for fast absorption calculations in 
the frequency range of 1-1000\,GHz while the CP98 model is a reduced model 
for a narrow frequency band around the 22\,GHz $\hzo$-line.

The total water vapor absorption ($\alphatot$) is in all the stated models 
described by a line absorption ($\alphal$) term and a continuum absorption 
($\alphac$) term: 
\begin{equation}
  \label{eq:h2o:totabs}
  \alphatot = \alphal + \alphac
\end{equation}
The main differences between the different models is the line shape used for 
$\alphal$ and the formulation of $\alphac$.

\levelb{Continua and Complete Absorption Models}
=======
It has to be emphasized that, $\alphal$ and $\alphac$ of different
models are not necessarily compatible and should therefore not be interchanged.
%
%
% ============================================================================
%
% 
\leveld{H2O-MPM87}
\label{leveld:mpm87}
% ==================
This version, which is described in \cite{liebeandlayton:87} and 
follows the general line of the MPM model to devide the total water vapor absorption, 
$\alphampmotot$, into a spectral line term, $\alphampmol$, and a 
continuum term not attributed to spectral lines, $\alphampmoc$:
\begin{equation}
  \label{eq:mpm87_abs}
  \alphampmotot = \alphampmol + \alphampmoc\hspace*{10mm}\mbox{dB/km}
\end{equation}
%
%
\levele{Water Vapor Line Absorption:}
\label{levele:h2o_mpm87_lines}
%-----------------------------------
The MPM87 \cite{liebeandlayton:87} water vapor line catalog consists 
of 30 lines from 22\,GHz up to 988\,GHz. The center frequencies and parameter 
values are listed in Table~\ref{tab:mpm87linelist}. To describe the line 
absorption, a set of three parameters ($\bek$ and $\bdk$) per line are used: two 
for the line strength and one for the line width. The total line 
absorption coefficient (in units of dB/km) is the sum over all 
individual line absorption coefficients\footnote{The factor 
  $0.1820 \cdot 10^{6}$ is equal to $(4\,\pi/c)\cdot 10\log{(e)}$
  (the term $(4\,\pi/c)$ comes from the definition of the absorption
  coefficient in terms of the dielectric constant and the term 
  $10\,\log{(e)}$ is due to the definition of the Decibel.) The
  velocity of light is defined as $c=2.9979\cdot 10^{-4}$\,km\,GHz. 
  The factor $10^{6}$ is incorporated into the line strength and 
  does therefore not appear in the pre-factor.}:
\begin{equation}
  \label{eq:mpm87:absline}
  \alphampmol = 0.1820 \cdot \nuk \cdot \phzo \cdot 
  \sum_{k}{\inten \cdot \shape}\hspace*{10mm}\mbox{dB/km}
\end{equation}
where $\inten$ is the line intensity described by the parameterization
\begin{equation}
  \label{eq:mpm87:strength}
  \inten = \bek \cdot \phzo \cdot \Theta^{3.5} 
           \cdot \exp{(\bzk \cdot [1-\Theta])}\hspace*{10mm}\mbox{kHz}
\end{equation}
with $\nuk$ as the line center frequency, $\phzo$ the water
vapor partial pressure and $\Theta = 300\,\mbox{K}/T$.\\
The line shape function, $\shape$, in Eq.~(\ref{eq:mpm87:absline}) 
is the standard Van~Vleck-Weisskopf (VVW) function, given by:
\begin{eqnarray}
% Van Vleck-Weisskopf function
  \label{eq:mpm87:VVW}
  \shape & = & \left(\frac{\nu}{\nuk}\right) \cdot 
               \left[\frac{\gamk}{(\nu - \nuk)^2 + \gamk^2} + 
                     \frac{\gamk}{(\nu + \nuk)^2 + \gamk^2}\right]\\
\end{eqnarray}
The pressure broadened line width, $\gamk$, is calculated with the 
single parameter $\bdk$ in the following way:
\begin{equation}
  \label{eq:mpm87:gamma}
  \gamk = \bdk \cdot 
          (4.80 \cdot \phzo \cdot \Theta^{1.1} + \pda \cdot
          \Theta^{0.6})\hspace*{10mm}\mbox{GHz}
\end{equation}
where $\pda$ is the partial pressure of dry air ($\pda=\ptot-\phzo$). 
The parameterizations of $\inten$ and $\gamk$ are already in use for the 
early version of MPM81 \cite{liebe:81}.\\
%
\begin{longtable}{rrrrr}
 K & K & K & K & K \kill
%
% --------------------- only begin of table ------------------------------
 \hline
       & $\nu_k$ & $\bek$   & $\bzk$ & $\bdk$  \\
% //      [GHz]    [kHz/kPa]   [1]     [GHz/kPa]
 $k$   & {\sf [GHz]}  & {[$\frac{\sf kHz}{\sf kPa}$]} & {\sf [1]} & 
 {[$\frac{\sf GHz}{\sf kPa}$]}\\
 \hline
 \endfirsthead
% --------------------- every page begin of table ------------------------
 \hline
  $k$  & $\nu_k$ & $\bek$ & $\bzk$ & $\bdk$ \\
 \hline
 \endhead
% --------------------- every page end of table ------------------------
 K & K & K & K & K \kill
 \hline
 \caption[]{(continued)}\\
 \endfoot
% --------------------- only end of table ------------------------------
 K & K & K & K & K \kill 
 \hline
 \caption{List of H$_2$O spectral lines and their spectroscopic 
   parameters (H$_2$O-air mixture) for the MPM87 model \cite{liebeandlayton:87}.}
 \label{tab:mpm87linelist}
 \endlastfoot
% --------------------- body of table  ----------------------------------  
% //         0           1           2       3      
% //         f0          b1          b2      b3     
% //        [GHz]       [kHz/kPa]   [1]    [GHz/kPa]
%  const Numeric mpm87[30][4] = { 
1     &    22.235080&    0.1090&  2.143&   27.84$\cdot$ 10$^{-3}$\\
2     &    67.813960&    0.0011&  8.730&   27.60$\cdot$ 10$^{-3}$\\
3     &   119.995940&    0.0007&  8.347&   27.00$\cdot$ 10$^{-3}$\\
4     &   183.310117&    2.3000&  0.653&   31.64$\cdot$ 10$^{-3}$\\
5     &   321.225644&    0.0464&  6.156&   21.40$\cdot$ 10$^{-3}$\\
6     &   325.152919&    1.5400&  1.515&   29.70$\cdot$ 10$^{-3}$\\
7     &   336.187000&    0.0010&  9.802&   26.50$\cdot$ 10$^{-3}$\\
8     &   380.197372&   11.9000&  1.018&   30.36$\cdot$ 10$^{-3}$\\
9     &   390.134508&    0.0044&  7.318&   19.00$\cdot$ 10$^{-3}$\\
10    &   437.346667&    0.0637&  5.015&   13.70$\cdot$ 10$^{-3}$\\
11    &   439.150812&    0.9210&  3.561&   16.40$\cdot$ 10$^{-3}$\\
12    &   443.018295&    0.1940&  5.015&   14.40$\cdot$ 10$^{-3}$\\
13    &   448.001075&   10.6000&  1.370&   23.80$\cdot$ 10$^{-3}$\\
14    &   470.888947&    0.3300&  3.561&   18.20$\cdot$ 10$^{-3}$\\
15    &   474.689127&    1.2800&  2.342&   19.80$\cdot$ 10$^{-3}$\\
16    &   488.491133&    0.2530&  2.814&   24.90$\cdot$ 10$^{-3}$\\
17    &   503.568532&    0.0374&  6.693&   11.50$\cdot$ 10$^{-3}$\\
18    &   504.482692&    0.0125&  6.693&   11.90$\cdot$ 10$^{-3}$\\
19    &   556.936002&  510.0000&  0.114&   30.00$\cdot$ 10$^{-3}$\\
20    &   620.700807&    5.0900&  2.150&   22.30$\cdot$ 10$^{-3}$\\
21    &   658.006500&    0.2740&  7.767&   30.00$\cdot$ 10$^{-3}$\\
22    &   752.033227&  250.0000&  0.336&   28.60$\cdot$ 10$^{-3}$\\
23    &   841.073593&    0.0130&  8.113&   14.10$\cdot$ 10$^{-3}$\\
24    &   859.865000&    0.1330&  7.989&   28.60$\cdot$ 10$^{-3}$\\
25    &   899.407000&    0.0550&  7.845&   28.60$\cdot$ 10$^{-3}$\\
26    &   902.555000&    0.0380&  8.360&   26.40$\cdot$ 10$^{-3}$\\
27    &   906.205524&    0.1830&  5.039&   23.40$\cdot$ 10$^{-3}$\\
28    &   916.171582&    8.5600&  1.369&   25.30$\cdot$ 10$^{-3}$\\
29    &   970.315022&    9.1600&  1.842&   24.00$\cdot$ 10$^{-3}$\\
30    &   987.926764&  138.0000&  0.178&   28.60$\cdot$ 10$^{-3}$\\
\hline
% -----------------------------------------------------------------------  
\end{longtable}
%
%
\levele{Water Vapor Continuum Absorption:}
\label{levele:h2o_mpm87_cont}
%----------------------------------------
The water vapor continuum absorption coefficient in MPM87, $\alphampmoc$, 
is determined from laboratory measurements at 137.8\,GHz by Liebe 
and Layton covering the following parameter range:\\
\begin{tabular}{lr}
temperature          & 282-316\,K\\
relative humidity    & 0-95\,\%\\
dry air pressure     & 0 - 160\,kPa\\ 
\end{tabular}\\
The mathematical expression of $\alphampmoc$ is derived from the far wing 
approximation of the line absorption and is expressed as follows
\begin{equation} 
  \label{eq:mpm87:cont}
  \alphampmoc = \nu^2 \cdot \phzo \cdot 
                (\cso \cdot \phzo \cdot \Theta^{\xs} + 
                 \cdo \cdot \pda  \cdot \Theta^{\xf}),
\end{equation}
with the continuum parameter set $\cso$, $\cdo$, $\xs$, and $\xf$. 
The determined values of the continuum parameters are:
\begin{center}
\begin{tabular}{ccrr}
\hline
\multicolumn{1}{c}{$\cso$} & \multicolumn{1}{c}{$\cdo$} & $\xs$ & $\xd$ \\
\multicolumn{2}{c}{[10$^{-6}$(dB/km)/(hPa$\cdot$GHz)$^2$]} & [1] & [1] \\
\hline
6.496 & 0.206 & 10.5 & 3.0 \\
\hline
\end{tabular}
\end{center}
%
%
\levele{ARTS Workspace Variables and Methods for H2O-MPM87}
% ---------------------------------------------------------
\begin{center}
\begin{tabular}{ll}
\hline
\multicolumn{2}{c}{ARTS tag definition}\\
name      & comment \\
H2O-MPM87 & MPM87 is defined as {\it isotope} of $\hzo$\\
\hline
\multicolumn{2}{c}{ARTS workspace variables}\\
name & comment \\
 f\_mono & input frequency grid \\
 p\_abs  & input pressure grid \\
 t\_abs  & input temperature grid\\
 vmr    & input $\hzo$ volume mixing ratio\\
\hline
\multicolumn{2}{c}{ARTS methods}\\
function           & method\\
xsec\_continuum\_tag & called by absCalc \\
\hline
\multicolumn{2}{c}{calling tree}\\
MPM87H2OAbsModel & $\Leftarrow$~xsec\_continuum\_tag~$\Leftarrow$~absCalc\\
\end{tabular}
\end{center}

\levele{ARTS control file:}
% -------------------------

\begin{verbatim}
#
tgsDefine{
      [ 
        "N2",
        "H2O-MPM87"
      ] 
}
#
\end{verbatim}
$\vdots$
\begin{verbatim}
#
# MPM87 H2O absorption model (lines + continuum)
cont_descriptionAppend{
    name       = "H2O-MPM87"
    parameters = [ ]
}
#
\end{verbatim}
$\vdots$
\begin{verbatim}
# Set the physical H2O profile from the H2O profile in vmrs:
h2o_absSet{}

# Set the physical N2 profile from the N2 profile in vmrs:
n2_absSet{}
\end{verbatim}
$\vdots$
%
%
% ============================================================================
%
% 
\leveld{H2O-MPM89}
\label{leveld:mpm89}
% ==================
%
MPM89 is described in \cite{liebe:89} and follows the general line 
of the MPM model to devide the total water vapor absorption, 
$\alphampmmtot$, into a spectral line term, $\alphampmml$, and a continuum 
term not attributed to spectral lines, $\alphampmmc$:
\begin{equation}
  \label{eq:mpm89_abs}
  \alphampmmtot = \alphampmml + \alphampmmc\hspace*{10mm}\mbox{dB/km}
\end{equation}
All the absorption coefficients are calculated in units of \mbox{dB/km}.
%
%
\levele{Water Vapor Line Absorption:}
\label{levele:h2o_mpm89_lines}
%-----------------------------------
The MPM89 water vapor line catalog consists of the same 30 lines 
like MPM87 from 22\,GHz up to 988\,GHz. The center frequencies and parameter 
values are listed in Table~\ref{tab:mpm89linelist}. To describe the line 
absorption, a set of six parameters ($\bek$ and $\bsk$) per line are used: two 
for the line strength and four for the line width. The total line 
absorption coefficient (in units of dB/km) is the sum over all
individual line absorption coefficients\footnote{see footnote for
  MPM97 line absorption}:
\begin{equation}
  \label{eq:mpm89:absline}
  \alphampmml = 0.1820 \cdot \nuk \cdot \phzo \cdot 
  \sum_{k}{\inten \cdot \shape}\hspace*{10mm}\mbox{dB/km}
\end{equation}
where $\inten$ is the line intensity described by the parameterization
\begin{equation}
  \label{eq:mpm89:strength}
  \inten = \bek \cdot \phzo \cdot \Theta^{3.5} 
           \cdot \exp{(\bzk \cdot [1-\Theta])}\hspace*{10mm}\mbox{kHz}
\end{equation}
whit $\nuk$ as the line center frequency, $\phzo$ the water
vapor partial pressure and $\Theta = 300\,\mbox{K}/T$.\\
The line shape function, $\shape$, in Eq.~(\ref{eq:mpm89:absline}) 
is the standard Van Vleck-Weisskopf (VVW) function, given by 
\begin{eqnarray}
% Van Vleck-Weisskopf function
  \label{eq:mpm89:VVW}
  \shape & = & \left(\frac{\nu}{\nuk}\right) \cdot 
               \left[\frac{\gamk}{(\nu - \nuk)^2 + \gamk^2} + 
                     \frac{\gamk}{(\nu + \nuk)^2 + \gamk^2}\right]\\
\end{eqnarray}
where the pressure broadened line width, $\gamk$, is calculated as
\begin{equation}
  \label{eq:mpm89:gamma}
  \gamk = \bdk \cdot 
         (\bfk \cdot \phzo \cdot \Theta^{\bsk} + 
                     \pda  \cdot \Theta^{\bvk})
        \cdot 10^{-3}\hspace*{10mm}\mbox{GHz}
\end{equation}
with $\pda=\ptot-\phzo$ as the dry air partial pressure. 
The only difference between MPM87 and MPM89 with respect to the line 
absorption is the parameterization of the pressure broadened line
width, $\gamk$, which is calculated with the four parameters $\bdk$ to
$\bsk$ in the case of MPM89 whereas in MPM87 a single parameter
($\bdk$) is used (see Eq.~(\ref{eq:mpm87:gamma})).
%
\begin{longtable}{rrrrrrrr}
 K & K & K & K & K & K & K & K \kill
%
% --------------------- only begin of table ------------------------------
 \hline
    & $\nu_k$ & $\bek$ & $\bzk$ & $\bdk$ & $\bvk$ & $\bfk$ & $\bsk$ \\
%    [GHz]     [kHz/kPa]   [1]   [MHz/kPa]  [1]    [1]    [1]
 $k$& {\sf [GHz]}  & {[$\frac{\sf kHz}{\sf kPa}$]} & {\sf [1]} & 
 {[$\frac{\sf MHz}{\sf kPa}$]} & {\sf [1]} & {\sf [1]} & {\sf [1]} \\
 \hline
 \endfirsthead
% --------------------- every page begin of table ------------------------
 \hline
  $k$  & $\nu_k$ & $\bek$ & $\bzk$ & $\bdk$ & $\bvk$ & $\bfk$ & $\bsk$ \\
 \hline
 \endhead
% --------------------- every page end of table ------------------------
 K & K & K & K & K & K & K & K \kill
 \hline
 \caption[]{(continued)}\\
 \endfoot
% --------------------- only end of table ------------------------------
 K & K & K & K & K & K & K & K \kill
 \hline
 \caption{List of H$_2$O spectral lines and their spectroscopic 
   parameters (H$_2$O-air mixture) for the MPM89 model \cite{liebe:89}.}
 \label{tab:mpm89linelist}
 \endlastfoot
% --------------------- body of table  ----------------------------------  
%            0           1        2       3        4      5      6
%            f0          b1       b2      b3       b4     b5     b6
%          [GHz]     [kHz/kPa]   [1]   [MHz/kPa]  [1]    [1]    [1]
%  const Numeric mpm89[30][7] = { 
1    &    22.235080&    0.1090&  2.143&   28.11&   0.69&  4.80&  1.00\\
2    &    67.813960&    0.0011&  8.735&   28.58&   0.69&  4.93&  0.82\\
3    &   119.995940&    0.0007&  8.356&   29.48&   0.70&  4.78&  0.79\\
4    &   183.310074&    2.3000&  0.668&   28.13&   0.64&  5.30&  0.85\\
5    &   321.225644&    0.0464&  6.181&   23.03&   0.67&  4.69&  0.54\\
6    &   325.152919&    1.5400&  1.540&   27.83&   0.68&  4.85&  0.74\\
7    &   336.187000&    0.0010&  9.829&   26.93&   0.69&  4.74&  0.61\\
8    &   380.197372&   11.9000&  1.048&   28.73&   0.69&  5.38&  0.84\\
9    &   390.134508&    0.0044&  7.350&   21.52&   0.63&  4.81&  0.55\\
10    &   437.346667&    0.0637&  5.050&   18.45&   0.60&  4.23&  0.48\\
11    &   439.150812&    0.9210&  3.596&   21.00&   0.63&  4.29&  0.52\\
12    &   443.018295&    0.1940&  5.050&   18.60&   0.60&  4.23&  0.50\\
13    &   448.001075&   10.6000&  1.405&   26.32&   0.66&  4.84&  0.67\\
14    &   470.888947&    0.3300&  3.599&   21.52&   0.66&  4.57&  0.65\\
15    &   474.689127&    1.2800&  2.381&   23.55&   0.65&  4.65&  0.64\\
16    &   488.491133&    0.2530&  2.853&   26.02&   0.69&  5.04&  0.72\\
17    &   503.568532&    0.0374&  6.733&   16.12&   0.61&  3.98&  0.43\\
18    &   504.482692&    0.0125&  6.733&   16.12&   0.61&  4.01&  0.45\\
19    &   556.936002&  510.0000&  0.159&   32.10&   0.69&  4.11&  1.00\\
20    &   620.700807&    5.0900&  2.200&   24.38&   0.71&  4.68&  0.68\\
21    &   658.006500&    0.2740&  7.820&   32.10&   0.69&  4.14&  1.00\\
22    &   752.033227&  250.0000&  0.396&   30.60&   0.68&  4.09&  0.84\\
23    &   841.073593&    0.0130&  8.180&   15.90&   0.33&  5.76&  0.45\\
24    &   859.865000&    0.1330&  7.989&   30.60&   0.68&  4.09&  0.84\\
25    &   899.407000&    0.0550&  7.917&   29.85&   0.68&  4.53&  0.90\\
26    &   902.555000&    0.0380&  8.432&   28.65&   0.70&  5.10&  0.95\\
27    &   906.205524&    0.1830&  5.111&   24.08&   0.70&  4.70&  0.53\\
28    &   916.171582&    8.5600&  1.442&   26.70&   0.70&  4.78&  0.78\\
29    &   970.315022&    9.1600&  1.920&   25.50&   0.64&  4.94&  0.67\\
30    &   987.926764&  138.0000&  0.258&   29.85&   0.68&  4.55&  0.90\\
\hline
% -----------------------------------------------------------------------  
\end{longtable}

\levele{Water Vapor Continuum Absorption:}
\label{levele:h2o_mpm89_cont}
%----------------------------------------
The continuum absorption coefficient in MPM89, $\alphampmmc$, 
is the same as in MPM87 (see Sec. \ref{levele:h2o_mpm87_cont} for 
details):\\
\begin{equation} 
  \label{eq:mpm89:cont}
  \alphampmmc = \nu^2 \cdot \phzo \cdot 
                (\cso \cdot \phzo \cdot \Theta^{\xs} + 
                 \cdo \cdot \pda  \cdot \Theta^{\xf}),
\end{equation}
with\\
\begin{tabular}{lcl}
$\cso$   & = & 6.496$\cdot$10$^{-6}$(dB/km)/(hPa$\cdot$GHz)$^2$\\
$\xs$    & = & 10.5\\
$\cdo$   & = & 0.206$\cdot$10$^{-6}$(dB/km)/(hPa$\cdot$GHz)$^2$\\
$\xd$    & = & 3.0\\
\end{tabular}
%
%
\levele{ARTS Workspace Variables and Methods for H2O-MPM89}
% ---------------------------------------------------------
\begin{center}
\begin{tabular}{ll}
\hline
\multicolumn{2}{c}{ARTS tag definition}\\
name      & comment \\
H2O-MPM89 & MPM89 is defined as {\it isotope} of $\hzo$\\
\hline
\multicolumn{2}{c}{ARTS workspace variables}\\
name & comment \\
 f\_mono & input frequency grid \\
 p\_abs  & input pressure grid \\
 t\_abs  & input temperature grid\\
 vmr    & input $\hzo$ volume mixing ratio\\
\hline
\multicolumn{2}{c}{ARTS methods}\\
function           & method\\
xsec\_continuum\_tag & called by absCalc \\
\hline
\multicolumn{2}{c}{calling tree}\\
MPM89H2OAbsModel & $\Leftarrow$~xsec\_continuum\_tag~$\Leftarrow$~absCalc\\
\end{tabular}
\end{center}

\levele{ARTS control file:}
% -------------------------

\begin{verbatim}
#
tgsDefine{
      [ 
        "N2",
        "H2O-MPM89"
      ] 
}
#
\end{verbatim}
$\vdots$
\begin{verbatim}
#
# MPM89 H2O absorption model (lines + continuum)
cont_descriptionAppend{
    name       = "H2O-MPM89"
    parameters = [ ]
}
#
\end{verbatim}
$\vdots$
\begin{verbatim}
# Set the physical H2O profile from the H2O profile in vmrs:
h2o_absSet{}

# Set the physical N2 profile from the N2 profile in vmrs:
n2_absSet{}
\end{verbatim}
$\vdots$
%
%
% ============================================================================
%
% 
\leveld{H2O-MPM93}
\label{leveld:mpm93}
% ==================
This version, which is described in \cite{liebeandlayton:87} and 
follows the general line of the MPM model to devide the total 
water vapor absorption, $\alphampmntot$, into a spectral line 
term, $\alphampmnl$, and a continuum term not attributed to 
spectral lines, $\alphampmnc$:
\begin{equation}
  \label{eq:mpm93_abs}
  \alphampmntot = \alphampmnl + \alphampmnc\hspace*{10mm}\mbox{dB/km}
\end{equation}
The continuum absorption is parameterized like a
resonant spectral line of $\hzo$, a so-called pseudo-line. This is a 
fundamental change in the parameterization of the water vapor
continuum in respect to all older versions of MPM, which makes it 
quite complicate to compare the different versions, especially to 
distinguish a self- and foreign broadening term in the continuum.
%
%
\levele{Water Vapor Line Absorption:}
\label{levele:h2o_mpm87_lines}
%-----------------------------------
The water vapor line spectrum of MPM93 \cite{liebeetal:93} 
consists of 34 lines below 1\,THz (four more than in MPM89 and MPM87). 

\levele{Water Vapor Line Absorption:}
\label{levele:h2o_mpm93_lines}
%-----------------------------------
To describe the MPM93 water vapor line absorption, a set of six parameters 
($\bek$ and $\bdk$) per line are used: two for the line strength and 
four for the line width. The total line absorption coefficient 
(in units of dB/km) is the sum over all individual line absorption 
coefficients\footnote{see footnote for MPM97 line absorption}:
\begin{equation}
  \label{eq:mpm93:absline}
  \alphampmnl = 0.1820 \cdot \nuk \cdot \phzo \cdot 
  \sum_{k}{\inten \cdot \shape}\hspace*{10mm}\mbox{dB/km}
\end{equation}
where $\inten$ is the line intensity described by the parameterization
\begin{equation}
  \label{eq:mpm93:strength}
  \inten = \bek \cdot \phzo \cdot \Theta^{3.5} 
           \cdot \exp{(\bzk \cdot [1-\Theta])}\hspace*{10mm}\mbox{kHz}
\end{equation}
with $\nuk$ as the line center frequency, $\phzo$ the water
vapor partial pressure and $\Theta = 300\,\mbox{K}/T$.\\
The line shape function, $\shape$, in Eq.~(\ref{eq:mpm87:absline}) 
is the standard Van~Vleck-Weisskopf (VVW) function, given by:
\begin{eqnarray}
% Van Vleck-Weisskopf function
  \label{eq:mpm93:VVW}
  \shape & = & \left(\frac{\nu}{\nuk}\right) \cdot 
               \left[\frac{\gamk}{(\nu - \nuk)^2 + \gamk^2} + 
                     \frac{\gamk}{(\nu + \nuk)^2 + \gamk^2}\right]\\
\end{eqnarray}
The pressure broadened line width, $\gamk$, is calculated with the 
single parameter $\bdk$ in the following way:
\begin{equation}
  \label{eq:mpm93:gamma}
  \gamk = \bdk \cdot 
          (4.80 \cdot \phzo \cdot \Theta^{1.1} + \pda \cdot
          \Theta^{0.6})\hspace*{10mm}\mbox{GHz}
\end{equation}
where $\pda$ is the partial pressure of dry air ($\pda=\ptot-\phzo$). 

The parameterizations of $\inten$ was already in use for the early 
version of MPM81 \cite{liebe:81}. The expression for $\gamk$ is the
same as in MPM89. The main difference between MPM93 and MPM89 
concerning the water vapor line absorption is the updated line catalog.
%
%
\begin{longtable}{rrrrrrrrrr}
 K & K & K & K & K & K & K & K & K & K \kill
%
% --------------------- only begin of table ------------------------------
 \hline
    & $\nu_k$ & $\bek$ & $\bzk$ & $\bdk$ & $\bdkp$ 
 & $\bvk$ & $\bvkp$ & $\bfk$ & $\bsk$ \\
 $k$   & {\sf [GHz]}  & {[$\frac{\sf kHz}{\sf hPa}$]} & {\sf [1]} & 
 {[$\frac{\sf MHz}{\sf hPa}$]} & {[$\frac{\sf MHz}{\sf hPa}$]} & 
 {\sf [1]} & {\sf [1]} & {\sf [1]} & {\sf [1]} \\
 \hline
 \endfirsthead
% --------------------- every page begin of table ------------------------
 \hline
    & $\nu_k$ & $\bek$ & $\bzk$ & $\bdk$ & $\bdkp$ 
 & $\bvk$ & $\bvkp$ & $\bfk$ & $\bsk$ \\
 \hline
 \endhead
% --------------------- every page end of table ------------------------
 K & K & K & K & K & K & K & K & K & K \kill
 \hline
 \caption[]{(continued)}\\
 \endfoot
% --------------------- only end of table ------------------------------
 K & K & K & K & K & K & K & K & K & K \kill
 \hline
 \caption{List of used H$_2$O spectral lines and their spectroscopic 
   coefficients of H$_2$O in air for the MPM models. The last 
   separated line is the unphysical pseudo-line used in MPM93. 
   The lines which are marked with a "$^+$" were not in the MPM87/MPM89 
   line catalog. The data is from \cite{liebeetal:93}. The parameters
   $b'_{3,k}$ are calculated values from Table\,3 of
   \cite{abaueretal:89} for a mixture of $\hzo-\nz$. Because of the
   missing line parameters for the 552\,GHz line in
   \cite{abaueretal:89}, the same value is assumed as for the 547\,GHz line.}
 \label{tab:mpm93linelist}
 \endlastfoot
% --------------------- body of table  ----------------------------------  
1      & 22.235080  & 0.01130 & 2.143 & 2.811 & 2.903 & 4.80 & 4.648 & 0.69 & 1.00 \\
2      & 67.803960  & 0.00012 & 8.735 & 2.858 & 3.083 & 4.93 & 4.570 & 0.69 & 0.82 \\
3      & 119.995940 & 0.00008 & 8.356 & 2.948 & 3.173 & 4.78 & 4.441 & 0.70 & 0.79 \\
4      & 183.310091 & 0.24200 & 0.668 & 3.050 & 3.173 & 5.30 & 5.095 & 0.64 & 0.85 \\
5      & 321.225644 & 0.00483 & 6.181 & 2.303 & 2.498 & 4.69 & 4.324 & 0.67 & 0.54 \\ 
6      & 325.152919 & 0.14990 & 1.540 & 2.783 & 2.993 & 4.85 & 4.509 & 0.68 & 0.74 \\
7      & 336.222601 & 0.00011 & 9.829 & 2.693 & 2.909 & 4.74 & 4.397 & 0.69 & 0.61 \\ 
8      & 380.197372 & 1.15200 & 1.048 & 2.873 & 3.083 & 5.38 & 5.014 & 0.54 & 0.89 \\
9      & 390.134508 & 0.00046 & 7.350 & 2.152 & 2.333 & 4.81 & 4.437 & 0.63 & 0.55 \\
10     & 437.346667 & 0.00650 & 5.050 & 1.845 & 1.980 & 4.23 & 3.942 & 0.60 & 0.48 \\
11     & 439.150812 & 0.09218 & 3.596 & 2.100 & 2.258 & 4.29 & 3.990 & 0.63 & 0.52 \\
12     & 443.018295 & 0.01976 & 5.050 & 1.860 & 1.995 & 4.23 & 3.944 & 0.60 & 0.50 \\
13     & 448.001075 & 1.03200 & 1.405 & 2.632 & 2.775 & 4.84 & 4.591 & 0.66 & 0.67 \\
14     & 470.888947 & 0.03297 & 3.599 & 2.152 & 2.318 & 4.57 & 4.242 & 0.66 & 0.65 \\
15     & 474.689127 & 0.12620 & 2.381 & 2.355 & 2.535 & 4.65 & 4.320 & 0.65 & 0.64 \\
16     & 488.491133 & 0.02520 & 2.853 & 2.602 & 2.805 & 5.04 & 4.675 & 0.69 & 0.72 \\
17     & 503.568532 & 0.00390 & 6.733 & 1.612 & 1.733 & 3.98 & 3.702 & 0.61 & 0.43 \\
18     & 504.482692 & 0.00130 & 6.733 & 1.612 & 1.733 & 4.01 & 3.730 & 0.61 & 0.45 \\
19$^+$ & 547.676440 & 0.97010 & 0.114 & 2.600 & 2.798 & 4.50 & 4.182 & 0.70 & 1.00 \\
20$^+$ & 552.020960 & 1.47700 & 0.114 & 2.600 & 2.798 & 4.50 & 4.182 & 0.70 & 1.00 \\
21     & 556.936002 & 48.74000& 0.159 & 3.210 & 3.443 & 4.11 & 3.832 & 0.69 & 1.00 \\
22     & 620.700807 & 0.50120 & 2.200 & 2.438 & 2.618 & 4.68 & 4.358 & 0.71 & 0.68 \\
23$^+$ & 645.866155 & 0.00713 & 8.580 & 1.800 & 1.936 & 4.00 & 3.719 & 0.60 & 0.50 \\
24     & 658.005280 & 0.03022 & 7.820 & 3.210 & 3.443 & 4.14 & 3.860 & 0.69 & 1.00 \\
25     & 752.033227 & 23.96000& 0.396 & 3.060 & 3.285 & 4.09 & 3.810 & 0.68 & 0.84 \\
26     & 841.053973 & 0.00140 & 8.180 & 1.590 & 1.755 & 5.76 & 5.218 & 0.33 & 0.45 \\
27     & 859.962313 & 0.01472 & 7.989 & 3.060 & 3.285 & 4.09 & 3.068 & 0.68 & 0.84 \\
28     & 899.306675 & 0.00605 & 7.917 & 2.985 & 3.195 & 4.53 & 4.232 & 0.68 & 0.90 \\
29     & 902.616173 & 0.00426 & 8.432 & 2.865 & 3.083 & 5.10 & 4.739 & 0.70 & 0.95 \\
30     & 906.207325 & 0.01876 & 5.111 & 2.408 & 2.550 & 4.70 & 4.438 & 0.70 & 0.53 \\
31     & 916.171582 & 0.83400 & 1.442 & 2.670 & 2.865 & 4.78 & 4.455 & 0.70 & 0.78 \\
32$^+$ & 923.118427 & 0.00869 & 10.220& 2.900 & 3.118 & 5.00 & 4.650 & 0.70 & 0.80 \\
33     & 970.315022 & 0.89720 & 1.920 & 2.550 & 2.760 & 4.94 & 4.564 & 0.64 & 0.67 \\
34     & 987.926764 & 13.21000& 0.258 & 2.985 & 3.195 & 4.55 & 4.251 & 0.68 & 0.90 \\
\hline
 & $\nu^*$ & $\beks$ & $\bzks$ & $\bdks$ & $\bdkps$ & $\bvks$ &
 $\bvkps$ & $\bfks$ & $\bsks$\\
 & {\sf [GHz]}  & {[$\frac{\sf kHz}{\sf hPa}$]} & {\sf [1]} & 
 {[$\frac{\sf MHz}{\sf hPa}$]} & {[$\frac{\sf MHz}{\sf hPa}$]} & 
 {\sf [1]} & {\sf [1]} & {\sf [1]} & {\sf [1]} \\
\hline
 & 1780.000000 & 2230.00000 & 0.952 & 17.620 & 19.021 & 30.50 & 28.254 & 2.00 & 5.00 \\
% -----------------------------------------------------------------------  
\end{longtable}
%
%
\levele{The MPM93 Continuum Parameterization:}
\label{levele:mpm93:contabs}
%-----------------------------------------------
Liebe used measured total absorption coefficient data to fit the 
parameters of pseudo water vapor line ($\beks$--$\bsks$). 
The fitted line center frequency for this pseudo-line is at 
1.780 THz and therefore out of the valid frequency region of MPM 
itself. Table~\ref{tab:mpm93linelist} lists the values of this set 
of parameters. It is remarkable that all these parameters are much 
larger compared to physical lines, except for $\bzks$, the parameter 
which governs the exponential temperature behavior of the line strength:
\begin{eqnarray}
  \label{eq:mpm93:contabs}
  \alphampmnc & = & 0.1820 \cdot \frac{\beks}{\nucc} \cdot \phzo \cdot 
                \Theta^{3.5} \cdot e^{\bzks\cdot(1-\Theta)} \cdot 
                \nu^2 \cdot F_c(\nu)\\
%
  \label{eq:mpm93:contabs1}
  F_c(\nu) & = & \left[\frac{\gamc}{(\nucc+\nu)^2+\gamc^2} + 
                       \frac{\gamc}{(\nucc-\nu)^2+\gamc^2}\right]\\
%
  \label{eq:mpm93:contabs2}
  \gamc & = &  \bdks \cdot 
          \left(\bvks \cdot \phzo \cdot \Theta^{\bsks}~~+~~ 
                            \pda  \cdot \Theta^{\bfks}\right)\\
\nonumber
\end{eqnarray}
The magnitude of the pseudo-line width is shown for four different 
cases in Table\,\ref{tab:mpm_psl_broad}. With respect to a frequency
in the microwave range, the line shape function, $F_c(\nu)$, can be 
approximated by
\begin{equation}
 \label{eq:mpm93:contfapp1}
 F_c(\nu) \approx 2 \cdot \frac{\gamc}{\nucc^2}
\end{equation}
Inserting Eq.~(\ref{eq:mpm93:contfapp1}) into Eq.~(\ref{eq:mpm93:contabs})
shows the quadratic frequency dependence of the MPM93 continuum
similar to the continuum parameterization expressed in MPM89 and earlier 
versions. By additionally approximating the temperature dependence to the 
simple form
\begin{eqnarray}
  \xs\cdot\ln{(\Theta)} & = & 
  \ln{\left(\Theta^{3.5} \cdot e^{\bzks\cdot(1-\Theta)}\right)}\nonumber\\
%
  \xs  & = & 3.5 +
  \bzks\cdot\frac{1-\Theta}{\ln{(\Theta)}}\nonumber\\
%
  \xs & \approx & 3.5 -\bzks = 2.55\hspace*{5mm}\mbox{with}\hspace*{2mm}
                 \ln{(\Theta)}\approx(\Theta-1)\\
\nonumber
\end{eqnarray}
one can rearrange the pseudo-line continuum to fit Eq.~(\ref{eq:mpm87:cont})
(denoted by MPM93$^*$). The deduced continuum parameter set is:
\begin{tabular}{lcl}
 $\cso$  & = & 7.73$\cdot$10$^{-8}$~dB/km/hPa$^2$~GHz$^2$\\
 $\xs$   & = & 7.55\\
 $\cdo$  & = & 0.253$\cdot$10$^{-8}$~dB/km/hPa$^2$~GHz$^2$\\ 
 $\xd$   & = & 4.55 \\
\end{tabular}
The MPM93$^*$ continuum parameters $\cso$ and $\cdo$ are 20\,\% and 
15\,\% larger, respectively, than in the case of MPM87/MPM89. 
Compared with the R98 model, $\cso$ is 1\,\% smaller and
$\cdo$ is 7\,\% increased. A very good agreement is between CP98 and
MPM93$^*$ in the case of $\cdo$. Large discrepancies exist for the
temperature exponents $\xs$ and $\xd$ between MPM93$^*$ and earlier
versions of the same model. The exponent $\xs$ is in MPM93$^*$ only
60\,\% of the corresponding value in MPM89 and the temperature
dependence of the $\hzo$-air term is significant larger than for
earlier MPM versions. This reduction of $\xs$ is mainly due to 
additional measurements considered in MPM93 
(Refs. \cite{beckerautler:46,godonetal:92}), while the continuum 
parameters in MPM87/MPM89 are determined by a single laboratory 
measurement at 138\,GHz.
%
\begin{table}[!htb]
\begin{center}
\begin{tabular}{lrrr}
\hline
 & \multicolumn{2}{c}{contribution} & \multicolumn{1}{c}{total} \\
 & \multicolumn{1}{c}{$\hzo$--$\hzo$} & \multicolumn{1}{c}{$\hzo$--air} & \\
\hline
$\gamc$(200\,K) & 40.8\,GHz & 80.4\,GHz & 121.2\,GHz\\
$\gamc$(300\,K) &  5.4\,GHz & 23.0\,GHz &  28.4\,GHz\\
\hline
\end{tabular}
\caption{Magnitude of the line width of the pseudo-line of the
  continuum term in MPM93. Assumed is a total pressure of 1000\,hPa
  and a water vapor partial pressure of 10\,hPa.}
\label{tab:mpm_psl_broad}
\end{center}
\end{table} 
%
%
\levele{ARTS Workspace Variables and Methods for H2O-MPM93}
% ---------------------------------------------------------
\begin{center}
\begin{tabular}{ll}
\hline
\multicolumn{2}{c}{ARTS tag definition}\\
name      & comment \\
H2O-MPM93 & MPM93 is defined as {\it isotope} of $\hzo$\\
\hline
\multicolumn{2}{c}{ARTS workspace variables}\\
name & comment \\
 f\_mono & input frequency grid \\
 p\_abs  & input pressure grid \\
 t\_abs  & input temperature grid\\
 vmr    & input $\hzo$ volume mixing ratio\\
\hline
\multicolumn{2}{c}{ARTS methods}\\
function           & method\\
xsec\_continuum\_tag & called by absCalc \\
\hline
\multicolumn{2}{c}{calling tree}\\
MPM93H2OAbsModel & $\Leftarrow$~xsec\_continuum\_tag~$\Leftarrow$~absCalc\\
\end{tabular}
\end{center}

\levele{ARTS control file:}
% -------------------------

\begin{verbatim}
#
tgsDefine{
      [ 
        "N2",
        "H2O-MPM93"
      ] 
}
#
\end{verbatim}
$\vdots$
\begin{verbatim}
#
# MPM93 H2O absorption model (lines + continuum)
cont_descriptionAppend{
    name       = "H2O-MPM93"
    parameters = [ ]
}
#
\end{verbatim}
$\vdots$
\begin{verbatim}
# Set the physical H2O profile from the H2O profile in vmrs:
h2o_absSet{}

# Set the physical N2 profile from the N2 profile in vmrs:
n2_absSet{}
\end{verbatim}
$\vdots$
%
%
% ============================================================================
%
% 
\leveld{H2O-CP98}
\label{leveld:cp98}
% ================
%
\levele{Line Absorption}
\label{levele:cp98_line}
%-----------------------
The water vapor absorption module of CP98 \cite{cruzpol:98} is based on 
MPM87 with the main difference that the line catalog consists of only a 
single line at $\nu_{\sf o}=$\,22\,GHz. 
The contributions from the other lines is put into the water vapor 
continuum part. The line absorption (in units of Neper/km) is therefore 
very quickly calculated according to the formula
\begin{eqnarray}
  \label{eq:cp98:lineabs}
  \alphacpl &=& 0.0419 \cdot \nu^2 \cdot T_L \cdot T_S\hspace*{10mm}\mbox{Np/km}\\
  \mbox{with} & & \nonumber\\
  T_L       &=& 0.0109 \cdot C_L \cdot \phzo \cdot \Theta^{3.5} 
                \cdot \exp{(2.143\cdot[1-\Theta])}\nonumber\\
  T_S       &=& \frac{1}{\nu_{\sf o}} \cdot 
                \left[\frac{\gamma}{(\nu-\nu_{\sf o})^2+\gamma^2}~+~
                \frac{\gamma}{(\nu+\nu_{\sf o})^2+\gamma^2}\right]\nonumber\\
  \gamma    &=& 0.002784 \cdot C_W \cdot (\pda \cdot \Theta^{0.6}+ 
                4.8 \cdot \phzo \cdot \Theta^{1.1}) \nonumber\\
\end{eqnarray}
where $\phzo$ and $\pda$ are the partial pressure of water vapor and dry
air in units of hPa, respectively. The numbers correspond to the line
parameters form MPM87 for this special line and the factors  $C_L$ and $C_W$
are adjustable scaling factors to match the model with the
measurements. Best agreement is obtained with 
$C_L=$\,1.0639$\pm$0.016 and $C_W=$\,1.0658$\pm$0.0096. The
correlation between these two scaling factors was found to be
negligible.

The main reason why the Cruz-Pol model (CP98) considers only one line
lies in the fact that CP98 is especially designed for the data analysis
in the 22\,GHz region. The determination of the scaling factors was therefore
performed with ground based radiometer data in the frequency range of
22 to 32 \,GHz from different locations\footnote{The data were
  recorded at San Diego, California (11. December 1991) and West Pal
  Beach, Florida (8.-21. March 1992)} in the USA.
%
\levele{Water Vapor Continuum Absorption:}
\label{levele:cp98_cont}
%-----------------------------------------
The continuum absorption coefficient in CP98, $\alphampmmc$, 
is the same as in MPM87/MPM89 (see Sec. \ref{levele:h2o_mpm87_cont} for 
details) but multiplied with a constant factor of $C_c$=1.2369$\pm$0.155 determined 
from the above stated measurements:\\
\begin{eqnarray} 
  \label{eq:cp98:cont}
  \alphacpc &=& \nu^2 \cdot \phzo \cdot 
                (\cso \cdot \phzo \cdot \Theta^{\xs} + 
                 \cdo \cdot \pda  \cdot \Theta^{\xf})\\
            &=& C_c \cdot \alphampmoc
\nonumber
\end{eqnarray}
with\\
\begin{tabular}{lcl}
$\cso$   & = & 8.04$\cdot$10$^{-8}$(dB/km)/(hPa$\cdot$GHz)$^2$\\
$\xs$    & = & 10.5\\
$\cdo$   & = & 0.254$\cdot$10$^{-8}$(dB/km)/(hPa$\cdot$GHz)$^2$\\
$\xd$    & = & 3.0\\
\end{tabular}
%
%
\levele{ARTS Workspace Variables and Methods for H2O-MPM93}
% ---------------------------------------------------------
\begin{center}
\begin{tabular}{ll}
\hline
\multicolumn{2}{c}{ARTS tag definition}\\
name      & comment \\
H2O-MPM93 & MPM93 is defined as {\it isotope} of $\hzo$\\
\hline
\multicolumn{2}{c}{ARTS workspace variables}\\
name & comment \\
 f\_mono & input frequency grid \\
 p\_abs  & input pressure grid \\
 t\_abs  & input temperature grid\\
 vmr    & input $\hzo$ volume mixing ratio\\
\hline
\multicolumn{2}{c}{ARTS methods}\\
function           & method\\
xsec\_continuum\_tag & called by absCalc \\
\hline
\multicolumn{2}{c}{calling tree}\\
MPM93H2OAbsModel & $\Leftarrow$~xsec\_continuum\_tag~$\Leftarrow$~absCalc\\
\end{tabular}
\end{center}

\levele{ARTS control file:}
% -------------------------

\begin{verbatim}
#
tgsDefine{
      [ 
        "N2",
        "H2O-MPM93"
      ] 
}
#
\end{verbatim}
$\vdots$
\begin{verbatim}
#
# MPM93 H2O absorption model (lines + continuum)
cont_descriptionAppend{
    name       = "H2O-MPM93"
    parameters = [ ]
}
#
\end{verbatim}
$\vdots$
\begin{verbatim}
# Set the physical H2O profile from the H2O profile in vmrs:
h2o_absSet{}

# Set the physical N2 profile from the N2 profile in vmrs:
n2_absSet{}
\end{verbatim}
$\vdots$
%
%
% ============================================================================
%
% 
\leveld{H2O-PWR98}
\label{leveld:pwr98}
% ==================
The water vapor continuum formulation of \citet{pwr:98} is a re-investigation 
of the existing models MPM87/MPM89, MPM93, and CKD\_2.1 especially for 
the frequency region below 1-1000\,GHz. in the context of the available
laboratory and atmospheric data \citep{abaueretal:89, abaueretal:93, 
abaueretal:95, beckerautler:46, englishetal:94, godonetal:92,
liebe:84, liebeandlayton:87, westwateretal:90}.

Rosenkranz adopted the structure of MPM89 for his improved model (R98). 
However, some important differences exist compared with MPM89:
\begin{itemize}
\item the water vapor line catalogs are different 
\item the R98 uses the Van~Vleck--Weisskopf line shapefunction with 
      cutoff and MPM89 without cutoff
\end{itemize}

\levele{Water Vapor Line Absorption:}
\label{levele:pwr98_line}
%------------------------
The local line absorption is defined as 
\begin{eqnarray} 
 \label{eq:r98absline}
 \alphapwrl &=& N_{H_2O} \cdot \sum_k \inten \cdot \shapec \nonumber\\
            &=& N_{H_2O} \cdot \sum_k \inten \cdot 
                \left [\shapefp + \shapefm \right]\hspace*{10mm}\mbox{Np/km}
\end{eqnarray}
where $N_{H_2O}$ is the number density of water molecules, $\nu$ the
frequency and $S$ the line intensity, calculated from the HITRAN92
data base \citet{rothman:92}. Considered for this re-investigation are 
15 lines with a frequency lower than 1\,THz as listed in 
Table~\ref{tab:r98linelist}.

The line shape function $\shapec$ has a cutoff frequency, $\nucut$,
and a baseline subtraction similar to the CKD model \cite{clough:89}.
The introduction of a cutoff frequency has two advantages: (1) the
cutoff avoids applying the line shape to distant frequencies where the 
line form is theoretically not well understood and (2) the cutoff also
establishes a limit to the summation in Eq.~(\ref{eq:r98absline}) where lines
far away from the cutoff limit do not contribute to the sum.  
The Rosenkranz formulation uses the same value for
the cutoff frequency as the CKD model:
\begin{equation} 
 \label{cutoff}
 \nucut = 750\mbox{ GHz}
\end{equation}
%
The explicit mathematical form of the line shape function is defined 
in such a way that in the limit $\nucut \rightarrow \infty$ the 
combination of Eq.~(\ref{eq:r98absline}) with the line shape function would 
be equivalent to a Van Vleck--Weisskopf \citep{vanvleck:45} line shape: 
\begin{equation}
 \label{eq:r98lineshape}
 \hspace*{-5mm}\shapefpm = 
   \left \{ \begin{array}{r@{\quad:\quad}l} 
   \left (\displaystyle{\frac{\nu}{\nuk}}\right )^2 
   \displaystyle{\frac{\gamk}{\pi}} 
   \left \{ \displaystyle{\frac{1}{(\nu \mp \nuk)^2 + \gamk^2}} - 
   \displaystyle{\frac{1}{\nucut^2 + \gamk^2}} \right \}
   & |\nu \pm \nuk| < \nucut \\ 
   0 & |\nu \pm \nuk| \geq \nucut
                       \end{array} \right.
\end{equation}
$\nuk$ is the line center frequency and $\gamk$ the line
half width, which is calculated according to 
\begin{equation}
 \label{eq:r98gamma}
 \gamk = \ws \cdot \phzo \cdot \Theta^{\xs} + 
         \wf \cdot \pda  \cdot \Theta^{\xf}\hspace*{10mm}\mbox{GHz}
\end{equation}
with $\phzo$ and $\pda$ as the partial pressure of water vapor and of 
dry air, respectively. The line depending parameters $\ws$, $\xs$, 
$\wf$, and $\xf$ are listed in Table~\ref{tab:r98linelist} and the 
dimensionless parameter $\Theta$ is defined as $\Theta$\,=\,300\,K/$T$.

Because of the structural similarity to MPM89, the line broadening 
parameters differ only in minor respects from the values used therein 
(only the parameters $x_{\sf s,1}$, $w_{\sf f,2}$ and $\sf w_{\sf s,2}$ 
are significantly different).
%
\begin{table}[!htb]
\begin{center}
\begin{tabular}{rrrrrr}
 \hline
 index &  $\nuk$      & $\wf$     & $\xf$ & $\ws$     & $\xs$ \\
   k   &  [GHz]       & [GHz/kPa] & [1]   & [GHz/kPa] & [1] \\ 
 \hline
   1   &   22.2351    & 0.00281   & 0.69  & 0.01349   &  0.61 \\
   2   &  183.3101    & 0.00281   & 0.64  & 0.01491   &  0.85 \\
   3   &  321.2256    & 0.00230   & 0.67  & 0.01080   &  0.54 \\
  4    &  325.1529    & 0.00278   & 0.68  & 0.01350   &  0.74 \\
  5    &  380.1974    & 0.00287   & 0.54  & 0.01541   &  0.89 \\
  6    &  439.1508    & 0.00210   & 0.63  & 0.00900   &  0.52 \\
  7    &  443.0183    & 0.00186   & 0.60  & 0.00788   &  0.50 \\
  8    &  448.0011    & 0.00263   & 0.66  & 0.01275   &  0.67 \\
  9    &  470.8890    & 0.00215   & 0.66  & 0.00983   &  0.65 \\
  10   &  474.6891    & 0.00236   & 0.65  & 0.01095   &  0.64 \\
  11   &  488.4911    & 0.00260   & 0.69  & 0.01313   &  0.72 \\
  12   &  556.9360    & 0.00321   & 0.69  & 0.01320   &  1.00 \\
  13   &  620.7008    & 0.00244   & 0.71  & 0.01140   &  0.68 \\
  14   &  752.0332    & 0.00306   & 0.68  & 0.01253   &  0.84 \\
  15   &  916.1712    & 0.00267   & 0.70  & 0.01275   &  0.78 \\
  \hline
\end{tabular}
\end{center}
  \caption{Line parameters of the Rosenkranz absorption model (R98) 
  (values taken from \citet{pwr:98}).}
\label{tab:r98linelist}
\end{table}
%
%
\levele{Water Vapor Continuum Absorption:}
\label{levele:pwr98_cont}
%----------------------------
The continuum absorption in R98 has the same functional dependence on frequency,
presure, and temperature like in MPM87/MPM89 (see Sec. \ref{levele:h2o_mpm87_cont}
for details):
\begin{equation} 
  \label{eq:r98:cont}
  \alphapwrc = \nu^2 \cdot \phzo \cdot 
               (\cso \cdot \phzo \cdot \Theta^{\xs} + 
                \cdo \cdot \pda  \cdot \Theta^{\xf})
\end{equation}
with\\
\begin{tabular}{lcl}
$\cso$   & = & 7.80$\cdot$10$^{-8}$(dB/km)/(hPa$\cdot$GHz)$^2$\\
$\xs$    & = & 7.5\\
$\cdo$   & = & 0.236$\cdot$10$^{-8}$(dB/km)/(hPa$\cdot$GHz)$^2$\\
$\xd$    & = & 3.0\\
\end{tabular}
The main difference to the MPM versions are the values of these 
parameters, since Rosenkranz used additional data to fit his set of 
parameters. A second point is the cutoff in the line shape of the line 
absorption calculation. Since this cutoff decreases the line absorption 
in the window regions, the continuum absorption tends to compensate this 
decrease to get the same total absorption as withouot cutoff. This effects 
mainly the parameters $\cso$ and $\cdo$ but has also an influence in the 
temperature dependence and therefore on $\xs$ and $\xd$.
%
%
\levele{ARTS Workspace Variables and Methods for H2O-R98}
% ---------------------------------------------------------
\begin{center}
\begin{tabular}{ll}
\hline
\multicolumn{2}{c}{ARTS tag definition}\\
name      & comment \\
H2O-MPM93 & MPM93 is defined as {\it isotope} of $\hzo$\\
\hline
\multicolumn{2}{c}{ARTS workspace variables}\\
name & comment \\
 f\_mono & input frequency grid \\
 p\_abs  & input pressure grid \\
 t\_abs  & input temperature grid\\
 vmr    & input $\hzo$ volume mixing ratio\\
\hline
\multicolumn{2}{c}{ARTS methods}\\
function           & method\\
xsec\_continuum\_tag & called by absCalc \\
\hline
\multicolumn{2}{c}{calling tree}\\
MPM93H2OAbsModel & $\Leftarrow$~xsec\_continuum\_tag~$\Leftarrow$~absCalc\\
\end{tabular}
\end{center}

\levele{ARTS control file:}
% -------------------------

\begin{verbatim}
#
tgsDefine{
      [ 
        "N2",
        "H2O-MPM93"
      ] 
}
#
\end{verbatim}
$\vdots$
\begin{verbatim}
#
# MPM93 H2O absorption model (lines + continuum)
cont_descriptionAppend{
    name       = "H2O-MPM93"
    parameters = [ ]
}
#
\end{verbatim}
$\vdots$
\begin{verbatim}
# Set the physical H2O profile from the H2O profile in vmrs:
h2o_absSet{}

# Set the physical N2 profile from the N2 profile in vmrs:
n2_absSet{}
\end{verbatim}
$\vdots$
%
%
% ============================================================================
%
%
\leveld{Suspended Cloud Liquid Water and Ice Particles}
\label{subsec:lipartabs}
%------------------------------------------------------
% 
The MPM93 model provides beside the absorption model of air also 
an absorption model for suspended liquid water droplets and ice 
particles \cite{liebe:89b,liebeetal:91,hufford:91,liebeetal:93}. 
The model is applicable for the Rayleigh regime, for which the
relation  
\begin{equation}
 \label{eq:MPM93ralrel}
 r < \frac{\lambda}{20}
\end{equation} 
between the radius of the suspended particle, $r$, and the 
wavelength, $\lambda$ holds (See Ref. \cite{brussaard:95}, 
page 81 for details). For example the particle radius must be 
smaler than 750\,$\mu$m for a frequenxy of 20\,GHz and smaler than 
23\,$\mu$m for $\nu$=650\,GHz. Considering Salby \cite{salby:96}, 
this criterium is for $\nu$=20\,GHz nearly for every aerosol and 
cloud class -- except cirrus -- satisfied. But one has to bear in mind
that these values have a wide range of variability, e.~g. Salby
state for the mean particle radius for stratus, cumulus, and nimbus
clouds a range of 10-1000\,$\mu$m and that the particle radius
distribution is highly asymmetric.

With respect to the imaginary part of the complex refractivity, a 
unified parameterization of liquid and ice particle absorption 
is formulated:
\begin{eqnarray}
  \alpha & = & 0.1820 \cdot \nu \cdot\imn\hspace*{20mm}
               \mbox{dB/km}\\
    \imn & = & \frac{3}{2} \cdot \frac{w}{m} \cdot 
               \left[\frac{3\cdot
                   \ime}{(\ree+2)^2~+~(\ime)^2}\right]\nonumber\\
  \nonumber
\end{eqnarray}
%
The difference between liquid water droplets and ice particles is
put in the expressions for the complex permittivities
(i. e. the relative dielectric constant) which depend on frequency and
temperature:\\
$\bullet$ complex permittivity for suspended liquid water droplets:
\begin{eqnarray}
  \ree       & = & \epsilon_o~-~\nu^2\cdot
                   \left[\frac{\epsilon_o-\epsilon_1}{\nu^2+\gamma_1^2}~+~
                   \frac{\epsilon_1-\epsilon_2}{\nu^2+\gamma_1^2}\right]\nonumber\\
%
  \ime       & = & \nu\cdot\left[\gamma_1\cdot
                   \frac{\epsilon_o-\epsilon_1}{\nu^2+\gamma_1^2}~+~
                   \gamma_2\cdot\frac{\epsilon_1-\epsilon_2}{\nu^2+\gamma_1^2}\right]\nonumber\\
%
  \epsilon_o & = & 77.66 + 103.3\cdot(\Theta-1)\nonumber\\
  \epsilon_1 & = & 0.0671\cdot\epsilon_o\nonumber\\
  \epsilon_2 & = & 3.52\nonumber\\
%
  \gamma_1   & = & 20.20 - 146\cdot(\Theta-1) + 316\cdot(\Theta-1)^2\,\mbox{GHz}\nonumber\\
  \gamma_2   & = & 39.8\cdot\gamma_1\,\mbox{GHz}\nonumber\\
%
  \Theta     & = & 300\,\mbox{K}~/~T\nonumber\\
%
  w          & = & 0.0~\mbox{to}~5.0~\mbox{g}/\mbox{m}^3 \nonumber\\
  m          & = & 1.0~\mbox{g}/\mbox{cm}^3\nonumber\\
  \nonumber
\end{eqnarray}
$\bullet$ complex permittivity for ice crystals the parameterization:
\begin{eqnarray}
  \ree    & = & 3.15\nonumber\\
%
  \ime    & = & \frac{a}{\nu}~+~b\cdot\nu\nonumber\\
%
  a       & = & (\Theta-0.171)\cdot\exp{(17.0-22.1\cdot\Theta)}\nonumber\\
  b       & = & \left[\left(\frac{0.233}{1-0.993/\Theta}\right)^2 + 
                \frac{6.33}{\Theta} - 1.31\right]\cdot10^{-5}\nonumber\\
%
  \Theta  & = & 300\,\mbox{K}~/~T\nonumber\\
%
  w       & = & 0.0~\mbox{to}~1.0~\mbox{g}/\mbox{m}^3 \nonumber\\
  m       & = & 0.916~\mbox{g}/\mbox{cm}^3\nonumber\\
  \nonumber
\end{eqnarray}
%
The absorption is direct proportional to the liquid or ice water
content $w$ and inversely proportional the the density of a single
particle $m$. Like the mean particle radius, the liquid and ice water
content have a high variability. This variability is also reflected in
the literature, Table \ref{tab:lwc} summarizes some values without the
aim of completeness.
\begin{table}[!htb]
\begin{center}
\begin{tabular}{lllll}
\hline
              &       & \multicolumn{2}{c}{water content} & \\
 cloud        & class & \multicolumn{1}{c}{liquid} & \multicolumn{1}{c}{ice} & reference \\
              &       & \multicolumn{1}{c}{(g/m$^3$)} & \multicolumn{1}{c}{(g/m$^3$)} & \\
\hline
 stratus      & St    & 0.15         &    & Salby \cite{salby:96}\\
              &       & 0.09--0.9    &    & Seinfeld and Pandis \cite{seinfeld:98}\\
              &       & 0.28--0.3    &    & Hess et al. \cite{hess:98}\\
              &       & 0.29         &    & Kneizys et al. \cite{abreu:96}\\
 nimbostratus & Ns    & 0.4          &    & Salby \cite{salby:96}\\
              &       & 0.65         &    & Kneizys et al. \cite{abreu:96}\\
 altostratus  & As    & $<$0.01--0.2 &    & Seinfeld and Pandis \cite{seinfeld:98}\\
              &       & 0.41         &    & Kneizys et al. \cite{abreu:96}\\
 stratocumulus& Sc    & 0.3          &    & Salby \cite{salby:96}\\
              &       & $<$0.1--0.7  &    & Seinfeld and Pandis \cite{seinfeld:98}\\
              &       & 0.15         &    & Kneizys et al. \cite{abreu:96}\\
 cumulus      & Cu    & 0.5          &    & Salby \cite{salby:96}\\
              &       & 0.26--0.44   &    & Hess et al. \cite{hess:98}\\
              &       & 1.00         &    & Kneizys et al. \cite{abreu:96}\\
 cumulonimbus & Cb    & 2.5          &    & Salby \cite{salby:96}\\
 Cirrus       & Ci    &              & 0.025 & Salby \cite{salby:96}\\
              &       &              & 0.00193--0.0260 & Hess et al. \cite{hess:98}\\
              &       &              & 3.128$\cdot$10$^{-4}$--0.06405 & Kneizys et al. \cite{abreu:96}\\
 trop. Cirrus & Cs    &              & 0.2   & Salby \cite{salby:96}\\
\hline
\end{tabular}
\caption{Stated values for the liquid and ice water content of several 
  cloud classes from different sources.}
\label{tab:lwc}
\end{center}
\end{table}
%
%
%
\levele{ARTS Workspace Variables and Methods for H2O-MPM93}
% ---------------------------------------------------------
\begin{center}
\begin{tabular}{ll}
\hline
\multicolumn{2}{c}{ARTS tag definition}\\
name      & comment \\
H2O-MPM93 & MPM93 is defined as {\it isotope} of $\hzo$\\
\hline
\multicolumn{2}{c}{ARTS workspace variables}\\
name & comment \\
 f\_mono & input frequency grid \\
 p\_abs  & input pressure grid \\
 t\_abs  & input temperature grid\\
 vmr    & input $\hzo$ volume mixing ratio\\
\hline
\multicolumn{2}{c}{ARTS methods}\\
function           & method\\
xsec\_continuum\_tag & called by absCalc \\
\hline
\multicolumn{2}{c}{calling tree}\\
MPM93H2OAbsModel & $\Leftarrow$~xsec\_continuum\_tag~$\Leftarrow$~absCalc\\
\end{tabular}
\end{center}

\levele{ARTS control file:}
% -------------------------

\begin{verbatim}
#
tgsDefine{
      [ 
        "N2",
        "H2O-MPM93"
      ] 
}
#
\end{verbatim}
$\vdots$
\begin{verbatim}
#
# MPM93 H2O absorption model (lines + continuum)
cont_descriptionAppend{
    name       = "H2O-MPM93"
    parameters = [ ]
}
#
\end{verbatim}
$\vdots$
\begin{verbatim}
# Set the physical H2O profile from the H2O profile in vmrs:
h2o_absSet{}

# Set the physical N2 profile from the N2 profile in vmrs:
n2_absSet{}
\end{verbatim}
$\vdots$
%
%
% ============================================================================
%
%
\levelc{Complete Oxygen Models}
\label{levelc:02_models}
% =====================
%
Since the Maxwell equations are symmetric in the electric and
magnetic fields, electric as well as magnetic dipole transitions 
are both possible although magnetic dipoles are in general some
orders of magnitudes weaker and therefore not relevant in
atmospheric radiative transfer models. An exception to this is the complex 
around 60\,GHz of the paramagnetic oxygen magnetic dipole transitions. 
This bulk of lines arise due to the fact that for rotational 
quantum numbers $K>1$ the allowed transitions \mbox{$\Delta J = \pm$1} 
have an energy gap of approximately 60\,GHz.\\
The most frequently used absorption model for this absorption effect is that of
Liebe, Rosenkranz, and Hufford \cite{liebeetal:92} (also reported in 
\cite{pwr:93} with a slightly different parameterization).

For oxygen -- like for water vapor -- the total absorption 
($\alphatot$) is modelled as the line absorption ($\alphal$) plus a  
continuum absorption ($\alphac$):
\begin{equation}
  \label{eq:o2:totabs}
  \alphatot = \alphal + \alphac
\end{equation}
It has to be emphasized that, $\alphal$ and $\alphac$ of different
models are not necessarily compatible and should therefore not be interchanged.
%
%
% ============================================================================
%
%
\leveld{O2-PWR93}
\label{leveld:02_pwr98}
% =====================
%
\levele{Resonant Oxygen Absorption}
\label{levele:02_pwr98_line}
%----------------------------------
The oxygen absorption model of Rosenkranz is described in \cite{pwr:93}. It 
is based on the investigations made by Liebe, Rosenkranz, and Hufford 
\cite{liebeetal:92}. The FORTRAN77 computer programme of Rosnekranz for 
the $\oz$ absorption calculation can be downloaded via anonymous ftp from 
mesa.mit.edu/phil/lbl\_rt.
%
\levele{Oxygen Line Absorption:}
\label{levele:o2_r93_lines}
%-------------------------------
The oxygen line catalog has 40 lines from which 33 lines build the 
complex around 60\,GHz. The parameterization of the line absorption,
$\alphapwrl$, is:
\begin{eqnarray}
% line ansorption:
  \alphapwrl & = & \frac{n_{\sf O_2}}{\pi} \cdot 
                   \sum_{k=1}^{40}{S_k(T) \cdot F(\nu,\nu_k)}\\
%
% \mbox{with} &   &\nonumber\\
%
% line intensity:
 \mbox{line intensity:} & & \nonumber\\
      S_k(T) & = & S_k(300\,{\sf K})~~/~~\exp{(b_k \cdot \Theta)}\nonumber\\
% line shape:
 \mbox{line shape function:} & & \nonumber\\
F(\nu,\nu_k) & = & \left(\frac{\nu}{\nuk}\right)^2 \cdot 
                   \left[\frac{\Gamma_k+(\nu-\nuk)\cdot Y_k}
                              {(\nu-\nu_k)^2+\Gamma_k^2}~~+~~
                         \frac{\Gamma_k-(\nu+\nuk)\cdot Y_k}
                              {(\nu+\nu_k)^2+\Gamma_k^2}\right] \nonumber\\
% line width:
 \mbox{line width:} & & \nonumber\\
    \Gamma_k & = & w_k \cdot \left(          \pda  \cdot \Theta^{0.8} + 
                                   1.1 \cdot \phzo \cdot \Theta \right) \nonumber\\
% line coupling:
 \mbox{line coupling:} & & \nonumber\\
         Y_k & = & \pdair \cdot \Theta^{0.8} \cdot 
                   \left[ y_k + (\Theta-1) \cdot v_k \right]\nonumber\\
% O2 number density:
 \mbox{number density of $\oz$:} & & \nonumber\\
           n_{\sf O_2} & = & (0.20946 \cdot \pdair)/(k_B \cdot T)\nonumber\\
           \nonumber
\end{eqnarray}
%
where $Q_{e,v}$ are the electronic and vibrational partition functions, 
respectively, and $S_k(300\,{\sf K})$ denotes the reference line
intensity. All model parameters are tabulated in Table \ref{tab:o2_pwrlinelist}
Refs. \cite{pwr:93,liebeetal:92}.
%
\begin{longtable}{|l||r|r|r|r|r|r|}
 K & K & K & K & K & K & K \kill
%
% --------------------- only begin of table ------------------------------
 \hline
 index & 
 $\nuk$ & 
 $S_k(300\,{\sf K})$ & 
 $b_k$ & 
 $w_k$  & 
 $y_k$ & 
 $v_k$ \\
 $k$   & 
 {\sf [GHz]}  & 
 {\sf [cm$^2$\,Hz]} & 
 {\sf [1]} & 
 {[$\frac{\sf MHz}{\sf hPa}$]} & 
 {[$\frac{\sf 10{^{-3}}}{\sf hPa}$]} & 
 {[$\frac{\sf 10{^{-3}}}{\sf hPa}$]} \\
 \hline
 \hline
 \endfirsthead
% --------------------- every page begin of table ------------------------
 \hline
 index & 
 $\nuk$ & 
 $S_k(300\,{\sf K})$ & 
 $b_k$ & 
 $w_k$  & 
 $y_k$ & 
 $v_k$ \\
 \hline
 \hline
 \endhead
% --------------------- every page end of table ------------------------
 K & K & K & K & K & K & K \kill
 \hline
 \caption[]{(continued)}\\
 \endfoot
% --------------------- only end of table ------------------------------
 K & K & K & K & K & K & K \kill
 \hline
 \caption{List of $\oz$ spectral lines of the Rosenkranz absorption 
          model \cite{pwr:93}.}
 \label{tab:pwr02line}
 \endlastfoot
% --------------------- body of table ----------------------------------  
1  & 118.7503  & .2936$\cdot$\,10$^{-14}$ & .009 & 1.63 & -0.0233 & 0.0079 \\
2  & 56.2648 & .8079$\cdot$\,10$^{-15}$ & .015 & 1.646 & 0.2408 & -0.0978 \\
3  & 62.4863 & .2480$\cdot$\,10$^{-14}$ & .083 & 1.468 & -0.3486 &  0.0844 \\
4  & 58.4466 & .2228$\cdot$\,10$^{-14}$ & .084 & 1.449 & 0.5227 & -0.1273 \\
5  & 60.3061 & .3351$\cdot$\,10$^{-14}$ & .212 & 1.382 & -0.5430 & 0.0699 \\
6  & 59.5910 & .3292$\cdot$\,10$^{-14}$ & .212 & 1.360 & 0.5877 & -0.0776 \\
7  & 59.1642 & .3721$\cdot$\,10$^{-14}$ & .391 & 1.319 & -0.3970 & 0.2309 \\
8  & 60.4348 & .3891$\cdot$\,10$^{-14}$ & .391 & 1.297 & 0.3237 & -0.2825 \\
9  & 58.3239 & .3640$\cdot$\,10$^{-14}$ & .626 & 1.266 & -0.1348 &  0.0436 \\
10 & 61.1506 & .4005$\cdot$\,10$^{-14}$ & .626 & 1.248 & 0.0311 & -0.0584 \\
11 & 57.6125 & .3227$\cdot$\,10$^{-14}$ & .915 & 1.221 & 0.0725 & 0.6056 \\
12 & 61.8002 & .3715$\cdot$\,10$^{-14}$ & .915 & 1.207 & -0.1663 & -0.6619 \\
13 & 56.9682 & .2627$\cdot$\,10$^{-14}$ & 1.260 & 1.181 & 0.2832 & 0.6451 \\
14 & 62.4112 & .3156$\cdot$\,10$^{-14}$ & 1.260 & 1.171 & -0.3629 & -0.6759 \\
15 & 56.3634 & .1982$\cdot$\,10$^{-14}$ & 1.660 & 1.144 & 0.3970 &  0.6547 \\
16 & 62.9980 & .2477$\cdot$\,10$^{-14}$ & 1.665 & 1.139 & -0.4599 & -0.6675 \\
17 & 55.7838 & .1391$\cdot$\,10$^{-14}$ & 2.119 & 1.110 & 0.4695 & 0.6135 \\
18 & 63.5685 & .1808$\cdot$\,10$^{-14}$ & 2.115 & 1.108 & -0.5199 & -0.6139 \\
19 & 55.2214 & .9124$\cdot$\,10$^{-15}$ & 2.624 & 1.079 & 0.5187 & 0.2952 \\
20 & 64.1278 & .1230$\cdot$\,10$^{-14}$ & 2.625 & 1.078 & -0.5597 & -0.2895 \\
21 & 54.6712 & .5603$\cdot$\,10$^{-15}$ & 3.194 & 1.05 & 0.5903 & 0.2654 \\
22 & 64.6789 & .7842$\cdot$\,10$^{-15}$ & 3.194 & 1.05 & -0.6246 & -0.2590 \\
23 & 54.1300 & .3228$\cdot$\,10$^{-15}$ & 3.814 & 1.02 & 0.6656 & 0.3750 \\
24 & 65.2241 & .4689$\cdot$\,10$^{-15}$ & 3.814 & 1.02 & -0.6942 & -0.3680 \\
25 & 53.5957 & .1748$\cdot$\,10$^{-15}$ & 4.484 & 1.00 & 0.7086 & 0.5085 \\
26 & 65.7648 & .2632$\cdot$\,10$^{-15}$ & 4.484 & 1.00 & -0.7325 & -0.5002 \\
27 & 53.0669 & .8898$\cdot$\,10$^{-16}$ & 5.224 & .97 & 0.7348 & 0.6206 \\
28 & 66.3021 & .1389$\cdot$\,10$^{-15}$ & 5.224 & .97 & -0.7546 & -0.6091 \\
29 & 52.5424 & .4264$\cdot$\,10$^{-16}$ & 6.004 & .94 & 0.7702 & 0.6526 \\
30 & 66.8368 & .6899$\cdot$\,10$^{-16}$ & 6.004 & .94 & -0.7864 & -0.6393 \\
31 & 52.0214 & .1924$\cdot$\,10$^{-16}$ & 6.844 & .92 & 0.8083 & 0.6640 \\
32 & 67.3696 & .3229$\cdot$\,10$^{-16}$ & 6.844 & .92 & -0.8210 & -0.6475 \\
33 & 51.5034 & .8191$\cdot$\,10$^{-17}$ & 7.744 & .89 & 0.8439 & 0.6729 \\
34 & 67.9009 & .1423$\cdot$\,10$^{-16}$ & 7.744 & .89 & -0.8529 & -0.6545 \\
35 & 368.4984 & .6460$\cdot$\,10$^{-15}$ & .048 & 1.92 & 0.0000 & 0.0000 \\
36 & 424.7631 & .7047$\cdot$\,10$^{-14}$ & .044 & 1.92 & 0.0000 & 0.0000 \\
37 & 487.2494 & .3011$\cdot$\,10$^{-14}$ & .049 & 1.92 & 0.0000 & 0.0000 \\
38 & 715.3932 & .1826$\cdot$\,10$^{-14}$ & .145 & 1.81 & 0.0000 & 0.0000 \\
39 & 773.8397 & .1152$\cdot$\,10$^{-13}$ & .141 & 1.81 & 0.0000 & 0.0000 \\
40 & 834.1453 &  .3971$\cdot$\,10$^{-14}$ & .145 & 1.81 & 0.0000 & 0.0000 \\
\hline
\end{longtable}

\begin{longtable}{|l||r|r|r|r|r|r|}
 K & K & K & K & K & K & K \kill
%
% --------------------- only begin of table ------------------------------
 \hline
 index & 
 $\nuk$ & 
 $S_k(300\,{\sf K})$ & 
 $b_k$ & 
 $w_k$  & 
 $y_k$ & 
 $v_k$ \\
 $k$   & 
 {\sf [GHz]}  & 
 {\sf [cm$^2$\,Hz]} & 
 {\sf [1]} & 
 {[$\frac{\sf MHz}{\sf hPa}$]} & 
 {[$\frac{\sf 10{^{-3}}}{\sf hPa}$]} & 
 {[$\frac{\sf 10{^{-3}}}{\sf hPa}$]} \\
 \hline
 \hline
 \endfirsthead
% --------------------- every page begin of table ------------------------
 \hline
 index & 
 $\nuk$ & 
 $S_k(300\,{\sf K})$ & 
 $b_k$ & 
 $w_k$  & 
 $y_k$ & 
 $v_k$ \\
 \hline
 \hline
 \endhead
% --------------------- every page end of table ------------------------
 K & K & K & K & K & K & K \kill
 \hline
 \caption[]{(continued)}\\
 \endfoot
% --------------------- only end of table ------------------------------
 K & K & K & K & K & K & K \kill
 \hline
 \caption{List of $\oz$ spectral lines of the Rosenkranz absorption 
          model \cite{pwr:93}.}
 \label{tab:pwr02line}
 \endlastfoot
% --------------------- body of table ----------------------------------  
1  & 118.7503  & .2936$\cdot$\,10$^{-14}$ & .009 & 1.63 & -0.0233 & 0.0079 \\
2  & 56.2648 & .8079$\cdot$\,10$^{-15}$ & .015 & 1.646 & 0.2408 & -0.0978 \\
3  & 62.4863 & .2480$\cdot$\,10$^{-14}$ & .083 & 1.468 & -0.3486 &  0.0844 \\
4  & 58.4466 & .2228$\cdot$\,10$^{-14}$ & .084 & 1.449 & 0.5227 & -0.1273 \\
5  & 60.3061 & .3351$\cdot$\,10$^{-14}$ & .212 & 1.382 & -0.5430 & 0.0699 \\
6  & 59.5910 & .3292$\cdot$\,10$^{-14}$ & .212 & 1.360 & 0.5877 & -0.0776 \\
7  & 59.1642 & .3721$\cdot$\,10$^{-14}$ & .391 & 1.319 & -0.3970 & 0.2309 \\
8  & 60.4348 & .3891$\cdot$\,10$^{-14}$ & .391 & 1.297 & 0.3237 & -0.2825 \\
9  & 58.3239 & .3640$\cdot$\,10$^{-14}$ & .626 & 1.266 & -0.1348 &  0.0436 \\
10 & 61.1506 & .4005$\cdot$\,10$^{-14}$ & .626 & 1.248 & 0.0311 & -0.0584 \\
11 & 57.6125 & .3227$\cdot$\,10$^{-14}$ & .915 & 1.221 & 0.0725 & 0.6056 \\
12 & 61.8002 & .3715$\cdot$\,10$^{-14}$ & .915 & 1.207 & -0.1663 & -0.6619 \\
13 & 56.9682 & .2627$\cdot$\,10$^{-14}$ & 1.260 & 1.181 & 0.2832 & 0.6451 \\
14 & 62.4112 & .3156$\cdot$\,10$^{-14}$ & 1.260 & 1.171 & -0.3629 & -0.6759 \\
15 & 56.3634 & .1982$\cdot$\,10$^{-14}$ & 1.660 & 1.144 & 0.3970 &  0.6547 \\
16 & 62.9980 & .2477$\cdot$\,10$^{-14}$ & 1.665 & 1.139 & -0.4599 & -0.6675 \\
17 & 55.7838 & .1391$\cdot$\,10$^{-14}$ & 2.119 & 1.110 & 0.4695 & 0.6135 \\
18 & 63.5685 & .1808$\cdot$\,10$^{-14}$ & 2.115 & 1.108 & -0.5199 & -0.6139 \\
19 & 55.2214 & .9124$\cdot$\,10$^{-15}$ & 2.624 & 1.079 & 0.5187 & 0.2952 \\
20 & 64.1278 & .1230$\cdot$\,10$^{-14}$ & 2.625 & 1.078 & -0.5597 & -0.2895 \\
21 & 54.6712 & .5603$\cdot$\,10$^{-15}$ & 3.194 & 1.05 & 0.5903 & 0.2654 \\
22 & 64.6789 & .7842$\cdot$\,10$^{-15}$ & 3.194 & 1.05 & -0.6246 & -0.2590 \\
23 & 54.1300 & .3228$\cdot$\,10$^{-15}$ & 3.814 & 1.02 & 0.6656 & 0.3750 \\
24 & 65.2241 & .4689$\cdot$\,10$^{-15}$ & 3.814 & 1.02 & -0.6942 & -0.3680 \\
25 & 53.5957 & .1748$\cdot$\,10$^{-15}$ & 4.484 & 1.00 & 0.7086 & 0.5085 \\
26 & 65.7648 & .2632$\cdot$\,10$^{-15}$ & 4.484 & 1.00 & -0.7325 & -0.5002 \\
27 & 53.0669 & .8898$\cdot$\,10$^{-16}$ & 5.224 & .97 & 0.7348 & 0.6206 \\
28 & 66.3021 & .1389$\cdot$\,10$^{-15}$ & 5.224 & .97 & -0.7546 & -0.6091 \\
29 & 52.5424 & .4264$\cdot$\,10$^{-16}$ & 6.004 & .94 & 0.7702 & 0.6526 \\
30 & 66.8368 & .6899$\cdot$\,10$^{-16}$ & 6.004 & .94 & -0.7864 & -0.6393 \\
31 & 52.0214 & .1924$\cdot$\,10$^{-16}$ & 6.844 & .92 & 0.8083 & 0.6640 \\
32 & 67.3696 & .3229$\cdot$\,10$^{-16}$ & 6.844 & .92 & -0.8210 & -0.6475 \\
33 & 51.5034 & .8191$\cdot$\,10$^{-17}$ & 7.744 & .89 & 0.8439 & 0.6729 \\
34 & 67.9009 & .1423$\cdot$\,10$^{-16}$ & 7.744 & .89 & -0.8529 & -0.6545 \\
35 & 368.4984 & .6460$\cdot$\,10$^{-15}$ & .048 & 1.92 & 0.0000 & 0.0000 \\
36 & 424.7631 & .7047$\cdot$\,10$^{-14}$ & .044 & 1.92 & 0.0000 & 0.0000 \\
37 & 487.2494 & .3011$\cdot$\,10$^{-14}$ & .049 & 1.92 & 0.0000 & 0.0000 \\
38 & 715.3932 & .1826$\cdot$\,10$^{-14}$ & .145 & 1.81 & 0.0000 & 0.0000 \\
39 & 773.8397 & .1152$\cdot$\,10$^{-13}$ & .141 & 1.81 & 0.0000 & 0.0000 \\
40 & 834.1453 &  .3971$\cdot$\,10$^{-14}$ & .145 & 1.81 & 0.0000 & 0.0000 \\
\hline
% -----------------------------------------------------------------------  
\end{longtable}




\levele{Oxygen Continum Absorption:}
\label{levele:02_pwr98_cont}
%--------------------------------------
As pointed out by Van~Vleck \cite{vv:87}, the standard theory for
non-resonant absorption is that of Debye (see also Ref. \cite{townes:55}). 
The Debye line shape is obtained from the VVW line shape function of 
Eq.~(\ref{eq:def_vvw}) by the limiting case\footnote{The factor 
  $(\nu/\nuk)^2$ of Eq.~(\ref{eq:def_vvw}) is in this case set to $\nu^2$.}
$\nuk \rightarrow 0$.
Both, Liebe \cite{liebeetal:93} and Rosenkranz \cite{pwr:93} adopt the
Debye theory for their models. The only difference is the formulation
of the line broadening, where the influence of water vapor is treated 
differently\footnote{in Ref. \cite{liebeetal:93} is a mistake in the formula
  for $\gamma$. According to this text $\gamma =
  w\cdot\pda\cdot\Theta^{0.8}$ while in the source code of MPM93 
  Eq. (\ref{eq:abs_cont_pwr_o2_2}) is implemented.}: 
\begin{eqnarray}
  \label{eq:abs_cont_pwr_o2}
  \alphac &=&  C \cdot \pda \cdot \Theta^2 \cdot 
             \frac{\nu^2 \cdot \gamma}{\nu^2+\gamma^2}\\
%
  \label{eq:abs_cont_pwr_o2_1}
  \gamma &=&  w \cdot (\pda \cdot \Theta^{0.8} + 1.1 \cdot \phzo \cdot
  \Theta)\hspace*{5mm}\mbox{:~Rosenkranz}\\
  \label{eq:abs_cont_pwr_o2_2}
  \gamma &=&  w \cdot (\pda + \phzo) \cdot \Theta^{0.8}\hspace*{17.25mm}\mbox{:~MPM93}\\
\nonumber
\end{eqnarray}
The values for the parameters are $C = 1.11\cdot 10^{-5}$ dB/km/(hPa\,GHz) and 
$w = 5.6 \cdot 10^{-4}$ GHz/hPa, respectively. This absorption
term is proportional to the collision frequency of a single oxygen molecule
and thus proportional to the dry air pressure\footnote{The absorption
  due to weakly bound complexes of $\oz$--$X$ with $X=\hzo,~\nz$ is 
  treated separately and therefore not included in this Debye
  formula.}.

%
%
% ============================================================================
%
% 
\leveld{O2-MPM93}
%
%
% ============================================================================
%
%
\levelc{ARTS Workspace Variables and Methods}
%
%
% ============================================================================
%
%
\levelb{Continuum Absorption Models}
\label{levelb:ContAbsMod}
%===================================

There should be some general introduction here.

The headings here are tentative, TBD by Thomas.
\begin{equation} 
  \label{eq:r98:cont}
  \alphac = \nu^2 \cdot \phzo \cdot 
            (\cso \cdot \phzo \cdot \Theta^{\xs} + 
             \cdo \cdot \pda  \cdot \Theta^{\xf})
\end{equation}

\begin{table}[!hb]
  \begin{center}
  \begin{tabular}{lrrrrr}
    \hline
    model  & \multicolumn{1}{c}{$\cso$} & 
             \multicolumn{1}{c}{$\xs$}  & 
             \multicolumn{1}{c}{$\cdo$} & 
             \multicolumn{1}{c}{$\xd$}  & 
             ref.\\
           & \multicolumn{1}{c}{$\left[\frac{\mbox{dB/km}}
                               {\mbox{hPa}^2~\mbox{GHz}^2}\right]$} & 
             \multicolumn{1}{c}{$[1]$} & 
             \multicolumn{1}{c}{$\left[\frac{\mbox{dB/km}}
                               {\mbox{hPa}^2~\mbox{GHz}^2}\right]$} & 
             \multicolumn{1}{c}{$[1]$} & \\
    \hline
    MPM85  & 4.92$\cdot$10$^{-8}$ & 2.5 & 0.255$\cdot$10$^{-8}$  & -0.5 & \cite{liebe:84}\\
    MPM87  & 6.50$\cdot$10$^{-8}$ & 7.5 & 0.206$\cdot$10$^{-8}$  & 0.0 & \cite{liebeandlayton:87}\\
    MPM89  & 6.50$\cdot$10$^{-8}$ & 7.5 & 0.206$\cdot$10$^{-8}$  & 0.0 & \cite{liebe:89}\\
    CP98   & 8.04$\cdot$10$^{-8}$ & 7.5 & 0.254$\cdot$10$^{-8}$ & 0.0 & \cite{cruzpol:98}\\ 
    R98    & 7.80$\cdot$10$^{-8}$ & 4.5 & 0.236$\cdot$10$^{-8}$  & 0.0 & \cite{pwr:98}\\
    \hline
    MPM93$^*$ & 7.73$\cdot$10$^{-8}$ & 4.55 & 0.253$\cdot$10$^{-8}$  & 1.55 & \cite{liebeetal:93}\\
    \hline
 \end{tabular}
\end{center}
 \caption{Values of the sets of continuum parameters. The last line (MPM93$^*$)
   represents an approximation of the pseudo-line continuum of MPM93
   in the form of Eq.~\ref{eq:abs_cont_pwr} for the microwave range as
   described in paragraph \ref{sec:mpm93:contabs}.}
 \label{tab:wvcontparam}
\end{table}
%
%
% ============================================================================
%
%
\levelc{Water Vapor Continuum Models}
\label{leveld:h2o_ContMod}
%------------------------------------
%
%
% ============================================================================
%
% 
\levelc{Dry air Models ($\oz,~\nz~\coz$)}
%
%
% ============================================================================
%
% 
\leveld{O2-SelfContPWR93}
\label{leveld:o2_pwr98_cont}
%---------------------------
%
%
% ============================================================================
%
% 
\leveld{O2-SelfContMPM93}
\label{leveld:o2_mpm93_cont}
%---------------------------
%
%
% ============================================================================
%
% 
\leveld{N2-SelfContPWR93}
\label{leveld:n2_pwr98_cont}
%---------------------------
%
%
%
% ============================================================================
%
% 
\leveld{N2-SelfContMPM93}
\label{leveld:n2_mpm93_cont}
%---------------------------
%
%
% ============================================================================
%
% 
\leveld{$\coz$-SelfContPWR93}
\label{leveld:co2_pwr98_s_cont}
%------------------------------
%
%
% ============================================================================
%
% 
\leveld{CO2-ForeignContPWR93}
\label{leveld:co2_pwr98_f_cont}
%------------------------------
%
%
% ============================================================================
%
%
\levelc{ARTS Workspace Variables and Methods}
%
%
% ============================================================================

 



%%% Local Variables: 
%%% mode: latex 
%%% TeX-master: "uguide" 
%%% End:

