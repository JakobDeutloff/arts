%
% To start the document, use
%  \levela{...}
% For lover level, sections use
%  \levelb{...}
%  \levelc{...}
%
\levela{Gas Absorption}
 \label{sec:absorption}


%
% Document history, format:
%  \starthistory
%    date1 & text .... \\
%    date2 & text .... \\
%    ....
%  \stophistory
%
\starthistory
  2001-07-05 & Template created by Stefan Buehler.\\
\stophistory


%
% Symbol table, format:
%  \startsymbols
%    ... & \verb|...| & text ... \\
%    ... & \verb|...| & text ... \\
%    ....
%  \stopsymbols
%
%
%\startsymbols
%  -- & -- & -- \\
% \label{symtable:wfuns}     
%\stopsymbols


Some general introduction here, also explaining the structure of this chapter.



\levelb{Line Absorption}
%-----------------------
\label{sec:line_absorption}

Some introduction here, how line by line absorption is calculated in
principle. (Should define intensity, partition function, line shape, etc.)

\levelc{Line Catalogues} 

Mostly describe the ARTS internal format, but also briefly list the
other catalogues that can be read by ARTS

\levelc{Species specific data} 

Molecular mass, etc. 

\levelc{Partition Functions}

These are strictly also species specific data, but they seem to deserve
their own section.

\levelc{Line Shape Functions}


\levelc{ARTS Workspace Variables and Methods}



\levelb{Continua and Complete Absorption Models}
%-----------------------------------------------

There should be some general introduction here.

The headings here are tentative, TBD by Thomas.

\levelc{H2O Models}

\levelc{Dry air Models}

\levelc{ARTS Workspace Variables and Methods}


 



%%% Local Variables: 
%%% mode: latex 
%%% TeX-master: "uguide" 
%%% End:

