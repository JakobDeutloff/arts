%
% To start the document, use
%  \chapter{...}
% For lover level, sections use
%  \section{...}
%  \subsection{...}
%
\chapter{Vectors, matrices, tensors, and arrays}
%-------------------------------------------------------------------------
\label{sec:matpack}


%
% Document history, format:
%  \starthistory
%    date1 & text .... \\
%    date2 & text .... \\
%    ....
%  \stophistory
%
\starthistory
  030807 & Sparse added by Mattias Ekstr\"om. \\
  030109 & Documentation for using jokers without Range added by Stefan Buehler.\\
  020516 & Tensors added by Stefan Buehler.\\
  011018 & Created and written by Stefan Buehler.\\
\stophistory


This section describes how arbitrary-rank "tensor" classes are
implemented in ARTS and how their objects can be used.
Furthermore it describes how arrays of arbitrary type can be constructed and used.

\section{Implementation files}
The arbitrary-rank "tensor" classes are implemented in the
\verb|src/matpack/| folder.  The implementation is done through 
template programming, meaning that the logic for how to use the objects
is similar even though the rank may change.  Common for all of these
types though is that the rank of the object has to be known at compile time.

There are two template classes that represents
objects owning their own data.  The template class that knows both its rank and exact size at
compile time is implemented in \fileindex{matpack\_constexpr.h}.  The
template class that only knows its rank at compile time but has to allocate memory
at runtime is implemented in \fileindex{matpack\_data.h}.

There are also two template classes representing a view of the data.  The template class that
views a piece of data with known rank and size is implemented also in \fileindex{matpack\_constexpr.h}.
The template class that views data of known rank but unknown size at compile time is implemented
in \fileindex{matpack\_view.h}.

All four of these will be described more below.

The template class Array (also described below) is implemented in the file \fileindex{array.h}.

The \builtindoc{Sparse} class is described in the file \fileindex{matpack\_sparse.h}.

\section{Arbitrary rank "tensor" --- matpack}
This section described ranked data in ARTS.  We call template classes of our arbitrary data
template class with rank above 2 "Tensor", though we fully admit that they do not conform to all the ideas
that a mathematical tensor would.  The intent is that they should, but we are limited by the
use that we have had of them, so further extensions that come at no cost to the quality of
the current implementation is highly welcome.  We use the term "Vector" to describe
a rank 1 realization of the template class.  We use the term "Matrix" to 
describe a rank 2 realization of the template class.  These rank 1 and 2 objects do more closely
follow the mathematical description of vectors and matrices, such as supporting the dot-product and
matrix-vector, and matrix-matrix multiplications.

As the most common underlying type
to perform computations on in ARTS is \builtindoc{Numeric}, the rank 1 class representing
\builtindoc{Numeric} is called \builtindoc{Vector} and the rank 2 class is called \builtindoc{Matrix}.
We do not name types above rank 7.  Ranked classes representing \builtindoc{Numeric} between rank 3
and rank 7 are called \builtindoc{Tensor3}, \builtindoc{Tensor4}, \builtindoc{Tensor5}, \builtindoc{Tensor6},
and \builtindoc{Tensor7}, respectively.  These are all the named types, each of which have their
sizes known only at runtime.  It is important to be aware that re-interpolating a \builtindoc{Tensor7}
requires a rank 14 object, however, so the information in this section also applies to versions of 
the class that are only named internally.

Under the hood, the implementation for these classes is based on C++23 \shortcode{std::mdspan}
and the C++26 proposal for \shortcode{std::submdspan}.  The intent is that you should be able to
interact with these objects as if they were actually of the template class \shortcode{std::mdspan}.
Note that neither of these C++ standards are actually available as of yet (nb., 2023-02-28), so the
details under the hood relies on the Kokkos implementation (which represents what got accepted for C++23
and what is proposed for C++26). 

The underlying data type when the size of the data is known at compile time is \shortcode{std::array}
and the underlying data type when the size of the data is known only at runtime is \shortcode{std::vector}.
Again, it is our intent that we should expose as much of these the underlying data type's behavior in the interface to our
own classes as possible.

In short there are two intents of these codes:
\begin{enumerate}
\item Represent any data that we might need in ARTS in a concise, fast, and mathematically meaningful way.
\item Offer a functional interface that matches the C++ standard interface as closely as possible.
\end{enumerate}

\subsection{Defining the template classes}
These template classes can own data or view data.  The size of the data can be known at compile time or not.
This subsection goes through the common type and their used names inside ARTS.

\subsubsection{matpack\_constant\_data --- owning and constant size}
The data owning template class for the type that knows its size at compile time is called
\begin{code}
template <typename T, Index... alldim>
struct matpack_constant_data;
\end{code}
It will hold a type \shortcode{T} that can be any type that can be help by \shortcode{std::array}
and it will have a rank \shortcode{N} that is as many \builtindoc{Index} that you define in the place
of \shortcode{alldim}.  Here are a few examples to help you understand:
\begin{code}
// An object "a" that holds 5 Numeric as a vector:
matpack_constant_data<Numeric, 5> a;

// An object "a2" that holds 1,2,3,4,5 as Numeric as a vector:
matpack_constant_data<Numeric, 5> a2{1,2,3,4,5};

// An object "b" that holds 6 Index as a 3x2 matrix:
matpack_constant_data<Index, 3, 2> b;

// An object "b" that holds 1,2;3,4;5,6 Index as a 3x2 matrix:
matpack_constant_data<Index, 3, 2> b{1,2,3,4,5,6};

// An object "c" that holds 120 Complex as a 5x4x3x2 tensor 4:
matpack_constant_data<Complex, 5, 4, 3, 2> c;
\end{code}
Note that while the name might be confusing, the "constant" part here refers to "constant size".  The
"const" and "constexpr" properties of C++ still has to be applied to make the data actually not mutable.

These are the named version of these types inside ARTS:
\begin{itemize}
\item THERE ARE NO TYPES EXPLICITLY NAMED YET
\end{itemize}

\subsubsection{matpack\_constant\_view --- non-owning and constant size}
The template class that does not own its data but points at a compile time constant size of data
is called
\begin{code}
template <typename T, bool constant, Index... alldim>
struct matpack_constant_view;
\end{code}
Both \shortcode{T} and \shortcode{alldim} are the same as above.  The new boolean argument \shortcode{constant}
is held to tell the type if it can mutate its data or not.  Generally, you should try to not construct
a type like this manually but instead rely on access-patterns to a higher or same rank \shortcode{matpack\_constant\_data} or \shortcode{matpack\_constant\_view}
to generate the type as need be.  You can do so by using the access or call operators provided by 
\shortcode{matpack\_constant\_data}.  Note also that you can always create a \shortcode{constant==true}
object if you have a \shortcode{constant==false} object, but that this is irreversible within the type
system.

These are the named version of these types inside ARTS:
\begin{itemize}
\item THERE ARE NO TYPES EXPLICITLY NAMED YET
\end{itemize}

\subsubsection{matpack\_data --- owning and runtime size}
The data owning template class that does not know its size but its rank at compile time is called
\begin{code}
template <typename T, Index N> class matpack_data;
\end{code}
Unlike the other data owning object, the size of the data is invariant in most operations by
keeping the underlying allocated data constant.  This has to be dealt with on a function-by-function
basis to keep the rest of the class working.
As for the constant data type, the value type will be a \shortcode{T} --- that is possible to be held by a \shortcode{std::vector} ---
but the rank is defined explicitly as \shortcode{N}.  Rank 0 objects are not allowed.
The sizes of the object has to at some point be given during runtime. Here are a few examples to help you understand:
\begin{code}
// An object "a" as a rank 1 vector of unknown size:
matpackt_data<Numeric, 1> a;

// An object "a2" as a rank 1 vector of current size 5:
matpackt_data<Numeric, 1> a2(5);

// An object "a3" as a rank 1 vector of current size n:
extern Index n;
matpackt_data<Numeric, 1> a3(n);

// An object "b" that holds 6 Index as a 3x2 matrix:
matpack_data<Index, 2> b(3, 2);

// An object "b2" that holds 6 Index as a 3x2 matrix all of value 1:
matpack_data<Index, 2> b2(3, 2, 1);

// An object "c" that holds 120 Complex as a 5x4x3x2 tensor 4:
matpack_data<Complex, 4> c(5, 4, 3, 2);
\end{code}
Note that the same number of \builtindoc{Index} as the rank of the object has to
be given to the constructor for the created object to have an immediate size.
If an additional argument is given, it is converted to type \shortcode{T} and
set as the value for all of the elements of the tensor.
Since the data is only known at compile time, these objects can all be resized
calling the member function \shortcode{resize}.  From the example above, calling
\shortcode{a.resize(5);} gives the object the size 5.

These are the named version of these types inside ARTS:
\begin{center}
\begin{tabular}{llc}
\\\hline
Name & Value Type \shortcode{T} & Rank \shortcode{N} \\\hline
\builtindoc{Vector}  & \builtindoc{Numeric} & 1 \\\hline
\builtindoc{Matrix}  & \builtindoc{Numeric} & 2 \\\hline
\builtindoc{Tensor3} & \builtindoc{Numeric} & 3 \\\hline
\builtindoc{Tensor4} & \builtindoc{Numeric} & 4 \\\hline
\builtindoc{Tensor5} & \builtindoc{Numeric} & 5 \\\hline
\builtindoc{Tensor6} & \builtindoc{Numeric} & 6 \\\hline
\builtindoc{Tensor7} & \builtindoc{Numeric} & 7 \\\hline
\shortcode{ComplexVector} & \shortcode{Complex} & 1 \\\hline
\shortcode{ComplexMatrix} & \shortcode{Complex} & 2 \\\hline
\end{tabular}
\end{center}

\subsubsection{matpack\_view --- non-owning and runtime size}
Likewise, the template class that does not own its data but points to runtime allocated set of data
is called
\begin{code}
template <typename T, Index N, bool constant, bool strided>
class matpack_view;
\end{code}
All of \shortcode{T}, \shortcode{N} and \shortcode{constant} are as defined above but because of
both legacy and practical reasons, we need an additional boolean argument here: \shortcode{strided}.
Sometimes the memory layout in hardware of data is not \textit{contiguous} or \textit{exhaustive},
that is you cannot expect that as long as you are within the total size of the object, that the
next data point is always offset by the same step size.  A good example of a strided vector
is all the real values from a complex-valued vector; standard implementations of complex values
guarantees that the real value is next to the imaginary value in memory, so the next real value
is atleast 2 steps away.  In analogy for \shortcode{matpack\_constant\_view}, the recommended way to create
a \shortcode{matpack\_view} is same or higher ranks of \shortcode{matpack\_data} or \shortcode{matpack\_view}.
We also allow viewing a single \shortcode{T} as the underlying via explicit conversion.
Also as for \shortcode{matpack\_constant\_view}, you can generate a constant view from a non-constant
but never the reverse.  The same is true for strided views: you are able to generate a strided view
from a exhaustive view but never a exhaustive view from a strided view.  Be aware that a \shortcode{matpack\_data}
type is neither constant nor strided, which is why these properties are left to
the type system and not to the type itself.

These are the named version of these types inside ARTS:
\begin{center}
\begin{tabular}{llccc}
\\\hline
Name & Value Type \shortcode{T} & Rank \shortcode{N} & \shortcode{constant} & \shortcode{strided} \\\hline
\shortcode{VectorView}  & \builtindoc{Numeric} & 1 & false & true \\\hline
\shortcode{MatrixView}  & \builtindoc{Numeric} & 2 & false & true \\\hline
\shortcode{Tensor3View} & \builtindoc{Numeric} & 3 & false & true \\\hline
\shortcode{Tensor4View} & \builtindoc{Numeric} & 4 & false & true \\\hline
\shortcode{Tensor5View} & \builtindoc{Numeric} & 5 & false & true \\\hline
\shortcode{Tensor6View} & \builtindoc{Numeric} & 6 & false & true \\\hline
\shortcode{Tensor7View} & \builtindoc{Numeric} & 7 & false & true \\\hline
\shortcode{ComplexVectorView} & \shortcode{Complex} & 1 & false & true \\\hline
\shortcode{ComplexMatrixView} & \shortcode{Complex} & 2 & false & true \\\hline
\shortcode{ConstVectorView}  & \builtindoc{Numeric} & 1 & true & true \\\hline
\shortcode{ConstMatrixView}  & \builtindoc{Numeric} & 2 & true & true \\\hline
\shortcode{ConstTensor3View} & \builtindoc{Numeric} & 3 & true & true \\\hline
\shortcode{ConstTensor4View} & \builtindoc{Numeric} & 4 & true & true \\\hline
\shortcode{ConstTensor5View} & \builtindoc{Numeric} & 5 & true & true \\\hline
\shortcode{ConstTensor6View} & \builtindoc{Numeric} & 6 & true & true \\\hline
\shortcode{ConstTensor7View} & \builtindoc{Numeric} & 7 & true & true \\\hline
\shortcode{ConstComplexVectorView} & \shortcode{Complex} & 1 & true & true \\\hline
\shortcode{ConstComplexMatrixView} & \shortcode{Complex} & 2 & true & true \\\hline
\shortcode{ExhaustiveVectorView}  & \builtindoc{Numeric} & 1 & false & false \\\hline
\shortcode{ExhaustiveMatrixView}  & \builtindoc{Numeric} & 2 & false & false \\\hline
\shortcode{ExhaustiveTensor3View} & \builtindoc{Numeric} & 3 & false & false \\\hline
\shortcode{ExhaustiveTensor4View} & \builtindoc{Numeric} & 4 & false & false \\\hline
\shortcode{ExhaustiveTensor5View} & \builtindoc{Numeric} & 5 & false & false \\\hline
\shortcode{ExhaustiveTensor6View} & \builtindoc{Numeric} & 6 & false & false \\\hline
\shortcode{ExhaustiveTensor7View} & \builtindoc{Numeric} & 7 & false & false \\\hline
\shortcode{ExhaustiveComplexVectorView} & \shortcode{Complex} & 1 & false & false \\\hline
\shortcode{ExhaustiveComplexMatrixView} & \shortcode{Complex} & 2 & false & false \\\hline
\shortcode{ExhaustiveComplexTensor3View} & \shortcode{Complex} & 3 & false & false \\\hline
\shortcode{ExhaustiveConstVectorView}  & \builtindoc{Numeric} & 1 & true & false \\\hline
\shortcode{ExhaustiveConstMatrixView}  & \builtindoc{Numeric} & 2 & true & false \\\hline
\shortcode{ExhaustiveConstTensor3View} & \builtindoc{Numeric} & 3 & true & false \\\hline
\shortcode{ExhaustiveConstTensor4View} & \builtindoc{Numeric} & 4 & true & false \\\hline
\shortcode{ExhaustiveConstTensor5View} & \builtindoc{Numeric} & 5 & true & false \\\hline
\shortcode{ExhaustiveConstTensor6View} & \builtindoc{Numeric} & 6 & true & false \\\hline
\shortcode{ExhaustiveConstTensor7View} & \builtindoc{Numeric} & 7 & true & false \\\hline
\shortcode{ExhaustiveConstComplexVectorView} & \shortcode{Complex} & 1 & true & false \\\hline
\shortcode{ExhaustiveConstComplexMatrixView} & \shortcode{Complex} & 2 & true & false \\\hline
\shortcode{ExhaustiveConstComplexTensor3View} & \shortcode{Complex} & 3 & true & false \\\hline
\end{tabular}
\end{center}

\subsection{Access operations of the template classes}
The access patterns we allow currently are through these types
\begin{itemize}
\item \builtindoc{Index} -- an integer, e.g., "row 5"
\item \shortcode{Range} -- a strided range of values, e.g., "row 1 until row 5, every other item"
\item \shortcode{Joker} -- an entire dimension, e.g., "all rows"
\end{itemize}
Since the constant size data and view types cannot be strided, they make use only of \builtindoc{Index} and \shortcode{Joker}.
The runtime size data make use of all three types.
If the access operation would logically reduce the rank to 0, instead of returning a rank 0 object,
something of the underlying value type \shortcode{T} is returned.  Generally, this is a copy of the value if the
object is constant or a reference to the value if the object is mutable.
Otherwise, if the access operation does not completely reduce the rank of the object, a view of the remaining rank is returned.
Don't worry, there are examples below.

\subsubsection{Constant size access}
There are only two ways to access constant data:
\begin{code}
// "a" from earlier example:
matpack_constant_data<Numeric, 5> a;
a[0] = 3;  // Set the first element of "a" to three
a(1) = 4;  // Set the second element of "a" to four

// "b" from earlier example
matpack_constant_data<Index, 3, 2> b;
b(0, 0) = 3;  // Set row 0 col 0 of b to 3
b(0, 1) = 4;  // Set row 0 col 1 of b to 4

// c is now a matpack_constant_view<Index, false, 2>
auto c = b(1, joker);

// Set the second element of "c" to five. b(1, 1) is also 5.
c[1] = 5;

// We can only access from the left to remain exhaustive so
// we can just skip all the jokers and do
auto d = b[2];
// So that the type of "d" is the same as the type of "c"

const auto e = a;  // Create a constant of "a"

// We can do the same accessing as before to create a
// matpack_constant_view<Index, true, 2>
auto f = e[1];

// But if we try to write to the data we get a compile-error
// The error is difficult to predict, but it will either say
// that writing to a constant is not allowed or that the
// requirements of "not constant" is not fulfilled
f[0] = 2;  // compilation error
\end{code}
The same access pattern works for the template class that owns and 
the template class that does not own the view of the data

\subsubsection{Runtime size access}
The access operators for the runtime sized data can be any combination of the 
three accessing types above or just the square-bracket to access the inner-view.
Note that the way you access an object may change the type returned in subtle ways.
Examples:
\begin{code}
// For a Vector
Vector a(5);
a[joker] = Vector(5, 1);  // Set all of "a" to 1
a[1] = 3;  // Set the second element of "a" to 3
a[Range(2, 2, 1)] = 4;  // Set the 3rd+4th element of "a" to 4

// What are the return types?
a[0];  // This is a Numeric!
a[joker];  // This is an ExhaustiveVectorView!
a[Range(2, 2, 1)];  // This is a VectorView!

// For a Matrix:
Matrix b(3, 2);
b(0, 0) = 10;  // Sets b at row 0 and col 0 to 10
b[1] = 3;  // Set b(1, 0) and b(1, 1) to 3
b(joker, 1) = 2;  // Set b(0, 1), b(1, 1), and b(2, 1) to 2

// What are the return types?
b(0, 0);  // This is a Numeric!
b(joker, joker);  // This is an ExhaustiveMatrixView!
b(1, joker);
b[1];  // These are ExhaustiveVectorView!
b(joker, 0);  // This is a VectorView!
b(1, Range(0, 1));  // This is a VectorView! 
\end{code}
These access concepts are extendible the any rank.  Note that the last example contains
exhaustive data layout but the returned type is not exhaustive.  That is because any access
with range can only be determined at runtime if it is strided or not, but the interface
is strict with regards to types in C++.

It is up to the developer to ensure that you never access out-of-bounds elements.
To get the extent of a dimension we have these member functions
\begin{itemize}
\item \shortcode{extent(i)} --- get the size of the i:th left-most dimension, e.g., rows for $i=1$ on a matrix
\item \shortcode{nelem} --- get the size of a rank 1 tensor
\item \shortcode{ncols} --- get the size of the right-most dimension of a rank 2 or higher tensor
\item \shortcode{nrows} --- get the size of the right-most bar 1 dimension of a rank 2 or higher tensor
\item \shortcode{npages} --- get the size of the right-most bar 2 dimension of a rank 3 or higher tensor
\item \shortcode{nbooks} --- get the size of the right-most bar 3 dimension of a rank 4 or higher tensor
\item \shortcode{nshelves} --- get the size of the right-most bar 4 dimension of a rank 5 or higher tensor
\item \shortcode{nvitrines} --- get the size of the right-most bar 5 dimension of a rank 6 or higher tensor
\item \shortcode{nlibraries} --- get the size of the right-most bar 6 dimension of a rank 7 or higher tensor
\item \shortcode{shape} --- get the sizes of all dimensions of a rank \shortcode{N} tensor as a \shortcode{N}-long array of indices
\end{itemize}
To get the stride of a given dimension, we offer the member function \shortcode{stride(i)} for the stride of dimension \shortcode{i}
or \shortcode{strides} to get the all the strides of a rank \shortcode{N} tensor as a \shortcode{N}-long array of indices.

\subsection{Iterate over elements for faster execution time}
All template classes in this part of matpack gives 2 very important and very different types
of iterators: elementwise and subviews.  This allows us to write composable code that works
for any and all types.  For instance
\begin{code}
Vector a({1, 2, 3, 4, 5});  // "a" is {1, 2, 3, 4, 5}
for (auto&& x: a) {
  x = sin(x);  // update x to the sin of itself
}  // "a" is {sin(1), sin(2), sin(3), sin(4), sin(5)}
\end{code}
Note that we have a shorter way of writing this that also could generate better runtime:
\begin{code}
Vector a({1, 2, 3, 4, 5});  // "a" is {1, 2, 3, 4, 5}
matpack::transform(a, sin, a);
\end{code}
You can read the implementation of transform in \fileindex{matpack\_math.h}.  Here the input and output is the same, and both are \builtindoc{Vector}, but this works for any types as long as their total size is the same.  Even if the two types have a different rank...

For matrices, this works a little bit different:
\begin{code}
Matrix b(3, 2, 1); // A matrix of {1, 1; 1, 1; 1, 1}
for (auto&& a: b) {
  a[0] = 3;  // Set the first element of "a" to 3
}  // Will now have a matrix of {3, 1; 3, 1;, 3; 1}
\end{code}
This happens because the \shortcode{begin} and \shortcode{end}
iterator pairs are dereferenced to a lower rank view.
This works for any higher rank tensor as well, so a tensor of rank 5 views
the pair of iterators as a tensor view of rank 4, for example.
Remember that constness and stridedness is preserved or decayed-towards in these operations, so the
iterators of a constant view can only ever be constant and the iterator of a 
strided view can only ever return a strided view when dereferenced.  But iterators of non-constant
and non-strided can be dereferenced to constant or strided views.

Note also that sometimes we just want to do something with all of the elements.  For rank 1 tensors, this is the same,
but rank 2 and higher ranked tensors behave differently.
This is actually the trick to \shortcode{transform} above, we have a pair of iterators \shortcode{elem\_begin} and \shortcode{elem\_end}
that loops over the individual elements of the tensor.
Several functions and members of the template classes make use of this feature.  Again, the recommendation is to see \fileindex{matpack\_math.h} as it uses many of these iterators for solving various simple 
transformation and reduction problems (e.g., \shortcode{sum}, \shortcode{min}, \shortcode{transform}, etc...). 

The subsection also claims that you get faster execution times using these iterators.  This is of course necessary to
test on a case-by-case basis.  The main benefit we have found in our testing is for exhaustive data and views.  In
several settings, the compiler seems to emit SIMD code for these exhaustive types.  Depending on your operating
system and compiler settings, this means some operations may be 2-4 times faster for exhaustive objects when
operated upon by element-wise iterator logic.  If you use index access or view iterators some additional calculations
to get to the internal value element is needed, which might prevent such SIMD optimizations.

\subsection{Mathematical and logical operations}
All of the normal element-wise mathematical operations are available for objects of the same rank and shape as well as for scalar operations:
\begin{code}
Vector a({1,2,3,4,5});
a += 1;  // a = {2,3,4,5,6}
a -= 2;  // a = {0,1,2,3,4}
a *= 3;  // a = {0,3,6,9,12}
a /= 2;  // a = {0,1.5,3,4.5,6}

Matrix b(2, 2, 1);  // b = {1,1;1,1}
Matrix c(2, 2, 2);  // c = {2,2;2,2}
b += b;  // b = {2,2;2,2}
c *= c;  // c = {4,4;4,4}
b /= c;  // b = {0.5,0.5;0.5,0.5}
c -= b;  // c = {3.5,3.5;3.5,3.5}
\end{code}

We also offer more linear algebra implementations in the \fileindex{lin\_alg.h}, \fileindex{lin\_alg.cc}, \fileindex{matpack\_math.h}, and \fileindex{matpack\_eigen.h} files.  The first two of these files support operations such as LU decomposition, solving of linear systems, matrix inversions and diagonalizations, matrix exponents, and similar.  The \fileindex{matpack\_math.h} file contains matrix-vector, matrix-matrix, and dot-product like features as well as a few statistical functions.  The \fileindex{matpack\_eigen.h} file wraps vectors and matrices with the Eigen3 library so that more direct math works.  Be careful using these convenience functions as they can slow down code that would be faster if the direct LAPACK-call is used.

\subsubsection{Matrix multiplication:}
\begin{code}
// Matrix-Vector:
Vector b(a.nrows()), c(a.ncols());
mult(b,a,c);                           // b = a * c

// Matrix-Matrix:
Matrix d(a.nrows(),5), e(a.ncols(),5);
mult(d,a,e);                           // d = a * e
\end{code}

Note, that the result is put in the first argument, consistent with
the general ARTS policy, but different from the old MTL based
multiplication function. Furthermore note, that as you can see from
the first example, a Vector is always considered to be a 1-column
Matrix.

\textbf{Important: The matrices or vectors that you give for the three
arguments must not overlap, or you will get garbage.} In particular,
this means that
\begin{code}
mult(x,y,x);            // x = y * x FORBIDDEN!!!
\end{code}
does not work. No, even worse: It works, but it gives the wrong
result.  The reason for this behavior is that the result is
constructed in the first argument variable. If that is also an input
variable it will change while it is multiplied, which will lead to a
different result.  There is no efficient way to detect overlap, so the
only way to allow input and output arguments to be identical would be
to use another internal dummy variable to store the result. However,
this would be much less efficient.

Another thing: You can use transpose, of course. These two examples
should obviously give the same result:

\begin{code}
// Define b and c as in first example above.
mult(c,transpose(a),b);                      // c = a' * b

// Vector-Matrix:
mult(transpose(c),transpose(b),a);           // c' = b' * a
\end{code}

\subsection{Notes on minimizing runtime overhead}
This is just a list of reminders that anyone coming back to read this should be aware of.  There is
no specific order to these recommendations, and they are probably hardware bound, so take these with
a grain of salt but look at the list as a "best practices" example.

\paragraph{Exhaustive or not?}  During the implementation of the modern iteration of \shortcode{matpack}, we
found that exhaustive views of the data often perform between 2-5 times faster than strided views.  The cost
is of course that you have to implement the function twice if you even end up having to call it with a
strided view.  Your choice!


\section{Arrays}
%-------------------------------------------------------------------------
\label{sec:matpack:arrays}

The template class \typeindex{Array} can be used to make arrays out of
anything. I do not know a good definition for `array', but I guess
anybody who has written a computer program in any programming language
is familiar with the concept. Of course, it is rather similar to the
concept of a Vector, just missing all the mathematical functionality
like Matrix-Vector multiplication and sub-range access.

The implementation of our \shortcode{Array} class is based on the STL class
\shortcode{std::vector}, whereas the implementation of our \builtindoc{Vector}
class is done from scratch. So the two implementations are completely
independent. Nevertheless, I tried to make \shortcode{Array} behave
consistently with \builtindoc{Vector}, as much as possible. There are a number
of important differences, though, hopefully sufficiently explained in
this part. A short summary of important differences:

\begin{itemize}
\item An Array can contain elements of any type, whereas a Vector
  always contains elements of type Numeric.
\item No mathematical functionality for Array (no sub-ranges (nothing
  like VectorView); no +=, -=, *=, /=; no scalar product; no
  \shortcode{transform} function; no \shortcode{mult} function; no
  \shortcode{transpose} function).
\item On the other hand, resizing (for example adding to the end) of
  an Array is ok. (See the \shortcode{push\_back} method below.) It is still
  rather expensive, though, at least for large Arrays. 
\end{itemize}

\subsection{Constructing an Array}
%-------------------------------------------------------------------------
You can construct an object of an Array class like this:

\begin{code}
Array<Index>  a;        // Empty Array of class Index.

Array<String> b(5);     // String Array with 5
                        // elements. Without initialization, 
                        // elements contain random values.
Array<String> c(5,"x"); // The same, but fill with "x".

Array<Index>  d=a;      // Make d a copy of a;
Array<String> a{"ARTS",
                "is",
                "great"}; // Creates an array of String
                          // with these 3 elements.
\end{code}

There are already a lot of predefined Array classes. The naming
convention for them is: \typeindex{ArrayOfIndex}, \typeindex{ArrayOfString},
etc.. Normally you should use these predefined classes. But if you want
to define an Array of some uncommon type, you can do it with `$<>$',
as in the above examples.

\begin{code}

\end{code}

\subsection{What you can do with an Array}
%-------------------------------------------------------------------------

All examples below assume that \shortcode{a} is an ArrayOfString.

\subsubsection{Resize:}
\begin{code}
a.resize(5);
\end{code}

This adjusts the size of \shortcode{a} to 5. Resizing is more efficiently
implemented than for Vector, but still expensive.

\subsubsection{Get the number of elements:}
\begin{code}
cout << a.nelem();  // Just as for Vector.
\end{code}

In particular, note that the return type of this method is
\builtindoc{Index}, just as for Vector. This is an extension compared to
std::vector, which just has a method \shortcode{size()} that returns the
positive integer type \shortcode{size\_t}.

\subsubsection{Element access:}
\begin{code}
cout << a[3];   // Print 4th element.
a[0] = "Hello"; // Assign string "Hello" to first element.
\end{code}

In other words, this works just like for Vector.

\subsubsection{Copying Arrays:}

This works also the same as for Vector. The size of the target must
match! In this respect, I have modified the behavior with respect to
the underlying std::vector, which has different copy semantics.

\subsubsection{Assigning a scalar of the base type:}
\begin{code}
a = "Hello";    // Assign string "Hello" to all elements.
\end{code}

\subsubsection{Append to the end:}
\begin{code}
a.push_back("Hello"); // Adds this new element at the
                      // end of a.
\end{code}

This can be an expensive operation, especially for large Arrays.
Therefore, use it with care. Actually, the \funcindex{push\_back} method
comes from the \shortcode{std::vector} class that Array is based on. You
can do a lot more with \shortcode{std::vector}, all of which also works
with \shortcode{Array}. However, to explain the Standard Template Library
is beyond the scope of this text. You can read about it in C++ or even
dedicated STL textbooks.

\section{Sparse matrices}
%-------------------------------------------------------------------------
\label{sec:matpack:sparse}

The class \typeindex{Sparse} implements the mathematical concept of a
matrix, same as Matrix does, but the data is stored in a
different manner. Sparse offers a memory saving storage when
most of the matrix is filled with zeros. This means that:
\begin{itemize}
\item A Sparse contains floating point values of type \builtindoc{Numeric}.
\item The values are arranged in rows and columns in the same ways as for
  ordinary matrices, in \emph{row-major} order.
\item A Sparse can be multiplied with a Vector, a Matrix or with another
  Sparse.
\item There exist no views for Sparse.
\item Resizing a Sparse is expensive and should be avoided.
\end{itemize}

To calculate the maximum number of non-zero elements for efficient storage,
take the product of number of columns and number of rows, subtract the
number of columns plus one and then divide by two,
\mbox{($nnz \leq 0.5\times (ncols\times nrows - (ncols+1)$).}

\subsection{Constructing a Sparse}
%-------------------------------------------------------------------------
You can construct an object of class Sparse in any of these ways:

\begin{code}
Sparse a;          // Create empty Sparse.
Sparse b(3,4);     // Create Sparse with 3 rows
                   // and 4 columns. When
                   // created like this it will
                   // contain only zeros, i.e.
                   // be an empty Sparse.

Sparse d=c;        // Make d a copy of c.
\end{code}

\subsection{What you can do with a Sparse}
%-------------------------------------------------------------------------
All examples below assume that \shortcode{a} is a Sparse.

\subsubsection{Identity matrix:}
\begin{code}
a.resize(10,10);
id_mat(a);
\end{code}
This sets \shortcode{a} to be the identity matrix of size $10 \times 10$ (10 rows
and 10 columns). Using this function is much faster than setting the
diagonal elements to one by yourself. Note that a must be a square matrix.

\subsubsection{Resize:}
\begin{code}
a.resize(5,10);
\end{code}
This makes \shortcode{a} a $5 \times 10$ Sparse (5 rows, 10 columns). Note that the
previous content will be completely lost. The new Sparse will be empty.

\subsubsection{Get the number of rows, columns or non-zero elements:}
\begin{code}
cout << a.nrows();
cout << a.ncols();
cout << a.nnz();
\end{code}

\subsubsection{Element access:}
There are two different ways to access individual elements. One used for
read only and one for read and write. The distinction is necessary since
the read and write method creates elements if they don't already exist.
Note that we use 0-based indexing. For reading only use:
\begin{code}
cout << a.ro(3,4);  // Print that element. If it
                    // it doen't exist a zero will
                    // be printed.
cout << a(0,0);     // Short version of the above.
\end{code}

For reading and writing, such as assigning values to elements, use:
\begin{code}
a.rw(0,0) = 1.5;    // Assigns the value 1.5 to the
                    // first row and first column.
cout << a.rw(0,0);  // Also returns the value of the
                    // first row and first column,
                    // if the element doesn't exist
                    // it will be created and set
                    // to zero.
\end{code}

\subsubsection{Copying Matrices:}
\begin{code}
Sparse b;
b = a;
\end{code}
%
The copying of matrices is implemented as deep copy. That means that the complete
object is duplicated including all elements in the matrix. The resulting matrices
are completely independent of each other, but depending on a this may require
considerable amount time and memory.

\subsubsection{Transpose:} The function \shortcode{transpose} works a bit
differently for Sparse than for Vector and Matrix. This is due to the
fact that we don't have any views for Sparse. Thus,
\shortcode{transpose} for a Sparse creates a new Sparse variable that
contains the transpose of the original Sparse, whereas
\shortcode{transpose} for a Matrix just creates a transposed view of
the original Matrix.

The target variable for the transposed Sparse has to have the right
dimensions before the function is called.
\begin{code}
Sparse b(a.ncols(),a.nrows());
transpose(b,a);     // Make b the transpose of a.
                    // Note the argument order!
\end{code}

\subsubsection{Matrix addition and subtraction:}

The sums and differences of sparse matrices \textbf{with the same
dimensions} can be computed as follows:
%
\begin{code}
Sparse b(a.nrows(),a.ncols());
Sparse c(a.nrows(),a.ncols());

add( c, a, b ); // c = a + b
a += b;       // a = a + b

sub( c, a, b ); // c = a - b
a -= b;       // a = a - b
\end{code}
%

\subsubsection{Scaling of sparse matrices:}

Sparse matrices can be scaled by scalar factors as follows:
%
\begin{code}
a *= 2.0;  // a = 2.0 * a
a /= 2.0;  // a = 0.5 * a
\end{code}
%
Note that the \shortcode{/=} scales the matrix by the reciprocal of the
given scalar factor.

\subsubsection{Matrix multiplication:}
\begin{code}
// Sparse-Vector
Vector b(a.nrows()), c(a.ncols());
mult(b,a,c);        // b = a * c

// Sparse-Matrix
Matrix d(a.nrows(),5), e(a.ncols(),5);
mult(d,a,e);        // d = a * e

// Sparse-Sparse
Sparse f(a.nrows(),5), g(a.ncols(),5);
mult(f,a,g);        // f = a * g
\end{code}
The result is put in the first argument, consistent with the Matrix class.
Note that for the Sparse -- Matrix multiplication the output is a Matrix.
\textbf{Important: As for Matrix, the matrices or vectors
that you give for the three arguments must not overlap, or you will get
garbage.}




%%% Local Variables:
%%% mode: latex
%%% TeX-master: "uguide"
%%% End:

