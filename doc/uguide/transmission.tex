\chapter{Transmission calculations}
 \label{sec:trans}

\starthistory
 120921 & A first, but incomplete, version (Patrick Eriksson). Partly based on
 text written by Bengt Rydberg.\\
\stophistory

\graphicspath{{Figs/transmission/}}

The term ``transmission calculations'' refers here to situations when
atmospheric and surface emission can be neglected. These calculations can be
divided into two main types. The first one is when just the transmission of the
atmosphere is diagnosed (Sec.~\ref{sec:transmission}). The observation geometry
is given exactly as for emission simulations, by a position and a
line-of-sight.

The second main type is radio link budgets (Sec.~\ref{sec:rlinks}). For
this case, the propagation path is determined solely by the position of the
transmitter and the receiver. That is, the user does not need to set a
line-of-sight of the sensor (receiver), it is determined by the transmitter
position. A characterisation of a radio link normally involves several
attenuation terms not encountered for passive measurements, such as free space
loss and defocusing. Faraday rotation is an additional physical mechanism that
is of special concern for active microwave devices (it can normally be
neglected at the higher frequencies used for passive observations).




\section{Pure transmission calculations}
%===================
\label{sec:transmission}

This section discusses the \wsmindex{iyTransmissionStandard} workspace method,
that is relevant if you want to calculate the transmission through the
atmosphere for a given position and line-of-sight. The set-up is largely the
same as for simulations involving emission, such as that the observation
geometry is defined by \wsvindex{sensor\_pos} and \wsvindex{sensor\_los} (or
\wsvindex{rte\_pos} and \wsvindex{rte\_los} if \builtindoc{iyCalc} is used).
The first main difference is that \wsaindex{iy\_transmitter\_agenda} replaces
\wsaindex{iy\_space\_agenda} and \wsaindex{iy\_surface\_agenda}. The second
main difference is that handling of cloud scattering is built-in and the need
to define \wsaindex{iy\_cloudbox\_agenda} vanishes. These differences appear
inside \wsaindex{iy\_main\_agenda}. Further,
\builtindoc{blackbody\_radiation\_agenda} can be left undefined).

As for emission measurements (Sec.~\ref{sec:fm_defs:calcproc},
Algorithm~\ref{alg:fm_defs:iyCSagenda}) the first main operation is to
determine the propagation path through the atmosphere, but this is here done
without considering the cloud-box (it is simple deactivated during this step).
The possible ``radiative backgrounds'' are accordingly the surface or space,
i.e.\ where the propagation path starts. 

The next step is to call \builtindoc{iy\_transmitter\_agenda}. It should be
noted that the same agenda is called independently if the radiative background
is space or the surface. It is up to the user to decide if these two cases
shall be distinguished in some manner (no workspace method for this task exists
yet). For these calculations, the standard choice for
\builtindoc{iy\_transmitter\_agenda} is \wsmindex{MatrixUnitIntensity}. If this
workspace method is used, the output of \builtindoc{iyTransmissionStandard}
shows you the fraction of unpolarised radiation that is transmitted through the
atmosphere, and the polarisation state of the transmitted part.

The radiative transfer expression applied is
(cf.~Eq.~\ref{eq:fm_defs:vrte_step})
\begin{equation}
  \label{eq:trans:vrte}
  \StoVec_{i+1} = e^{-\bar{\ExtMat}s} \StoVec_i 
\end{equation}
where the extinction matrix is determined in the same manner as for emission
cases (Sec.~\ref{sec:fm_defs:rad_bkgr}). In situations where the matrix
\ExtMat\ is diagonal, the scalar form shown in Eq.~\ref{eq:fm_defs:rte_step2})
is used.

The method determines automatically if any part of the propagation path is
inside the cloud-box (if active). If this is the case, particle extinction is
included in \ExtMat, following the same path step averaging as applied for
pure absorbing species. For this method, scattering is purely a loss mechanism,
there is no gain by scattering into the line-of-sight

A related concern is the treatment of the surface. In standard usage of the
method, there is no signal reflected by the surface. The radiative transfer
calculations are started at the surface.

See the built-in documentation (\builtindoc{iyTransmissionStandard}) for a full
list of possible auxiliary quantities. These data include quantities that make
it possible to determine the transmission for different parts of the
propagation path. For example, the state of \builtindoc{iy} at each point of
the propagation path can be extracted, and also the transmission matrix
(Eq.~\ref{eq:rte:transmat}) for each path step is accessible.





\section{Radio link calculations}
%===================
\label{sec:rlinks}

\subsection{Practical usage}
%
This type of calculations is handled by \wsmindex{iyRadioLink}. A sensor
position must be specified, as usual. The ``sensor'' takes the place as the
receiver of the radio link. The observation angles of the receiver are not user
input (if any given they are just ignored), they are determined by the position
of the transmitter. This position is given as \wsvindex{transmitter\_pos} or
\wsvindex{rte\_pos2}, dependent on if \builtindoc{yCalc} or \builtindoc{iyCalc}
is used. Specification of receiver and transmitter properties, beside their
positions, is discussed in Section~\ref{sec:rlinks:oview}. The quantities
returned by \builtindoc{iyRadioLink} are also described in
Section~\ref{sec:rlinks:oview}.

It is allowed to use \builtindoc{iyRadioLink} together with pure geometrical
propagation paths, but this option does not give realistic results as the this
set-up misses the impact of ``defocusing'' (Sec.~\ref{sec:rlink:defoc}). With
refraction, the propagation path can not be determined in a strict analytical
manner and some search algorithm is required. ARTS contains so far just a quite
simple algorithm for this purpose, activated by applying
\wsmindex{ppathFromRtePos2} for \wsaindex{ppath\_agenda}. In short, the search
is started by determining the geometrical path. The closest distance between
this path and the receiver is converted to an estimated correction for the
``line-of-sight'' of the transmitter, the corresponding refracted path is
determined and a new correction term is obtained. The procedure is repeated
until a (stringent) hard-coded convergence criterion is fulfilled.

   
\subsection{Definitions}
%===================
\label{sec:rlinks:oview}

The radiance law is not applicable for the signal from a point source.
The power received, $p_r$, by an antenna with an effective aperture size
$A_e$, located a distance $l$ from a transmitter, with power
$p_t$ and gain $g_t$, can be expressed as \citet[e.g.][]{ippolito:satco:08}
\begin{equation}
  \label{eq:rlink:start}
  p_r(l) = t_a\frac{p_tg_t}{4\pi l^2}A_e,
\end{equation}
where $t_a$ is the transmission due to losses caused by the atmosphere. This
total atmospheric loss ($t_a$) is the product between losses due to defocusing,
$t_d$ (Sec.~\ref{sec:rlink:defoc}) and gaseous and particle extinction, $t_e$
(Sec.~\ref{sec:rlink:atmext}):
\begin{equation}
  t_a = t_d t_e.
\end{equation}
The factor
\begin{equation}
  \label{eq:rlink:fspl}
  t_f = \frac{1}{4\pi l^2}
\end{equation}
represents the pure geometrical dilution of the power, following the inverse
square law. This effect is encountered also for propagation in free space, and,
accordingly, it is denoted as the \emph{free space loss}. Please note that
other definitions of this quantity exists, see Section~\ref{sec:rlinks:free}.

With these definitions, Equation~\ref{eq:rlink:start} can be written as
\begin{equation}
  \label{eq:rlink:pr}
  p_r(l) = t_d t_e t_f p_t',
\end{equation}
where $p_t'=p_tg_tA_e$, a term discussed further in
Section~\ref{sec:rlinks:free}.

The method \wsmindex{iyRadioLink} reports $p_r$ as \builtindoc{iy}, i.e.\ $p_r$
is treated as the main quantities. All loss terms ($t_d$, $t_e$ and $t_f$) can
be obtained as auxiliary data (\builtindoc{iy\_aux}). The auxiliary variables
include also extra path delay, bending and Faraday rotation. Some of these
terms can be obtained as attenuation/rotation (per length) along the
propagation path. 

Note the distinction between ``loss'' and ``attenuation''. The first term
refers here to path-integrated transmission factors, while the second term
refers to local attenuation coefficients. That is, losses are dimensionless
values, between 0 and 1, while the unit of attenuation is $1/m$. See the
following sections for details.

The quantity returned in \builtindoc{iy} is also what ends up in
\builtindoc{y}, outputted by \builtindoc{yCalc}. For increased flexibility, the
method \FIXME{Add name when implemented} allows the user to replace the content
of \builtindoc{iy} with some of the auxiliary variables. For example, for radio
link calculations bending angle can frequently be seen as the primary
observation, and this case can be handled by \FIXME{Add name when implemented}.
The constrain for allowing this swap of variables is that the auxiliary
variable to replace $p_r$ must have the same data dimensions. In practice, this
means that the auxiliary variable is a function of frequency, possibly being a
Stokes vector for each frequency. Variables fulfilling this constrain are:
defocusing loss ($t_d$), atmospheric extinction loss ($t_e$), free space loss
($t_f$), bending angle, extra path delay, and (total) Faraday rotation.


\subsection{Free space loss and sensor characteristics}
%===================
\label{sec:rlinks:free}

It is repeated that (total) free space loss is in ARTS defined as given
above in Equation~\ref{eq:rlink:fspl}:
\begin{displaymath}
  t_f = \frac{1}{4\pi l^2}.
\end{displaymath}
It shall be noted, \underline{and stressed}, that this deviates from what
appears to be a more common (even standard?) definition of free space loss,
that is $\left(\Wvl/(4 \pi l)\right)^2$, where \Wvl\ is the wavelength of the
signal. This expression encompasses the relationship between the receiver's
effective antenna area and its gain ($A_e=g_r\Wvl^2/(4\pi)$). This later
definition was avoided to keep atmospheric and sensor effects clearly separated
in ARTS,

Hence, the definition of free space loss has consequences for how to specify
sensor characteristics. A possible choice is to treat pure multiplicative
factors such as $p_t$, $g_t$ and $A_e$ separately, and let ARTS handle the pure
radiative propagation effects. In this case, $p_t'$ (Eq.~\ref{eq:rlink:pr})
shall be set to unity (\wsmindex{MatrixUnitIntensity} handles this case). If
e.g. Faraday rotation or particle scattering is considered,
\builtindoc{stokes\-dim} must be $>1$ and also the polarisation of the
transmitted signal is of concern, and must be included in the specification of
$p_t'$ (e.g.\ $p_t'=[1,\pm1,0,0]$ when V or H is transmitted). The next step
towards a more complete treatment of the transmitter should be to include $p_t$
and $g_t$ in $p_t'$. In any case, \builtindoc{iyRadioLink} expects
\wsaindex{iy\_transmitter\_agenda} to return $p_t'$, for each frequency in
\builtindoc{f\_grid}.

So far ARTS has no dedicated support to handle characteristics of receivers in
radio link configurations. The first step here should be to consider $A_e$, but
the simplest way to achieve this is to include $A_e$ in $p_t'$ (and it is here
where the definition of free space loss comes into play, as with the standard
definition $p_t'$ should instead include $g_r$). Some functionality developed
for passive sensors could also be of use for radio links, but this has not yet
been tested practically (even less validated) and these options are not
commented further.

Setting $\Mpi=p_r$ (to be consistent in the nomenclature used elsewhere) and
taking the derivative of \(\Mpi(s)\) gives
\begin{equation}
\label{eq:fsplarts}
 \frac{\DiffD\Mpi(s)}{\DiffD s}=-\frac{2}{s}I(s).
\end{equation}
This equals the Beer-Lambert law with an attenuation coefficient of $2/s$. With
other words, the free space loss can be also treated as an ordinary extinction
term, and when ``free space attenuation'', $k_f$ is requested as an auxiliary
variable it is calculated as
\begin{equation}
 k_f=\frac{2}{s}.
\end{equation}
This extinction coefficient can be seen as non-linear, as it varies with the
distance from the transmitter. If the transmitter is placed inside the
atmosphere, $k_f$ becomes infinity for the path point corresponding to $l=0$.
The $k_f$-coefficient is not dependent on the definition differences of (total)
free space loss discussed above.


\subsection{Extra path delay}
%
In the absence of an atmosphere a signal sent from a transmitter in the
direction towards a receiver would follow a straight line, i.e. the shortest
geometric distance between the two instruments. Moreover, the signal would
propagate with the speed of light. The geometric distance, $l_g$, between the
transmitter and receiver is
\begin{equation}
l_g\equiv\int_{\mathrm{geometric}}1\DiffD s
\end{equation}
In the presence of an atmosphere, two changes follow,
the speed of the signal is retarded and the ray path gets 
bent. The optical path length is defined as
\begin{equation}
l_o\equiv\int_{\mathrm{ray}}\Rfr(s)\DiffD s.
\end{equation}
The geometrical length of the bent ray path is
\begin{equation}
l_r\equiv\int_{\mathrm{ray}}1\DiffD s.
\label{eq:rlink:lr}
\end{equation}
to recall what you probably heard during our physics lessons, some quotes from
the Wikipedia entry on optical path length: \emph{Fermat's principle states
  that the path light takes between two points is the path that has the minimum
  optical path length. An electromagnetic wave that travels a path of given
  optical path length arrives with the same phase shift as if it had travelled
  a path of that physical length in a vacuum.}

The total delay, compared if there would be vacuum between transmitter and
receiver, expressed in units of length, is
\begin{equation}
d \equiv l_o-l_g \ge 0,
\end{equation}
which can be expressed also as
\begin{equation} 
d = (l_o-l_r) + (l_r-l_g) = d_{n}+d_{g},
\end{equation}
where $d_{n}$ is the along-path delay due to decreased propagation velocity,
and $d_{g}$ is delay due to deviation from geometric propagation (refraction).

The (total) extra path delay, in units of seconds, is
\begin{equation} 
\Delta t = \frac{d}{c},
\end{equation}
and is provided by \builtindoc{iyRadioLink} as an auxiliary variable. In
addition, $d_n/c$ can be obtained, then denoted as ``local path retardation''
and have the unit s/m.


\subsection{Bending angle}
%
The bending angle is a measure on the deviation from geometric propagation.
Using the nomenclature defined in Figure~\ref{fig:rlink:bangle}, the bending
angle, $\alpha$, can be calculated as \citep[e.g.][Eq.~6]{schreiner:99}
\begin{displaymath}
  \alpha = \phi_{\mathrm{GPS}} + \phi_{\mathrm{LEO}} + \theta - \pi.
\end{displaymath}
This equation is derived assuming what in ARTS is denoted as a 1D atmosphere.
ARTS operates with zenith angles and the equation above becomes
\begin{figure*}[!tb]
 \begin{center}
  \includegraphics*[width=0.7\hsize]{bangle.png}\\
  \caption{Illustration of the bending angle of a satellite-to-satellite radio
    occultation \citep[copied from][Fig.~2]{schreiner:99}.}
 \end{center}
 \label{fig:rlink:bangle}
\end{figure*}
\begin{equation} 
  \alpha = \aZntAng{t} - \aZntAng{r} + \theta,
  \label{eq:rlink:bangle}
\end{equation}
where \aZntAng{t} and \aZntAng{r} is the zenith angle of propagation path, at
the transmitter and the receiver, respectively. This equation is applied for
all atmospheric dimensionalities, hence, any bending in the horizontal plane,
that can occur for 3D, is neglected.

A remark here is that the ARTS' zenith angles for the path are the ones to
observe the propagating radiation, and
e.g.\ $\aZntAng{r}=\pi-\phi_{\mathrm{LEO}}$. Equation~\ref{T-eq:ppath:dlat} in
\theory (note the difference in notation) shows that that without refraction
$\theta=\aZntAng{t} - \aZntAng{r}$, and Equation~\ref{eq:rlink:bangle} is
consistent with a zero bending angle for propagation in vacuum. Consequently,
ARTS should theoretically return a bending angle of zero if geometrical path
calculations are selected, but due to limited calculation precision a small
deviation from zero can be found.

The bending along the path [rad/m] is expressed as
\begin{equation} 
  \frac{\DiffD \alpha}{\DiffD s} = \frac{\DiffD \ZntAng}{\DiffD s} - 
                                   \frac{\DiffD \theta }{\DiffD s},
  \label{eq:rlink:dbangleds}
\end{equation}
where $\DiffD\ZntAng/\DiffD s$ is the change in zenith angle along the path,
and $\DiffD\theta/\DiffD s$ is the same for the angular distance to the
transmitter, both calculated by local polynomial fits of the angles.



\subsection{Defocusing}
\label{sec:rlink:defoc}
%
A special effect for radio links is defocusing. This effect is caused by the
fact that refraction varies over the wavefront. In short, defocusing occurs if
neighbouring ray-paths diverge more quickly than for free space propagation.
This is the typical situation for limb sounding and due to the vertical
variation, causing an additional divergence of the transmitted power. The
opposite, focusing, is also possible. In fact, already for an ideal spherically
symmetric atmosphere a focusing effect takes place in the horizontal
dimension, caused by the curvature of the planet. In this case, the middle
point of the beam experiences the highest refractive index and has then the
lowest propagation speed. This, in addition to symmetry around the middle
point, counteracts partly the free space divergence of the power.

The defocusing effect of a transmitted signal with intensity \(I_{0}\) can be
described as \citep{haugstad:78:turbu,kursinski:00:thegp}
\begin{equation}
\label{eq:foc1}
I=I_{0}M_{\alpha}M_{\beta},
\end{equation}
where \(M_{\alpha}\) and \(M_{\beta}\) is the vertical and horizontal
defocusing factors, respectively. In general, the defocusing effect is fairly
complex in a modelling perspective, although for special cases the defocusing
factors can be calculated analytically. Such analytic expressions can be used
for "test cases" of more general solutions. For a signal transmitted in a limb
geometry through a spherical symmetric atmosphere (i.e.\ 1D), with an
exponential scale height \(H\), \citep{haugstad:78:turbu}
\begin{eqnarray}
 M_{\alpha} &=& \left[1+\frac{\alpha L_{r}}{H} \right]^{-1}, \nonumber \\
 M_{\beta}  &=& \left[1-\frac{\alpha L_{r}}{R} \right]^{-1}, \nonumber
\end{eqnarray}
where \(\alpha\) is the bending angle (Eq.~\ref{eq:rlink:bangle}), \(L_{r}\) is
the distance from the tangent point to the receiver, and \(R\) is the planet
radius. An alternative form of the equations above, also
also valid for an atmosphere with a non-exponential scale height, given in
\citep{kursinski:00:thegp}, is
\begin{eqnarray}
 \label{eq:kur1}
 M_{\alpha} &=& \left[1+\frac{\DiffD \alpha}{\DiffD a} 
               \frac{{L_{t}L_{r}}}{L_{r}+L_{t}} \right]^{-1}, \\
\label{eq:kur2}
 M_{\beta} &=& \left[1-\frac{\alpha}{R}
              \frac{{L_{t}L_{r}}}{L_{r}+L_{t}} \right]^{-1},
\end{eqnarray}
where \(L_{t}\) is the distance from the tangent point to the receiver and
\(a\)=\(\Rfr\Rds\sin(\ZntAng)\) is denoted as the impact parameter (and is
equal to the ``propagation path constant'' defined in
Eq.~\ref{T-eq:ppath:snellspherical} of \theory).

Default in \wsmindex{iyRadioLink} is to calculate defocusing following
Equations~\ref{eq:kur1} and \ref{eq:kur2}. However, those expressions are not
general (as mentioned, they assume 1D) and do not allow calculation of a
defocusing attenuation term (i.e. the change in defocusing loos per length
unit). For these reasons, an alternative, pure numerical, algorithm has been
implemented and is being tested. In short, propagation paths are calculated for
small shifts in zenith angle, and also azimuth angle for 3D. The distance
between these paths are determined, for common lengths along the bent path
following Equation~\ref{eq:rlink:lr}. The found distances are compared to the
expected distances according to free space propagation, and the ratios can be
determined to total defocusing and local defocusing attenuation. See further
Section~\ref{sec:rlink:defoc2}.




\subsection{Atmospheric extinction}
\label{sec:rlink:atmext}
%
Atmospheric absorption and scattering are handled exactly as for
\builtindoc{iyTransmissionStandard}, this including to only handle scattering
out-of-the propagation path.



\subsection{Faraday rotation}
\label{sec:rlink:farrot}
%
Not yet implemented.



\subsection{Justification for pure numerical defocusing algorithm}
\label{sec:rlink:defoc2}
%
The derivations of, for example, \citet{haugstad:78:turbu} were made using the
framework of geometrical optics. In this framework, the equation describing the
propagation direction of a ray can be written as
\begin{equation}
\label{eq:eik}
\Rfr\frac{\DiffD \bf{\Rds}}{\DiffD s}=\nabla S,
\end{equation}
where \(S\) (often called eikonal) is a function describing the wavefront
(\(S(\bf{\Rds})\)=constant represents a wave front). According to
\cite{born:80} the intensity ratio of any two points along a ray can be
expressed as
\begin{equation}
\frac{I_{2}}{I_{1}}=\frac{\Rfr_{2}}{\Rfr_{1}}
\exp \left(-\int_{s_{1}}^{s_{2}}\frac{\nabla^{2}S}{n}ds\right),
\end{equation}
where the integrand 
or the along path contribution to the intensity variation is
\begin{eqnarray}
\label{eq:def4}
\frac{\nabla^{2}S}{n} &=& \frac{\nabla \cdot \nabla S}{n}
=\frac{\nabla \cdot \left(\Rfr\frac{\DiffD \bf{\Rds}}{\DiffD s}\right)}{\Rfr} \nonumber \\
& = & 
\frac{\nabla \Rfr}{\Rfr} \cdot \frac{\DiffD \bf{\Rds}}{\DiffD s}
+\nabla \cdot \frac{\DiffD \bf{\Rds}}{\DiffD s}=
\frac{1}{\Rfr}\frac{\DiffD \Rfr}{\DiffD s}+\nabla \cdot \frac{\DiffD \bf{\Rds}}{\DiffD s}. 
\end{eqnarray}
The first term on the right hand side of Equation~\ref{eq:def4}
is related to the along path variation of refractive index and it is 
an unproblematic term in a modelling perspective.
The second term on the right hand side of Equation~\ref{eq:def4}
is the divergence of the vector pointing in the direction 
of the ray.  If the signal initially is travelling in the \(x\)-direction,
we have for small bending angles \citep{haugstad:78:turbu}
\begin{equation}
\label{eq:raydiv}
\nabla \cdot \frac{\DiffD \bf{\Rds}}{\DiffD s}\approx
\frac{\DiffD \alpha}{\DiffD z}+\frac{\DiffD \beta}{\DiffD y}
\end{equation}
 where \(\alpha\) and  \(\beta\) is the bending angle in the \(xz\)-plane
and \(xy\)-plane, respectively (\(\alpha\) and  \(\beta\) can in this
case be seen as  the \(z\) and \(y\) component of the propagation
direction vector).
The bending
angle is an integrated measure of the refractive index in 
the atmosphere traversed by the ray. That is, at a given point
along the propagation path, the bending angle depends
on the history of the path, and hence also Equation~\ref{eq:raydiv}.
This means, for example, that if we only perform calculations
along the propagation path and do not consider the history of the path, 
it will be problematic to approximate the \(\alpha\) term in 
Equation \ref{eq:raydiv} as
\begin{equation}
\frac{\DiffD \alpha(x,y,z)}{\DiffD z}\approx
\frac{\alpha(x,y,z+\Delta z)-\alpha(x,y,z)}{\Delta z},
\end{equation}
as \(\alpha(x,y,z+\Delta z)\) will not be known.
A solution to handle this term analytically, that is, the required 
integrals to solve, in a spherically symmetric atmosphere (and for small bending angles) is given in \citet{haugstad:78:turbu}. 
During the literature review no suitable solution for ARTS for
the general case was found.
This means that an alternative solution to the problem will be considered.

The algorithm to be implemented in ARTS to handle defocusing will be based on
basic geometrical optical principles, although one approximation that
significantly reduces the computation complexity will be applied. The theory
behind the algorithm, the algorithm itself, and the approximation will now be
described.

If we consider the wavefront of a narrow tube of all rays proceeding from an
area element \(\Delta A_{1}\) to a corresponding area element \(\Delta A_{2}\)
we have from the intensity law of geometrical optics that
\begin{equation}
I_{1}\Delta A_{1}=I_{2} \Delta A_{2}.
\end{equation}
If we now consider the case that the wavefront is plane when passing \(\Delta
A_{1}\) and there is a gradient in refractive index in the orthogonal direction
to the path to \(\Delta A_{2}\). We then have that, at a distance \(s\) from
\(\Delta A_{1}\),
\begin{equation}
I_{2}(s)=I_{1}\frac{\Delta A_{1}}{\Delta A_{2}(s)}=I_{1}M(s),
\end{equation}
where \(M(s)\) can be seen as the overall defocusing factor. Furthermore, the
change in intensity (at \(s=0\)) due to defocusing would then be
\begin{eqnarray}
\label{eq:defocloss}
\frac{\DiffD I_{2}(s=0)}{\DiffD s} &=&
-I_{1}(s=0)\frac{\Delta A_{1}}{\Delta A_{2}^{2}(s=0)}\frac{\DiffD}{\DiffD s}\Delta A_{2}(s=0)  \nonumber \\
&=& 
-I_{2}(s=0)\frac{1}{\Delta A_{1}}\frac{\DiffD}{\DiffD s}\Delta A_{2}(s=0).
\end{eqnarray}
The algorithm to handle the along path defocusing in ARTS will be approximated
by an algorithm that at each point along the propagation path treats the
incoming wavefront of a "tube" of rays to be plane and orthogonal to
propagation direction. At each point we will calculate the defocusing loss (or
gain) for such a "tube" of rays. The along path defocusing will be modelled
according to Equation~\ref{eq:defocloss}, that is, similar to how extinction is
modelled. The loss term can, with the approximation applied, be calculated
without knowledge of the history of the ray-path.


The numerical methodology to handle or calculate the loss term will now be
described. To estimate the defocusing along the path, let say at point
\(P_{1}\) with a propagation direction vector \(\mathbf{t}_{1}\), we make the
assumption that the wavefront of a "tube" rays is flat when entering this
point. We will then calculate how neighbouring ray-paths around \(P_{1}\) will
bend and change the area of the "tube" as they proceed. This will be done by
considering the bending of four neighbouring rays with points in the plane
orthogonal to the propagation direction (the plane defined by the point
\(P_{1}\) and with normal vector \(\mathbf{t}_{1}\)). Two of these points
considered will be displaced by a distance \(\pm 1/2\Delta l\) in the
"vertical" direction from \(P_{1}\) and we denote these by \(P_{1north}\) and
\(P_{1south}\). We denote the distance between those points by \(\Delta
l_{v}\). The other two points (\(P_{1east}\) and \( P_{1west}\)) will have a
horizontally displacement of \(\pm 1/2\Delta l\) from \(P_{1}\) , and we denote
this distance by \(\Delta l_{h}\). To be clear, all of these five rays are
assumed to have \(\mathbf{t}_{1}\) as propagation direction vector at the point
of concern, and the area enclosing the incoming rays is
\begin{equation}
\Delta A_{1}= \Delta l_{v}\Delta l_{h}=\Delta l^{2},
\end{equation}
and the loss term then reads
\begin{eqnarray}
\frac{1}{\Delta A_{1}}\frac{\DiffD}{\DiffD s}\Delta A_{1} &=&
%\frac{\Delta l_{h}}{\Delta l_{v} \Delta l_{h}}\frac{\DiffD}{\DiffD s}\Delta l_{v}+
%\frac{\Delta l_{l}}{\Delta l_{v} \Delta l_{h}}\frac{\DiffD}{\DiffD s}\Delta l_{h}
\frac{1}{\Delta l_{v} }\frac{\DiffD}{\DiffD s}\Delta l_{v}+
\frac{1}{\Delta l_{h}}\frac{\DiffD}{\DiffD s}\Delta l_{h} \nonumber \\
&\approx& \frac{\Delta l_{v}(\Delta s)/\Delta l_{v}-1}{\Delta s}+
 \frac{\Delta l_{h}(\Delta s)/\Delta l_{h}-1}{\Delta s},
\end{eqnarray}
where \(\Delta s\) is a small path length step in the propagation direction of the centre ray. 
That is, the loss term can be separated into two terms (one for the vertical defocusing and
one for the horizontal defocusing).

For each ray we calculate new propagation direction vectors, with the aim to find out the vertical and horizontal widths 
(\(\Delta l_{v}(\Delta s)\) and \(\Delta l_{h}(\Delta s)\)) 
of our "tube" of rays after propagating the centre ray a distance \(\Delta s\).
We should calculate the widths of our "tube" of rays at the wavefront,
which means that we must take the refractive indexes around the
considered points into account.
We propagate the centre ray an optical path length
of \(\Delta s\)\(n_{1}\), where \(n_{1}\) is the refractive index
at point \(P_{1}\). 
The surrounding rays should then also
be propagated an optical path length of \(\Delta s\)\(n_{1}\),
before we "measure" the widths of the tube at the wavefront.
In this way, each ray will arrive  to their new position 
simultaneously in time (or with the same phase shift).
That means for example that if we propagate the centre ray a distance 
\(\Delta s\),
the ray passing \(P_{1north}\) should be propagated
a distance  \(\Delta s \cdot n_{1} / n_{1north}\), where 
\(n_{1north}\) is the refractive index at point \( P_{1north}\).
The new position of the ray passing \(P_{1north}\) is then accordingly
\begin{equation}
P_{2north}=P_{1north}+{\bf{t}}_{1north}\Delta s \frac{n_{1}}{n_{1north}},
\end{equation}
where \(\mathbf{t}_{1north}\) is the updated propagation direction vector 
for the ray leaving point \(P_{1}\). The new positions \(P_{2south}\),\(P_{2east}\), and \(P_{2west}\)
are found in an analogous manner. 
We then have that
\begin{equation}
\Delta l_{v}(\Delta s)=|P_{2north}-P_{2south}|,
\end{equation}
where \(| |\) measures the distance between the two points, and 
the horizontal width is found in analogous manner.
