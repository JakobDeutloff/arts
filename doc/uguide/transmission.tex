\chapter{Transmission calculations}
 \label{sec:trans}


\starthistory
 120831 & Started (Patrick Eriksson).\\
\stophistory

The term ``transmission calculations'' refer here to situations when
atmospheric and surface emission can be neglected. These calculations can be
divided into two main types. The first one is when just the transmission of the
atmosphere is diagnosed (Sec.~\ref{sec:transmission}). The observation geometry
is here given exactly as for emission simulations, by a position and a
line-of-sight.

The second main type is radio link budgets (Sec.~\ref{sec:radiolinks}). In this
case, a core problem is to determine the propagation path between transmitter
and receiver. That is, the line-of-sight of the sensor (receiver) is not set by
the user, it is determined by the transmitter position. A characterisation of a
radio link normally involves several attenuation terms not encountered for
passive measurements, such as free space loss and defocusing.




\section{Pure transmission calculations}
%===================
\label{sec:transmission}

This section discusses the \wsmindex{iyTransmissionStandard} workspace method
and is relevant if you want to calculate the transmission through the
atmosphere for a given position and line-of-sight. The set-up is largely the
same as for simulations involving emission, such as that the observation
geometry is defined by \wsvindex{sensor\_pos} and \wsvindex{sensor\_los} (or
\wsvindex{rte\_pos} and \wsvindex{rte\_los} if \builtindoc{iyCalc} is used).
The first main difference is that \wsaindex{iy\_transmitter\_agenda} replaces
\wsaindex{iy\_space\_agenda} and \wsaindex{iy\_surface\_agenda}. The second
main difference is that handling of cloud scattering is built-in and the need
to define \wsaindex{iy\_cloudbox\_agenda} vanishes.

\dots 

\wsmindex{MatrixUnitIntensity} should be the standard choice for
\wsaindex{iy\_transmitter\_agenda}.

\builtindoc{blackbody\_radiation\_agenda} can be left undefined).





\section{Radio calculations}
%===================
\label{sec:radiolinks}


\subsection{Determining the propagation path}
%===================
\label{sec:radiolinks:ppath}

Use \wsmindex{ppathFromRtePos2}.


\subsection{Free space loss}
%===================
\label{sec:radiolinks:ppath}
The radiance law is not applicable for the signal from a point source.
According to \citet{ippolito:satco:08}, the signal intensity at a distance
\(s\), in free space, from a transmitting antenna, with a gain \(g_{t}\) is
defined as
\begin{equation}
 I(s)=\frac{p_{t}g_{t}}{4\pi s^{2}},
\end{equation}
where \(p_{t}\) is the transmitted power. The power \(p_{r}\) received by a
receiving antenna with an effective aperture \(A_{e}\) and gain \(g_{r}\)
located a distance \(s\) from the transmitting antenna is
\begin{equation}
\label{eq:fsl}
p_{r}(s)=\frac{p_{t}g_{t}}{4\pi s^{2}}A_{e}=\frac{p_{t}g_{t}}{4\pi s^{2}}
\frac{g_{r}\Wvl^{2}}{4\pi}=
p_{t}g_{t}g_{r}\left(\frac{\Wvl}{4\pi s} \right)^{2},
\end{equation}
where the last term  
to the right (or actually the reciprocal) is denoted as the
free space path loss (\(F\)):
\begin{equation}
F\equiv\frac{p_{t}g_{t}g_{r}}{p_{r}}=\left(\frac{4\pi s}{\Wvl}\right)^{2}.
\end{equation}
Taking the derivative of \(\Mpi(s)\) gives
\begin{equation}
\label{eq:fsplarts}
 \frac{\DiffD\Mpi(s)}{\DiffD s}=-\frac{2}{s}I(s).
\end{equation}
This equals the Beer-Lambert with an absorbance of $2/s$. With other words, the
free space loss can be treated as an ordinary extinction term. However, it
should be noted that this ``extinction coefficient'' is non-linear, it varies
with the distance from the transmitting antenna. 


\subsection{Extra path delay}
%
An electromagnetic wave that travels a path of given optical path 
length arrives with the same phase shift as if it had travelled a 
path of that physical length in a vacuum. 
In the absence of an atmosphere a signal
sent from a transmitter in the direction towards a receiver would
follow a straight line, i.e. the shortest geometric distance
between the two instruments. Moreover, the signal would propagate
with the speed of light.
The geometric distance \(D\) between the transmitter and receiver is
\begin{equation}
D\equiv\int_{straight\,line}1\DiffD s
\end{equation}
In the presence of an atmosphere, two changes follow,
the speed of the signal is retarded and the ray direction gets 
bent. 
The apparent bent ray-path length is
\begin{equation}
L\equiv\int_{bent\,ray-path}\Rfr(s)\DiffD s.
\end{equation}
One can also define the geometric bent ray-path length as
\begin{equation}
G\equiv\int_{bent\,ray-path}1\DiffD s.
\end{equation}
The total delay expressed in units of length is 
\begin{equation}
d \equiv L-D \ge 0,
\end{equation}
which can also be expressed as
\begin{equation} 
d=d_{a}+d_{g}=(L-G)+(G-D)
\end{equation}
where \(d_{a}\) is the along-path delay
and \(d_{g}\) is the geometric delay.


%%% Local Variables: 
%%% mode: latex
%%% TeX-master: "arts_user"
%%% End: 
