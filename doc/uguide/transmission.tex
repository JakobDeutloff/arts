\chapter{Transmission calculations}
 \label{sec:trans}


\starthistory
 120921 & A first, but incomplete, version (Patrick Eriksson).\\
\stophistory

The term ``transmission calculations'' refers here to situations when
atmospheric and surface emission can be neglected. These calculations can be
divided into two main types. The first one is when just the transmission of the
atmosphere is diagnosed (Sec.~\ref{sec:transmission}). The observation geometry
is given exactly as for emission simulations, by a position and a
line-of-sight.

The second main type is radio link budgets (Sec.~\ref{sec:radiolinks}). For
this case, the propagation path is deermined solely by the position of the
transmitter and the receiver. That is, the user does not need to set a
line-of-sight of the sensor (receiver), it is determined by the transmitter
position. A characterisation of a radio link normally involves several
attenuation terms not encountered for passive measurements, such as free space
loss and defocusing. Faraday rotation is an additional physical mechanism that
is of special concern for active microwave devices (it can normally be
neglected at the higher frequencies used for passive observations).




\section{Pure transmission calculations}
%===================
\label{sec:transmission}

This section discusses the \wsmindex{iyTransmissionStandard} workspace method,
that is relevant if you want to calculate the transmission through the
atmosphere for a given position and line-of-sight. The set-up is largely the
same as for simulations involving emission, such as that the observation
geometry is defined by \wsvindex{sensor\_pos} and \wsvindex{sensor\_los} (or
\wsvindex{rte\_pos} and \wsvindex{rte\_los} if \builtindoc{iyCalc} is used).
The first main difference is that \wsaindex{iy\_transmitter\_agenda} replaces
\wsaindex{iy\_space\_agenda} and \wsaindex{iy\_surface\_agenda}. The second
main difference is that handling of cloud scattering is built-in and the need
to define \wsaindex{iy\_cloudbox\_agenda} vanishes. These differences appear
inside \wsaindex{iy\_main\_agenda}. Further,
\builtindoc{blackbody\_radiation\_agenda} can be left undefined).

As for emission measurements (Sec.~\ref{sec:fm_defs:calcproc},
Algorithm~\ref{alg:fm_defs:iyCSagenda}) the first main operation is to
determine the propagation path through the atmosphere, but this is here done
without considering the cloud-box (it is simple deactivated during this step).
The possible ``radiative backgrounds'' are accordingly the surface or space,
i.e.\ where the propagation path starts. 

The next step is to call \builtindoc{iy\_transmitter\_agenda}. It should be
noted that the same agenda is called independently if the radiative background
is space or the surface. It is up to the user to decide if these two cases
shall be distinguished in some manner (no workspace method for this task exists
yet). For these calculations, the standard choice for
\builtindoc{iy\_transmitter\_agenda} is \wsmindex{MatrixUnitIntensity}. If this
workspace method is used, the output of \builtindoc{iyTransmissionStandard}
shows you the fraction of unpolarised radiation that is transmitted through the
atmosphere, and the polarisation state of the transmitted part.

The radiative transfer expression applied is
(cf.~Eq.~\ref{eq:fm_defs:vrte_step})
\begin{equation}
  \label{eq:trans:vrte}
  \StoVec_{i+1} = e^{-\bar{\ExtMat}s} \StoVec_i 
\end{equation}
where the extinction matrix is determined in the same manner as for emission
cases (Sec.~\ref{sec:fm_defs:rad_bkgr}). In situations where the matrix
\ExtMat\ is diagonal, the scalar form shown in Eq.~\ref{eq:fm_defs:rte_step2})
is used.

The method determines automatically if any part of the propagation path is
inside the cloud-box (if active). If this is the case, particle extinction is
included in \ExtMat, following the same path step averaging as applied for
pure absorbing species. For this method, scattering is purely a loss mechanism,
there is no gain by scattering into the line-of-sight

A related concern is the treatment of the surface. In standard usage of the
method, there is no signal reflected by the surface. The radiative transfer
calculations are started at the surface.

See the built-in documentation (\builtindoc{iyTransmissionStandard}) for a full
list of possible auxiliary quantities. These data include quantities that make
it possible to determine the transmission for different parts of the
propagation path. For example, the state of \builtindoc{iy} at each point of
the propagation path can be extracted, and also the transmission matrix
(Eq.~\ref{eq:rte:transmat}) for each path step is accessible.





\section{Radio link calculations}
%===================
\label{sec:radiolinks}

\subsection{Practical usage}
%
This type of calculations are handled by \wsmindex{iyRadioLink}. A sensor
position must be specified, as usual. The ``sensor'' takes the place as the
reciever of the radio link. The observation angles of the reciever are not user
input (if any given they are just ignored), they are determined by the position
of the transmitter. This position is given as \wsvindex{transmitter\_pos} or
\wsvindex{rte\_pos2}, dependent on if \builtindoc{yCalc} or \builtindoc{iyCalc}
is used. Specification of reciever and transmitter properties, beside their
positions, is discussed in Section~\ref{sec:radiolinks:oview}. The quantities
returned by \builtindoc{iyRadioLink} are also described in the same section
(\ref{sec:radiolinks:oview}).

It is allowed to use \builtindoc{iyRadioLink} together with pure geometrical
propagation paths, but this option does not give realistic results as the this
set-up misses the impact of ``defocusing'' (Sec.~\ref{sec:rlink:defoc}). With
refrcation, the propgation path can not be determined in a strict analytical
manner and some search algorithm is required. ARTS contains so far just a quite
simple algorithm for this purpose, activated by applying
\wsmindex{ppathFromRtePos2} for \wsaindex{ppath\_agenda}. In short, the search
is started by determining the geometrical path. The closest distance between
this path and the receiver is converted to an estimated correction for the
``line-of-sight'' of the transmitter, the corresponding refracted path is
determined and a new correction term is obtained. The procedure is repeated
until a (stringent) hard-coded convergence criterion is fulfilled.

   
\subsection{Definitions}
%===================
\label{sec:radiolinks:oview}

The power received, $p_r$, by an antenna with an effective aperture size
$A_e$, located a distance $l$ from a transmitter, with power
$p_t$ and gain $g_t$, can be expressed as
\begin{equation}
  \label{eq:rlink:start}
  p_r(l) = t_a\frac{p_tg_t}{4\pi l^2}A_e,
\end{equation}
where $t_a$ is the transmission due to losses caused by the atmosphere. This
total atmospheric loss ($t_a$) is the product between losses due to defocusing,
$t_d$ (Sec.~\ref{sec:rlink:defoc}) and gaseous and particle extinction, $t_e$
(Sec.~\ref{sec:rlink:atmext}):
\begin{equation}
  t_a = t_d t_e.
\end{equation}
The factor
\begin{equation}
  \label{eq:rlink:fspl}
  t_f = \frac{1}{4\pi l^2}
\end{equation}
represents the pure geometrical dilution of the power, following the inverse
square law. This effect is ecnountered also for propagation in free space, and
it is denoted as the \emph{free space loss}. 

However, it shall be noted that this deviates from what appears to be the
standard definition of free space loss, that is $\left(\Wvl/(4 \pi
l)\right)^2$, where \Wvl\ is the wavelength of the signal. This expression
encompasses the relationship between the reciever's effective antenna area and
its gain, and to keep atmospheric and sensor effects clearly seperated in ARTS,
\builtindoc{iyRadioLink} defines free space loss according to
Equation~\ref{eq:rlink:fspl}.

\begin{equation}
  p_r(l) = t_d t_e \left(\frac{1}{4\pi l} \right)^2 p_t',
\end{equation}


\subsection{Free space loss}
%===================
\label{sec:radiolinks:ppath}
The radiance law is not applicable for the signal from a point source.
According to \citet{ippolito:satco:08}, the signal intensity at a distance
\(s\), in free space, from a transmitting antenna, with a gain \(g_{t}\) is
defined as
\begin{equation}
 I(s)=\frac{p_{t}g_{t}}{4\pi s^{2}},
\end{equation}
where \(p_{t}\) is the transmitted power. The power \(p_{r}\) received by a
receiving antenna with an effective aperture \(A_{e}\) and gain \(g_{r}\)
located a distance \(s\) from the transmitting antenna is
\begin{equation}
\label{eq:fsl}
p_{r}(s)=\frac{p_{t}g_{t}}{4\pi s^{2}}A_{e}=\frac{p_{t}g_{t}}{4\pi s^{2}}
\frac{g_{r}\Wvl^{2}}{4\pi}=
p_{t}g_{t}g_{r}\left(\frac{\Wvl}{4\pi s} \right)^{2},
\end{equation}
where the last term  
to the right (or actually the reciprocal) is denoted as the
free space path loss (\(F\)):
\begin{equation}
F\equiv\frac{p_{t}g_{t}g_{r}}{p_{r}}=\left(\frac{4\pi s}{\Wvl}\right)^{2}.
\end{equation}
Taking the derivative of \(\Mpi(s)\) gives
\begin{equation}
\label{eq:fsplarts}
 \frac{\DiffD\Mpi(s)}{\DiffD s}=-\frac{2}{s}I(s).
\end{equation}
This equals the Beer-Lambert with an absorbance of $2/s$. With other words, the
free space loss can be treated as an ordinary extinction term. However, it
should be noted that this ``extinction coefficient'' is non-linear, it varies
with the distance from the transmitting antenna. 



\subsection{Bending angle and extra path delay}
%
An electromagnetic wave that travels a path of given optical path 
length arrives with the same phase shift as if it had travelled a 
path of that physical length in a vacuum. 
In the absence of an atmosphere a signal
sent from a transmitter in the direction towards a receiver would
follow a straight line, i.e. the shortest geometric distance
between the two instruments. Moreover, the signal would propagate
with the speed of light.
The geometric distance \(D\) between the transmitter and receiver is
\begin{equation}
D\equiv\int_{straight\,line}1\DiffD s
\end{equation}
In the presence of an atmosphere, two changes follow,
the speed of the signal is retarded and the ray direction gets 
bent. 
The apparent bent ray-path length is
\begin{equation}
L\equiv\int_{bent\,ray-path}\Rfr(s)\DiffD s.
\end{equation}
One can also define the geometric bent ray-path length as
\begin{equation}
G\equiv\int_{bent\,ray-path}1\DiffD s.
\end{equation}
The total delay expressed in units of length is 
\begin{equation}
d \equiv L-D \ge 0,
\end{equation}
which can also be expressed as
\begin{equation} 
d=d_{a}+d_{g}=(L-G)+(G-D)
\end{equation}
where \(d_{a}\) is the along-path delay
and \(d_{g}\) is the geometric delay.



\subsection{Defocusing}
\label{sec:rlink:defoc}
%
\dots



\subsection{Atmospheric extinction}
\label{sec:rlink:atmext}
%
\dots



\subsection{Faraday rotation}
\label{sec:rlink:farrot}
%
\dots

