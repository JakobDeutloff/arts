\chapter{Transmission calculations}
 \label{sec:trans}


\starthistory
 120921 & A first, but incomplete, version (Patrick Eriksson).\\
\stophistory

The term ``transmission calculations'' refers here to situations when
atmospheric and surface emission can be neglected. These calculations can be
divided into two main types. The first one is when just the transmission of the
atmosphere is diagnosed (Sec.~\ref{sec:transmission}). The observation geometry
is given exactly as for emission simulations, by a position and a
line-of-sight.

The second main type is radio link budgets (Sec.~\ref{sec:radiolinks}). For
this case, the propagation path is deermined solely by the position of the
transmitter and the receiver. That is, the user does not need to set a
line-of-sight of the sensor (receiver), it is determined by the transmitter
position. A characterisation of a radio link normally involves several
attenuation terms not encountered for passive measurements, such as free space
loss and defocusing.




\section{Pure transmission calculations}
%===================
\label{sec:transmission}

This section discusses the \wsmindex{iyTransmissionStandard} workspace method,
that is relevant if you want to calculate the transmission through the
atmosphere for a given position and line-of-sight. The set-up is largely the
same as for simulations involving emission, such as that the observation
geometry is defined by \wsvindex{sensor\_pos} and \wsvindex{sensor\_los} (or
\wsvindex{rte\_pos} and \wsvindex{rte\_los} if \builtindoc{iyCalc} is used).
The first main difference is that \wsaindex{iy\_transmitter\_agenda} replaces
\wsaindex{iy\_space\_agenda} and \wsaindex{iy\_surface\_agenda}. The second
main difference is that handling of cloud scattering is built-in and the need
to define \wsaindex{iy\_cloudbox\_agenda} vanishes. These differences appear
inside \wsaindex{iy\_main\_agenda}. Further,
\builtindoc{blackbody\_radiation\_agenda} can be left undefined).

As for emission measurements (Sec.~\ref{sec:fm_defs:calcproc},
Algorithm~\ref{alg:fm_defs:iyCSagenda}) the first main operation is to
determine the propagation path through the atmosphere, but this is here done
without considering the cloud-box (it is simple deactivated during this step).
The possible ``radiative backgrounds'' are accordingly the surface or space,
i.e.\ where the propagation path starts. 

The next step is to call \builtindoc{iy\_transmitter\_agenda}. It should be
noted that the same agenda is called independently if the radiative background
is space or the surface. It is up to the user to decide if these two cases
shall be distinguished in some manner (no workspace method for this task exists
yet). For these calculations, the standard choice for
\builtindoc{iy\_transmitter\_agenda} is \wsmindex{MatrixUnitIntensity}. If this
workspace method is used, the output of \builtindoc{iyTransmissionStandard}
shows you the fraction of unpolarised radiation that is transmitted through the
atmosphere, and the polarisation state of the transmitted part.

The radiative transfer expression applied is
(cf.~Eq.~\ref{eq:fm_defs:vrte_step})
\begin{equation}
  \label{eq:trans:vrte}
  \StoVec_{i+1} = e^{-\bar{\ExtMat}s} \StoVec_i 
\end{equation}
where the extinction matrix is determined in the same manner as for emission
cases (Sec.~\ref{sec:fm_defs:rad_bkgr}). In situations where the matrix
\ExtMat\ is diagonal, the scalar form shown in Eq.~\ref{eq:fm_defs:rte_step2})
is used.

The method determines automatically if any part of the propagation path is
inside the cloud-box (if active). If this is the case, particle extinction is
included in \ExtMat, following the same path step averaging as applied for
pure absorbing species. For this method, scattering is purely a loss mechanism,
there is no gain by scattering into the line-of-sight

A related concern is the treatment of the surface. In standard usage of the
method, there is no signal reflected by the surface. The radiative transfer
calculations are started at the surface.

See the built-in documentation (\builtindoc{iyTransmissionStandard}) for a full
list of possible auxiliary quantities. These data include quantities that make
it possible to determine the transmission for different parts of the
propagation path. For example, the state of \builtindoc{iy} at each point of
the propagation path can be extracted, and also the transmission matrix
(Eq.~\ref{eq:rte:transmat}) for each path step is accessible.





\section{Radio calculations}
%===================
\label{sec:radiolinks}

The \wsmindex{iyRadioLink} method is in development and at this moment just
some theory is provided:


\subsection{Overview}
%===================
\label{sec:radiolinks:oview}

The power received, $p_r$, by an antenna with an effective aperture size
$A_e$ and gain $g_r$, located a distance $l$ from a transmitter, with power
$p_t$ and gain $g_t$, can be written as
\begin{equation}
  p_r(l) = t_a\frac{p_tg_t}{4\pi l^2}A_e=t_a\frac{p_tg_t}{4\pi l^{2}}
  \frac{g_r\Wvl^2}{4\pi}=t_ap_tg_tg_r\left(\frac{\Wvl}{4\pi l} \right)^2,
\end{equation}
where $t_a$ is the transmission due to losses caused by the atmosphere and
\Wvl\ is the wavelength of the signal. The total atmospheric loss is the
product between loss due to defocusing, $t_d$ (Sec.~?) and gaseous and particle
extinction, $t_e$ (Sec.~?):
\begin{equation}
  t_a = t_d t_e.
\end{equation}
Xxx
\begin{equation}
  p_r(l) = t_d t_e \left(\frac{1}{4\pi l} \right)^2 p_t',
\end{equation}




\subsection{Determining the propagation path}
%===================
\label{sec:radiolinks:ppath}

Use \wsmindex{ppathFromRtePos2}.


\subsection{Free space loss}
%===================
\label{sec:radiolinks:ppath}
The radiance law is not applicable for the signal from a point source.
According to \citet{ippolito:satco:08}, the signal intensity at a distance
\(s\), in free space, from a transmitting antenna, with a gain \(g_{t}\) is
defined as
\begin{equation}
 I(s)=\frac{p_{t}g_{t}}{4\pi s^{2}},
\end{equation}
where \(p_{t}\) is the transmitted power. The power \(p_{r}\) received by a
receiving antenna with an effective aperture \(A_{e}\) and gain \(g_{r}\)
located a distance \(s\) from the transmitting antenna is
\begin{equation}
\label{eq:fsl}
p_{r}(s)=\frac{p_{t}g_{t}}{4\pi s^{2}}A_{e}=\frac{p_{t}g_{t}}{4\pi s^{2}}
\frac{g_{r}\Wvl^{2}}{4\pi}=
p_{t}g_{t}g_{r}\left(\frac{\Wvl}{4\pi s} \right)^{2},
\end{equation}
where the last term  
to the right (or actually the reciprocal) is denoted as the
free space path loss (\(F\)):
\begin{equation}
F\equiv\frac{p_{t}g_{t}g_{r}}{p_{r}}=\left(\frac{4\pi s}{\Wvl}\right)^{2}.
\end{equation}
Taking the derivative of \(\Mpi(s)\) gives
\begin{equation}
\label{eq:fsplarts}
 \frac{\DiffD\Mpi(s)}{\DiffD s}=-\frac{2}{s}I(s).
\end{equation}
This equals the Beer-Lambert with an absorbance of $2/s$. With other words, the
free space loss can be treated as an ordinary extinction term. However, it
should be noted that this ``extinction coefficient'' is non-linear, it varies
with the distance from the transmitting antenna. 


\subsection{Extra path delay}
%
An electromagnetic wave that travels a path of given optical path 
length arrives with the same phase shift as if it had travelled a 
path of that physical length in a vacuum. 
In the absence of an atmosphere a signal
sent from a transmitter in the direction towards a receiver would
follow a straight line, i.e. the shortest geometric distance
between the two instruments. Moreover, the signal would propagate
with the speed of light.
The geometric distance \(D\) between the transmitter and receiver is
\begin{equation}
D\equiv\int_{straight\,line}1\DiffD s
\end{equation}
In the presence of an atmosphere, two changes follow,
the speed of the signal is retarded and the ray direction gets 
bent. 
The apparent bent ray-path length is
\begin{equation}
L\equiv\int_{bent\,ray-path}\Rfr(s)\DiffD s.
\end{equation}
One can also define the geometric bent ray-path length as
\begin{equation}
G\equiv\int_{bent\,ray-path}1\DiffD s.
\end{equation}
The total delay expressed in units of length is 
\begin{equation}
d \equiv L-D \ge 0,
\end{equation}
which can also be expressed as
\begin{equation} 
d=d_{a}+d_{g}=(L-G)+(G-D)
\end{equation}
where \(d_{a}\) is the along-path delay
and \(d_{g}\) is the geometric delay.


%%% Local Variables: 
%%% mode: latex
%%% TeX-master: "arts_user"
%%% End: 
