\chapter{Radiative transfer basics}
 \label{sec:rte_basics}


 \starthistory
 130212 & Started (Patrick Eriksson).\\
 \stophistory

 This chapter introduces some basic radiative transfer nomenclature and
 equations. The radiative transfer equation is here presented in quite general
 terms, while special cases and solutions are discussed in later parts of the
 document. 


\section{Stokes dimensionality}
%==============================================================================
\label{sec:fm_defs:polarisation}

The full polarisation state of radiation can be described by the Stokes vector,
and is the formalism applied in ARTS. The vector can be defined in different
ways, but it has always four elements. The Stokes vector, \StoVec, is here
written as
\begin{equation}
  \label{eq:fm_defs:stokevec}
  \StoVec = \left[
  \begin{array}{c}
   \StoI\\ \StoQ \\ \StoU\\ \StoV
  \end{array}
  \right],
\end{equation}
where the first component (\StoI) is the full intensity of the
radiation, the second component (\StoQ) is the difference between
vertical and horizontal polarisation, the third component (\StoU) is the
difference for $\pm$45$^\circ$ polarisation and the last component
(\StoV) is the difference between left and right circular polarisation.
That is:
\begin{eqnarray}
  \StoI &=&   \Iv + \Ih = \Ipff + \Imff = \Ilhc + \Irhc, \\
  \StoQ &=&   \Iv - \Ih,                                 \\
  \StoU &=&   \Ipff - \Imff,                             \\
  \StoV &=&   \Ilhc - \Irhc,                             
\end{eqnarray}
where \Iv, \Ih, \Ipff, and \Imff\ are the intensity of the component linearly
polarised at the vertical, horizontal, +45\degree\ and -45\degree\ direction,
respectively, and \Irhc, and \Ilhc\ are the intensity for the right- and
left-hand circular components. Further details on polarisation and the
definition of the Stokes vector are found in \theory,
Section~\ref{T-sec:polarization}.

ARTS is a fully polarised forward model, but can be run with a smaller
number of Stokes components. The selection is made with the workspace
variable \wsvindex{stokes\_dim}. For example, gaseous absorption and
emission are in general unpolarised, and if not scattering has to be
considered it is sufficient to only include the first Stokes
components in the simulations. To include higher order Stokes
components results only, in this case, in slower calculations. The
general case is here denoted as \textindex{vector radiative transfer},
while \textindex{scalar radiative transfer} refers to the case when
only the first Stokes component is considered.
 


\section{The radiative transfer equation}
%==============================================================================
\label{sec:rteq}

The radiative transfer problem can only be expressed in general terms as a
differential equation. One version for vector radiative transfer is
\begin{equation}
    \label{eq:VRTE0}
  \frac {\DiffD\StoVec(\Frq,\PPos,\PDir)}{\DiffD s} =
    -\ExtMat(\Frq,\PPos,\PDir) \StoVec(\Frq,\PPos,\PDir) +
    \VctStl{j}_e(\Frq,\PPos,\PDir) + \VctStl{j}_s(\Frq,\PPos,\PDir),  
\end{equation}
where \Frq\ is frequency, \PPos\ represents the atmospheric position, \PDir\ is
the propagation direction (at \PPos), $s$ is distance along \PDir, $\ExtMat$ is
the extinction matrix, $\VctStl{j}_e$ is the emission at the point
and $\VctStl{j}_s$ represents the scattering from other directions into the
propagation direction.


\subsection{Main cases}
\label{sec:rteq:cases}
%
One of the general assumptions in ARTS is that local thermodynamic equilibrium
(LTE) can be assumed. If we for the moment assume that scattering can be
totally neglected then Equation~\ref{eq:VRTE0} becomes
\begin{equation}
  \label{eq:VRTE1}
  \frac{\DiffD\StoVec(\Frq,\PPos,\PDir)}{\DiffD s} =
    \ExtMat_a(\Frq,\PPos,\PDir)\left[ \VctStl{s}- \StoVec(\Frq,\PPos,\PDir)
    \right],
\end{equation}
where $\ExtMat_a$ is denoted as the absorption matrix (to flag that it contains
no contribution from scattering) and \VctStl{s} can be seen as the emission
source vector, defined as
\begin{equation}
  \VctStl{s} = [B(\Frq,\PPos),0,0,0]^T,
\end{equation}
where $B$ is the Planck function, describing blackbody radiation. Cases, where
Equation~\ref{eq:VRTE1} is valid, are in ARTS denoted as ``clear-sky''
radiative transfer (implying LTE if nothing else is stated). The discussion of
such radiative transfer calculations is continued in Section~\ref{sec:rte}. The
atmospheric quantities contributing to $\ExtMat_a$ already for clear-sky
conditions are denoted as ``absorbing species''.

Some additional conditions are required to put scattering into the picture. If
scattering is of incoherent and elastic nature, the extension of 
Equation~\ref{eq:VRTE1} is
\begin{eqnarray}
  \label{eq:VRTE2}
  \frac {\DiffD\StoVec(\Frq,\PPos,\PDir)}{\DiffD s} &=&
    -\ExtMat(\Frq,\PPos,\PDir) \StoVec(\Frq,\PPos,\PDir) +
    \ExtMat_a(\Frq,\PPos,\PDir)\VctStl{s} + \\ \nonumber
    &&+ \int_{4\pi} \DiffD\PDir' \PhaMat(\Frq,\PPos,\PDir,\PDir')
    \StoVec(\Frq,\PPos,\PDir'),
\end{eqnarray}
where $\PhaMat$ is the scattering (or phase) matrix (and
$\ExtMat\neq\ExtMat_a$). The aerosol, cloud or precipitation objects that
cause \PhaMat\ to deviate from zero (and contribute to \ExtMat\ and $\ExtMat_a$)
are simply denoted as ``particles'', and the solution of
Equation~\ref{eq:VRTE2} is referred to as ``scattering calculations''. ARTS
includes several modules to handle scattering, introduced in the last part of
this document.

Equation~\ref{eq:VRTE2} is discussed quite thoroughly in \ref{T-sec:rte_theory}
of \theory. This including that for some conditions also the ``n2-law of
radiance'' must be considered to obtain completely exact results. For further
discussion of this issue, see first of all Section~\ref{sec:fm_defs:unit}.



\subsection{Some comments and nomenclature}
\label{sec:rteq:names}
%
The equations above treat a single frequency and a single direction, at a
time, and can be said to describe monochromatic pencil beam radiative
transfer. For simplicity, the frequency and direction are left out from many of
the equations in this user guide. 
Some further information around the involved radiative quantities:
\begin{description}
\item[\StoVec] As workspace variable denoted as \wsvindex{iy}. The variable can
  hold \StoVec\ for a range of frequencies, but covers only radiative transfer
  along a single propagation path. Please note the distinction to \wsvindex{y},
  that holds radiances from several frequencies and pencil beams, possibly
  weighted with some instrument responses.
%\item[\ExtMat] In the most general case, this matrix can cover three
%  different physical mechanisms: absorption, scattering and magneto-optical
%  effects. The sum of absorption and scattering is denoted as extinction.
%  An example on magneto-optical effect is Faraday rotation, that is not
%  changing the radiation intensity, but the polarisation state.
%
%  Inside ARTS, \ExtMat\ treated as a sum of two parts:
%  $\ExtMat=\ExtMat_a\ExtMat_p$, where the first part, $\ExtMat_a$, corresponds
%  to the ``absorbing species'', while the second part, $\ExtMat_a$, covers the
%  extinction of the (scattering) ``particles''. As workspace variables,
%  $\ExtMat_a$ and $\ExtMat_p$ are called \wsvindex{propmat\_clearsky} and
%  \wsvindex{propmat\_particle}, respectively.

\item[B] The corresponding workspace variable is
  \wsvindex{blackbody\_radiation}.

%\item[$\VctStl{j}_e$] is denoted as the \wsvindex{emission\_vector}. 



\end{description}
