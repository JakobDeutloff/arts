\chapter{Radiative transfer basics}
 \label{sec:rte_basics}

\starthistory
 130218 & First version by Patrick Eriksson.\\
\stophistory

This chapter introduces some basic radiative transfer nomenclature and
equations. The radiative transfer equation is here presented in general terms,
while special cases and solutions are discussed in later parts of the document.



\section{Stokes dimensionality}
%==============================================================================
\label{sec:fm_defs:polarisation}

The full polarisation state of radiation can be described by the Stokes vector,
and is the formalism applied in ARTS. The vector can be defined in different
ways, but it has always four elements. The Stokes vector, \StoVec, is here
written as
\begin{equation}
  \label{eq:fm_defs:stokevec}
  \StoVec = \left[
  \begin{array}{c}
   \StoI\\ \StoQ \\ \StoU\\ \StoV
  \end{array}
  \right],
\end{equation}
where the first component (\StoI) is the full intensity of the
radiation, the second component (\StoQ) is the difference between
vertical and horizontal polarisation, the third component (\StoU) is the
difference for $\pm$45$^\circ$ polarisation and the last component
(\StoV) is the difference between left and right circular polarisation.
That is:
\begin{eqnarray}
  \StoI &=&   \Iv + \Ih = \Ipff + \Imff = \Ilhc + \Irhc, \\
  \StoQ &=&   \Iv - \Ih,                                 \\
  \StoU &=&   \Ipff - \Imff,                             \\
  \StoV &=&   \Ilhc - \Irhc,                             
\end{eqnarray}
where \Iv, \Ih, \Ipff, and \Imff\ are the intensity of the component linearly
polarised at the vertical, horizontal, +45\degree\ and -45\degree\ direction,
respectively, and \Irhc, and \Ilhc\ are the intensity for the right- and
left-hand circular components. Further details on polarisation and the
definition of the Stokes vector are found in \theory,
Section~\ref{T-sec:polarization}.

ARTS is a fully polarised forward model, but can be run with a smaller number
of Stokes components. The selection is made with the workspace variable
\wsvindex{stokes\_dim}. For example, gaseous absorption and emission are in
general unpolarised, and if not particle and surface scattering have to be
considered it is sufficient to only include the first Stokes components in the
simulations (ie.\ \builtindoc{stokes\_dim} set to 1). In this case, to include
higher order Stokes components results only in slower calculations. Simulations
where \builtindoc{stokes\_dim} is two or higher are denoted as
\textindex{vector radiative transfer}, while \textindex{scalar radiative
  transfer} refers to the case when only the first Stokes component is
considered.
 



\section{The radiative transfer equation}
%==============================================================================
\label{sec:rteq}

The radiative transfer problem can only be expressed in a general manner as a
differential equation. One version for vector radiative transfer is
\begin{equation}
  \label{eq:VRTE0}
  \frac {\DiffD\StoVec(\Frq,\PPos,\PDir)}{\DiffD \PpathLng} =
    -\ExtMat(\Frq,\PPos,\PDir) \StoVec(\Frq,\PPos,\PDir) +
    \aSrcVec{e}(\Frq,\PPos,\PDir) + \aSrcVec{s}(\Frq,\PPos,\PDir),  
\end{equation}
where \Frq\ is frequency, \PPos\ represents the atmospheric position, \PDir\ is
the propagation direction (at \PPos), \PpathLng\ is distance along \PDir,
\ExtMat\ is the propagation matrix, \aSrcVec{e}\ represents the emission at
the point, and \aSrcVec{s}\ covers the scattering from other directions into
the propagation direction.



\subsection{Propagation effects}
\label{sec:rteq:propmat}

Three mechanisms contribute to the elements of the propagation matrix:
absorption, scattering and magneto-optical effects. Absorption and scattering
can together be denoted as extinction, referring to that these two mechanisms
result in a decrease of the intensity (\StoI). A common name for \ExtMat\ is
also the extinction matrix. The extinction processes also affect the \StoQ,
\StoU\ and \StoV\ elements of the Stokes vector. If the degree of polarisation
(\DgrPlr)
\begin{equation}
  \label{eq:degreeofpol}
  \DgrPlr = \frac{\sqrt{\StoQ^2+\StoU^2+\StoV^2}}{\StoI}
\end{equation}
is kept constant or not depend on symmetry properties of the attenuating media.
For example, absorption of atmospheric gases is not changing \DgrPlr\ as long
as the molecules have no preferred orientation, which is a valid assumption
beside when there is a significant Zeeman effect. Non-polarising absorption
corresponds to that the propgation matrix can be written as $\AbsCoef\IdnMtr$,
where \AbsCoef\ is the absorption coefficient and \IdnMtr\ is the identity
matrix.

There are also examples on effects that change the polarisation state of the
Stokes vector without affecting \StoI. These effects are caused by an
interaction with the magnetic field, and are thus denoted magneto-optical.
Ionospheric Faraday rotation can (approximately) be seen as a pure
magneto-optical effect, while the Zeeman effect cause both non-isotropic
absorption and has magneto-optical aspects.

As Equation~\ref{eq:VRTE0} indicates, there are two source mechanisms that can
act to increase the intensity: emission and scattering.


\subsection{Absorbing species and particles}
\label{sec:rteq:absspecies}

The complexity of the radiative transfer is largely dependent on whether scattering
must be considered or not. For this reason, ARTS operates with two classes of
atmospheric matter:
\begin{description}
\item[Absorbing species] This class covers atmospheric matter for which
  scattering can be neglected. The set of ``species'' to consider is described
  by the workspace variable \wsvindex{abs\_species}, and the associated
  atmospheric fields are gathered into \wsvindex{vmr\_field}. As the name
  indicates, this later variable is mainly containing volume mixing ratio (VMR)
  data, but as this unit is not applicable in all cases also other units are
  accepted. That is, the unit for the fields varies, for gases it is VMR, while
  particle mass concentrations are given in $kg/m^3$ and electron density in
  $m^{-3}$. 
\item[Particles] This second class treats all matter causing significant
  scattering (and likely also adding to the absorption). The amount of
  scattering matter is given as particle number density fields ($m^{-3}$),
  denoted as \wsvindex{pnd\_field}. The \wsvindex{pnd\_field} is provided per
  scattering element (see Chapter \ref{sec:clouds} for definition). The
  corresponding optical properties of the particles are given by
  \wsvindex{scat\_data} containing one set of single scattering properties
  for each scattering element. Particles can be grouped into scattering
  species, each characterized by a mass density ($kg/m^3$) or flux ($kg/(m^2s$)
  field that is converted into particle number density fields before solving the
  radiative transfer equation.
\end{description}
Atmospheric quantities are not hard-coded to belong to any of these matter
classes, as the practical division depends on the conditions of the
simulations. The general rule is that for the shorter the wavelengths, a higher
number of atmospheric constituents must be treated as ``particles''. For
thermal infrared and microwave calculations, molecules and electrons (and
likely also aerosols) can be treated as absorbing species. It can also be
possible to place some hydrometeors in this class. For example,
non-precipitating liquid clouds can be treated as purely absorbing in the
microwave region.
Related to the division between absorbing species and particles is:
\begin{description}
\item[Clear-sky] In ARTS, the term ``clear-sky'' refers to the case when the
  influence of ``particles'' can be ignored, and only ``absorbing species'' are
  of relevance. Hence, a clear-sky calculation can involve e.g.\ cloud water
  droplets, but on the condition that the wavelength is such that scattering
  can be neglected.
\end{description}
The propagation effects of absorbing species and particles are kept separated.
The total propagation matrix is 
\begin{equation}
  \label{eq:propmattotal}
  \ExtMat = \aAbsMat{a} + \aExtMat{p}, 
\end{equation}
where $a$ and $p$ refers to absorbing species and particles, respectively, and
the symbol \AbsMat\ is used for the first term to remind about that the only
extinction process covered by this matrix is absorption (but including
magneto-optical effects). As workspace variable, \aAbsMat{a}\ is denoted as
\wsvindex{propmat\_clearsky}, obtained by the
\wsaindex{propmat\_clearsky\_agenda} agenda. \aExtMat{p}\ is obtained
internally from \builtindoc{scat\_data}.



\subsection{Emission and absorption vectors}
\label{sec:rteq:evec}

One of the general assumptions in ARTS is that local thermodynamic equilibrium
(LTE) can be assumed. For the moment there exists no method in ARTS to handle
deviations from LTE, but this can be changed in the future.
On the condition of LTE, the emission vector ($\VctStl{j}_e$) in
Equation~\ref{eq:VRTE0} can be written as
\begin{equation}
  \label{eq:evec1}
  \aSrcVec{e} = \Planck \AbsVec,
\end{equation}
where \Planck\ is the Planck function (a scalar value), describing blackbody
radiation. The related workspace variables are \wsvindex{blackbody\_radiation}
and \wsaindex{blackbody\_radiation\_agenda}. 

The quantity \AbsVec\ is denoted as the absorption vector. For clear-sky
calculations the absorption vector is given by \aAbsMat{a}, as in this case the
emission vector can be calculated as
\begin{equation}
  \label{eq:evec2}
  \aSrcVec{e} = \aAbsMat{a}\EmsVec,
\end{equation}
where \EmsVec\ can be seen as the emission source vector, defined as
\begin{equation}
  \EmsVec = [\Planck,0,0,0]^T.
\end{equation}
Hence, as only the first element of \EmsVec\ is non-zero, the absorption
vector is in this case equal to the first column of \aAbsMat{a}.
The absorption vector can not be extracted from \aExtMat{p}\ as this
propagation matrix covers also scattering and \wsvindex{scat\_data}
must also contain such data.

In summary, the total absorption vector in ARTS is obtained as
\begin{equation}
  \label{eq:evec3}
  \AbsVec = \aAbsVec{a} + \aAbsVec{p}
\end{equation}
where \aAbsVec{a}\ is the first column of \aAbsMat{a}\, and \aAbsVec{p}\ is the
absorption vector due to particles.




\subsection{Main cases}
\label{sec:rteq:cases}
%
The two equations below are discussed thoroughly in \ref{T-sec:rte_theory} of
\theory. This including that for some conditions also the ``n2-law of
radiance'' must be considered to obtain completely exact results (see also
Section~\ref{sec:fm_defs:unit}).

The equations below treat a single frequency and a single direction,
at a time, and can be said to describe monochromatic pencil beam radiative
transfer. For simplicity, the frequency and direction are left out from many of
the equations in this user guide.

\subsubsection{Clear-sky radiative transfer}
%
If we for the moment assume that scattering can be totally neglected, then
Equation~\ref{eq:VRTE0} can be simplified to
\begin{equation}
  \label{eq:VRTE1}
  \frac{\DiffD\StoVec(\Frq,\PPos,\PDir)}{\DiffD \PpathLng} =
  \aAbsMat{a}(\Frq,\PPos,\PDir)\left[ \EmsVec(\Frq,\PPos,\PDir)- 
  \StoVec(\Frq,\PPos,\PDir)\right] \quad 
  \left( = -\aAbsMat{a}\StoVec + \Planck\aAbsVec{a}\right).
\end{equation}
Cases where Equation~\ref{eq:VRTE1} is valid, are in ARTS denoted as clear-sky
radiative transfer (implying LTE if nothing else is stated). The discussion of
such radiative transfer calculations is continued in Section~\ref{sec:rte}.


\subsubsection{Radiative transfer with scattering}
%
Some additional conditions are required to put scattering into the picture. If
scattering is of incoherent and elastic nature, the extension of 
Equation~\ref{eq:VRTE1} is
\begin{eqnarray}
  \label{eq:VRTE2}
  \frac {\DiffD\StoVec(\Frq,\PPos,\PDir)}{\DiffD \PpathLng} &=&
    -\ExtMat(\Frq,\PPos,\PDir) \StoVec(\Frq,\PPos,\PDir) +
     \Planck\aAbsVec{a}(\Frq,\PPos,\PDir) + \\ \nonumber
    &&+ \int_{4\pi} \PhaMat(\Frq,\PPos,\PDir,\PDir')
    \StoVec(\Frq,\PPos,\PDir')\,\DiffD\PDir' ,
\end{eqnarray}
where $\PhaMat$ is the scattering (or phase) matrix.  ARTS includes several
modules to handle scattering, introduced in the last part of this document.\\





