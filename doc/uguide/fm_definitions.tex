%
% To start the document, use
%  \levela{...}
% For lover level, sections use
%  \levelb{...}
%  \levelc{...}
%
\levela{The forward model: concepts, definitions and overview}
 \label{sec:fm_defs}

%
% Document history, format:
%  \starthistory
%    date1 & text .... \\
%    date2 & text .... \\
%    ....
%  \stophistory
%
\starthistory
  020314 & Layout made by Patrick Eriksson.\\
\stophistory


%
% Symbol table, format:
%  \startsymbols
%    ... & \verb|...| & text ... \\
%    ... & \verb|...| & text ... \\
%    ....
%  \stopsymbols
%
%
\startsymbols
  \Ind           & -                 & vector/matrix/tensor index           \\
  \aInd{\Lat}    & -                 & the \Ind:th latitude                 \\
  \VctLng        & -                 & vector length or size of matrix/tensor for a dimension \\
  \aVctLng{\Lat} & -                 & length of the latitude grid \\
  \Prs           & \verb|p|          & pressure                             \\
  \PrsAlt        & \verb|pz|         & pressure altitude                    \\
  \Rds           & \verb|r|          & radius from the centre of the coordinate system         \\
  \Alt           & \verb|z|          & geometrical altitude above the geoid \\
  \Lat           & \verb|alpha|      & latitude                             \\
  \Lon           & \verb|beta|       & longitude                            \\
  \ZntAng        & \verb|psi|        & zenith angle                         \\
  \AzmAng        & \verb|omega|      & azimuthal angle                      \\
 \label{symtable:fm_defs}     
\stopsymbols



This chapter introduces terms and concepts of ARTS as a forward model,
in contrast to the previous chapter that describes ARTS as a computer
programme. While the content of the previous chapter is specific for
ARTS, as the way to use a forward model programme differ normally
significantly from one implementation to another, this chapter is of
more general nature. Most of the quantities treated here should be
part of any forward model of the same complexity as ARTS, where only
details regarding the definition should differ. The aim of this
chapter is to give an overview of the forward model and to describe
important terms and concepts, in such way that, the content of this user
guide can be fully appreciated and that you shall understand how to
generate a control file for your simulation problem.



\levelb{Mathematics}
%==============================================================================
\label{sec:fm_defs:math}

[* A short introduction to the vectors, matrices and tensors should be
found here. It should be described what the terms means, the order in
which dimensions are given, how sizes are specified here in the guide,
how different dimensionalities are handled etc. 

Vectors, matrices and tensors are indexed by roundbraces, e.g.
\begin{equation}
  \Alt(\aInd{\Prs},\aInd{\Lat},\aInd{\Lon})
\end{equation}
This is not consistent with how vectors are indexed in ARTS, but it
would be confusing to use square braces, this as square braces are
used for sizes. A size is given as $
[\aVctLng{\Prs},\aVctLng{\Lat},\aVctLng{\Lon}]$, or the size:
\begin{equation}
  \Alt \SzeSmb [\aVctLng{\Prs},\aVctLng{\Lat},\aVctLng{\Lon}]
\end{equation}
The concept of rows, columns, pages etc. should be described so a
particular dimension can be identified in a consistent manner.

The indexing is 0-based as in ARTS. *]



\levelb{The atmosphere}
%==============================================================================
\label{sec:fm_defs:atmosphere}


\levelc{Atmospheric dimensionality}
%===================
\label{sec:fm_defs:atmdim}

The structure of the atmosphere can be defined to have different
degree of complexity, the \qindex{atmospheric dimensionality}. There
exist three levels for the complexity of the atmosphere, 1D, 2D and
3D, where 1D and 2D can be treated as special cases of 3D. The
significance of these different atmospheric dimensionalities, and the
coordinate systems used, are described below. The atmospheric
dimensionality is selected by setting the workspace variable
\verb|atmosphere_dim| to a value between 1 and 3. Variables for which
the size depends on the atmospheric dimensionality are checked, when
used, to have a size consistent with \verb|atmosphere_dim|.

\leveld{\qindex{3D}} In this, the most general, case, the atmospheric
fields vary in all three spatial coordinates, as in a true atmosphere.
A spherical coordinate system is used where the dimensions are radius
(\Rds), latitude (\Lat) and longitude (\Lon), and a position is given
as $(\Lon,\Lat,\Rds)$. With other words, the standard way to specify a
geographical position is followed. However, the way to specify the
radial position differs depending on the context, which is described
in Section~\ref{sec:fm_defs:altitudes}. The valid range for latitudes
is [-90\degree,+90\degree], where +90\degree corresponds to the North
pole etc. Longitudes are counted from the Greenwich meridian with
positive values towards the east. Longitudes can have values from
-360\degree to +360\degree. When the difference between the last and
first value of the longitude grid is $\geq 360\degree$ then the whole
globe is considered to be covered. The user must ensure that the
atmospheric fields for \Lon and $\Lon+360\degree$ are equal. If a
point of propagation path is found to be outside the range of the
longitude grid, this will results in an error if not the whole globe
is covered. Otherwise, the longitude is shifted with 360\degree in
the relevant direction.

\leveld{\qindex{1D}} A 1D atmosphere can be described as being
spherically symmetric. The term 1D is used here for simplicity and
historical reasons, not because it is a true 1D case (a strictly 1D
atmosphere would just extend along a line). A spherical symmetry means
that atmospheric fields and the ground extend in all three dimensions,
but they have no latitude and longitude variation. This means that,
for example, atmospheric fields vary only as a function of altitude
and the ground constitutes the surface of a sphere. The radial
coordinate is accordingly sufficient when dealing with atmospheric
quantities, but the angular distance between the sensor and a point
along the propagation path can be of interest, for example when
determining the cross-link between two satellites (a fact that shows
that this is not a true 1D case). A polar coordinate system is for
this reason used when describing propagation paths, where the
coordinate additional to the radius gives the angular distance inside
the viewing plane between the sensor and the point of interest (see
further Section~[**]).  This latter coordinate system coincides with
the one used for 2D where the sensor position is set to be the zero
point for latitude. A 1D atmosphere is shown in
Figure~\ref{fig:fm_defs:1d}.

\leveld{\qindex{2D}} In contrast to the 1D and 3D cases, a 2D
atmosphere extends only inside a plane. A spherical coordinate system
is accordingly not needed and a polar system, consisting of a radial
and an angular coordinate, is used. The 2D case is most likely used
for satellite measurements where the atmosphere is observed inside the
orbit plane. The angular coordinate corresponds then to the angular
distance along the satellite track, but the coordinate is for
simplicity denoted as the latitude. The zero point for the 2D latitude
is arbitrary. No lower and upper limit exists for the 2D latitude, and
this allows that measurements from several subsequent orbits can be
simulated as one unit. The atmosphere is treated to be undefined
outside the considered plane. A 2D atmosphere is shown in
Figure~\ref{fig:fm_defs:2d}.

 \begin{figure}[!t]
  \begin{center}
   \includegraphics*[width=0.95\hsize]{Figs/atm_dim_1d}
   \caption{ A schematic cross-section of a 1D atmosphere. The atmosphere is 
     here spherically symmetric. This means that the radius of the
     geoid, the ground and all the pressure surfaces is constant. The
     cloud box extends from the ground up to the thin solid line. The
     upper limit of the cloud box coincides always with a pressure
     surface.  }
   \label{fig:fm_defs:1d}  
  \end{center}
 \end{figure}
 % This figure was produced by the Matlab function mkfigs_atm_dims.

 \begin{figure}[!t]
  \begin{center}
   \includegraphics*[width=0.95\hsize]{Figs/atm_dim_2d}
   \caption{ Schematic of a 2D atmosphere. The radii of the geoid, the ground
     and all the pressure surfaces are for 2D atmospheres allowed to have a
     latitude variation. The limits of
     the cloud box coincide always with grid box boundaries. Plotting
     symbols as in Figure~\ref{fig:fm_defs:1d}. }
   \label{fig:fm_defs:2d}
  \end{center}
 \end{figure}
 % This figure was produced by the Matlab function mkfigs_atm_dims.


\levelc{Altitude coordinates}
%===================
\label{sec:fm_defs:altitudes}

\leveld{Pressure \index{pressure} and \qindex{pressure altitude}} The
main altitude coordinate is pressure. This is most clearly manifested
by the fact that the vertical atmospheric grid consists of surfaces
with equal pressure. The vertical grid is consistently denoted as the
pressure grid and the corresponding workspace variable is called
\verb|p_grid|. The choice of having pressure as main altitude
coordinate results in that atmospheric quantities are retrieved as a
function pressure, not as a function of geometrical altitude.

However, a basic assumption in ARTS is that atmospheric quantities
(temperature, geometric altitude, species VMR etc.) vary linear with
the logarithm of the pressure. This corresponds roughly to assuming a
linear variation with altitude. To obtain a more intuitive unit for the
logarithm of the pressure, the quantity pressure altitude, \PrsAlt, is
introduced, and it is defined as
\begin{equation}
  \PrsAlt = \aPrsAlt{s}(\log_{10}\aPrs{0} - \log_{10}\Prs)
 \label{eq:fm_defs:prsalt}
\end{equation}
where \aPrsAlt{s} and \aPrs{0} are arbitrary parameters (set in
\verb|arts.h|), but suitably selected in such way that the pressure
altitude corresponds roughly with the geometrical altitude.  This
correspondance is achieved if the two parameters are treated as a
fixed scale height (for decreases of the pressure with a factor of 10)
and the surface pressure, respectively. Values around
$\aPrsAlt{s}=15.5$~km and $\aPrs{0}=1013$~hPa are suitable for
simulations dealing with the Earth's atmosphere.

\leveld{Radius \index{radius}} Beside pressure, geometrical altitudes
are needed to determine the propagation path through the atmosphere
etc. The main geometrical altitude coordinate is the distance to the
centre of the coordinate system used, the radius. This is a natural
consequence of using a spherical or polar coordinate system. The
radius is used inside ARTS for all geometrical calculations and to
store the position of the sensor (Section~ \ref{sec:fm_defs:sensor}).

\leveld{Geometrical altitude \index{geometrical altitude}} The term
geometrical altitude signifies here the difference in radius between a
point and the geoid (Section~\ref{sec:fm_defs:geoid}) along the vector
to the centre of the coordinate system (see for example
Equation~\ref{eq:fm_defs:zground}). Hence, the geometrical altitude is
not measured along the local zenith direction (the normal to the
reference geoid). Geometrical altitudes are mainly used to
facilitate the input of the ground altitude, the sensor position and
the altitude of the pressure surfaces. This is the case as these
quantities are known rather with respect to the geoid than with
respect to the Earth's centre.


\levelc{Atmospheric grids and fields} \index{grid} \index{field}
%===================
\label{sec:fm_defs:grids}

As mentioned above, the vertical grid of the atmosphere consists of a
set of layers with equal pressure, the pressure grid (\verb|p_grid|).
This grid must of course always be specified. It is not allowed that
there is an altitude gap between the ground and the lowermost pressure
surface.  That is, the ground pressure must be smaller than the
pressure of the lowermost vertical grid surface. On the hand, it is
not necessary to match the ground and the first pressure surface, the
pressure grid can extend below the ground level. The upper end of the
pressure grid gives the practical upper limit of the atmosphere as
vacuum is assumed above. With other words, no absorption and
refraction take place above the uppermost pressure surface.

A latitude grid (\verb|alpha_grid|) must be specified for 2D and 3D.
For 2D, the latitudes shall be treated as the angular distance along
the orbit track, as described above in
Section~\ref{sec:fm_defs:atmdim}.  The latitude angle is throughout
calculated for the vector going from the centre of the coordinate
system to the point of concern. Hence, the latitudes here correspond
to the definition of the geocentric latitude, and not geodetic
latitudes (see Section~\ref{sec:ppath:geoid}). This is an accordance
to the definition of geometric altitudes found above. 
For 3D, a longitude grid (\verb|beta_grid|) must also be specified.
Valid ranges for latitude and longitude values are given in
Section~\ref{sec:fm_defs:atmdim}. 

The atmosphere is treated to be undefined outside the latitude and
longitude ranges covered by the grids, if not the whole globe is
covered. This results in that a propagation path is not allowed to
cross a latitude or longitude end face of the atmosphere, if such
exists, it can only enter or leave the atmosphere through the top of
the atmosphere (the uppermost pressure level). See further
Sections~\ref{sec:fm_defs:atmdim} and \ref{sec:fm_defs:rte}. The
volume (or area for 2D) covered by the grids is denoted as the
\qindex{model atmosphere}.

If the longitude and latitude grids are not used for the selected
atmospheric dimensionality, then the longitude grid (for 1D and 2D)
and the latitude grid (for 1D) are set to be empty, but the when
dealing with the size of variables the grid length shall be treated to
be one ($\aVctLng{\Lat}=1$ and/or $\aVctLng{\Lon}=1$). For example,
the matrix describing the geoid (see Section~\ref{sec:fm_defs:geoid})
has for 1D the size $[1,1]$.

The basic atmospheric quantities are represented by their values at
each crossing of the involved grids (indicated by thick dots in
Figure~\ref{fig:fm_defs:2d}), or for 1D at each pressure surface
(thick dots in Figure~\ref{fig:fm_defs:1d}). This representation is
denoted as the field of the quantity. The field must, at least, be
specified for the geometric altitude of the pressure surfaces
(\verb|z_field|), the temperature (\verb|t_field|) and considered
atmospheric species (\verb|????|).

The fields are assumed to be piece-wise linear functions vertically
(with pressure altitude as the vertical coordinate,
Section~\ref{sec:fm_defs:altitudes}), and along the latitude and
longitude edges of 2D and 3D grid boxes. For points inside 2D and 3D
grid boxes, multidimensional linear interpolation is applied (that is,
bilinear interpolation for 2D etc.). Note especially that this is also
valid for the field of geometrical altitudes (\verb|z_field|).  Fields
are rank-3 tensors, for example \verb|z_field| has the dimensions
$[\aVctLng{\Prs},\aVctLng{\Lat},\aVctLng{\Lon}]$.  That means each
field is like a book, with one page for each pressure grid point, one
row for each latitude grid point, and one column for each longitude
grid point. In the 1-D case there is just one row and one column on each
page.



\levelc{The geoid and the ground}
%===================
\label{sec:fm_defs:geoid}

The geoid, \aRds{\odot}, is an imaginary surface used as a reference
surface when specifying the ground altitude and the altitude of
pressure surfaces. Any shape of the geoid is allowed but a smoothly
varying geoid is the natural choice, with the centres of the geoid and
the coordinate system coinciding. The geoid should normally be set to
the reference ellipsoid for some global geodetic datum, such as
WGS-84. For further reading on geoid ellipsoids and WGS-84, see
Section~\ref{sec:ppath:geoid}.

Inside ARTS, the geoid is represented as a matrix (\verb|r_geoid|),
holding the geoid radius for each crossing of the latitude and
longitude grids. The size of the matrix is accordingly $\aRds{\odot}
\SzeSmb [\aVctLng{\Lon},\aVctLng{\Lat}]$ and the geoid radius for a
specific position is $\aRds{\odot}(\aInd{\Lon},\aInd{\Lat})$. The
geoid is not defined outside the ranges covered by the latitude and
longitude grids, with the exception for 1D where the geoid by
definition is a full sphere. 
The ground altitude, \aAlt{g}, is given as the geometrical altitude
above the geoid. The radius for the ground is accordingly
\begin{equation}
  \aRds{g} = \aRds{\odot} + \aAlt{g}
 \label{eq:fm_defs:zground}
\end{equation}
As described in
Section~\ref{sec:fm_defs:grids}, it is not allowed that there is a gap
between the ground and the lowermost pressure level.

The ARTS variable for the ground altitude (\verb|z_ground|) is a
matrix of the same size as the geoid matrix.  For 1D, the ground is a
sphere by definition (as the geoid), while for 2D and 3D any shape is
allowed and a rough model of the ground topography can be made. For
example, for limb sounding into the troposphere, it could be of
importance to capture the intersection of the antenna beam by the
Himalayas, and maybe other mountain ridges.  However, it should be
noted that ground reflections are treated in a simplified manner and
accurate results cannot be expected beside when some conditions are
fulfilled (see Section~\ref{sec:fm_defs:groundrefl}).


\levelc{The cloud box}
%===================
\label{sec:fm_defs:cloudbox}

In order to save computational time, scattering calculations are
limited to the part of the atmosphere containing clouds and other
scattering objects (beside the ground). The atmospheric region in
which scattering shall be considered is denoted as the cloud box, and
it is discussed here as it acts as an additional atmospheric limit
when calculating propagation paths (see Section~\ref{sec:fm_defs:rte}). 

The cloud box is defined in such way that a propagation path 
entering the cloud box at one position has not crossed the cloud box
boundary at any other location (remember that there is no scattering
outside the cloud box). This requires that the lower limit of the
cloud box is throughout the ground. This is the case as if there would
be a gap between the ground and the cloud box, paths entering the
cloud box from below after a ground reflection have passed through the
cloud box on the way down, at least for propagations paths close to nadir.

Beside the lower vertical limit, the cloud box is defined to be
rectangular in the used coordinate system, with limits exactly at
points of the involved grids. This means that the upper limit of the
cloud box equals a pressure surface. For 2D, the angular extension of
the cloud box is between two points of the latitude grid
(Figure~\ref{fig:fm_defs:2d}), and likewise for 3D but then also with
a longitude extension between two grid points.

There exists in fact a small risk for 2D and 3D that a propagation
path into the cloud box has earlier crossed the box boundaries. This
can happen for propagation paths close to the nadir direction and if
the ground is tilted in such way that the surface of the ground points
towards the cloud box. The first crossing of the cloud box is
neglected for such cases, the cloud box is simply turned off when
determing the radiation field going into the cloud box.


\levelb{The sensor and data reduction}
%==============================================================================
\label{sec:fm_defs:sensor}

The instrument that detects the simulated radiation is denoted as the
sensor\index{sensor, the}. The forward model is constructed in such
way that a sensor must exists. For cases when only monochromatic
pencil beam radiation is of interest, the positions and directions for
which the radiation shall be calculated are given by specifying an
imagery sensor with infinite frequency and angular resolution. The
workspace variables for the sensor that always must be specified are
\verb|sensor_pos|, \verb|sensor_los|, \verb|sensor_pol|, \verb|antenna_dim|,
\verb|antenna_psi_grid| and \verb|Ha| and \verb|Hd|. These variables
are presented below, where they are first discussed individually and
then later together when it is described how different measurement
sequences are modelled in the most practical way. Remaining sensor
workspace variables are described last in this section.


\levelc{Sensor position \index{sensor position}}
%===================
\label{sec:fm_defs:sensorpos}

The observation positions of the sensor are stored in
\verb|sensor_pos|. This is a matrix where each row corresponds to a
sensor position. The number of columns in the matrix equals the
atmospheric dimensionality (1 column for 1D etc.). The columns of the
matrix (from first to last) are radius, latitude and longitude. 

Accordingly, row $i$ of \verb|sensor_pos| for a 3D case is
$(\aRds{i},\aLat{i},\aLon{i})$. The sensor position can be set to any
value, but the resulting propagation paths (also dependent on
\verb|sensor_los|) must be valid with respect to the model atmosphere
(see Section~\ref{sec:fm_defs:rte}). An obviously incorrect choice is to
place the senor below the ground altitude.

The fact that the sensor position can be given any value results in
that the radius must be used in \verb|sensor_pos|, in contrast to
\verb|z_ground| and \verb|z_field| where the altitude above the geoid
is applied. This is the case as the sensor can for 2D and 3D be
placed outside the covered latitude and longitude ranges, thus
outside the defined geoid and the geometrical altitude is undefined.

The sensor is treated to be motionless when calculating the spectrum,
or spectra, for each given observation position. One or several
spectra can be calculated for each position as described below.


\levelc{Line-of-sight\index{line-of-sight}}
%===================
\label{sec:fm_defs:los}

The viewing direction of the sensor, the line-of-sight, is described
with two angles, the zenith angle (\ZntAng) and the azimuth angle
(\AzmAng). The zenith angle exists for all atmospheric
dimensionalities, while the azimuthal angle only is defined for 3D.
The term line-of-sight is not only used in connection with the sensor,
it is also used to describe the local propagation direction along the
path taken by the observed radiation (Section~\ref{sec:fm_defs:rte}).
The zenith and azimuthal angles are defined identical in these two
contexts. This is expected as the position of the sensor is the end
point of the propagation path. The line-of-sight of propagation paths
is defined in the direction a photon travels to reach the sensor,
while the sensor line-of-sight is the direction the antenna is pointed
to receive the photons. This means that the sensor and path
line-of-sights (at the sensor) are parallel but go in opposite
directions. As a true sensor has a limited spatial resolution
(described by the antenna pattern), theoretically there are an
infinite number of line-of-sights associated with the sensor, but in
the forward model spectra are only calculated for a discrete set of
directions. If a sensor line-of-sight is mentioned without any
comments, it refers to the direction in which the centre of the
antenna pattern is directed.

The \qindex{zenith angle}, \ZntAng, is simply the angle between the
line-of-sight and the zenith direction. It should be mentioned that
the zenith and nadir directions are here defined to be along the line
passing the centre of the coordinate system and the point of concern
(Section~\ref{sec:ppath:geoid}). A nadir observation,
$\ZntAng=180\degree$, is thus a measurement towards the centre of the
coordinate system. The maximum absolute value of a zenith angle is
180\degree. For 1D and 3D, positive and negative zenith angles are
treated identically, while for 2D a distinction is made. In the case
of 2D, positive and negative zenith angles mean that the viewing
direction is towards higher and lower latitudes, respectively.

The \qindex{azimuthal angle}, \AzmAng, is given with respect to the
plane going through the north and south poles of the coordinate system
$(\Lat=\pm90\degree)$ and the sensor. The valid range is
$[-180\degree,180\degree]$ where angles are counted clockwise and
0\degree means that the viewing or propagation direction is north-wise.
Hence, +90\degree means that the direction of concern goes eastward.

The line-of-sight for propagation paths is discussed further in
Section~[**]. The sensor line-of-sights are stored in
\verb|sensor_los|. This workspace variable is a matrix, where the
first column holds zenith angles and the second column is azimuthal
angles. For 1D and 2D there is only one column in the matrix, while
for 3D a row $i$ of the matrix is $(\aZntAng{i},\aAzmAng{i})$. The
number of rows for \verb|sensor_los| must be the same as for
\verb|sensor_pos|.


\levelc{Sensor polarisation \index{sensor polarisation}}
%===================
\label{sec:fm_defs:sensorpol}

[* How shall this be handled? \verb|sensor_pol| *]


\levelc{Sensor characteristics \index{sensor characteristics} and
  \qindex{data reduction}}  
%===================
\label{sec:fm_defs:sensorchar}

Sensor characteristics is used here as a comprehensive term for the
response of all sensor parts that affect how the field of
monochromatic pencil beam intensities are translated to the recorded
spectrum. For example, the antenna pattern, the side-band filtering
and response of the spectrometer channels are normally the most
important charcteristics for a microwave heterodyne radiometer. Any
processing of the spectral data that takes place before the retrieval
is denoted as data reduction. The most common processing is to
represent the original spectra with a smaller set of values, that is,
a reduction of the data size. The most common data reduction
techniques is binning and Hotelling transformation by an eigenvector
expansion (Section~[**]).

The influence of sensor characteristics and data reduction is in ARTS
incorporated by transfer matrices, as described in
Section~\ref{sec:formalism:sensor}. The application of these transfer
matrices assumes that each step is a linear operation, which should be
the case for the response of the parts of a well designed instrument.
A possible non-linear data reduction must be handled seperately.

The creation of sensor and data reduction transfer matrices is not
(yet?) included in ARTS and no workspace variables exist to describe
the indivudual sensor parts etc. The sensor and data reduction are
described as a series of blocks, each having its own transfer matrix.
It is necessary to introduce a number of transfer matrices for various
reasons. At least three matrices are needed, denoted as \aSnsMtr{a},
\aSnsMtr{b} and \aSnsMtr{c}.



\levelc{How to model a \qindex{measurement sequence}?} 
%===================
\label{sec:fm_defs:scandef}


\levelc{Remaining sensor workspace variables} 
%===================
\label{sec:fm_defs:sensorvars}

\verb|antenna_psi_grid|, \aSnsMtr{b}, \aSnsMtr{c}


\levelb{Absorption and refractive index}
%==============================================================================
\label{sec:fm_defs:absorption}

[* Considerations for the calculation of absorption and refractive index ...
Tags, line-by-line, absorption models etc. *]



\levelb{Clear sky radiative transfer}
%==============================================================================
\label{sec:fm_defs:rte}

\newcommand{\Int}{{\bf I}}
\newcommand{\Ext}{{\bf K}}
\newcommand{\Abs}{{\bf a}}
\newcommand{\Sca}{{\bf Y}}
\newcommand{\Dir}{{\bf n}}
\newcommand{\Path}{{\bf s}}
\newcommand{\Planck}{{\bf B}}
\newcommand{\Freq}{\nu}
\newcommand{\Dep}{(\Dir,\Freq)}

\startsymbols
  \Ind           & -                 & vector/matrix/tensor index           \\
  \Int           & 
 \label{symtable:fm_defs_rt}     
\stopsymbols



\levelc{Radiative transfer equation and important quantities}

The radiative transfer equation in the absence of cloud particles is
\citep{mishchenko00:_light_scatt_nonsp_partic}: 

\begin{equation}
  \label{eq:rte_4stokes_no_scat}
  \frac{d\Int\Dep}{d\Path} = - \Ext\Dep \Int\Dep + \Abs\Dep\Planck(T)
\end{equation}
where the meaning of the quantities in this equation is as follows:

\leveld{The specific intensity vector $\Int$}



 where $I$ is the monochromatic pencil beam intensity, $l$ distance
 along the line of sight (LOS), $l_1$ the point of the considered part
 of the LOS furthest away from the sensor, $l_2$ the closest point of
 the LOS, $I_1$ the intensity at $l_1$, $\kappa$ the total absorption
 along the LOS and $\sigma$ the source function.\footnote{The symbols
   $\kappa$ and $\sigma$ are used here for the absorption and the source 
   function \emph{along} the LOS. The more commonly used symbols, $k$ and
   $S$, respectively, are used below to express the variables as
   functions of altitude.}


\levelc{Ground reflections}
%===================
\label{sec:fm_defs:groundrefl}

[* Discuss limitations for the used scheme and when OK results can be
expected. The slope of the ground is considered! Can we include
diffuse ground reflections in a simple way? *]



\levelb{Scattering}
%==============================================================================
\label{sec:fm_defs:scattering}

[* Special quantities and considerations for scattering ... *]


%%% Local Variables: 
%%% mode: latex
%%% TeX-master: "uguide"
%%% End: 
