\levela{Surface emission and reflections}
 \label{sec:surface}


\starthistory
  040427 & Started by Patrick Eriksson. \\
\stophistory


Basic facts regarding the treatment of surface emission and
reflections are given in Section~\ref{sec:fm_defs:groundrefl}.  Let us
start with a simple example in order to explain the usage of the
workspace variables describing the surface properties. We will here
assume that the surface has an isotropic scattering, where all
downwelling radiation is reflected. This assumption is made for all
polarisation states. We assume further a 1D simulation and that the
downwelling radiation shall be calculated for nine zenith angles. The
relevant workspace variables should then be set as follows:
  
 \wsvindex{ground\_emission}: A matrix (of correct size) of zeros.

 \wsvindex{ground\_los}: A vector of length 9, covering the zenith
 angle range. A possible choice would be [5,15,25,\dots,85].
 
 \wsvindex{ground\_refl\_coeffs}: Each reflection matrix is a diagonal
 matrix with the value 1/9 throughout on the diagonal. That is, all
 elements with index (:,:,i,i) is 1/9. Size matching
 \artsstyle{ground\_los}, \artsstyle{f\_grid} and
 \artsstyle{stokes\_dim}


\levelb{The dielectric constant and the refractive index}

 The properties of a material are reported either as the relative
 dielectric constant, $\epsilon$, or the refractive index, $n$. Both
 these quantities can be complex and are related as
 \begin{equation}
   \label{eq:rte_eps2n}
   n = \sqrt{\epsilon}.
 \end{equation}


\levelb{Specular reflections}
 \label{sec:rte:surface:specular}
 
 If the surface is sufficiently smooth, radiation will be
 reflected/scattered only in the complementary angle, specular
 reflection. Required smoothness for assuming specular reflection is
 normally estimated by the Rayleigh criterion:
 \begin{equation}
   \label{eq:rte:rayleigh}
   \Delta h < \frac{\Wvl}{8\cos\theta_1}
 \end{equation}
 where $\Delta h$ is the root mean square variation of the surface
 height, \Wvl\ the wavelength and $\theta_1$ the angle between the
 surface normal and the incident direction of the radiation. The
 criterion can also be defined with the factor 8 replaced with a lower
 integer number.
 
 The reflection coefficient for the amplitude of the electromagnetic
 wave for vertical ($R_v$) and horizontal ($R_v$) polarisation is
 for a flat surface given by the Fresnel equations:
 \begin{eqnarray}
   \label{eq:rte_fresnel}
   R_v &=& \frac{n_2\cos\theta_1-n_1\cos\theta_2}
                                           {n_2\cos\theta_1+n_1\cos\theta_2} \\
   R_h &=& \frac{n_1\cos\theta_1-n_2\cos\theta_2}
                                           {n_1\cos\theta_1+n_2\cos\theta_2} \\
 \end{eqnarray}
 [* check if correct *] where $n_1$ is refractive index for
 the medium where the incoming radiation is propagating, $\theta_1$ is
 the incident angle (measured from the local surface normal) and $n_2$
 is the refractive index of the reflecting medium. The angle
 $\theta_2$ is the propagation direction for the transmited part, and
 is given by Snell's law:
 \begin{equation}
   \label{eq:rte:snell}
   \Re(n_1)\sin\theta_1 = \Re(n_2)\sin\theta_2.
 \end{equation}
 where $\Re(\cdot)$ denotes the complex real part.
 For cases where medium 1 is air, $n_1$ can (in this context) be set to 1.
 
 The power reflection coefficients are converted to an intensity
 reflection coefficient as
 \begin{equation}
   \label{eq:rte:R2r}
   r = |R|^2,
 \end{equation}
 where $|\!\cdot\!|$ denotes the absolute value. Note that $R$ can be
 complex, while $r$ is always real.
 
 The surface reflection can be seen as a scattering event and
 Section~\ref{sec:polarization:ampmatrix} can be used to derive the
 reflection matrix values. The scattering amplitude functions of
 Equation~\ref{eq:polarisation:ampmatrix1} are simply
 \begin{eqnarray}
   S_2 &=& R_v, \\
   S_1 &=& R_h, \\
   S_3 = S_4 &=& =0.
 \end{eqnarray}
 This leads to that the transformation matrix for a specular ground
 reflection is (compare to \citet[Sec.\ 5.4.3]{liou:02})
 \begin{equation}
   \label{eq:rte:specular_matrix}
   \MtrStl{F} =
      \left[\begin{array}{cccc}
        \frac{r_v+r_h}{2}&\frac{r_v-r_h}{2}&0&0\\
        \frac{r_v-r_h}{2}&\frac{r_v+r_h}{2}&0&0\\
     0&0&\frac{R_hR_v^\ast+R_vR_h^\ast}{2}&i\frac{R_hR_v^\ast-R_vR_h^\ast}{2}\\
     0&0&i\frac{R_vR_h^\ast-R_hR_v^\ast}{2}&\frac{R_hR_v^\ast+R_vR_h^\ast}{2}\\
      \end{array}
      \right].
 \end{equation}
 If the downwelling radiatio is unpolarised, the reflected part of the
 upwelling radiation is
 \begin{equation}
   \MtrStl{F}
   \left[ \begin{array}{c} I\\0\\0\\0 \end{array} \right] =
   \left[ \begin{array}{c} I(r_v+r_h)/2\\I(r_v-r_h)/2\\0\\0 
   \end{array} \right].
 \end{equation}
 as expected.

 The workspace variable \artsstyle{ground\_los} shall here
 of course be set to have the length 1. The specular direction is
 calculated by the internal function \funcindex{ground\_specular\_los}.

 [* Fill in what workspace methods that match this section. *]


\levelc{Surface emission}
 \label{sec:rte:surface:emission}
 
 [* Can we relate \artsstyle{ground\_refl\_coeffs} and
 \artsstyle{ground\_emission}? 

 The emission for specular reflection should be 
 \begin{equation}
    \left[\begin{array}{c}
      B\left(1-\frac{r_v+r_h}{2}\right) \\
      B\frac{r_h-r_v}{2} \\
      0\\0
    \end{array}\right]
 \end{equation}
 Correct? *]

 
\levelc{Surface properties}
 \label{sec:rte:surfprop}
 
 This section discusses the propertiesof various surface materials and
 conditions.
 
\levelc{Sea water}
 \label{sec:rte:surfprop:water}

 What is the standard model for $\epsilon$?

 



%%% Local Variables: 
%%% mode: latex
%%% TeX-master: "main"
%%% TeX-master: "uguide"
%%% End: 
