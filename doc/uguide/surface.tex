\chapter{Surface emission and reflections}
 \zlabel{sec:surface}


\starthistory
  050613 & First version finished by Patrick Eriksson. \\
\stophistory

An introduction to the treatment of surface emission and reflections
is given in Section~\zref{sec:fm_defs:surface}. The methods developed
to handle different surface properties all set the variables
\wsvindex{surface\_emission}, \wsvindex{surface\_los} and
\wsvindex{surface\_rmatrix}.

Let us start with a simple example in
order to explain the usage of these workspace variables. We will here
assume that all downwelling radiation is reflected. This assumption is
made for all polarisation states. We assume further a 1D simulation,
that the downwelling radiation shall be calculated for nine zenith
angles and that all downwelling directions contribute equally (which
is not a realistic assumption). The relevant workspace
variables should then be set as follows:
 
 \wsvindex{surface\_emission}: A matrix (of correct size) of zeros.

 \wsvindex{surface\_los}: A vector of length 9, covering the zenith
 angle range. A possible choice would be [5,15,25,\dots,85].

 \wsvindex{surface\_rmatrix}: Each reflection matrix is a diagonal
 matrix with the value 1/9 throughout on the diagonal. That is, all
 elements with index (:,:,i,i) is 1/9. Size matching
 \artsstyle{surface\_los}, \artsstyle{f\_grid} and
 \artsstyle{stokes\_dim}


\section{The geoid and the surface}
%===================
\zlabel{sec:fm_defs:geoid}

The \textindex{geoid} is an imaginary surface used as a
reference when specifying the surface altitude and the altitude
of pressure levels. Any shape of the geoid is allowed but a smoothly
varying geoid is the natural choice, with the enters of the geoid and
the coordinate system coinciding. The geoid should normally be set to
the reference ellipsoid for some global geodetic datum, such as
WGS-84. For further reading on geoid ellipsoids and WGS-84, see
Section~\zref{sec:ppath:geoid}.

Inside ARTS, the geoid is represented as a matrix
(\wsvindex{r\_geoid}), holding the geoid radius, \aRds{\odot}, for
each crossing of the latitude and longitude grids,
$\aRds{\odot}=\aRds{\odot}(\Lat,\Lon)$. The geoid is not defined
outside the ranges covered by the latitude and longitude grids, with
the exception for 1D where the geoid by definition is a full sphere.
The surface altitude, \aAlt{g}, is given as the geometrical altitude
above the geoid. The radius for the surface is accordingly
\begin{equation}
  \aRds{g} = \aRds{\odot} + \aAlt{g}
 \zlabel{eq:fm_defs:zsurface}
\end{equation}
As already mentioned, a gap between the surface and the 
lowermost pressure level is not allowed.

The ARTS variable for the \textindex{surface altitude}
(\wsvindex{z\_surface}) is a matrix of the same size as the geoid
matrix. For 1D, the surface is a sphere by definition (as the geoid),
while for 2D and 3D any shape is allowed and a rough model of the
surface topography can be made. The treatment of surface emission
and reflectivity is discussed in Section~\zref{sec:fm_defs:surface}.




\section{The dielectric constant and the refractive index}
 
 The properties of a material are reported either as the relative
 dielectric constant, $\epsilon$, or the refractive index, $n$. Both
 these quantities can be complex and are related as
 \begin{equation}
   \zlabel{eq:surface_eps2n}
   n = \sqrt{\epsilon}.
 \end{equation}


\section{Relating reflectivity and emissivity}
 \zlabel{sec:surface:surface:ref2emi}
 
 Kirchoff's law applied to thermodynamics states that under conditions
 of local thermodynamic equilibrium, thermal emission has to be equal
 to absorption \citep[page 215]{ulaby:81}. This is a consequence of
 the fact that there must exist a radiation equilibrium between an
 object and its surrounding, if it is surrounded by a blackbody having
 the same temperature (with no physical contact). 
 
 Thermodynamic equilibrium can be assumed for natural surfaces, as
 long as there exist no strong temperature gradients. The Kirchoff law
 can then be used to relate the reflectivity and emissivity of a
 surface. For rough surfaces the scattering properties must be
 integrated over the half sphere (above the surface) to determine the
 emissivity \citep[see e.g.][Eq.\ 4.186]{ulaby:81}. For specular
 reflections (defined below) and scalar radiative transfer
 calculations, the emissivity $e$ is
 \begin{equation}
  \zlabel{eq:e=1-r}
   e = 1 - r,
 \end{equation}
 where $r$ is the reflective (power reflection coefficient) of the
 surface.  Equation \zref{eq:e=1-r} is valid for each polarisation state
 individually \citep[Eq.\ 4.190a]{ulaby:81}.

 We have then that
 \begin{equation}
  \Mpi^\mathrm{up} = \Mpi^\mathrm{down}r + (1-r)B,
 \end{equation}
 where $\Mpi^\mathrm{up}$ is upwelling radiation, $\Mpi^\mathrm{down}$
 is downwelling radiation and $B$ is the magnitude of blackbody
 radiation. As expected, if $\Mpi^\mathrm{down}=B$, also
 $\Mpi^\mathrm{up}$ equals $B$.  Expressing the last observation using
 vector nomenclature gives
 \begin{equation}
   \left[\begin{array}{c} B \\ 0 \\0\\0 \end{array}\right] =
  \MtrStl{R} \left[\begin{array}{c} B \\ 0 \\0\\0 \end{array}\right] + 
  \VctStl{b},
 \end{equation}
 where $\MtrStl{R}$ is the matrix (4\,x\,4) correspondence to the
 scalar reflectivity, describing the properties of the surface
 reflection. The vector \VctStl{b} is the surface emission, that
 can be expressed as
 \begin{equation}
  \zlabel{eq:surface:bvector} 
  \VctStl{b} = (\MtrStl{1}-\MtrStl{R})
      \left[\begin{array}{c} B \\ 0 \\0\\0 \end{array}\right],
 \end{equation}
 where \MtrStl{1} is the identity matrix. 


\section{Specular reflections}
 \zlabel{sec:surface:surface:specular}
 
 If the surface is sufficiently smooth, radiation will be
 reflected/scattered only in the complementary angle, specular
 reflection. Required smoothness for assuming specular reflection is
 normally estimated by the Rayleigh criterion:
 \begin{equation}
   \zlabel{eq:surface:rayleigh}
   \Delta h < \frac{\Wvl}{8\cos\theta_1}
 \end{equation}
 where $\Delta h$ is the root mean square variation of the surface
 height, \Wvl\ the wavelength and $\theta_1$ the angle between the
 surface normal and the incident direction of the radiation. The
 criterion can also be defined with the factor 8 replaced with a lower
 integer number.
 
 The complex reflection coefficient for the amplitude of the
 electromagnetic wave for vertical ($R_v$) and horizontal ($R_v$)
 polarisation is for a flat surface (if the relative magnetic
 permeability ($\mu_r$) of both media is 1) given by the Fresnel equations:
 \begin{eqnarray}
   \zlabel{eq:surface_fresnel}
   R_v &=& \frac{n_2\cos\theta_1-n_1\cos\theta_2}
                                           {n_2\cos\theta_1+n_1\cos\theta_2} \\
   R_h &=& \frac{n_1\cos\theta_1-n_2\cos\theta_2}
                                           {n_1\cos\theta_1+n_2\cos\theta_2} 
 \end{eqnarray}
 where $n_1$ is refractive index for the medium where the reflected
 radiation is propagating, $\theta_1$ is the incident angle (measured
 from the local surface normal) and $n_2$ is the refractive index of
 the reflecting medium. The angle $\theta_2$ is the propagation
 direction for the transmitted part, and is given by Snell's law:
 \begin{equation}
   \zlabel{eq:surface:snell}
   \Re(n_1)\sin\theta_1 = \Re(n_2)\sin\theta_2.
 \end{equation}
 where $\Re(\cdot)$ denotes the complex real part.
 For cases where medium 1 is air, $n_1$ can (in this context) be set to 1.

 The power reflection coefficients are converted to an intensity
 reflection coefficient as
 \begin{equation}
   \zlabel{eq:surface:R2r}
   r = |R|^2,
 \end{equation}
 where $|\!\cdot\!|$ denotes the absolute value. Note that $R$ can be
 complex, while $r$ is always real.

The surface reflection can be seen as a scattering event and
Section~\zref{sec:polarization:ampmatrix} can be used to derive the
reflection matrix values. The scattering amplitude functions of
Equation~\zref{eq:polarisation:ampmatrix1} are simply
\begin{eqnarray}
  S_2 &=& R_v, \\
  S_1 &=& R_h, \\
  S_3 = S_4 &=& =0.
\end{eqnarray}
This leads to that the transformation matrix for a specular surface
reflection is (compare to \citet[Sec.\ 5.4.3]{liou:02})
\begin{equation}
  \zlabel{eq:surface:specular_matrix}
  \MtrStl{R} =
     \left[\begin{array}{cccc}
       \frac{r_v+r_h}{2}&\frac{r_v-r_h}{2}&0&0\\
       \frac{r_v-r_h}{2}&\frac{r_v+r_h}{2}&0&0\\
    0&0&\frac{R_hR_v^\ast+R_vR_h^\ast}{2}&i\frac{R_hR_v^\ast-R_vR_h^\ast}{2}\\
    0&0&i\frac{R_vR_h^\ast-R_hR_v^\ast}{2}&\frac{R_hR_v^\ast+R_vR_h^\ast}{2}\\
     \end{array}
     \right].
\end{equation}
If the downwelling radiation is unpolarised, the reflected part of the
upwelling radiation is
\begin{equation}
  \MtrStl{R}
  \left[ \begin{array}{c} I\\0\\0\\0 \end{array} \right] =
  \left[ \begin{array}{c} I(r_v+r_h)/2\\I(r_v-r_h)/2\\0\\0 
  \end{array} \right].
\end{equation}
as expected.


If \MtrStl{R} is given by Equation~\zref{eq:surface:specular_matrix},
Equation \zref{eq:surface:bvector} gives that the surface emission  is
\begin{equation}
  \zlabel{eq:surface:specular_emission}
   \left[\begin{array}{c}
     B\left(1-\frac{r_v+r_h}{2}\right) \\
     B\frac{r_h-r_v}{2} \\
     0\\0
   \end{array}\right].
\end{equation}
In the case of specular reflections, \artsstyle{surface\_los} shall of
course be set to have the length 1. The specular direction is
calculated by the internal function
\funcindex{surface\_specular\_los}\footnote{Any tilt of the surface is
  neglected when determining the specular direction. If there would be
  any need to consider surface tilt, almost complete code for this
  task existed in \artsstyle{surface\_specular\_los} but was removed
  in version 1-1-876. The code can be obtained by e.g.\ checking out
  version 1-1-875.}.  Equations \zref{eq:surface:specular_matrix} and
\zref{eq:surface:specular_emission} give the values to put into
\artsstyle{surface\_rmatrix} and \artsstyle{surface\_emission}.



 



%%% Local Variables: 
%%% mode: latex
%%% TeX-master: "main"
%%% TeX-master: "uguide"
%%% End: 
