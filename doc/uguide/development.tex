%
% To start the document, use
%  \levela{...}
% For lover level, sections use
%  \levelb{...}
%  \levelc{...}
%
\levela{The art of developing ARTS}
 \label{sec:development}

%
% Document history, format:
%  \starthistory
%    date1 & text .... \\
%    date2 & text .... \\
%    ....
%  \stophistory
%
\starthistory
  020425 & Stefan Buehler: Put this part back in the AUG. Updated.\\
  011005 & Stefan Buehler: Fixed TeX warnings, updated. \\
  000728 & Stefan Buehler: Added stuff about build system and howto cut a release. \\
  000615 & Created by Stefan Buehler. For now, this is basically the
  former content of the file \verb|notes.txt|. \\
\stophistory

%
% Symbol table, format:
%  \startsymbols
%    ... & \verb|...| & text ... \\
%    ... & \verb|...| & text ... \\
%    ....
%  \stopsymbols
%
%

%
% Introduction
%
The aim of this section is to describe how the program is organized
and to give detailed instructions how to make extensions. That means,
it is addressed to the ARTS developers, not the users. If you only
want to use ARTS, you should not need to read it. \textbf{But if you
  want to make changes or additions, you should definitely read this
  carefully, since it can safe you a lot of work to understand how
  things are organized.}

\levelb{Organization}
%====================
\label{sec:development:org}
 
ARTS is written in C++ with the help of the GNU development tools
(Autoconf, Automake, etc.). It is organized in a similar manner as
most GNU packages. The top-level ARTS directory is either called
\verb|arts| or \verb|arts-x.y|, where x.y is the release number.
It contains various sub-directories, notably \verb|doc| for
documentation, \verb|src| for the C++ source code, \verb|ami| for the
MATLAB interface, and \verb|aii| for the IDL interface. The document
that you are reading right now, the ARTS User Guide, is located in
\verb|doc/uguide|.

There are two different versions of the ARTS package: The developers
version and the end-user version. Both contain the complete source
code, the only difference is that the developers version also includes
the CVS housekeeping data. If you want to join in the ARTS development
(which we of course encourage you to do), you should write an email to
the authors to obtain access to the developers version, which makes it
easier to merge your changes with the `official' ARTS program.
Furthermore, for serious development work you need a computer running
Unix, the GNU development tools, LaTeX, and the Doxygen program.  All
this is freely and easily available on the Internet, and, what is
more, all these tools are included in the standard linux
distributions like Suse and Redhat.

The end-user version contains everything that you need in order to
compile and install ARTS in a fairly automatic manner. The only
thing you should need is an ANSI-C++ compiler and the standard Unix
\verb|make| utility. Please see files \verb|arts/README| and
\verb|arts/INSTALL| for installation instructions. We are developing
with the GNU C++ compiler, no other compilers have been tried so
far.

\levelb{The ARTS build system}
%============================

As mentioned above, GNU tools are used to construct the ARTS
build system. A good introduction to the GNU build system can be found in:
\begin{quote}
  \footnotesize
  \verb|http://www.amath.washington.edu/~lf/tutorials/autoconf/|
\end{quote}
Using these tools makes a lot of things very easy, but also some
things slightly more complicated.

The most important thing to keep in mind is that an ARTS release
is not just a copy of the ARTS development tree. Instead there is a
special make target `dist' that you can use to cut a release. How this
is done in detail is described in Section \ref{sec:release}. Mostly,
the GNU tools are smart enough to figure out automatically what should
go into the release. However, this can be controlled by editing the
\verb|Makefile.am| files which can be found in almost all directories.

The support for documentation other than info and man pages is not
very good in the GNU system, so we had to use some tricks to make sure
that the Doxygen automatic documentation and the User Guide work as they
should. 

%\levelc{Configure options}
%%============================

% FIXME: Oliver, please put in a part here about the different
% ARTS specific configure options. (Particularly maintainer-mode.)

% Here are some interesting options for \verb|configure|:
%
%\begin{description}
%\item[--disable-warnings:] 
%  Compile without \verb|-Wall| on g++ compilers (by default warnings are on).
%\item[--disable-assert:] Include \verb|#define NDEBUG 1| in
%  \verb|config.h|.  The central switch to turn off all debugging
%  features (index range checking for vectors, the trace facility,
%  assertions,...) \textbf{Not yet implemented.}
%
%\end{description}


\levelc{Adding Directories or files}
%==================================

If you add directories or just files, you have to make sure that they
also go into the distribution. In some cases (e.g., program source
code files) this is done automatically. But if you add any other kind
of file, for example a data or a documentation file, you have to edit
the \verb|Makefile.am| file in that directory to make sure that your
stuff goes into the distribution. It is a good idea to always check
the release in order to see if the things you added are really there.

\levelb{Conventions}
%===================
\label{sec:development:conv}

Here are some general rules for ARTS programming:

\levelc{} Never use \verb|float| or \verb|double| explicitly, use the
type \verb|Numeric| instead.  This is set by \verb|configure| (to
\verb|double| by default). Thus, it is possible to compile the program
for \verb|float| by simply running configure with a different option.
%
% FIXME: Oliver, please say how.
%
In the same way, use Index for all integers. It can take on positive
or negative values.

\levelc{} Use \verb|Vector| and \verb|Matrix| for mathematical vectors
and matrices (with elements of type \verb|Numeric|). Use
\verb|Array<something>| to create an array of \verb|something|s. Commonly
used Arrays have been predefined, they have names like
\verb|ArrayOfString|, \verb|ArrayOfMatrix|, and so forth.

\levelc{Terminology}
Calculations are carried out in the so called workspace (WS), on
workspace variables (WSVs). A WSV is for example the variable
containing the absorption coefficients. The WSVs are manipulated by 
workspace methods (WSMs). The WSMs to use are specified in the
controlfile in the same order in which they will be
executed. 

\levelc{Global variables}
   Are not visible by default. To use them you have to declare them
   like this:
   \begin{quote}
   \verb|extern const Numeric PI;|
   \end{quote}
   which will make the global constant PI=3.14... available. Other important globals are:

   \begin{quote}
   \begin{tabular}{ll}
   \verb|full_name|&         Full name of the program, including version.\\
   \verb|parameters|&        All command line parameters.\\
   \verb|basename|&          Used to construct output file names.\\
   \verb|out_path|&          Output path.\\
   \verb|messages|&          Controls the verbosity level.\\
   \verb|wsv_data|&          WSV lookup data.\\
   \verb|wsv_group_names|&   Lookup table for the names of \emph{types} of WSVs.\\
   \verb|WsvMap|&            The map associated with \verb|wsv_data|. \\
   \verb|md_data|&           WSM lookup data.\\
   \verb|MdMap|&             The map associated with \verb|md_data|. \\
   \verb|workspace|&         The workspace itself.\\
   \verb|species_data|&      Lookup information for spectroscopic species.\\
   \verb|SpeciesMap|&        The map associated with \verb|species_data|.
   \end{tabular}
   \end{quote}
   The only exception from this rule are the output streams \verb|out0| to
   \verb|out3|, which are visible by default.

\levelc{Files}
Always use the \verb|open_output_file| and \verb|open_input_file|
functions to open files. This switches on exceptions, so that any
error occurring later on with this file will result in an
exception. (Currently not really implemented in the GNU compiler,
but please use it anyway.)

\levelc{Version numbers} 
The package version number is set in file \verb|configure.in| in the
top level ARTS directory. Always increase this when you do a CVS
commit, even for small changes. in such cases increase the last digit
by one. If you make a new distribution, increase the middle digit by
one and omit the last digit. If you make a bug-fix distribution, you
can add the last digit to indicate this. 

\levelc{Header files} 
The global header file \verb|arts.h| \emph{must} be included by every
file. Apart from that you have to see yourself what header files you
need. If you use functions from the C or C++ standard library, you
have to also include the appropriate header file.

\levelc{Documentation}
Doxygen is used to generate automatic source code documentation. See
\begin{quote}
  \verb|http://www.stack.nl/~dimitri/doxygen/|
\end{quote}
for information. There is a complete User manual there. At the moment
we only generate the output as HTML, although latex, man-page, and rtf
format is also possible. The HTML version is particularly useful for
source code browsing, since it includes the complete source code! You
should add Doxygen headers to the following:

\begin{enumerate}
\item Files
\item Classes (Including all private and public members)
\item Functions
\item Global Variables
\end{enumerate}

The documentation headers are comment blocks that look like the
examples below. They should be put above the \emph{definition} of a
function, i.e., in the \verb|.cc| file.  Some functions are defined in
the \verb|.h| file (e.g., inline member functions). In that case the
comment can be put in the \verb|.h| file.

There is an Emacs package (Doxymacs) that makes the insertion of
documentation headers particularly easy. You can find documentation of
this on the Doxymacs webpage: \verb|http://doxymacs.sourceforge.net/|.
To use it for ARTS (provided you have it), put the following in your
Emacs initialization file:

\begin{verbatim}
(require 'doxymacs)

(setq doxymacs-doxygen-style "Qt")

(defun my-doxymacs-font-lock-hook ()
  (if (or (eq major-mode 'c-mode) (eq major-mode 'c++-mode))
      (progn
        (doxymacs-font-lock)
        (doxymacs-mode))))

(add-hook 'font-lock-mode-hook 'my-doxymacs-font-lock-hook)

(setq doxymacs-doxygen-root "../doc/doxygen/html/")
(setq doxymacs-doxygen-tags "../doc/doxygen/arts.tag")
\end{verbatim}

The only really important lines are the first two, where the second
line is the one selecting the style of documentation. The next block
just turns on syntax highlighting for the Doxygen headers, which looks
nice. The last two lines are needed if you want to use the tag lookup
features (see Doxymacs documentation if you want to find out what this
is).  The package allows you to automatically insert headers. The
standard key-bindings are:
\begin{quote}
\begin{tabularx}{.8\hsize}{@{}lX}
\texttt{C-c d ?} & look up documentation for the symbol under the point.\\
\texttt{C-c d r} & rescan your Doxygen tags file.\\
\texttt{C-c d f} & insert a Doxygen comment for the next function.\\
\texttt{C-c d i} & insert a Doxygen comment for the current file.\\
\texttt{C-c d ;} & insert a Doxygen comment for a member variable on the current line (like M-;).\\
\texttt{C-c d m} & insert a blank multi-line Doxygen comment.\\
\texttt{C-c d s} & insert a blank single-line Doxygen comment.\\
\texttt{C-c d @} & insert grouping comments around the current region.\\
\end{tabularx}
\end{quote}
You can call the macros also by name, e.g., \verb|doxymacs-insert-file-comment|.

\leveld{File comment:}

Generated by \verb|doxymacs-insert-file-comment|.

\begin{verbatim}
/*!
\file   dummy.cc
\author Stefan Buehler <sbuehler@uni-bremen.de>
\date   Thu Apr 25 15:58:50 2002

\brief  A dummy file.

 This file has no purpose at all,
 it just servers as an example... 
*/
\end{verbatim}

\leveld{Function comment:}

Generated by \verb|doxymacs-insert-function-comment|.
If arguments are modified by the function you should
add `Output:' after the \verb|\param| command, just like for the
parameter \verb|a| in the example below. If a parameter is both input
and output, you should say `Output and Input:'. The documentation for
each parameter should start with a capital letter and end with a
period, like in the example below.

Author and date tags are not inserted by default, since they would be
overkill if you have many small functions. However, you should include
them for important functions. 

\begin{verbatim}
//! A dummy function.
/*! 
 This function has no purpose at all,
 it just serves as an example... 

\param  a Output: This parameter is modified by the
          function.
\param  b This is the other parameter.         
\return   Dummy value computed from a and b.         
*/
int dummy(int& a, int b);
\end{verbatim}

\leveld{Generic multi-line comment:}

Generated by \verb|doxymacs-insert-blank-multiline-comment|.

\begin{verbatim}
//! A dummy comment.
/*! 
 Some more elaborate description about this variable, 
 class, or whatever. 
*/
\end{verbatim}

\leveld{Generic single-line comment:}

Generated by \verb|doxymacs-insert-blank-singleline-comment|.

\begin{verbatim}
//! Short comment here.
\end{verbatim}


\levelb{Extending ARTS}
%======================
 \label{sec:development:extending}

\levelc{How to add a workspace variable}
%---------------------------------------

You should read Section{sec:agendas:wsvs} to understand what workspace
variables are. Here is just the practical description how a new
variable can be added.

\begin{enumerate}
\item Create a record entry in file \verb|workspace.cc|. (Just add
  another one of the \verb|wsv_data.push_back| blocks.) Take the
  already existing entries as templates. The ARTS concept works best
  if WSVs are only of a rather limited number of different types, so
  that generic WSMs can be used extensively, for example for IO.
      
  The name must be \emph{exactly} like you use it in the source code,
  because this is used to generate interface functions.
  
  Make sure that the documentation string you give explains the
  variable and its purpose well. \textbf{In particular, state the
    dimensions (in the case of matrices) and the units!} This string
  is used for the online documentation. Please take some time to write
  it carefully. Use the template at the beginning of function
  \verb|define_wsv_data()| in file \verb|workspace.cc| as a
  guideline. 

\item That's it!
\end{enumerate}


\levelc{How to add a workspace variable group}
%--------------------------------------------

You should read Section{sec:agendas:wsvs} to understand what workspace
variable groups are. Here is just the practical description how a new
group can be added.

\begin{enumerate}
\item Add a \verb|wsv_group_names.push_back("your_type")| function to
  the function \verb|define_wsv_group_names()| in \verb|groups.cc|. The
  name must be \emph{exactly} like you use it in the source code,
  because this is used to generate interface functions.
\item That's it! (But as stated above, use this feature wisely)
\end{enumerate}



\levelc{How to add a workspace method}
%-------------------------------------

You should read Section{sec:agendas:wsms} to understand what workspace
methods are. Here is just the practical description how a new
method can be added.

\begin{enumerate}
\item Create an entry in the function \verb|define_md_data| in file
  \verb|methods.cc|.  (Make a copy of an existing entry (one of the
  \verb|md_data.push_back(...)| blocks) and edit it to fit your new
  method.) Don't forget the documentation string! Please refer to the
  example at the beginning of the file to see how to format it.
\item Run:
  \verb|make|.
\item Look in \verb|auto_md.h|. There is a new function prototype
  \begin{quote}
    \verb|void <YourNewMethod>(...)|
  \end{quote}
\item Add your function to one of the \verb|.cc| files which contain method
  functions. Such files must have names starting with \verb|m_|. (See
  separate HowTo if you want to create a new source file.) The header
  of your function must be compatible with the prototype in \verb|md.h|.
\item Check that everything looks nice by running 
  \begin{quote}
    \verb|arts -d YourNewMethod|
  \end{quote}
  If necessary, change the documentation string.

\item Thats it!
\end{enumerate}


\levelc{How to add a source code file}
%---------------------------------------
\begin{enumerate}
\item Create your file. Names of files containing workspace methods should
  start with \verb|m_|.
\item You have to register your file in the file \verb|src/Makefile.am|.
  This file states which source files are needed for arts. Should be
  self-explanatory where you have to add your file. The above goes for
  source (\verb|.cc|) and header (\verb|.h|) files likewise.
\item Then go to the top level arts directory and run: \verb|reconf|.
\item Go to \verb|src| and run: \verb|cvs add <my_file>| to make your
  file known to CVS.
\end{enumerate}


\levelc{How to add an example file}
%---------------------------------------
\begin{enumerate}
\item Create your own example file. The filename should end with
  \verb|_example.arts.in|.
\item If your example uses files from the arts-data package, replace
  the path to the data package (e.g. \verb|/pool/lookup2/arts-data|)
  with \verb|@ac_arts_data@|. Configure will replace this with the
  correct path.
\item Add your file to the variable \verb|arts_examples| in the file\newline
  \verb|doc/examples/Makefile.am|.
\item Add your file to the AC\_OUTPUT list near the end of configure.in.
\item The next time when you call \verb|make| the \verb|.arts.in| file will
  be automatically converted to \verb|.arts|.
\end{enumerate}


\levelb{CVS issues}
%======================
 \label{sec:development:cvs}

The arts project is controlled by CVS. This section describes some
basic CVS commands. For more information see the extensive CVS
documentation or our own CVS Howto on:
\begin{quote}
  \verb|http://www.sat.uni-bremen.de/docs/|
\end{quote}




\levelc{How to check out arts}
%-----------------------------
\begin{enumerate}
\item Go to a temporary directory.
\item Run: \verb|cvs co -P arts|.
\end{enumerate}


\levelc{How to update (if you already have a copy)}
%--------------------------------------------------
\begin{enumerate}
\item Go to the top ARTS directory (called simply \verb|arts|).
\item Run: \verb|cvs update -P|
   
  \textbf{IMPORTANT!} Always update, before you start to make changes
  to the program, especially after a longer pause. If you edit an
  outdated copy, it will be a lot more work to bring your changes into
  the current copy of the program.
\end{enumerate}


\levelc{How to commit your changes}
%---------------------------------------
\begin{enumerate}
\item You should make sure that the program compiles and runs without
  obvious errors before you commit.
\item If you have created a new source file, make it known to CVS by
  running the command \verb|cvs add <my_file>| in the directory where
  the file resides.
  
  In general, when you run \verb|cvs update|, it will warn you about
  any files it doesn't know by marking them with a \verb|?|. Files
  that are created during the compilation process, but should not be
  part of the package are listed in the \verb|.cvsignore| files in
  each directory.
\item Have you added the documentation for your new features?
\item Increase the subversion number in file \verb|configure.in| in
  the top level ARTS directory.
\item Open the file \verb|ChangeLog| in the top level ARTS directory
  with your favorite editor.
  
  With Emacs, you can very easily add an entry by typing either
  \begin{quote}
    \verb|M-x add-change-log-entry|
  \end{quote}
  or \verb|C-x 4 a|.
  
  Specify the new version number and describe your changes.

  \textbf{These keystrokes work also while you are editing some other
    file in Emacs. Thus it is best to write your ChangeLog entry
    already while you work on a file}. Whenever you make a change to a
  file, there should be a ChangeLog Entry!
\item Make sure that you have saved all your files. Go to the top
  level ARTS directory and run: \verb|cvs commit|.
\item This will pop up an editor. Use the mouse to cut and paste the
  Change-Log message also to this editor window. Safe the file and exit
  the editor. If you made changes in different directories, another
  editor will pop up, already containing your message. Save again and
  exit. Do this until no more editors come up. (Note: This works well
  if you set
  \begin{quote}
    \verb|export EDITOR=xedit|
  \end{quote}
  in you shell startup file.
 
  With smart editors there can be problems, because they might
  refuse to safe your file if you haven't made changes to it. With
  xedit you just have to push the save button twice to override.
\item You have to give your version of the program a symbolic name, so
  that it can be retrieved later on if necessary. Do this by running:
  \verb|cvs tag arts-x-y-z| where x,y,z must be replace by the version
  numbers. You have to use dashes to separate the numbers, a point
  (\verb|.|) will not work.
\item Tell the other developers about it. The best way to do this is
  to send an email to \verb|arts-dev@sat.physik.uni-bremen.de|.
\end{enumerate}


\levelc{How to cut a release}
%----------------------------
\label{sec:release}
\begin{enumerate}
\item Change the release number in the file \verb|configure.in| in the
  top-level ARTS directory. (The line that you have to change is the
  one with \verb|AM_INIT_AUTOMAKE|.) Omit the subversion number (last digit).
\item Commit your changes (see other howto). 
\item In the top-level ARTS directory, run \verb|reconf|.
\item In the top-level ARTS directory, run \verb|make distcheck|. This
  will not only cut the release, but also immediately try to build
  it, to see if it works. Unless you are on a very fast machine, this
  may take a while. Maybe you should go and have a cup of coffee.
\item If all goes well, you can find the release inside the top-level
  ARTS directory as a file \verb|arts-x.y.tar.gz|, where x.y is the
  release number.
\item Check the release carefully by trying to build and install the
  program. 
\end{enumerate}


\levelc{How to move your arts working directory}
%----------------------------------------------
\textbf{Never try to move CVS directories!} Instead:
\begin{enumerate}
\item Commit your changes.
\item Go \emph{above} the top level ARTS directory.
\item Run: \verb|cvs release -d arts|.
  
  This will ask for confirmation, and if you say \verb|y| delete your
  working copy of arts.
\item Go to the directory where you want to have your ARTS copy in the
  future.
\item Check out a new copy (see other howto above).
\end{enumerate}

\levelb{Debugging (use of assert)}
%================================
 \label{sec:development:assert}

This section is taken more or less literally from the GNU tools manual
of Eleftherios Gkioulekas:
\begin{quote}
{\footnotesize
\verb|http://www.amath.washington.edu/~lf/tutorials/autoconf/|}
\end{quote}

The idea behind assert is simple. Suppose that at a certain point in
your code, you expect two variables to be equal.  If this expectation
is a precondition that must be satisfied in order for the subsequent
code to execute correctly, you must assert it with a statement like
this:
\begin{quote}
\verb|assert(var1 == var2);|
\end{quote}

In general assert takes as argument a boolean expression. If the
boolean expression is true, execution continues. Otherwise the
\verb|abort| system call is invoked and the program execution is
stopped. If a bug prevents the precondition from being true, then you
can trace the bug at the point where the precondition breaks down
instead of further down in execution or not at all.  The \verb|assert| call
is implemented as a C preprocessor macro, so it can be enabled or
disabled at will. One way to enable assertions is to include
\verb|assert.h|.
\begin{quote}
  \verb|#include <assert.h>|
\end{quote}
Then it's possible to disable them by defining the `NDEBUG' macro.

%
% FIXME: Oliver, please update
%
%In ARTS, assertions are turned on and off with the global NDEBUG
%preprocessor macro, which can be set or unset in file
%\verb|arts.h|. In the future there will be also a configure option to
%achieve this (FIXME: Update this).

During debugging and testing it is a good idea to leave assertions
enabled. However, for production runs it's best to disable them. If
your program crashes at an assertion, then the first thing you should
do is to find out where the error happens. To do this, run the program
under the \verb|gdb| debugger. First invoke the debugger:
\begin{quote}
\verb|gdb|
\end{quote}
Then load the executable and set a breakpoint at the \verb|exit|
system call:
\begin{quote}
  \verb|(gdb) file arts|\\
  \verb|(gdb) break exit| (or \verb|break __assert_fail|)
\end{quote}
Now run the program: 
\begin{quote}
  \verb|(gdb) run|
\end{quote}

Instead of crashing, under the debugger the program will be paused
when the \verb|exit| system call is invoked, and you will get back the
debugger prompt. Now type:
\begin{quote}
  \verb|(gdb) where| 
\end{quote}  
to see where the crash happened. You can use the \verb|print| command to
look at the contents of variables and you can use the \verb|up| and \verb|down|
commands to navigate the stack. For more information, see the GDB
documentation or type \verb|help| at the prompt of gdb.

For ARTS, the assertion failures mostly happen inside the Matrix /
Vector package (usually because you triggered a range check error,
i.e., you tried to read or write beyond array bounds). In this case the
\verb|up| command of GDB is particularly useful. If you give this a
couple of times you will finally end up in the part of your code that
caused the error.

Recommendation: In Emacs there is a special GDB mode. With this you
can very conveniently step through your code.




%%% Local Variables: 
%%% mode: latex
%%% TeX-master: "uguide"
%%% End: 
