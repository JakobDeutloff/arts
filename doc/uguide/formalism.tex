%
% To start the document, use
%  \levela{...}
% For lover level, sections use
%  \levelb{...}
%  \levelc{...}
%
\levela{Theoretical formalism}
 \label{sec:formalism}

%
% Document history, format:
%  \starthistory
%    date1 & text .... \\
%    date2 & text .... \\
%    ....
%  \stophistory
%
\starthistory
  000306 & Created by Patrick Eriksson. This section is to a large extent 
           based on \citet{eriksson:99} and \citet{eriksson:00a}. \\
\stophistory



%
% Symbol table, format:
%  \startsymbols
%    ... & \verb|...| & text ... \\
%    ... & \verb|...| & text ... \\
%    ....
%  \stopsymbols
%
%
\startsymbols
  \mpbi   & \verb|-|      & monochromatic pencil beam intensity      \\
  \f      & \verb|f|      & frequency                                \\
  \view   & \verb|view|   & viewing angle from zenith                \\
  \iv     & \verb|y|      & vector of monochromatic pencil beam intensities \\
  \y      & \verb|y|      & spectrum recorded by a sensor            \\
  \fm     & \verb|-|      & forward model                            \\
  \fma    & \verb|-|      & atmospheric part of \fm                  \\
  \fms    & \verb|-|      & sensor part of \fm                       \\
  \xt     & \verb|-|      & state vector (variables to be retrieved) \\
  $\xt_r$ & \verb|-|      & atmospheric part of \xt                  \\
  $\xt_s$ & \verb|-|      & sensor part of \xt                       \\
  $\xt_\merr$& \verb|-|   & part of \xt describing measurement errors   \\
  \bt     & \verb|-|      & forward model parameter vector           \\
  $\bt_r$ & \verb|-|      & atmospheric part of \bt                  \\
  $\bt_s$ & \verb|-|      & sensor part of \bt                       \\
  $\bt_\merr$& \verb|-|   & part of \bt describing measurement errors   \\
  \Kx     & \verb|K|      & state weighting function matrix          \\
  \Kb     & \verb|K|      & model parameter weighting function matrix\\  
 \label{symtable:formalism}     
\stopsymbols



%
% Introduction
%
In this section a theoretical framework for the forward model is
presented. The presentation follows \citet{rodgers:90}, but some
extensions are made, for example, the distinction between the
atmospheric and sensor parts of the forward model is also discussed
here.



\levelb{The forward model}
 \label{sec:formalism:fm}
 
 The radiative intensity, \mpbi, at a point in the atmosphere, $r$, for
 frequency \f\ and traversing in the direction, \view, is dependent
 on a variety of physical processes and continuous variables such as
 the temperature profile, $T$:

 \begin{equation}
   \mpbi = F(r,\f,\view,T,\dots)
 \end{equation} 
 To detect the spectral radiation some kind of sensor, having a finite
 spatial and frequency resolution, is needed, and the observed
 spectrum becomes a vector, \y, instead of a continuous function.
 The atmospheric radiative transfer is simulated by a computer model
 using a limited number of parameters as input, and the forward model,
 \fm, used in practice can be expressed as
 
 \begin{equation}
   \y = \fm(\xt_\fm,\bt_\fm) + \merr(\xt_\merr,\bt_\merr)
  \label{eq:formalism:fm}
 \end{equation}
 where $(\xt_\fm,\bt_\fm)$ and $(\xt_\merr,\bt_\merr)$ together give a
 total description of both the atmospheric and sensor states, and
 \merr\ is the measurement errors. The parameters are divided in such
 way that \xt, the state vector, contains the parameters to be
 retrieved, and the remainder is given by \bt, the model parameter
 vector. The total state vector is
 \begin{equation}
   \xt = \left[ \begin{array}{c} \xt_\fm \\ \xt_\merr \end{array} \right]
 \end{equation}
 and the total model parameter vector is
 \begin{equation}
   \bt = \left[ \begin{array}{c} \bt_\fm \\ \bt_\merr \end{array} \right]
 \end{equation}
 The actual forward model consists of either empirically determined
 relationships, or numerical counterparts of the physical
 relationships needed to describe the radiative transfer and sensor
 effects. The forward model described here is mainly of the latter
 type, but some parts are more based on empirical investigations, such
 as the parameterisations of continuum absorption. It should be noted
 that a possible data reduction is also part of the forward model.
  
 Both for the theoretical formalism and the practical implementation,
 it is suitable to make a separation of the forward model into two
 main sections, a first part describing the atmospheric radiative
 transfer for pencil beam (infinite spatial resolution) monochromatic
 (infinite frequency resolution) signals \citep{eriksson:99},

 \begin{equation}
   \iv = \fma(\xt_r,\bt_r)
  \label{eq:formalism:fma}
 \end{equation}
 and a second part modelling sensor characteristics,
 \begin{equation}
   \y = \fms(\iv,\xt_s,\bt_s) + \merr(\xt_\merr,\bt_\merr)
  \label{eq:formalism:fms}
 \end{equation}
 where \iv\ is the vector holding the spectral values for the
 considered set of frequencies and viewing angles (i.e.
 $\iv^i=I(\f^i,\view^i)$, where $i$ is the vector index), and
 $\xt_\fm$ and $\bt_\fm$ are separated correspondingly, that is,
 $\xt_\fm^T= [\xt_r^T,\xt_s^T]$ and $\bt_\fm^T= [\bt_r^T,\bt_s^T]$. The
 vectors \xt\ and \bt\ can now be expressed as
 \begin{equation}
   \xt = \left[ \begin{array}{c} \xt_r\\ \xt_s \\ \xt_\merr \end{array} \right]
 \end{equation}
 and
 \begin{equation}
   \bt = \left[ \begin{array}{c} \bt_r\\ \bt_s \\ \bt_\merr \end{array}\right],
 \end{equation}
 respectively.

 The subscripts of \xt\ and \bt\ are below omitted if the part of the
 vectors used is made clear by the context. 



\levelb{The sensor transfer matrix} 
 \label{sec:formalism:sensor}
  
 The modelling of the different sensor parts can be described by a
 number of of analytical expressions (see \citet{eriksson:97a}) that
 together makes the basis for the sensor model. These expressions are
 throughout linear operations and it possible, as suggested in
 \citet{eriksson:00a}, to implement the sensor model as a
 straightforward matrix multiplication:
 \begin{equation}
   \y = \Hm \iv + \merr
  \label{eq:formalism:H}
 \end{equation}
 where \Hm\ is here denoted as the sensor transfer matrix.  The matrix
 \Hm\ can  be set up to incorporate effects of a data reduction and
 the total transfer matrix is then
 \begin{equation}
   \Hm = \Hd \Hs
  \label{eq:formalism:Hs}
 \end{equation}
 as
 \begin{equation}
   \y = \Hd \y' = \Hd (\Hs \iv + \merr') = \Hm \iv + \merr
  \label{eq:formalism:datared}
 \end{equation}
 where \Hd\ is the reduction matrix, \Hs\ the sensor matrix, and $\y'$
 and $\merr'$ are the measurement vector and the measurement errors,
 respectively, before data reduction. The matrices \Hd\ and \Hs\ are
 described in Section \ref{sec:sensor} and \ref{sec:red}, respectively.



\levelb{Weighting functions} 
 \label{sec:formalism:wfuns}

 \levelc{Basics} 
 A weighting function is the partial derivative of the spectrum vector
 \y\ with respect to some variable used by the forward model.  As the
 input of the forward model is divided between \xt\ or \bt, the
 weighting functions are divided correspondingly between two matrices,
 the state weighting function matrix

 \begin{equation}
   \Kx = \frac{\partial \y}{\partial \xt}
  \label{eq:formalism:kx}
 \end{equation}
 and the model parameter weighting function matrix
 \begin{equation}
   \Kb = \frac{\partial \y}{\partial \bt}
  \label{eq:formalism:kb}
 \end{equation}
 For the practical calculations of the weighting functions, it is
 important to note that the atmospheric and sensor parts can be
 seperated. For example, if \xt\ only hold atmospheric and
 spectroscopic variables, \Kx\ can be expressed as

 \begin{equation}
   \Kx = \frac{\partial \y}{\partial \iv}\frac{\partial\iv}{\partial \xt} =
    \Hm\frac{\partial\iv}{\partial \xt}
  \label{eq:formalism:kx2}
 \end{equation}
 This equation shows that the new parts needed to calculate
 atmospheric weighting functions, are functions giving $\partial\iv /
 \partial \xt$ where \xt\ can represent the vertical profile of a
 species, atmospheric temperature, spectroscopic data etc.
 
 The practical calculation of weighting functions is discussed in
 detail in Sections \ref{sec:wfuns} and \ref{sec:wfuns_sens}.


 \levelc{Transformation between vector spaces}
 
  It could be of interest to transform a weighting function matrix from
  one vector space to another. The new vector, $\xt'$, is here
  assumed to be of length $n$ $(\xt' \in \msize{n}{1})$, while the original
  vector, \xt\ is of length $p$ $(\xt \in \msize{p}{1})$.  The
  relationship between the two vector spaces is described by a
  transformation matrix $\B$:
  \begin{equation}
    \xt = \B\xt'
  \end{equation}
  where $\B \in \msize{p}{n}$. For example, if $\xt'$ is assumed to be
  piecewise linear, then the columns of $\B$ contain tenth functions,
  that is, a function that are 1 at the point of interest and decreases
  linearly down to zero at the neighbouring points.  The matrix can
  also hold a reduced set of eigenvectors.
    
  The weighting function matrix corresponding to $\xt'$ is
  \begin{equation}
    \K_{\xt'} = \frac{\partial \y}{\partial \xt'}
  \end{equation}
  This matrix is related to the weighting function matrix of \xt\ (Eq.
  \ref{eq:formalism:kx}) as
  \begin{equation}
    \K_{\xt'}
      = \frac{\partial \y}{\partial \xt} \frac{\partial \xt}{\partial \xt'}
      = \frac{\partial \y}{\partial \xt} \B 
      = \Kx \B
  \end{equation}
  Note that
  \begin{equation}
    \K_{\xt'}\xt' = \Kx\B\xt' =  \Kx\xt
  \end{equation}
  However, it should be noted that this relationship only holds for
  those \xt\ that can be represented perfectly by some $\xt'$ (or vice
  versa), that is, $\xt=\B\xt'$, and not for all combinations of \xt\ 
  and $\xt'$.

  If $\xt'$ is the vector to be retrieved, we have that \citep{rodgers:90}
  \begin{equation}
    \xret' = \im(\y,\ct) = \tm(\xt,\bt,\ct)
  \end{equation}
  where \im\ and \tm\ are the inverse and transfer model, respectively.

  The contribution function matrix is accordingly
  \begin{equation}
    \Dy =  \frac{\partial \xret'}{\partial \y}
  \end{equation}
  that is, \Dy\ corresponds to $\K_{\xt'}$, not \Kx.
  
  We have now two possible averaging kernel matrices
  \begin{equation}
    \A_{\xt} 
      = \frac{\partial \xret'}{\partial \xt} 
      = \frac{\partial \xret'}{\partial \y} \frac{\partial \y}{\partial \xt}
      = \Dy \Kx
  \end{equation}
  \begin{equation}
    \A_{\xt'} 
      = \frac{\partial \xret'}{\partial \xt'} 
      = \frac{\partial\xret'}{\partial\y}\frac{\partial\y}{\partial\xt}
      \frac{\partial\xt}{\partial\xt'}
      = \Dy \K_{\xt'}
      = \A_{\xt} \B
  \end{equation}
  where $\A_{\xt} \in \msize{p}{n}$ and $\A_{\xt'} \in
  \msize{p}{p}$, that is, only $\A_{\xt'}$ is square. 




%%% Local Variables: 
%%% mode: latex
%%% TeX-master: "main"
%%% End: 
