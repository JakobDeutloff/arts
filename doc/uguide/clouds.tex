\graphicspath{{Figs/clouds/}}

\chapter{Description of clouds}
 \label{sec:clouds}

\starthistory
 050913 & Created and written by Claudia Emde\\ 
\stophistory

\section{Introduction}
\label{sec:clouds:intro}

In the Earth's atmosphere we find liquid water clouds consisting of
approximately spherical water droplets and cirrus clouds consisting of
ice particles of diverse shapes and sizes. We also find different
kinds of aerosols. In order to take into account this variety, the
model allows to define several \emph{scattering elements}.

A scattering element is either a specific single particle or a particle
ensemble, e.g., an ensemble following a certain size or shape distribution. The
scattering element can represent particles that are completely randomly
oriented, azimuthally randomly oriented or arbitrarily oriented. Each scattering
element is characterized by its single scattering properties and a field of
particle number densities.
For each grid point in the cloud box, the atmospheric volume that enclosed all
scattering particles, the single scattering properties of all scattering
elements weighted by their respective particle number density at this location
are summed up to derive the cloud ensemble optical properties.

The \wsvindex{scat\_data} structure contains the single scattering properties
($\langle\ExtMat_i\rangle$, $\langle\AbsVec_i\rangle$, and
$\langle\PhaMat_i\rangle$) for each of the scattering elements. 
In \wsvindex{scat\_data}, the single scattering properties are stored in
different coordinate systems, depending on the kind of particle. For instance, for
randomly oriented particles it makes sense to store the single scattering
properties in the so-called scattering frame in order to reduce memory
requirements. The following section describes in detail the
\typeindex{SingleScatteringData} class.

The number density field, \wsvindex{pnd\_field}, contains the number densities
of all scattering elements at all grid points within the cloudbox. Section
\ref{sec:clouds:size_distr} describes how to realize different kinds of size
distributions in the ARTS frame by defining appropriate particle number density
fields. \wsvindex{pnd\_field} can be read in from externally prepared data
files or be derived from mass density of flux fields provided to the model.
Section \ref{sec:clouds:implement} below describes the first. The latter can be
done using \wsmindex{pnd\_fieldCalcFromscat\_speciesFields} (\FIXME: document
here in AUG!).


\section{\textindex{Single scattering properties}}
\label{sec:clouds:ssp}

\subsection[Coordinate systems]{\textindex{Coordinate systems}: The
  \textindex{laboratory frame} and the \textindex{scattering frame} }
\label{sec:clouds:coordinate_sytems}

For radiative transfer calculations we need a coordinate system to
describe the direction of propagation. For this purpose we use the
laboratory frame, which is shown in
\theory, Figure \ref{T-fig:RT_theory_coordinates}.  The z-axis corresponds to the
local zenith direction and the x-axis points towards the
north-pole. The propagation direction is described by the local zenith
angle $\theta$ and the local azimuth angle $\phi$.  This coordinate
system is the most appropriate frame to describe the propagation
direction and the polarization state of the radiation.  However, in
order to describe scattering of radiation by a particle or a particle
ensemble, it makes sense to define another coordinate system taking
into consideration the symmetries of the particle or the scattering
medium, as one gets much simpler expressions for the single scattering
properties.  For macroscopically isotropic and mirror-symmetric
scattering media it is convenient to use the scattering frame, in
which the incidence direction is parallel to the z-axis and the x-axis
coincides with the scattering plane, that is, the plane through the
unit vectors \VctStl{\hat{n}^{\inc}}\ and \VctStl{\hat{n}^{\sca}}. The
scattering frame is illustrated in
Figure \ref{fig:scattering:part_frame}. For symmetry reasons the single
scattering properties defined with respect to the scattering frame can
only depend on the scattering angle $\Theta$,
\begin{equation}
  \label{eq:scat_angle}
  \Theta=\arccos(\hat{\PDir}^{\inc} \cdot {\hat{\PDir}^{\sca}}),
\end{equation}
between the incident and the scattering direction.

\begin{figure}[htbp]
 \begin{center}
   \includegraphics*[width=0.4\hsize]{part_frame}
   \caption{Illustration of the scattering frame. The z-axis coincides with the incident direction $\hat{\PDir}^{\inc}$. The scattering angle $\Theta$ is the angle between  $\hat{\PDir}^{\inc}$ and $\hat{\PDir}^{\sca}$.}
   \label{fig:scattering:part_frame}  
 \end{center}
\end{figure}

\subsection{Scattering data structure}
\label{sec:clouds:ARTS_SSP_structure}
 
The single scattering data is stored in a specific structure format, the
the \typeindex{SingleScatteringData} class\footnote{Definition resides in
\fileindex{optproperties.h}.}. The format allows space reduction due to
symmetry for certain special cases, e.g. random orientation or
horizontal alignment. The class consists of the following fields (compare also
Table \ref{tab:scattering:datastructure}):

\begin{itemize}
\item  {\sl String} \shortcode{ptype}: An attribute which contains
  information about the 
  data type, which is a classification of the scattering
  particle, specifically regarding its symmetry properties (randomly oriented,
  horizontally aligned, general case, \dots). This attribute is
  needed in the radiative transfer function to be able to extract
  the physical phase matrix, the physical extinction matrix and the
  physical absorption vector from the data. 
  
  Possible values of ptype are:
  
  \shortcode{"general"} \\
  \shortcode{"macroscopically\_isotropic"} \\
  \shortcode{"horizontally\_aligned"}
  
  A more detailed description of the different cases is given below.

\item {\sl String} \shortcode{description}: Here the scattering element
  can be characterized explicitly. For example, information on the size and
  shape of the particle or the respective distributions of a particle ensemble
  might be given. This can be a longer text describing how the scattering
  properties were generated. It should be formatted for direct printout to screen
  or file.
  
\item {\sl Vector} \shortcode{f\_grid}: Frequency grid [Unit: Hz].
  
\item {\sl Vector} \shortcode{T\_grid}: Temperature grid [Unit: K].
  
\item {\sl Vector} \shortcode{za\_grid}:
  \begin{enumerate}
  \item \shortcode{"general", }: Zenith angle grid (Range: 0.0\degree $\le$ za $\le$ 180.0\degree).
  \item \shortcode{"horizontally\_aligned"}: Scattering angle grid (Range: 0.0\degree $\le$ za $\le$ 180.0\degree).
  \end{enumerate}
  
\item {\sl Vector} \shortcode{aa\_grid}: Azimuth angle grid.
  \begin{enumerate}
  \item \shortcode{"general"}: Range: -180.0\degree $\le$ aa $\le$ 180.0\degree
  \item \shortcode{"macroscopically\_isotropic"}: Not needed, since optical properties depend only on
    scattering angle (dummy grid).
  \item \shortcode{"horizontally\_aligned"}: Only half of the grid is required (Range: 0.0\degree $\le$ aa $\le$ 180.0\degree)
  \end{enumerate}
  
  The angular grids have to satisfy the following conditions:
  \begin{itemize}
  \item They have to be equidistant. \FIXME{is this actually (still) true?}
  \item The value of the data must be the same for the first and the
    last grid-point \FIXME{of the azimuthal grid? doesn't make sense for zenith
    angle grid.}. This condition is required for the integration routine.
  \item If we only have to store a part of the grid, for example
    \shortcode{za\_grid} only from 0\degree to 90\degree, these two values
    (0\degree, 90\degree) must be grid-points.
  \end{itemize}
  
\item {\sl Tensor7} \shortcode{pha\_mat\_data}: Phase matrix data
  \EnsAvr\PhaMat\ [Unit: m$^2$]. The dimensions of the data array are:  
  
  \shortcode{[frequency temperature za\_sca aa\_sca za\_inc aa\_inc matrix\_element]}
  
  The order of matrix elements depends on the chosen case. For most
  cases we do not need all matrix elements (see description of cases
  below).

\item {\sl Tensor5} \shortcode{ext\_mat\_data}: Extinction matrix data
  \EnsAvr\ExtMat\ [Unit: m$^2$]. The dimensions are: 

  \shortcode{[frequency temperature za\_inc aa\_inc matrix\_element]}
  
  Again, the order of matrix elements depends on the chosen case.

\item {\sl Tensor5} \shortcode{abs\_vec\_data}: Absorption vector data
  \EnsAvr\AbsVec\ [Unit: m$^2$]. 
  
  The absorption vector is also precalculated. It could be calculated
  from extinction matrix and phase matrix. But this calculation takes
  long computation time, as it requires an angular integration over
  the phase matrix. For the cases with symmetries (e.g., random
  orientation) the data files will not become too large even if we
  store additionally the absorption vector. The dimensions are: 
  
 \shortcode{[frequency temperature za\_inc aa\_inc vector\_element]}
\end{itemize}

\begin{table}
\label{tab:scattering:datastructure}
\caption{Structure of single scattering data files}
\begin{flushleft}
\begin{tabular}{llll}
\hline
\multicolumn{1}{c}{Symbol}&Type&Dimensions&Description \\
\hline
  &enum& & ptype specification \\
  &String& & short description of the scattering element \\
\Frq & Vector & (\Frq) & frequency grid \\
\Tmp  & Vector & (\Tmp) & temperature grid \\
\ZntAng & Vector & (\ZntAng) & zenith angle grid \\
\AzmAng & Vector & (\AzmAng) & azimuth angle grid \\
\EnsAvr{\PhaMat}  & Tensor7 & (\Frq, \Tmp, \ZntAng, \AzmAng,
$\ZntAng'$, $\AzmAng'$, $i$ )  & phase matrix \\ 
\EnsAvr{\ExtMat} & Tensor5  & (\Frq, \Tmp, \ZntAng, \AzmAng, $i$ ) & extinction matrix \\
\EnsAvr{\AbsVec} & Tensor5 & (\Frq, \Tmp, \ZntAng, \AzmAng, $i$ ) & absorption vector\\
\hline
\end{tabular}
\end{flushleft}
\end{table}

\subsection{Definition of \textindex{ptypes}}
\label{sec:clouds:particle_types}

Ptype essentially classifies the scattering elements regarding their
orientation. As described above, this effects the optimal choice of the
coordinate system to represent the scattering element in. Below we give a more
detailed description of each possible ptype classification in the model.

\subsubsection{``macroscopically\_isotropic''}
The ptype value ``macroscopically\_isotropic'' refers to macroscopically
isotropic and mirror-symmetric scattering media. It covers totally randomly
oriented particles as well as spherical particles.
For this type of scattering media the optical properties are
calculated in the so-called scattering frame as shown in
Figure \ref{fig:scattering:part_frame}. In this coordinate 
system the z-axis corresponds to the incident direction and the
xz-plane coincides with the scattering plane. Using this frame only
the scattering angle, which is the angle between incident and
scattered direction is needed. Furthermore the number of matrix
elements of both matrices, phase matrix and extinction matrix, can be
reduced (see \citet{Mishchenko:02}, p.90). To calculate the
particle optical properties it is convenient to use Mishchenko's
T-matrix code for randomly oriented particles \citep{Mishchenko:98}
which returns the averaged phase matrix and extinction matrix. 
The only drawback is that the single scattering data has
to be transformed from the particle frame representation to the
laboratory frame representation. These transformations are described
in the appendix of \citet{emde05:_phdthesis}.

Only six elements of the transformed phase matrix, which is commonly
called scattering matrix \ScaMat, are different. Therefore the size of
\shortcode{pha\_mat\_data} is: 

\shortcode{[N\_f N\_T N\_za\_sca 1 1 1 6]}\\
The order of the matrix elements is as follows: {\sl F11, F12, F22,
  F33, F34, F44}\\
The extinction matrix is in this case diagonal and independent of
direction and polarization. That means that we need to store only one
element for each frequency. Hence the size of
\shortcode{ext\_mat\_data} is
 
\shortcode{[N\_f N\_T 1 1 1]}\\
The absorption vector is also direction and polarization
independent. Therefore the size of \shortcode{abs\_vec\_data} for this
case is the same as \shortcode{ext\_mat\_data}: 

\shortcode{[N\_f N\_T 1 1 1]}

\subsubsection{``horizontally\_aligned''}

The ptype value ``horizontally\_aligned'' refers to particles that are
aligned with respect to polar angle, specifically in horizontal direction, but
oriented randomly regarding the azimuthal angle. For these particles,
the angular dimension can be
reduced by one, if we rotate the coordinate system appropriately. For
this case we use the T-matrix code for single particles in fixed
orientation and average phase matrix and extinction matrix manually
like in the general case.

The phase matrix (and also extinction matrix and absorption vector)
become independent of the incident azimuth angle in this
frame. Furthermore, regarding the symmetry of this case, it can be
shown that for the scattered directions we need only half of the
angular grids, as the two halves must contain the same
data. \shortcode{pha\_mat\_data} therefore has the following size:

\shortcode{[N\_f N\_T N\_za\_sca N\_aa\_sca N\_za\_inc/2+1 1 16]}\\
We store \shortcode{za\_sca} for all grid points from 0\degree to 180\degree,
\shortcode{aa\_sca} from 0\degree to 
180\degree, and \shortcode{za\_inc} from 0\degree to 90\degree. This means that the
zenith angle grid 
has to include 90\degree as grid-point. The order of the matrix elements is
the same as in the general case. For this case it can be shown that the extinction matrix has only
three elements {\sl Kjj}, {\sl K12(=K21)}, and {\sl K34(=-K43)}. 
Because of azimuthal symmetry, it can not depend on the azimuth
angle. Hence the size of \shortcode{ext\_mat\_data} is 

\shortcode{[N\_f N\_T N\_za/2+1 1 3]}\\
The absorption coefficient vector has only two elements {\sl a1} and
{\sl a2}. This means that the size of \shortcode{abs\_vec\_data} is 

\shortcode{[N\_f N\_T N\_za/2+1 1 2]}

\subsubsection{``general''}

The ptype value ``general'' refers to arbitrarily shaped and oriented particles.
For those, generally no symmetries ecist, hence all 16 elements of
the phase matrix have to be stored. The average phase matrix has to be generated from
all individual phase matrices of the particles in the distribution
outside ARTS. The individual phase matrices are calculated using
Mishchenko's T-matrix code for single particles in fixed orientation
\citep{Mishchenko:00}. 
We have to store all elements for all angles in the grids. The size of
\shortcode{pha\_mat\_data} is therefore: 

\shortcode{[N\_f N\_T N\_za\_sca N\_aa\_sca N\_za\_inc N\_aa\_inc 16]}\\
The matrix elements have to be stored in the following order: {\sl Z11,
  Z12, Z13, Z14, Z21, Z22,~...} Seven extinction matrix elements are
independent (cp. \citet{Mishchenko:02}, p.55). The elements being equal for
single particles should still be the equal for a distribution as we
get the total extinction just by adding. Here we need only the
incoming grids, so the size of ext\_mat\_data is: 

\shortcode{[N\_f N\_T N\_za\_inc N\_aa\_inc 7]}\\
The absorption vector in general has four components (cp. Equation
(2.186) in \citet{Mishchenko:02}). The size of abs\_vec\_data is
accordingly: 

\shortcode{[N\_f N\_T N\_za\_inc N\_aa\_inc 4]}

\subsection{Generating single scattering properties}
The single scattering properties have to be pre-calculated before entering the
scattering solver. This can be done for example by Mie, T-Matrix or
Discrete dipole approximation methods. Different methods for the calculation of
single scattering properties are reviewed in \citet{emde05:_phdthesis}.

Calculations of the single scattering properties can be done by external tools
or by internally interfaced methods:
\begin{itemize}
\item
The ATMLAB package includes functions to generate single scattering
properties for spherical particles (Mie-Theory). 
\item
It is also convenient to use the Python module PyARTS, which has been
developed especially for ARTS and which is freely available at
\href{http://www.sat.ltu.se/arts/tools/}
{\url{http://www.sat.ltu.se/arts/tools/}}. This
module can be used to generate single scattering properties for
horizontally aligned as well as for randomly oriented particles in the
ARTS data-file-format. PyARTS has been developed by C. Davis, who has
implemented the Monte Carlo scattering algorithm in ARTS (see
\theory, Section \ref{T-sec:montecarlo}).
\item
The WSM \wsmindex{scat\_data\_singleTmatrix} provides an ARTS internal interface
to the the T-matrix code by \citet{Mishchenko:02}.
\end{itemize}

\section[Particle size distributions]
{Representation of the \textindex{particle size distribution}}
\label{sec:clouds:size_distr}

The particle size has an important impact on the scattering and
absorption properties of particle ensembles, e.g. clouds, as shown for instance
in \citep{emde04:_doit_jgr}.  Particle ensembles typically contain a whole range
of different particle sizes, which can be described by a size
distribution giving the number of particles per unit volume per unit
radius interval as a function of radius.  It is most convenient to
parameterize the size distribution by analytical functions, because in
this case optical properties can be calculated much faster than for
arbitrary size distributions. The T-matrix code for randomly oriented
particles includes several types of analytical size distributions,
e.g., the gamma distribution or the log-normal distribution.  This
section presents the size distribution parameterizations, which were
used for the ARTS simulations included in this thesis.

\subsection{Mono-disperse particle distribution}

The most simple assumption is, that all particles in an ensemble have
the same size.  In order to study scattering effects like polarization
or the influence of particle shape, it makes sense to use this most
simple assumption, because one can exclude effects resulting from the
particle size distribution itself.  

Along with the single scattering properties we need the particle
number density field, which specifies the number of particles per
unit volume at each grid point, for ARTS scattering simulations.  For
a given \imc\ and mono-disperse particles the particle number density
$n^p$ is simply
\begin{equation}
\label{eq:pnd_mono}
  n^p (\imc, r) =\frac{\imc}{m} = \frac{\imc}{V\rho}
    =\frac{3}{4\pi}\frac{\imc}{\rho r^3},  
\end{equation}
where $m$ is the mass of a particle, $r$ is its equal volume sphere
radius, $\rho$ is its density, and $V$ is its volume.
     
\subsection{Gamma size distribution}
\label{sec:clouds:gamma_distr}

A commonly used distribution for radiative transfer modeling in cirrus
clouds is the \emph{gamma distribution}
\begin{equation}
  n(r) = a  r^\alpha \exp(-br).
\label{eq:gamma_distr}
\end{equation}
The dimensionless parameter $\alpha$ describes the width of the
distribution. The other two parameters can be linked to the effective
radius \Reff\ and the ice mass content \imc\ as follows:
\begin{eqnarray}
  b &= \frac{\alpha+3}{\Reff},\\
  a &= \frac{\imc}{4/3\pi\rho b^{-(\alpha+4)}\Gamma[\alpha+4]},
\label{eq:gamma_coeff}
\end{eqnarray}
where $\rho$ is the density of the scattering medium and $\Gamma$ is
the gamma function. For cirrus clouds $\rho$ corresponds to the bulk
density of ice, which is approximately 917 kg/m$^3$.

Generally, the effective radius \Reff\ is defined as the average
radius weighted by the particles' cross-sectional area
\begin{equation}
  \label{eq:Reff}
  \Reff= \frac{1}{\EnsAvr{A}} \int_{r_{min}}^{r_{max}} {A(r) r n(r) \DiffD r},
\end{equation}
where $A$ is the area of the geometric projection of a particle. The
minimal and maximal particle sizes in the distribution are given by
$r_{min}$ and $r_{max}$ respectively. In the case of spherical
particles $A = \pi r^2$. The average area of the geometric projection
per particle \EnsAvr{A} is given by
\begin{equation}
  \EnsAvr{A} = \frac{\int_{r_{min}}^{r_{max}} {A(r) n(r) \DiffD r}}{\int_{r_{min}}^{r_{max}} {n(r) \DiffD r}}.
\end{equation}

The question is how well a gamma distribution can represent the true
particle size distribution in radiative transfer calculations. This
question is investigated by \citet{evans:98}. The authors come to the
conclusion that a gamma distribution represents the distribution of
realistic clouds quite well, provided that the parameters \Reff, \imc\ 
and $\alpha$ are chosen correctly. They show that setting $\alpha = 1$
and calculating only \Reff\ gave an agreement within 15\% in 90\% of
the considered measurements obtained during the First ISCCP Regional
Experiment (FIRE).  Therefore, for all calculations including gamma
size distributions for ice particles, $\alpha = 1$ was assumed.  

The particle number density for size distributions is obtained by
integration of the distribution function over all sizes:
\begin{eqnarray}
  n^p(\imc, \Reff) &= \int_0^\infty {n(r)\DiffD r} \\
 &= \int_0^\infty {a  r^\alpha \exp(-br)
    \DiffD r} = a \frac{\Gamma(\alpha+1)}{b^{\alpha+1}}.
\end{eqnarray}

After setting $\alpha$ = 1, inserting Equation \ref{eq:gamma_coeff} and
some simple algebra we obtain
\begin{equation}
  \label{eq:pnd_gamma}
  n^p(\imc, \Reff) = \frac{2}{\pi} \frac{\imc}{\rho \Reff^3}.
\end{equation}
Comparing Equation \ref{eq:pnd_mono} and \ref{eq:pnd_gamma}, we see
that the particle number density for mono-disperse particles with a
particle size of $R$ is smaller than the particle number density for
gamma distributed particles with $\Reff=R$. The reason is that in the
gamma distribution most particles are smaller than~\Reff.

\subsection[McFarquhar and Heymsfield parametrization]
{Ice particle size parameterization by McFarquhar and Heymsfield}
\label{sec:McFHey_distr}

A more realistic parameterization of tropical cirrus ice crystal size
distributions was derived by
\citet{mcfarquar97:_param_tropic_cirrus_ice_cryst}, who derived the
size distribution as a function of temperature and \imc. The
parameterization was made based on observations during the Central
Equatorial Pacific Experiment (CEPEX). Smaller ice crystals with an
equal volume sphere radius of less than 50\,\mum\ are parametrized
as a sum of first-order gamma functions:
\begin{equation}
  \label{eq:MH_gamma}
  n(r) = \frac{12\cdot \imc_{<50}\alpha^5_{<50} r}{\pi \rho
    \Gamma(5)} \exp(-2 \alpha_{<50} r), 
\end{equation}
where $\alpha_{<50}$ is a parameter of the distribution, and
$\imc_{<50}$ is the mass of all crystals smaller than 50\,\mum\ in
the observed size distribution.  Large ice crystals are represented
better by a log-normal function
\begin{eqnarray}
  \label{eq:MH_lognorm}
  n(r) = \frac{3\cdot\imc_{>50}}{\pi^{3/2}\rho \sqrt{2}
    \exp(3\mu_{>50}+(9/2)\sigma^2_{>50}) r \sigma_{>50} r_0^3}
  \nonumber \\
  \cdot \exp\left[-\frac{1}{2}\left(\frac{\log\frac{2r}{r_0} -
          \mu_{>50}}{\sigma_{>50}}\right)^2\right], 
\end{eqnarray}
where $\imc_{>50}$ is the mass of all ice crystals greater than
50\,\mum\ in the observed size distribution, $r_0$ = 1\,\mum\ is a
parameter used to ensure that the equation does not depend on the
choice of unit for r, $\sigma_{>50}$ is the geometric standard
deviation of the distribution, and $\mu_{>50}$ is the location of the
mode of the log-normal distribution.  The fitted parameters of the
distribution can be looked up in the article by
\citet{mcfarquar97:_param_tropic_cirrus_ice_cryst}.  The particle
number density field is obtained by numerical integration over a
discrete set of size bins. This parameterization of particle size has
been implemented in the PyARTS package, which was introduced in
Section \ref{sec:clouds:ARTS_SSP_structure}. Using PyARTS one can calculate the size
distributions, the corresponding single scattering properties and the
particle number density fields for given \imc\ and temperature.


\section{Implementation}
\label{sec:clouds:implement}

The workspace methods related to the description of clouds in ARTS are
implemented in the file \fileindex{m\_cloudbox.cc}.
Work space methods related to the optical properties of the clouds are
implemented in the file \fileindex{m\_optproperties.cc}. The coordinate system
transformations described above reside in the file
\fileindex{optproperties.cc}.

\subsection{Work space methods and variables}

The following controlfile section illustrates how a simple cloud can
be included in an ARTS calculation. 

First we have to define the cloudbox region, i.e. the region where
scattering objects are found. To do this we can use the method
\wsmindex{cloudboxSetManuallyAltitude}:
\begin{code}
cloudboxSetManuallyAltitude( cloudbox_on, cloudbox_limits,
                             atmosphere_dim, z_field,
                             lat_grid, lon_grid,
                             8000, 120000,
                             0, 0, 0, 0 )
\end{code}
If we want to do a simulation for a
cirrus cloud at an altitude from 9 to 11\,km the cloudbox limits can
be set to 8 and 12\,km. The latitude and longitude limits are set to
an arbitrary value for a 1D calculation. For 3D calculations they are
also needed. Alternatively one can use the method
\wsmindex{cloudboxSetManually}, where one has to provide pressure
instead of altitude limits. 
 
Now we have to specify the cloud particles inside the scattering
region:
\begin{code}
# Initialisation
ParticleTypeInit
# Only one scattering element is added in this example 
ParticleTypeAdd( scat_data, pnd_field_raw,
                 atmosphere_dim, f_grid,
                 "ssd_sphere_50um_macroscopically_isotropic.xml",
                 "pnd_sphere_50um_macroscopically_isotropic.xml" )
\end{code}
In the workspace method \wsmindex{ParticleTypeAdd} the single
scattering properties for one scattering element are read. The
generic input \shortcode{filename\_scat\_data} must be set to
the filename of a datafile containing the single scattering data (class
\typeindex{SingleScatteringData}) in xml-format. The generic input
\shortcode{filename\_pnd\_field} must contain the filename of the
corresponding particle number density field in xml-format (class
\typeindex{GField3}). If the cloud is composed of several
different scattering elements \wsmindex{ParticleTypeAdd} can be used
repeatedly for all scattering elements, for instance to the
randomly oriented spherical particles from above one could add horizontally aligned
cylindrical particles:
\begin{code}
ParticleTypeAdd( scat_data, pnd_field_raw,
                 atmosphere_dim, f_grid,
                 "ssd_cylinder_30um_horizontally_aligned.xml",
                 "pnd_cylinder_30um_horizontally_aligned.xml" )
\end{code}
Alternatively it is possible to use the method
\wsmindex{ParticleTypeAddAll}, which is convenient to generate a size
distribution using several size bins. In this case one needs to define
one scattering element for each size bin. If the num,ber of size bins is large,
the control file
becomes very lengthy when using \wsmindex{ParticleTypeAdd}
repeatedly. \wsmindex{ParticleTypeAddAll} requires as input an array
of string including the filenames of the single scattering data files for all
individual scattering elements
and the variable \wsvindex{pnd\_field\_raw} which includes the particle
number density fields for all scattering elements at once. Using this function, one
has to make sure that the order of the filenames containing the single
scattering data corresponds to the order of the particle number
density fields in \wsvindex{pnd\_field\_raw}.
After reading the data the workspace variable \wsvindex{pnd\_field} is
calculated using \wsmindex{pnd\_fieldCalcFrompnd\_field\_raw}: 
\begin{code}
# Calculate the particle number density field
pnd_fieldCalcFrompnd_field_raw
\end{code}

The definition of the single scattering data along with the
corresponding particle number density fields is common in both
scattering modules, the DOIT module described in 
Chapter \ref{sec:scattering:doit} and the Monte Carlo module in
Chapter \ref{sec:scattering:mc}.


%%% Local Variables: 
%%% mode: latex
%%% TeX-master: "uguide"
%%% End: 
