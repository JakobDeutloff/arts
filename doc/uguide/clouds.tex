\chapter{Description of clouds}
 \label{sec:clouds}

\starthistory
 050913 & Created and written by Claudia Emde\\ 
\stophistory

\section{Introduction}

In the Earth's atmosphere we find liquid water clouds consisting of
approximately spherical water droplets and cirrus clouds consisting of
ice particles of diverse shapes and sizes. We also find different
kinds of aerosols. In order to take into account this variety, the
model allows to define several \emph{particle types}. A particle type
is either a specified particle or a specified particle distribution,
for example a particle ensemble following a gamma size distribution.
The particles can be completely randomly oriented, azimuthally
randomly oriented or arbitrarily oriented. For each particle type
being a part of the modeled cloud field, a data file containing the
single scattering properties ($\langle\ExtMat_i\rangle$,
$\langle\AbsVec_i\rangle$, and $\langle\PhaMat_i\rangle$), and the
appropriate particle number density field is required. The particle
number density fields are stored as \typeindex{GriddedField3}, including the
field stored in a three-dimensional tensor and also the appropriate
atmospheric grids (pressure, latitude and longitude grid). For each
grid point in the cloud box the single scattering properties are
averaged using the particle number density fields.  In the scattering
database the single scattering properties are not always stored in the
same coordinate system. For instance for randomly oriented particles
it makes sense to store the single scattering properties in the
so-called scattering frame in order to reduce memory requirements. 
This chapter describes in detail the \typeindex{SingleScatteringData}
class. 

\section{Single scattering properties}
\label{sec:scattering:ssp}

\subsection[Coordinate systems]{Coordinate systems: The laboratory frame and the scattering frame}
\label{sec:scattering:coordinate_sytems}

For radiative transfer calculations we need a coordinate system to
describe the direction of propagation. For this purpose we use the
laboratory frame, which has been introduced in
\secref{sec:fm_defs} and which is also shown in \figref{fig:RT_theory_coordinates}.
The z-axis corresponds to the local zenith direction and the x-axis
points towards the north-pole. The propagation direction is described
by the local zenith angle $\theta$ and the local azimuth angle $\phi$.
This coordinate system is the most appropriate frame to describe the
propagation direction and the polarization state of the radiation.
However, in order to describe scattering of radiation by a particle or
a particle ensemble, it makes sense to define another coordinate
system taking into consideration the symmetries of the particle or the
scattering medium, as one gets much simpler expressions for the single
scattering properties.  For macroscopically isotropic and
mirror-symmetric scattering media it is convenient to use the
scattering frame, in which the incidence direction is parallel to the
z-axis and the x-axis coincides with the scattering plane, that is,
the plane through the unit vectors \VctStl{\hat{n}^{\inc}}\ and
\VctStl{\hat{n}^{\sca}}. The scattering frame is illustrated in
\figref{fig:scattering:part_frame}. For symmetry reasons the single scattering
properties defined with respect to the scattering frame can only
depend on the scattering angle $\Theta$,
\begin{equation}
  \label{eq:scat_angle}
  \Theta=\arccos(\hat{\PDir}^{\inc} \cdot {\hat{\PDir}^{\sca}}),
\end{equation}
between the incident and the scattering direction.

\begin{figure}[htbp]
 \begin{center}
   \includegraphics*[width=0.4\hsize]{./Figs/scattering/part_frame}
   \caption{Illustration of the scattering frame. The z-axis coincides with the incident direction $\hat{\PDir}^{\inc}$. The scattering angle $\Theta$ is the angle between  $\hat{\PDir}^{\inc}$ and $\hat{\PDir}^{\sca}$.}
   \label{fig:scattering:part_frame}  
 \end{center}
\end{figure}

\subsection{Scattering datafile stucture}
\label{sec:scattering:ARTS_SSP_structure}
 
The single scattering properties are pre-calculated, for example by
using the T-matrix 
code by \citet{Mishchenko:02}, and stored in data-files. Different
methods for the calculation of single scattering properties are
reviewed in \citet{emde05:_phdthesis}. 

The format of the scattering database allows space reduction due to
symmetry for certain special cases, e.g. random orientation or
horizontal alignment. The file format is xml. The data is stored in a
class called \typeindex{SingleScatteringData}, which resides in
the files \fileindex{optproperties.h}. The class consists of the
following 
fields (compare also \tabref{tab:scattering:datastructure}):

\begin{itemize}
\item  {\sl enum} \artsstyle{ptype}: An attribute which contains
  information about the 
  data type, which is the classification of the kind of hydrometeor
  species (randomly oriented, general case~...). This attribute is
  needed in the radiative transfer function to be able to extract
  the physical phase matrix, the physical extinction matrix and the
  physical absorption vector from the data. 
  
  Possible values of ptype are:
  
  \artsstyle{PTYPE\_GENERAL} = 10 \\
  \artsstyle{PTYPE\_MACROS\_ISO} = 20\\
  \artsstyle{PTYPE\_HORIZ\_AL} = 30\\
  
  A more detailed description of the different cases is given below.

\item {\sl String} \artsstyle{description}: Here the particle type
  should be specified 
  explicitly. We can have the case randomly oriented particles, but
  furthermore we also have to specify the exact particle properties
  (i.e. size and shape distribution). This can be a longer text
  describing how the scattering properties were generated. It should
  be formated for direct printout to screen or file.
  
\item {\sl Vector} \artsstyle{f\_grid}: Frequency grid [Unit: Hz].
  
\item {\sl Vector} \artsstyle{T\_grid}: Temperature grid [Unit: K].
  
\item {\sl Vector} \artsstyle{za\_grid}:
  \begin{enumerate}
  \item \artsstyle{p10, p30}: Zenith angle grid (Range: 0.0\degree $\le$ za $\le$ 180.0\degree).
  \item \artsstyle{p20}: Scattering angle grid (Range: 0.0\degree $\le$ za $\le$ 180.0\degree).
  \end{enumerate}
  
\item {\sl Vector} \artsstyle{aa\_grid}: Azimuth angle grid.
  \begin{enumerate}
  \item \artsstyle{p10}: Range: -180.0\degree $\le$ aa $\le$ 180.0\degree
  \item \artsstyle{p20}: Not needed, since optical properties depend only on
    scattering angle (dummy grid).
  \item \artsstyle{p30}: Only half of the grid is required (Range: 0.0\degree $\le$ aa $\le$ 180.0\degree)
  \end{enumerate}
  
  The angular grids have to satisfy the following conditions:
  \begin{itemize}
  \item They have to be equidistant.
  \item The value of the data must be the same for the first and the
    last grid-point. This condition is required for the integration
    routine.
  \item If we only have to store a part of the grid, for example
    \artsstyle{za\_grid} only from 0\degree to 90\degree, these two values (0\degree, 90\degree) must be
    grid-points. 
  \end{itemize}
  
\item {\sl Tensor7} \artsstyle{pha\_mat\_data}: Phase matrix data
  \EnsAvr\PhaMat\ [Unit: m$^2$]. The dimensions of the data array are:  
  
  \artsstyle{[frequency temperature za\_sca aa\_sca za\_inc aa\_inc matrix\_elment]}
  
  The order of matrix elements depends on the chosen case. For most
  cases we do not need all matrix elements (see description of cases
  below).

\item {\sl Tensor5} \artsstyle{ext\_mat\_data}: Extinction matrix data
  \EnsAvr\ExtMat\ [Unit: m$^2$]. The dimensions are: 

  \artsstyle{[frequency temperature za\_inc aa\_inc matrix\_element]}
  
  Again, the order of matrix elements depends on the chosen case.

\item {\sl Tensor5} \artsstyle{abs\_vec\_data}: Absorption vector data
  \EnsAvr\AbsVec\ [Unit: m$^2$]. 
  
  The absorption vector is also precalculated. It could be calculated
  from extinction matrix and phase matrix. But this calculation takes
  long computation time, as it requires an angular integration over
  the phase matrix. For the cases with symmetries (e.g., random
  orientation) the data files will not become too large even if we
  store additionally the absorption vector. The dimensions are: 
  
 \artsstyle{[frequency temperature za\_inc aa\_inc vector\_element]}
\end{itemize}

\begin{table}
\label{tab:scattering:datastructure}
\caption{Structure of single scattering data files}
\begin{flushleft}
\begin{tabular}{llll}
\hline
\multicolumn{1}{c}{Symbol}&Type&Dimensions&Description \\
\hline
  &enum& & specification of particle type \\
  &String& & short description of particle type \\
\Frq & Vector & (\Frq) & frequency grid \\
\Tmp  & Vector & (\Tmp) & temperature grid \\
\ZntAng & Vector & (\ZntAng) & zenith angle grid \\
\AzmAng & Vector & (\AzmAng) & azimuth angle grid \\
\EnsAvr{\PhaMat}  & Tensor7 & (\Frq, \Tmp, \ZntAng, \AzmAng,
$\ZntAng'$, $\AzmAng'$, $i$ )  & phase matrix \\ 
\EnsAvr{\ExtMat} & Tensor5  & (\Frq, \Tmp, \ZntAng, \AzmAng, $i$ ) & extinction matrix \\
\EnsAvr{\AbsVec} & Tensor5 & (\Frq, \Tmp, \ZntAng, \AzmAng, $i$ ) & absorption vector\\
\hline
\end{tabular}
\end{flushleft}
\end{table}

\subsection{Definition of particle types}
\label{sec:scattering:ARTS_SSP_structure}

\subsubsection{Macroscopically isotropic and mirror-symmetric scattering
  media (p20)}
For macroscopically isotropic and mirror-symmetric scattering media
(totally randomly oriented particles) the optical properties are
calculated in the so-called scattering frame as shown in
\figref{fig:scattering:part_frame}. In this coordinate 
system the z-axis corresponds to the incident direction and the
xz-plane coincides with the scattering plane. Using this frame only
the scattering angle, which is the angle between incident and
scattered direction is needed. Furthermore the number of matrix
elements of both matrices, phase matrix and extinction matrix, can be
reduced (see \citet{Mishchenko:02}, p.90). To calculate the
particle optical properties it is convenient to use Mishchenko's
T-matrix code for randomly oriented particles \citep{Mishchenko:98}
which returns the averaged phase matrix and extinction matrix. 
The only drawback is that the single scattering data has
to be transformed from the particle frame representation to the
laboratory frame representation. These transformations are described
in the appendix of \citet{emde05:_phdthesis}.

Only six elements of the transformed phase matrix, which is commonly
called scattering matrix \ScaMat, are different. Therefore the size of
\artsstyle{pha\_mat\_data} is: 

\artsstyle{[N\_f N\_T N\_za\_sca 1 1 1 6]}\\
The order of the matrix elements is as follows: {\sl F11, F12, F22,
  F33, F34, F44}\\
The extinction matrix is in this case diagonal and independent of
direction and polarization. That means that we need to store only one
element for each frequency. Hence the size of
\artsstyle{ext\_mat\_data} is
 
\artsstyle{[N\_f N\_T 1 1 1]}\\
The absorption vector is also direction and polarization
independent. Therefore the size of \artsstyle{abs\_vec\_data} for this
case is the same as \artsstyle{ext\_mat\_data}: 

\artsstyle{[N\_f N\_T 1 1 1]}

\paragraph{Horizontally aligned plates and columns (p30)}

For particle distributions of horizontally aligned plates and columns
that are oriented randomly in the azimuth the angular dimension can be
reduced by one, if we rotate the coordinate system appropriately. For
this case we use the T-matrix code for single particles in fixed
orientation and average phase matrix and extinction matrix manually
like in the general case.

The phase matrix (and also extinction matrix and absorption vector)
become independent of the incident azimuth angle in this
frame. Furthermore, regarding the symmetry of this case, it can be
shown that for the scattered directions we need only half of the
angular grids, as the two halfs must contain the same
data. \artsstyle{pha\_mat\_data} therefore has the following size:

\artsstyle{[N\_f N\_T N\_za\_sca N\_aa\_sca N\_za\_inc/2+1 1 16]}\\
We store \artsstyle{za\_sca} for all grid points from 0\degree to 180\degree,
\artsstyle{aa\_sca} from 0\degree to 
180\degree, and \artsstyle{za\_inc} from 0\degree to 90\degree. This means that the
zenith angle grid 
has to include 90\degree as grid-point. The order of the matrix elements is
the same as in the general case. For this case it can be shown that the extinction matrix has only
three elements {\sl Kjj}, {\sl K12(=K21)}, and {\sl K34(=-K43)}. 
Because of azimuthal symmetry, it can not depend on the azimuth
angle. Hence the size of \artsstyle{ext\_mat\_data} is 

\artsstyle{[N\_f N\_T N\_za/2+1 1 3]}\\
The absorption coefficient vector has only two elements {\sl a1} and
{\sl a2}. This means that the size of \artsstyle{abs\_vec\_data} is 

\artsstyle{[N\_f N\_T N\_za/2+1 1 2]}

\paragraph{General case (p10)}

If there are no symmetries at all we have to store all 16 elements of
the phase matrix. The average phase matrix has to be generated from
all individual phase matrices of the particles in the distribution
outside ARTS. The individual phase matrices are calculated using
Mishchenko's T-matrix code for single particles in fixed orientation
\citep{Mishchenko:00}. 
We have to store all elements for all angles in the grids. The size of
\artsstyle{pha\_mat\_data} is therefore: 

\artsstyle{[N\_f N\_T N\_za\_sca N\_aa\_sca N\_za\_inc N\_aa\_inc 16]}\\
The matrix elements have to be stored in the following order: {\sl Z11,
  Z12, Z13, Z14, Z21, Z22,~...} Seven extinction matrix elements are
independent (cp. \citet{Mishchenko:02}, p.55). The elements being equal for
single particles should still be the equal for a distribution as we
get the total extinction just by adding. Here we need only the
incoming grids, so the size of ext\_mat\_data is: 

\artsstyle{[N\_f N\_T N\_za\_inc N\_aa\_inc 7]}\\
The absorption vector in general has four components (cp. Equation
(2.186) in \citet{Mishchenko:02}). The size of abs\_vec\_data is
accordingly: 

\artsstyle{[N\_f N\_T N\_za\_inc N\_aa\_inc 4]}

\paragraph{Generating single scattering properties}
It is very convenient to use the PYTHON module PyARTS, which has been
developed especially for ARTS and which is freely available at
\href{http://www.sat.uni-bremen.de/cgi-bin/cvsweb.cgi/PyARTS/}
{\url{http://www.sat.uni-bremen.de/cgi-bin/cvsweb.cgi/PyARTS/}}. This
module can be used to generate single scattering properties for
horizontally aligned as well as for randomly oriented particles in the
ARTS data-file-format. PyARTS has been developed by C. Davis, who has
implemented the Monte Carlo scattering algorithm in ARTS (see
\secref{sec:montecarlo}).
The ATMLAB package includes functions to generate single scattering
properties for spherical particles (Mie-Theorie). 


\section{Particle number density fields}

\section{Implementation}

\FIXME{Write down what is implemented in which files}
\subsection{Work space methods and variables}

\subsubsection{How to read scattering data}

\FIXME{include ParticleTypeInit/Add common for MC and DOIT}

\subsubsection{How to use the data in ARTS}

\FIXME{refer to DOIT and MC chpters since different functions are
  used. }

%%% Local Variables: 
%%% mode: latex
%%% TeX-master: "uguide"
%%% End: 
