
%--- Packages to use
%
\usepackage[]{fancyhdr}   
\usepackage[]{natbib}
\usepackage{alltt}
\usepackage{times}
\usepackage{lscape}         % landscape mode of a single page
\usepackage[]{longtable}    % allow tables longer than one page
\usepackage{makeidx}        % index of terms
\usepackage{tabularx}       % allows line breaking in table columns
\usepackage{algorithm}      % for describing algorithms with pseudo-code
\usepackage{algorithmic}
\usepackage{ifthen}
\usepackage{ifpdf}
\usepackage{xr-hyper}
\usepackage{fancyvrb}
\usepackage{xcolor}
\usepackage{amsmath}
\usepackage{minted}


%--- Margins
%
\voffset-1.5cm
\headheight16pt
\headsep1.1cm
\textheight23cm
\hoffset-1.3cm
\oddsidemargin2.2cm
\textwidth14.0cm


%--- Headings
%
\pagestyle{fancy}
\renewcommand{\chaptermark}[1]{\markboth{#1}{}}
\renewcommand{\sectionmark}[1]{\markright{\thesection\ #1}}
\fancyhf{}
\fancyhead[LE,RO]{\small{\sc\thepage}}
\fancyhead[LO]{\small{\scshape\rightmark}}
\fancyhead[RE]{\small{\scshape\leftmark}}
\renewcommand{\headrulewidth}{0.5pt}
\renewcommand{\footrulewidth}{0pt}
\fancypagestyle{plain}{%
  \fancyhead{}
  \renewcommand{\headrulewidth}{0pt}
}  


%--- Some layout commands
%
\sloppy
\raggedbottom
\hbadness=10000
\makeindex
\bibliographystyle{agu}


%--- Some fixes

%- To avoid hyperref error:
\newcommand{\theHalgorithm}{\theHchapter.\arabic{algorithm}}

%- Width of the caption in longtable:
\setlength{\LTcapwidth}{0.9\textwidth} 

%- Change brace type for comments in algorithmic
\renewcommand{\algorithmiccomment}[1]{(#1)}


%--- Symbol definitions
%
% This file defines the general math macros.
% Mathematical symbols used only once, or for a particular purpose, 
% should not be included here. Note that the scalar quantities exist 
% in a subscript version, and it is not necessary to define macros for
% all possible subscripts of a variable.
% A lot of macros have been defined for the Rodgers formalism is it
% used extensively.

% If you add new definitions, please try to follow the rules below to
% get a naming scheme as consistent as possible. Just check how a 
% similar macro is defined and use it as an example.

% Patrick Eriksson 2001-03-13

%-----------------------------------------------------------------------------

% Most of the macros are named by putting 3-letters acronyms together. 
% The acronyms are mainly formed by taking the first letter and the two 
% first following consonants. The list below shows the used acronyms.
% Please, if you introduce a new acronym, add it to this list.
%
% altitude     Alt
% angel        Ang
% average      Avr
% azimuthal    Azm
% constant     Cns
% contribution Cnt
% covariance   Cvr
% derivative   Drv
% error        Err
% frequency    Frq
% forward      Frw
% function     Fnc
% identity     Idn
% inverse      Inv
% kernel       Krn
% latitude     Lat       (exception from general naming scheme)
% length       Lng
% longitude    Lon       (exception from general naming scheme)
% matrix       Mtr
% measurement  Msr
% model        Mdl
% partial      Prt
% population   Ppl
% pressure     Prs
% radius       Rds
% retrieval/ed Rtr
% sensor       Sns
% size         Sze
% space        Spc
% state        Stt
% style        Stl
% symbol       Smb
% temperature  Tmp
% transfer     Trf
% transpose    Trp
% wavelength   Wvl
% vector       Vct
% zenith       Znt
%
% a priori                              Apr
% monochromatic pencil beam intensity   Mpi
% optical thickness                     Oth
% weighting function                    Wfn

% For other terms or features, use as far as possible complete strings to get 
% macros with clear names.


%--- General math ------------------------------------------------------------

% Vector style
\newcommand{\VctStl}[1]     {\ensuremath{\mathbf{#1}}}

% Matrix style
\newcommand{\MtrStl}[1]     {\ensuremath{\mathbf{#1}}}

% The identity matrix
\newcommand{\IdnMtr}        {\MtrStl{1}}  

% Scalar or matrix inverse
\newcommand{\Inv}           {^{-1}}  

% Vector or matrix transpose
\newcommand{\Trp}           {^T}  

% Size symbol
\newcommand{\SzeSmb}        {\ensuremath{\in}}  

% Vector space
\newcommand{\VctSpc}[1]     {\ensuremath{\mathbf{R}^#1}}

% Matrix space
\newcommand{\MtrSpc}[2]     {\ensuremath{\mathbf{R}^{#1 \times #2}}}

% Vector length (simple)
\newcommand{\VctLng}        {\ensuremath{n}}  
\newcommand{\aVctLng}[1]    {\ensuremath{n_{#1}}}  

% Index (to vector, matrix ...)
\newcommand{\Ind}           {\ensuremath{i}}  
\newcommand{\aInd}[1]       {\ensuremath{i_{#1}}}  

% Differential d
\newcommand{\DiffD}         {\ensuremath{\mathrm{d}}}  

% Partial d
\newcommand{\PartD}         {\ensuremath{\partial}}  

% Real and Imaginary part
\renewcommand{\Re}            {\ensuremath{\mathrm{Re}}}
\renewcommand{\Im}            {\ensuremath{\mathrm{Im}}}

% Ensemble average
\newcommand{\EnsAvr}[1]        {\ensuremath{\left\langle #1 \right\rangle}}

% Absolute Value
\newcommand{\Abs}[1]          {\ensuremath{\left| #1 \right| }}   

% *10^#1
\newcommand{\topowerten}[1]   {\ensuremath{\cdot10^{#1}}}

% degrees
\newcommand{\degree}          {\ensuremath{^\circ}}

% 1/2
\newcommand{\half} {\ensuremath{\textstyle\frac{1}{2}}}


%--- Physical constants ------------------------------------------------------

% Speed of light in vaccum [m/s]
\newcommand{\speedoflight}   {\ensuremath{c}}  

% Planck constant [Js]
\newcommand{\planckCns}      {\ensuremath{h}}  

% Boltzmann constant [J/K]
\newcommand{\boltzmannCns}   {\ensuremath{k_b}}  

% Avogadro's number [molec/kg]
\newcommand{\avogadrosCns}   {\ensuremath{N_a}}  


%--- The Rodgers formalism ---------------------------------------------------

% True (natural) forward model
\newcommand{\trueFrwMdl}       {\ensuremath{F}}  

% Discrete forward model
\newcommand{\FrwMdl}           {\ensuremath{\mathcal{F}}}  

% Inverse model  
\newcommand{\InvMdl}           {\ensuremath{\mathcal{I}}}  

% Transfer model  
\newcommand{\TrfMdl}           {\ensuremath{\mathcal{T}}}  

% A priori symbol
\newcommand{\AprSmb}           {\ensuremath{_a}}  

% Measurement vector
\newcommand{\MsrVct}           {\VctStl{y}}  

% Vector of monochromatic pencil beam intensities
\newcommand{\MpiVct}           {\VctStl{i}}  

% Vector of monochromatic pencil beam intensities with subscript
\newcommand{\aMpiVct}[1]       {\MpiVct\ensuremath{_{#1}}}

% Measurement error vector
\newcommand{\MsrErrVct}        {\ensuremath{\varepsilon}}  

% State vector 
\newcommand{\SttVct}           {\VctStl{x}}  

% A priori state vector 
\newcommand{\AprSttVct}        {\SttVct\AprSmb}

% Retrieved state vector
\newcommand{\RtrVct}           {\ensuremath{\hat{\SttVct}}}

% A state vector with subscript
\newcommand{\aSttVct}[1]       {\SttVct\ensuremath{_{#1}}}

% A state vector with subscript and transpose
\newcommand{\aSttVctTrp}[1]    {\SttVct\ensuremath{_{#1}\Trp}}

% Forward model parameter vector
\newcommand{\FrwMdlVct}        {\VctStl{b}}  

% A priori forward model parameter vector 
\newcommand{\AprFrwMdlVct}     {\FrwMdlVct\AprSmb}

% A forward model parameters vector with subscript
\newcommand{\aFrwMdlVct}[1]    {\FrwMdlVct\ensuremath{_{#1}}}

% A forward model parameters vector with subscript and transpose
\newcommand{\aFrwMdlVctTrp}[1] {\FrwMdlVct\ensuremath{_{#1}\Trp}}

% Inverse model parameters
\newcommand{\InvMdlVct}        {\VctStl{c}}  

% Weighting function matrix
\newcommand{\WfnMtr}           {\MtrStl{K}}

% A weighting function matrix with a subscript
\newcommand{\aWfnMtr}[1]       {\WfnMtr\ensuremath{_{#1}}}

% A weighting function matrix with a subscript and transpose
\newcommand{\aWfnMtrTrp}[1]    {\WfnMtr\ensuremath{_{#1}\Trp}}

% Contribution function matrix
\newcommand{\CtrFncMtr}        {\MtrStl{D_y}}  

% Averaging kernel matrix
\newcommand{\AvrKrnMtr}        {\MtrStl{A}}  

% A averaging kernel matrix with subscript
\newcommand{\aAvrKrnMtr}[1]    {\AvrKrnMtr\ensuremath{_{#1}}}

% Sensor (and data reduction) matrix
\newcommand{\SnsMtr}           {\MtrStl{H}} 

% Sensor (and data reduction) matrix with subscript.
\newcommand{\aSnsMtr}[1]       {\SnsMtr\ensuremath{_{#1}}} 

% Transformation between vector spaces
\newcommand{\VctTrfMtr}        {\MtrStl{B}}  

% Population matrix
\newcommand{\PplMtr}           {\MtrStl{\Sigma}}

% A population matrix with a subscript
\newcommand{\aPplMtr}[1]       {\PplMtr\ensuremath{_{#1}}}

% A population matrix with a subscript and invererse
\newcommand{\aPplMtrTrp}[1]    {\PplMtr\ensuremath{_{#1}\Inv}}

% Covariance matrix
\newcommand{\CvrMtr}           {\MtrStl{S}}

% A covariance matrix with a subscript
\newcommand{\aCvrMtr}[1]       {\CvrMtr\ensuremath{_{#1}}}

% A covariance matrix with a subscript and invererse
\newcommand{\aCvrMtrTrp}[1]    {\CvrMtr\ensuremath{_{#1}\Inv}}


% --- Special functions ------------------------------------------------------

% The Planck function
\newcommand{\Planck}     {\ensuremath{B}}  


% --- General scalar quantities ----------------------------------------------
%
% All quantities shall have a subscript version named as aXxx.

% Altitude (above geoid)
\newcommand{\Alt}        {\ensuremath{z}}  
\newcommand{\aAlt}[1]    {\ensuremath{z_{#1}}}

% Azimuthal angle
\newcommand{\AzmAng}     {\ensuremath{\omega}}  
\newcommand{\aAzmAng}[1] {\ensuremath{\omega_{#1}}}  

% Frequency
\newcommand{\Frq}        {\ensuremath{\nu}}  
\newcommand{\aFrq}[1]    {\ensuremath{\nu_{#1}}}  

% Wavelength
\newcommand{\Wvl}        {\ensuremath{\lambda}}  
\newcommand{\aWvl}[1]    {\ensuremath{\lambda_{#1}}}  

% Latitude
\newcommand{\Lat}        {\ensuremath{\alpha}}  
\newcommand{\aLat}[1]    {\ensuremath{\alpha_{#1}}}  

% Length along the propagation path
\newcommand{\PpathLng}        {\ensuremath{l}}  
\newcommand{\aPpathLng}[1]    {\ensuremath{l_{#1}}}  

% Longitude
\newcommand{\Lon}        {\ensuremath{\beta}}  
\newcommand{\aLon}[1]    {\ensuremath{\beta_{#1}}}  

% Monochromatic pencil beam intensity
\newcommand{\Mpi}        {\ensuremath{I}}  
\newcommand{\aMpi}[1]    {\ensuremath{I_{#1}}}  

% Pressure altitude
\newcommand{\Oth}        {\ensuremath{\tau}}  
\newcommand{\aOth}[1]    {\ensuremath{\tau_{#1}}}  

% Pressure
\newcommand{\Prs}        {\ensuremath{P}}  
\newcommand{\aPrs}[1]    {\ensuremath{P_{#1}}}  

% Pressure altitude
\newcommand{\PrsAlt}     {\ensuremath{\zeta}}  
\newcommand{\aPrsAlt}[1] {\ensuremath{\zeta_{#1}}}  

% Radius
\newcommand{\Rds}        {\ensuremath{r}}  
\newcommand{\aRds}[1]    {\ensuremath{r_{#1}}}  

% Refractive index
\newcommand{\Rfr}        {\ensuremath{n}}  
\newcommand{\aRfr}[1]    {\ensuremath{n_{#1}}}  
\newcommand{\RealRfr}    {\ensuremath{n'}}  
\newcommand{\ImagRfr}    {\ensuremath{n''}}  

% Speed
\newcommand{\Spd}        {\ensuremath{v}}  
\newcommand{\aSpd}[1]    {\ensuremath{v_{#1}}}

% Temperature
\newcommand{\Tmp}        {\ensuremath{T}}  
\newcommand{\aTmp}[1]    {\ensuremath{T_{#1}}}  

% Zenith angle
\newcommand{\ZntAng}     {\ensuremath{\psi}}  
\newcommand{\aZntAng}[1] {\ensuremath{\psi_{#1}}}  

% Winds
\newcommand{\Wind}        {\ensuremath{v}}  
\newcommand{\WindWE}      {\ensuremath{v_u}}  
\newcommand{\WindSN}      {\ensuremath{v_v}}  
\newcommand{\WindVe}      {\ensuremath{v_w}}  


% --- Quantities concerning scattering  -------------------

% Total extinction matrix
\newcommand{\ExtMat}    {\ensuremath{{\bf K}}}
\newcommand{\aExtMat}[1]{\ensuremath{{\bf K}_{#1}}}

% Absorption matrix
\newcommand{\AbsMat}    {\ensuremath{{\bf A}}}
\newcommand{\aAbsMat}[1]{\ensuremath{{\bf A}_{#1}}}

% Total absorption vector
\newcommand{\AbsVec}    {\ensuremath{{\bf a}}}
\newcommand{\aAbsVec}[1]{\ensuremath{{\bf a}_{#1}}}     

% Phase matrix
\newcommand{\PhaMat}    {\ensuremath{{\bf Z}}}
\newcommand{\aPhaMat}[1]{\ensuremath{{\bf Z}_{#1}}}

% Scattering matrix
\newcommand{\ScaMat}    {\ensuremath{{\bf F}}}
\newcommand{\aScaMat}[1]{\ensuremath{{\bf F}_{#1}}}

% Stokes vector
\newcommand{\StoVec}    {\ensuremath{{\bf I}}}
\newcommand{\aStoVec}[1]{\ensuremath{{\bf I}_{#1}}}
\newcommand{\StoI}      {\ensuremath{I}}
\newcommand{\aStoI}[1]  {\ensuremath{I_{#1}}}
\newcommand{\StoQ}      {\ensuremath{Q}}
\newcommand{\StoU}      {\ensuremath{U}}
\newcommand{\StoV}      {\ensuremath{V}}


% Propagation direction and position
\newcommand{\PDir}      {\ensuremath{{\bf \hat{n}}}}
\newcommand{\PPos}      {\ensuremath{{\bf r}}}

% Particle density
\newcommand{\PDen}      {\ensuremath{n^p}}

% Radiation field
\newcommand{\IFld}      {\ensuremath{{\mathcal I}}}
% Radiation field index
\newcommand{\aIFld}[1]  {\ensuremath{{\mathcal I}^{(#1)}}}

% Scattered field
\newcommand{\SFld}      {\ensuremath{{\mathcal S}}}

% Scattered field index
\newcommand{\aSFld}[1]  {\ensuremath{{\mathcal S}^{(#1)}}}

% Scattering Integral vector
\newcommand{\SVec}      {\ensuremath{{\bf S}}}

% Amplitude matrix
\newcommand{\AmpMat}    {\ensuremath{{\bf S}}}

% Phase matrix
\newcommand{\TraMat}    {\ensuremath{{\bf T}}}
\newcommand{\aTraMat}[1]{\ensuremath{{\bf T}_{#1}}}

% Amplitude matrix index
\newcommand{\IAmp}      {\ensuremath{i_{amp}}}

% Single extinction matrix
\newcommand{\SExMat}    {\ensuremath{{\bf L}}}
\newcommand{\aSExMat}[1]{\ensuremath{{\bf L}^{#1}}}     

% Scattered field
\newcommand{\ScaInt}    {\ensuremath{{\bf S}}}

% Identity matrix
\newcommand{\IdnMat}    {\ensuremath{{\bf E}}}

% Scattering zenith angle
\newcommand{\ScaZa}     {\ensuremath{{\psi_s}}}

% Scattering azimuth angle
\newcommand{\ScaAa}     {\ensuremath{{\omega_s}}}

% Particle type index
\newcommand{\IPart}     {\ensuremath{{i_{part}}}}

%Inverse Wave Impendance
\newcommand{\InvImp} %
      {\ensuremath{\sqrt{\textstyle{\frac{\epsilon}{\mu}}}}}    

% Micrometer
\newcommand{\mum}          {\ensuremath{\mu m}}

% Incoming direction
\newcommand{\inc}       {\mathrm{inc}}

% Scattered direction
\newcommand{\sca}       {\mathrm{sca}}

% Ice mass content
\newcommand{\imc}       {\ensuremath{IMC}}

% effective radius
\newcommand{\Reff}       {\ensuremath{R_{eff}}}


% --- Quantities concerning scalar gas absorption  -------------------

% Absorption coefficient
\newcommand{\AbsCoef}    {\ensuremath{{\alpha}}}
\newcommand{\aAbsCoef}[1]{\ensuremath{{\alpha}_{#1}}}

% Total absorption coefficient
\newcommand{\AbsCoefTot} {\aAbsCoef{\mbox{\footnotesize total}}}

% Absorption cross section
\newcommand{\AbsXsec}    {\ensuremath{{\kappa}}}
\newcommand{\aAbsXsec}[1]{\ensuremath{{\kappa}_{#1}}}

% Number density:
\newcommand{\Den}    {\ensuremath{{n}}}
\newcommand{\aDen}[1]{\ensuremath{{n}_{#1}}}


% --- Intensity for different polarisation components  -------------------

\newcommand{\Iv}      {\ensuremath{I_v}}
\newcommand{\Ih}      {\ensuremath{I_h}}
\newcommand{\Ipff}    {\ensuremath{I_{+45^\circ}}}
\newcommand{\Imff}    {\ensuremath{I_{-45^\circ}}}
\newcommand{\Irhc}    {\ensuremath{I_{rhc}}}
\newcommand{\Ilhc}    {\ensuremath{I_{lhc}}}


% --- Brightness temperature  -------------------

\newcommand{\BT}      {\aTmp{B}}


%--- Plotting line styles ----------------------------------------------------

\def \lsolid     {\mbox{------}}
\def \ldashed    {\mbox{--~--~--}}
\def \ldashdot   {\mbox{--~$\cdot$~--}}
\def \ldotted    {\mbox{$\cdot~\cdot~\cdot$}}


%%% Local Variables: 
%%% mode: plain-tex
%%% TeX-master: "uguide"
%%% End: 




%--- PDF/LaTeX specific options
\ifpdf
  \usepackage{graphicx}    % includegraphics
  \DeclareGraphicsExtensions{.pdf}
  \usepackage{color}
  \definecolor{DarkRed}{rgb}{0.5,0,0}
  \usepackage
    [pdftex,                         % or dvips
     colorlinks=true,
     linkcolor=DarkRed,
     citecolor=DarkRed,
     urlcolor=DarkRed,
%     pdftitle={ARTS User Guide},
%     pdfauthor={The ARTS development team},
%     pdfsubject={},
%     pdfkeywords={},
%     bookmarks=true,
%     bookmarksopen=false,
%     pdfpagemode=None,
%     plainpages=false,
%     pdfpagelabels
      ]
  {hyperref}
  \setcounter{tocdepth}{3}
\else
  \usepackage{graphicx}    % includegraphics
  \DeclareGraphicsExtensions{.eps}
  \setcounter{tocdepth}{1}
\fi


%--- Command definitions -----------------------------------------------------

%- Document history
\newcommand{\starthistory} {\begin{table}[b]  \begin{tabular}{l p{11cm}} 
                             \hline {\bf History} & \\ }
\newcommand{\stophistory}  {\end{tabular} \end{table} }


%- Symbol table
\newcommand{\startsymbols} {\begin{table} \begin{center} 
                            \caption{Examples of symbols used in this chapter,
                            the corresponding notation in the ARTS source code
                            and a short description of the quantity. }
                            \begin{tabular}{l l l}
                            {\bf Here} & {\bf In ARTS} & {\bf Description} 
                            \\ \hline \\ } 
\newcommand{\stopsymbols}  {\\ \hline \end{tabular} 
                           \end{center} \end{table}}      
\newcommand{\startsymbolswithunits} 
                   {\begin{table} \begin{center} 
                            \caption{Examples of symbols used in this chapter,
                            the corresponding notation in the ARTS source code
                            and a short description of the quantity. }
                   \begin{tabular}{l l l l}
                   {\bf Here} & {\bf Unit} & {\bf In ARTS} & {\bf Description} 
                   \\ \hline \\ } 
\newcommand{\stopsymbolswithunits} {\stopsymbols}

% ARTS Web address
\newcommand{\artsserver}{https://www.radiativetransfer.org/}
\newcommand{\artsdocsserver}{https://atmtools.github.io/arts-docs-master/}

% Docserver for this ARTS version
\newcommand{\docserver}{docserver/}

%- Command to link to a URL on the ARTS web server
% takes the subpath (without leading / !!!) as input
\newcommand{\artsurl}[1]{\href{\artsserver #1}{\artsserver #1}}
\newcommand{\artsdocsurl}[1]{\href{\artsdocsserver #1}{\artsdocsserver #1}}

%- Command to create link to ARTS built-in documentation. (Consider
% using \wsmindex, \wsvindex, etc., instead. They use this command
% implicitly.  But direct use may be useful if you use the same term
% several times in a short section, and don't want all of these
% occurrences to be in the index.)
% Underscores must be escaped by leading backslash!
\newcommand{\builtindoc}[1]{\href{\artsdocsserver\docserver all/#1}{#1}}

%- Command to write an internal ARTS variable, internal function, or
% file name with special style. Anything that does not have built-in
% documentation. Also for other things that are code,
% but inside the text. Use the "code" environment for longer pieces of
% code.
% Underscores must be escaped by leading backslash!
\newcommand{\shortcode}[1]{\texttt{#1}}

%- Define verbatim environment for arts code examples.
% (For longer pieces of code, for in-text use "\shortcode".)
% This is the only code command where you do not have to escape
% underscores. 
\DefineVerbatimEnvironment{code}{Verbatim}{fontsize=\small}


%- Commands for easy indexing of terms
%
% Underscores must be escaped by leading backslash!
%
% Index command to use when text and index reference are equal. Otherwise
% the normal \index command must be used.
\newcommand{\textindex}[1]{#1\index{#1}} 
%
% Index command to make index for a workspace method. It writes out the
% function name in verbatim style and makes an index reference.
\newcommand{\wsmindex}[1]{\builtindoc{#1}\index{workspace methods!#1}} 
%
% Index command to make index for workspace variable. Works as \wsmindex.
\newcommand{\wsvindex}[1]{\builtindoc{#1}\index{workspace variables!#1}}
%
% Index command to make index for workspace agenda. Works as \wsmindex.
\newcommand{\wsaindex}[1]{\builtindoc{#1}\index{workspace agendas!#1}} 
%
% Index command to make index for a ARTS file. Works as \wsmindex.
\newcommand{\fileindex}[1]{\shortcode{#1}\index{ARTS files!#1}}
%
% Index command to make index for an internal function. Works as \wsmindex.
\newcommand{\funcindex}[1]{\shortcode{#1}\index{internal ARTS functions!#1}}
%
% Index command to make index for a ARTS data structure. Works as \wsmindex.
\newcommand{\typeindex}[1]{\shortcode{#1}\index{data types!#1}}


%- For FIXMEs:
\newcommand{\FIXME}[1]{\textcolor{gray}{\bfseries FIXME: #1}}


%- Names of the different documentation documents:
\newcommand{\user}{\emph{ARTS User Guide}}
\newcommand{\developer}{\emph{ARTS Developer Guide}}
\newcommand{\theory}{\emph{ARTS Theory}}

%- Cope comments
\usemintedstyle{sas}
\definecolor{LightGray}{gray}{0.9}
\setminted[cpp]{
frame=lines,
framesep=2mm,
baselinestretch=1.2,
fontsize=\footnotesize,
autogobble,
label=C++
}
\setminted[python]{
frame=lines,
framesep=2mm,
baselinestretch=1.2,
fontsize=\footnotesize,
autogobble,
label=Python,
python3=true
}
\setmintedinline{bgcolor={}}

%------------------------------------------------------------------------------



%%% Local Variables: 
%%% mode: latex
%%% TeX-master: t
%%% End: 
