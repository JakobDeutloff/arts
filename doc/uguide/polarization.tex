%
% To start the document, use
%  \levela{...}
% For lover level, sections use
%  \levelb{...}
%  \levelc{...}
%
\levela{Polarization and Stokes Parameters}
 \label{sec:polarization}

%
% Document history, format:
%  \starthistory
%    date1 & text .... \\
%    date2 & text .... \\
%    ....
%  \stophistory
%
\starthistory
   260602 & Created and written by Christian Melsheimer.\\
\stophistory


%
% Symbol table, format:
%  \startsymbols
%    ... & \artsstyle{...} & text ... \\
%    ... & \artsstyle{...} & text ... \\
%    ....
%  \stopsymbols
%
%=====================================================================


%=====================================================================
% Definition of new commands:
% ====================================================================
\newcommand{\half} {\ensuremath{\textstyle\frac{1}{2}}}
\newcommand{\eVrt} {\ensuremath{\VctStl{e}_v}}
\newcommand{\eHor} {\ensuremath{\VctStl{e}_h}}
\newcommand{\eLh} {\ensuremath{\VctStl{e}_{LH}}}
\newcommand{\eRh} {\ensuremath{\VctStl{e}_{RH}}}
\newcommand{\ePls} {\ensuremath{\VctStl{e}_{+45\degree}}}
\newcommand{\eMin} {\ensuremath{\VctStl{e}_{-45\degree}}}
%\newcommand{\DirFrePr} {\ensuremath{(\PDir^\prime, \Frq)}}


The present version of ARTS implements the radiative transfer equation
in tensor form, i.e. for the 4 components of the Stokes vector, not
just for its first component, the intensity or radiance.
This means that the model can include polarization dependence in
absorption or scattering processes.
It is therefore necessary to give some details on the polarization of
radiation and the definition of
the Stokes parameters. (FIXME: more...)

\levelb{Plane Monochromatic Waves}
%=====================================================================
\label{sec:polarization:monochrom}

Plane monochromatic electromagnetic waves are commonly written in the form
\begin{equation} 
  \label{eq:polarization:e_field}
  \VctStl{E}(\VctStl{x},t) 
   = \left[E_v \atop E_h\right] e^{i(\VctStl{k}\VctStl{x} - \omega t)}
   = \left(E_v \eVrt +  E_h \eHor \right) 
      e^{i(\VctStl{k}\VctStl{x} - \omega t)}
\end{equation}
where $\VctStl{E}$ is the electric field vector, the subscripts $v$
and $h$ denote the components with vertical and horizontal
polarization, respectively. $E_v$ and $E_h$, the amplitudes, are
complex numbers, $\VctStl{k}$ and $\omega$ are the wavenumber vector
and the angular frequency, respectively, of the plane wave, and the
unit vectors $\eVrt = (1,0)\Trp$, $\eHor = (0,1)\Trp$.  It is always
implicitly understood that the actual, physical, electric field is the
real part of the above expression. Rewriting the complex amplitudes
$E_v$ and $E_h$ using real, non-negative amplitudes $a_v$ and $a_h$, and
phases $\delta_v$ and $\delta_h$,
\begin{equation}
  \label{eq:polarization:compl_ampl}
  E_v=a_v e^{i\delta_v}\mbox{ , }
  E_h=a_h e^{i\delta_h}
\end{equation}
the actual electric field vector $\VctStl{\tilde{E}}$ is
\begin{equation}
  \label{eq:polarization:actual_efield}
  \VctStl{\tilde{E}}(\VctStl{x},t) = \Re[\VctStl{E}(\VctStl{x},t)] 
    = \left[a_v\cdot \cos(\VctStl{k}\VctStl{x} - \omega t + \delta_v) 
                       \atop 
            a_h\cdot \cos(\VctStl{k}\VctStl{x} - \omega t + \delta_h) 
       \right] 
\end{equation}

In general, instruments do not measure the electric or magnetic field
vectors of an electromagnetic wave, but rather the time-averaged
intensity, i.e., the energy flux, $F$. This is the time-averaged Poynting
vector (which, in turn, is proportional to the square of the electric
field), thus:
\begin{eqnarray}
  F 
  &=& 
  \sqrt{\frac{\epsilon}{\mu}}\overline{(\VctStl{E}(\VctStl{x},t))^2}\\
   \nonumber
  &=&
  \sqrt{\frac{\epsilon}{\mu}}\left(
    a_v^2\overline{\cos^2(\VctStl{k}\VctStl{x} - \omega t + \delta_v)}
    + a_h^2\overline{\cos^2(\VctStl{k}\VctStl{x} - \omega t + \delta_h)}
   \right)
  \label{eq:polarization:intensity}
\end{eqnarray}
The overline denotes the time average
%, which is taken over one oscillation period $T$ ($T=2\pi\omega$). 
%Since the time average of
which for cosine squares is 1/2, thus:
\begin{equation}
  \label{eq:polarization:intensity2}
 F = 
  \half\sqrt{\frac{\epsilon}{\mu}}(
    a_v^2 + a_h^2)  
\end{equation}
Taking into account that for plane, monochromatic waves 
the time average always results in a factor
$\frac{1}{2}$, we can also directly write the intensity using the
electric field vector in complex notation
(Eq.~\ref{eq:polarization:e_field}).
\begin{eqnarray}
  \label{eq:polarization:intensity_from_complex}
  F &=&  \half\sqrt{\frac{\epsilon}{\mu}} \VctStl{E} (\VctStl{x},t) 
          \cdot \VctStl{E}^\ast (\VctStl{x},t)\\ \nonumber
    &=&   \half\sqrt{\frac{\epsilon}{\mu}}
          (E_v E_v^\ast + E_h E_h^\ast)
\end{eqnarray}
where the asterisk denotes complex conjugation.%
%\footnote{Note that the same argumentation works for ???}

Commonly the factor  $\frac{1}{2}\sqrt{\frac{\epsilon}{\mu}}$ is
omitted, and in addition to the flux, three more intensity quantities
are defined as in the following equations. They are called 
\emph{Stokes parameters}:
\begin{eqnarray}
  \label{eq:polarization:stokesparam_I}
  I &=&   E_v E_v^\ast + E_h E_h^\ast \\
  \label{eq:polarization:stokesparam_Q}
  Q &=&   E_v E_v^\ast - E_h E_h^\ast \\
  \label{eq:polarization:stokesparam_U}
  U &=& - E_v E_h^\ast - E_h E_v^\ast \\
  \label{eq:polarization:stokesparam_V}
  V &=& i(E_h E_v^\ast - E_v E_h^\ast)
\end{eqnarray}
Using the amplitude/phase notation from
Eq.~(\ref{eq:polarization:compl_ampl}),
we can rewrite the Stokes parameters as 
\begin{eqnarray}
  \label{eq:polarization:stokesparam_alt_I}
 I &=& a_v^2 + a_h^2\\
  \label{eq:polarization:stokesparam_alt_Q}
 Q &=& a_v^2 - a_h^2\\
  \label{eq:polarization:stokesparam_alt_U}
 U &=&  - a_v a_h \cos(\delta_v-\delta_h)\\
  \label{eq:polarization:stokesparam_alt_V}
 V &=&   a_v a_h \sin(\delta_v-\delta_h)
\end{eqnarray}
The Stokes parameters fully characterize the electromagnetic wave and
therefore contain the same information as the electric field vector
(except for one absolute phase).  Since instruments generally measure
intensities (fluxes), describing electromagnetic radiation by the
Stokes parameters is more practical than describing it by the electric
(or magnetic) field vector. Furthermore, the Stokes parameters are
always real numbers.

 
In order understand what the Stokes parameters mean, we have to go
back to the electric field vector and see what polarization state it
describes.  To do so, we look at the curve that the tip of the
physical electric field vector $\VctStl{\tilde{E}}$ describes with
time at a fixed position $\VctStl{x_0}$:
\begin{eqnarray}
  \tilde{E}_v (t) &=& a_v \cos(\Delta_v - \omega t)\\
  \tilde{E}_h (t) &=& a_h \cos(\Delta_h - \omega t)
\end{eqnarray}
where $\Delta_{v,h} = \VctStl{k}\VctStl{x_0} + \delta_{v,h}$. 
To see that this is an ellipse, we first split the cosines using
the addition theorem:
\begin{eqnarray}
  \label{eq:polarization:tip_of_fieldvec1}
  \tilde{E}_v (t) &=&   a_v \cos\Delta_v \cos(\omega t)
                      + a_v \sin\Delta_v \sin(\omega t)\\
  \label{eq:polarization:tip_of_fieldvec2}
  \tilde{E}_h (t) &=&   a_h \cos\Delta_h \cos(\omega t)
                      + a_h \sin\Delta_h \sin(\omega t)
\end{eqnarray}

In order to have the tip of $\VctStl{\tilde{E}}$ describe an ellipse with semi-major axis $a_0 \cos\beta$
and
semi-minor axis $a_0 \sin\beta$, where $a_0^2 = a_v^2 + a_h^2$, 
it should have the following form
\begin{eqnarray}
  \label{eq:polarization:ellipse_parallel}
 \tilde{E}_v (t) &=&   a_0 \sin\beta \cos(\omega t)\\
 \tilde{E}_h (t) &=&   a_0 \cos\beta \sin(\omega t)
\end{eqnarray}
Here $\beta$ must be between $-45\degree$ and $45\degree$: the tip of
the vector $\VctStl{\tilde{E}}$ describes a circle for $\beta = \pm
45\degree$ (circular polarization), oscillates along the $h$-axis for
$\beta = 0$ (linear polarization) and else describes an ellipse (cf.
Figure~\ref{fig:polarization:ellipse_aligned}). The sense of rotation
is counterclockwise for positive $\beta$ and clockwise for negative
$\beta$.
%which corresponds to left-handed
Since $|\tan\beta|$ is the ratio of the semi-minor and semi-major axes
of the ellipse (the ellipticity), $\beta$ is called the ellipticity
angle.
\begin{figure}[!h]
 \begin{center}
  \begin{minipage}[c]{0.9\textwidth}
   \begin{center}
    \includegraphics*[width=0.9\hsize]{Figs/pol_ellipse_aligned}
   \end{center}
  \end{minipage}
  \begin{minipage}[c]{0.9\textwidth}
   \caption{The ellipse that the electric field vector describes with
     time, with the major axis oriented along the $h$-axis.}
   \label{fig:polarization:ellipse_aligned}
  \end{minipage}
 \end{center}
\end{figure}   
Note that the semi-major axis is oriented along the positive $h$-axis.
To have the major axis of the ellipse enclose an arbitrary angle
$\zeta$ ($0 \leq \zeta < 180\degree$) with the $h$-axis, we apply a
rotation matrix and get the equation for an ellipse with arbitrary
shape (ellipticity) and orientation (cf.\
Figure~\ref{fig:polarization:ellipse_arbitrary}):
\begin{eqnarray}
  \label{eq:polarization:ellipse_rotated1}
 \tilde{E}_v (t) &=&  a_0(  \sin\beta \cos(\omega t) \cos\zeta
                           +\cos\beta \sin(\omega t) \sin\zeta )\\
  \label{eq:polarization:ellipse_rotated2}
 \tilde{E}_h (t) &=&  a_0( -\sin\beta \cos(\omega t) \sin\zeta
                           +\cos\beta \sin(\omega t) \cos\zeta )
\end{eqnarray}
%
\begin{figure}[!h]
 \begin{center}
  \begin{minipage}[c]{0.9\textwidth}
   \begin{center}
    \includegraphics*[width=0.9\hsize]{Figs/pol_ellipse_arbitrary}
   \end{center}
  \end{minipage}
  \begin{minipage}[c]{0.9\textwidth}
   \caption{The ellipse that the electric field vector describes with
     time, with the major axis oriented arbitrarily.}
   \label{fig:polarization:ellipse_arbitrary}
  \end{minipage}
 \end{center}
\end{figure}   

%FIXME: include image pol\_ellipse.fig here (similar to Fig.4, p.25 in
%\textit{Mishchenko} [2000])!

Now we want to establish a direct connection between the parameters
$\beta$ and $\zeta$ describing the shape (ellipticity) and orientation
of the polarization ellipse on the one hand, and the amplitudes $a_v$
and $a_h$ and phases $\delta_v$ and $\delta_h$ of the components of
the electric field vector on the other hand.  Comparing the
$\sin(\omega t)$ and $\cos(\omega t)$ terms in
Eqs.~(\ref{eq:polarization:ellipse_rotated1})%
-(\ref{eq:polarization:ellipse_rotated2}) with the corresponding terms
in Eqs.~(\ref{eq:polarization:tip_of_fieldvec1})%
-(\ref{eq:polarization:tip_of_fieldvec2}), we get:
\begin{eqnarray}
  \label{eq:polarization:corresp1a}
 a_v \cos\Delta_v &=& a_0 \sin\beta \cos\zeta\\
  \label{eq:polarization:corresp1b}
 a_v \sin\Delta_v &=& a_0 \cos\beta \sin\zeta
\end{eqnarray}
and 
\begin{eqnarray}
  \label{eq:polarization:corresp2a}
 a_h \cos\Delta_h &=& -a_0 \sin\beta \sin\zeta\\
  \label{eq:polarization:corresp2b}
 a_h \sin\Delta_h &=&  a_0 \cos\beta \cos\zeta
\end{eqnarray}
Multiplying Eq.~(\ref{eq:polarization:corresp1a}) with
Eq.~(\ref{eq:polarization:corresp2a}), and
Eq.~(\ref{eq:polarization:corresp1b}) with
Eq.~(\ref{eq:polarization:corresp2b}) and adding up the results, we get
\begin{equation}
  a_v a_h (\cos\Delta_v\cos\Delta_h + \sin\Delta_v\sin\Delta_h)
  = a_0^2 \sin\zeta\cos\zeta (\cos^2\beta - \sin^2\beta) 
\end{equation}
Using the addition theorems for sinusoidals and taking into account
that
$\Delta_v-\Delta_h = \delta_v-\delta_h$:
\begin{equation}
  \label{eq:polarization:sinzetacosbeta}
  \frac{a_v a_h}{a_0^2} \cos(\delta_v-\delta_h)
  = \half\sin(2\zeta)\cos(2\beta)
\end{equation}
In a similar way, subtracting the product of
Eq.~(\ref{eq:polarization:corresp1b}) with
Eq.~(\ref{eq:polarization:corresp2a}) from the product of
Eq.~(\ref{eq:polarization:corresp1a}) with
Eq.~(\ref{eq:polarization:corresp2b}) and adding up the results, we get
\begin{equation}
  \label{eq:polarization:sinbeta}
  -\frac{a_v a_h}{a_0^2} \sin(\delta_v-\delta_h)
  = \half\sin(2\beta)
\end{equation}
The above two equations tell us how to translate the amplitudes
($a_v$, $a_h$) and phases ($\delta_v$, $\delta_h$) of the vertical and
horizontal component of the electric field into the orientation and
shape of the ellipse that the tip of the electric field vector describes with
time.  We can obtain one further relation by subtracting the sum of
the squares of Eq.~(\ref{eq:polarization:corresp2a}) and
Eq.~(\ref{eq:polarization:corresp2b}) from the sum of the squares of
Eq.~(\ref{eq:polarization:corresp1a}) and
Eq.~(\ref{eq:polarization:corresp1b}):
\begin{equation}
  \label{eq:polarization:cosbetazeta}
 a_v^2 - a_h^2 =  -a_0^2 \cos(2\zeta)\cos(2\beta)
\end{equation}
Finally, we use the above 3 equations
(\ref{eq:polarization:sinzetacosbeta}), 
(\ref{eq:polarization:sinbeta}) and 
(\ref{eq:polarization:cosbetazeta}) to rewrite the Stokes parameters
(Eqs.~\ref{eq:polarization:stokesparam_alt_I}%
-\ref{eq:polarization:stokesparam_alt_V}) 
as
\begin{eqnarray}
  \label{eq:polarization:stokesparam_alt2_I}
 I &=& a_0^2\\
  \label{eq:polarization:stokesparam_alt2_Q}
 Q &=&  -a_0^2 \cos(2\zeta)\cos(2\beta)\\ 
  \label{eq:polarization:stokesparam_alt2_U}
 U &=& -a_0^2 \sin(2\zeta)\cos(2\beta)\\
  \label{eq:polarization:stokesparam_alt2_V}
 V &=& -a_0^2 \sin(2\beta)
\end{eqnarray}

Thus, we can get the orientation angle $\zeta$ of the ellipse from
\begin{equation}
  \label{eq:polarization:tan2zeta}
 \tan(2\zeta) = \frac{U}{V}
\end{equation}
Since $0 \leq 2\zeta < 360\degree$, there are 2 solutions for $\zeta$ for a
given pair $U,V$. This ambiguity is resolved by looking at
Eq.~\ref{eq:polarization:stokesparam_alt2_Q}, taking into account that
$|\beta| \leq 45\degree$ and thus $\cos(2\beta) \geq 0$:
The sign of $\cos(2\zeta)$ must be the same as the sign of $-Q$.

We get the ellipticity angle $\beta$  from
\begin{equation}
  \label{eq:polarization:tan2beta}
 \tan(2\beta) = - \frac{V}{(Q^2 + U^2)^{1/2}}  
\end{equation}

$I$ is the total intensity of the radiation, 
$Q$ is the difference in the intensity of the vertically and
horizontally polarized components
(cf. section~\ref{sec:polarization:measuring}, i.e. next section).
$I$ is always non-negative, and $Q$, $U$, and $V$ are between $+I$ and $-I$,
since they can be expressed as a product of $I$ with sines and/or
cosines
(Eqs. \ref{eq:polarization:stokesparam_alt2_Q}-%
\ref{eq:polarization:stokesparam_alt2_V}). 
Note also that the 4 Stokes parameters are not independent, since the
following equality applies:
\begin{equation}
  \label{eq:polarization:Isquare}
  I^2 = Q^2 + U^2 + V^2
\end{equation}

\levelb{Measuring Stokes Parameters}
%=====================================================================
\label{sec:polarization:measuring}
The three different ways given so far to write the Stokes parameters
(Eqs.~\ref{eq:polarization:stokesparam_I}ff.,
Eqs.~\ref{eq:polarization:stokesparam_alt_I}ff.,
Eqs.~\ref{eq:polarization:stokesparam_alt2_I}ff.)  are not very
helpful if we actually want to measure the Stokes parameters. So here
we are going to rewrite them while keeping in mind that most
instruments can just measure intensities of radiation.

We have seen above that the Stokes parameter $Q$ is the difference in
the intensity of the vertically and horizontally polarized components,
\begin{equation}
  \label{eq:polarization:Q_Idiff}
  Q = I_v - I_h
\end{equation}
where
\begin{eqnarray}
  \label{eq:polarization:Iv}
  I_v &=& E_v E_v^\ast\\
  \label{eq:polarization:Ih}
  I_h &=& E_h E_h^\ast
\end{eqnarray}
  
Thus if we measure $I_v$ and $I_h$ using -- for optical wavelengths --
a polarizer aligned with the $v$- and the $h$-axis, respectively, or
using -- for microwaves -- two appropriately aligned dipole antennas, we
can directly obtain $I$ by taking their sum and $Q$ by taking their
difference.

$U$ and $V$ can likewise be expressed as differences of intensities, but not
with respect to the linear base \eVrt and \eHor:

We recall Eq.~(\ref{eq:polarization:e_field}), omitting the
oscillatory term:
\begin{equation}
  \label{eq:polarization:E_linbase}
  \VctStl{E}= \left(E_v \eVrt +  E_h \eHor \right) 
\end{equation}

Now we want to write \VctStl{E} by two components along polarization axes at
$\pm 45\degree$ with respect to the $h$-axes. The basis vectors are
thus (cf.\ Figure~\ref{fig:polarization:e45})
\begin{eqnarray}
  \label{eq:polarization:e_p45}
  \ePls &=& \sqrt{\half} \left( \eHor - \eVrt \right) \\
  \label{eq:polarization:e_m45}
  \eMin &=& \sqrt{\half} \left( \eHor + \eVrt \right) 
\end{eqnarray}
and we get the field vector in this modified linear basis:
\begin{equation}
  \label{eq:polarization:E_lin45base}
  \VctStl{E} = \underbrace{
               \sqrt{\half}\left(E_v +  E_h \right)}_{E_{-45\degree}} 
               \eMin 
              + \underbrace{
               \sqrt{\half}\left( -E_v +  E_h \right)}_{E_{+45\degree}} 
               \ePls 
\end{equation}
%
\begin{figure}[!h]
 \begin{center}
  \begin{minipage}[c]{0.9\textwidth}
   \begin{center}
    \includegraphics*[width=0.4\hsize]{Figs/pol_e45}
   \end{center}
  \end{minipage}
  \begin{minipage}[c]{0.9\textwidth}
   \caption{Two sets of basis vectors for the linear basis.}
   \label{fig:polarization:e45}
  \end{minipage}
 \end{center}
\end{figure}   
%FIXME: Image showing \eVrt, \eHor, \ePls, \eMin.

%
With the definitions of intensities of the components,
\begin{eqnarray}
  \label{eq:polarization:Im45}
  I_{-45\degree} &=& E_{-45\degree} E_{-45\degree}^\ast\\
  \label{eq:polarization:Ip45}
  I_{+45\degree} &=& E_{+45\degree} E_{+45\degree}^\ast
\end{eqnarray}
we get for their difference:
\begin{eqnarray}
   I_{-45\degree} -  I_{+45\degree}
   &=&
     \half (E_v + E_h) (E_v^\ast + E_h^\ast)
    - \half (-E_v + E_h) (-E_v^\ast + E_h^\ast)\\ \nonumber
   &=&
   E_v E_h^\ast + E_h E_v^\ast
  \label{eq:polarization:Idiff_lin45base}
\end{eqnarray}
Therefore (cf. Eq.~\ref{eq:polarization:stokesparam_U})
\begin{equation}
  \label{eq:polarization:U_Idiff}
  U =   I_{+45\degree} -  I_{-45\degree}
\end{equation}
Thus if we measure $I_{+45\degree}$ and $I_{-45\degree}$ using -- for
optical wavelengths -- a polarizer aligned at $+45\degree$ and
$-45\degree$ with respect to the $h$-axis, respectively, or using --
for microwaves -- two appropriately aligned dipole antennas, we can
directly obtain $U$ by taking their difference.

In order to see how to measure the fourth Stokes parameter, $V$, we
have to transform to the circular basis, i.e.\ express \VctStl{E} by a
left-hand ($LH$) and a right-hand ($RH$) circularly polarized component. The
relevant equations:

Basis vectors
\begin{eqnarray}
  \label{eq:polarization:e_Lh}
  \eLh &=& \sqrt{\half} \left( \eVrt + i\eHor \right) \\
  \label{eq:polarization:e_Rh}
  \eRh &=& \sqrt{\half} \left( \eVrt - i\eHor \right) 
\end{eqnarray}
%
%FIXME: \eRh{} and \eLh{} have been  swapped one...\\
Field vector in circular base
\begin{equation}
  \label{eq:polarization:E_circbase}
  \VctStl{E} = \underbrace{
               \sqrt{\half}\left(E_v -  i E_h \right)}_{E_{LH}} 
               \eLh 
              + \underbrace{
               \sqrt{\half}\left(E_v +  i E_h \right)}_{E_{RH}} 
               \eRh 
\end{equation}
%
Intensity of the components
\begin{eqnarray}
  \label{eq:polarization:ILH}
  I_{LH} &=& E_{LH} E_{LH}^\ast\\
  \label{eq:polarization:IRH}
  I_{RH} &=& E_{RH} E_{RH}^\ast
\end{eqnarray}
%
Their difference
\begin{eqnarray}
   I_{LH} -  I_{RH}
   &=&
     \half (E_v - i E_h) (E_v^\ast + i E_h^\ast)
    - \half (E_v + i E_h) (E_v^\ast - i E_h^\ast)\\ \nonumber
   &=&
   i (E_v E_h^\ast -  E_h E_v^\ast)
  \label{eq:polarization:Idiff_circbase}
\end{eqnarray}
Therefore (cf. Eq.~\ref{eq:polarization:stokesparam_V}):
\begin{equation}
  \label{eq:polarization:V_Idiff}
  V =   I_{RH} -  I_{LH}
\end{equation}
%
Thus if we measure $I_{RH}$ and $I_{LH}$ using -- for microwaves --
appropriate helical beam antennas, we can directly obtain $V$ by
taking their difference.  Unfortunately, for optical wavelengths, we
cannot measure $I_{RH}$ and $I_{LH}$ directly with the help of
filters like  polarizers and retarders%
\footnote{A retarder allows the phase of two orthogonal components 
of light to be varied  with respect to each other}.  
However, a combination of a retarder and a polarizer can be used
to measure the sum of $I$ and $V$:

The light first passes through a retarder that delays the phase of
the horizontally polarized component by 90\degree{} with respect to
the phase of the vertically polarized component (a quarter-wave
plate).
A phase delay by 90\degree can be expressed as a multiplication of the
horizontal component by $i$, so the resulting
electric field vector $\VctStl{E}^\prime$ is
\begin{equation}
  \label{eq:polarization:E_retarded}
  \VctStl{E}^\prime = \left(E_v \eVrt +   i E_h \eHor \right) 
\end{equation}
The light then passes through a polarizer that is  aligned at
$-45\degree$ with respect to the $h$-axis. This means we have to
project $\VctStl{E}^\prime$ onto $\eMin$, resulting in
\begin{equation}
  \label{eq:polarization:E_retarded_polarized}
  \VctStl{E}^{\prime\prime} 
   = (\VctStl{E}^\prime \cdot \eMin)\eMin 
   = \sqrt{\half}\left(E_v  +  i E_h  \right) \eMin 
\end{equation}
Measuring the intensity now, we get
\begin{eqnarray}
  I^{\prime\prime} 
   &=&  \left| \VctStl{E}^{\prime\prime} \right| ^2 \\ \nonumber
   &=& \half\left(E_v +  i E_h  \right)  
     \left(E_v^\ast -  i E_h^\ast  \right)\\ \nonumber
   &=& \half \left( |E_v|^2 + |E_h|^2 - 
        i (E_v E_h^\ast -  E_h E_v^\ast )\right) \\ \nonumber
   &=& \half ( I + V )
  \label{eq:polarization:I_retarded_polarized}
\end{eqnarray}


Here is a summary of the Stokes parameters in terms of intensities of
orthogonal components:
\begin{eqnarray}
  \label{eq:polarization:stokesparam_Idiff_I}
  I &=& I_v + I_h = I_{-45\degree} + I_{+45\degree} = I_{RH} + I_{LH}\\
  \label{eq:polarization:stokesparam_Idiff_Q}
  Q &=&   I_v - I_h \\
  \label{eq:polarization:stokesparam_Idiff_U}
  U &=&  I_{+45\degree} - I_{-45\degree} \\
  \label{eq:polarization:stokesparam_Idiff_V}
  V &=&  I_{RH} - I_{LH}
\end{eqnarray}
We see that $Q$ and $U$ are both related to linear polarization, while
$V$ is related to circular polarization.




\levelb{Partial Polarization}
%=====================================================================
\label{sec:polarization:part_pol}
The equality   $I^2 = Q^2 + U^2 + V^2$
(Eq.~\ref{eq:polarization:Isquare}) 
%and in fact all of section~\ref{sec:polarization:monochrom} 
is valid for the ideal case of a 
monochromatic plane wave that is completely polarized, i.e. where the
amplitudes $a_v$ and
$a_h$ and the phases $\delta_v$ and $\delta_v$ are fixed and do not
vary with time. This means that the plane wave is emitted by one
coherent source.

In reality, the amplitudes and phases fluctuate, since the radiation
originates from several sources that do not emit radiation coherently,
or since the emission from one source is not coherent with time. This
means that the polarization state fluctuates as well and that in
addition we usually have a superposition of several incoherent waves.
Typically, such fluctuations have time scales that are longer than the
period ($2\pi/\omega$) of the oscillation, but that are still shorter
than the integration time of the instrument that measures the
radiation. Thus, the instrument measures a superposition of time averages
over of the fluctuating polarization. If the fluctuations are
completely random for all the sources or if the different sources are
incoherent (uncorrelated?), then there is no preferred(??)
orientation, ellipticity or handedness, and the radiation is called
unpolarized. If the fluctuations are not completely random, the
radiation is called partially polarized.

To quantify this rather heuristic argumentation, we express the
above-mentioned ideas in the language of the Stokes parameters:
The Stokes parameters $I$, $Q$, $U$, $V$
derived from measurements come from the superposition of
radiation from many sources and/or the average over emission events with individual Stokes
parameters $I_i$, $Q_i$, $U_i$, $V_i$.  Since the different
sources/emission events are incoherent, the Stokes parameters -- which
are intensity, not amplitude quantities -- can
simply be added up:
\begin{equation}
  \label{eq:polarization:summed_stokes}
  I = \sum_i I_i \: \mbox{, }\; 
  Q = \sum_i Q_i \: \mbox{, }\; 
  U = \sum_i U_i \: \mbox{, }\; 
  V = \sum_i V_i
\end{equation}
In the case of unpolarized radiation, i.e. when the amplitudes and
phases, or equivalently, the orientation angle $\zeta$ and the
ellipticity angle $\beta$ are completely random (FIXME: really
completely random? or uncorrelated, fluctuating?), $Q$, $U$, and $V$
each cancel out.

For each $i$, the equality $I_i^2 = Q_i^2 + U_i^2 + V_i^2$
holds. Thus, we get for partially polarized radiation:
\begin{eqnarray}
  I^2 &=& \left(\sum_i I\right)^2\\ \nonumber
      &\geq& \sum_i I_i^2 = \sum_i (Q_i^2  + U_i^2 + V_i^2)\\ \nonumber
      &=& Q^2 + U^2 + V^2
  \label{eq:polarization:stokes_inequality}
\end{eqnarray}
The inequality follows from  the mathematical fact that the square of
a sum of non-negative numbers is greater or equal than the sum of the
squares. 

Now we can define a measure for the degree of polarization, $p$:
\begin{equation}
  \label{eq:polarization:pol_degree}
  p = \frac{\sqrt{Q^2 + U^2 + V^2}}{I}
\end{equation}
%FIXME: Should the $p$ be squared (cf. equ.~\ref{eq:polarization:p_lin}
%and~\ref{eq:polarization:p_circ}?
For completely polarized radiation, $Q^2 + U^2 + V^2 = I^2$, so $p =
1$, and for unpolarized radiation, $Q = U = V = 0$, so $p = 0$.

Furthermore, it can be convenient to define the the polarized
component of radiation by
\begin{equation}
  \label{eq:polarization:pol_component}
  I_p^2 = Q^2 + U^2 + V^2
\end{equation}
and the unpolarized component as
\begin{equation}
  \label{eq:polarization:unpol_component}
  I_u = I - I_p
\end{equation}

In addition, we can define measures for the circularity and the
linearity of the polarization.
Recalling Eqs.~(\ref{eq:polarization:stokesparam_Idiff_U})%
~and~(\ref{eq:polarization:stokesparam_Idiff_V}),
we can define the degree of linear polarization, $p_{lin}$, as
\begin{equation}
  \label{eq:polarization:p_lin}
 p_{lin} = \frac{\sqrt{Q^2 + U^2}}{I} 
\end{equation}
and the the degree of circular polarization, $p_{circ}$, as
\begin{equation}
  \label{eq:polarization:p_circ}
 p_{circ} = \frac{V}{I} 
\end{equation}
%(Stephens p.61)

FIXME: Give some typical numbers for the relative magnitude of $I$,
$Q$, $U$, $V$
for realistic radiation in relevant cases ($I \gg Q \gg U,V$ for
atmosphere?)


%This is $S_{\mathrm{H_2 O}}$.

%This is $S_{H_2 O}$.

%This is $S_{O2}$.
%%% Local Variables: 
%%% mode: latex
%%% TeX-master: "uguide"
%%% End: 
% LocalWords:  ext matrix abs vec pha pnd sca lat lon za aa pt FIXME Eq LH RH
% LocalWords:  Eqs mishchenko scatt nonsp partic RTE Melsheimer overline
