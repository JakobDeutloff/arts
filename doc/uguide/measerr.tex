%
% To start the document, use
%  \levela{...}
% For lover level, sections use
%  \levelb{...}
%  \levelc{...}
%
\levela{Measurement errors}
 \label{sec:measerr}


%
% Document history, format:
%  \starthistory
%    date1 & text .... \\
%    date2 & text .... \\
%    ....
%  \stophistory
%
\starthistory
  000315 & Created and written by Patrick Eriksson.\\
\stophistory


%
% Symbol table, format:
%  \startsymbols
%    ... & \verb|...| & text ... \\
%    ... & \verb|...| & text ... \\
%    ....
%  \stopsymbols
%
%
%\startsymbols
%  -- & -- & -- \\
% \label{symtable:measerr}     
%\stopsymbols



%
% Introduction
%
Following Equation \ref{eq:formalism:fm},
\begin{eqnarray}
   \y = \fm + \merr, \nonumber
\end{eqnarray}
measurement errors, \merr\, are here defined as errors that are
additive to the spectrum, that is, not dependent on the actual spectrum.
Error sources falling into this category are thermal noise and
baseline ripples (there is a small influence of the magnitude of the
spectrum on the thermal noise but this effect is normally totally
negligible).

The term baseline ripple is used here as a common name for all instrumental
imperfections causing a distortion of the spectra, for example,
reflections inside the receiver, adding theoretically a sinusoidal term
to the spectrum.



\levelb{General}
 \label{sec:measerr:general}
 
 The sensor transfer matrix can be neglected when treating measurement
 errors as these errors are assumed to be additive to the spectra. On
 the other hand, a possible data reduction must be considered. This
 fact can also be understood by Equation \ref{eq:formalism:datared}:
 \begin{eqnarray}
   \y = \Hd \y' = \Hd (\Hs \iv + \merr') = \Hm \iv + \merr \nonumber
 \end{eqnarray}
 Using this equation, a measurement error WF can be
 written as
 \begin{equation}
    \K_{\xt}^p = \frac{\partial \y}{\partial \xt^p} 
               = \frac{\partial \merr}{\partial \xt^p}
               =  \Hd \frac{\partial \merr'}{\partial \xt^p}
  \label{eq:measerr:kx}
 \end{equation}
 Accordingly, quantities connected with the measurement errors shall be
 multiplicated with the data reduction matrix \Hd, this in contrast to
 the atmospheric WFs where the total reduction sensor matrix must be applied
 (Eq. \ref{eq:formalism:kx2}).



\levelb{Thermal noise}
 \label{sec:measerr:tn}
 
 The nature of the thermal noise differs from all other variables and
 error sources. The most distinct feature of the thermal noise is the
 low correlation between the measurements channels, in fact, the
 thermal noise is normally assumed to be totally uncorrelated. Such an
 assumption results in that a variable for each channel would be
 needed to model, or to fit, the measurement noise, and this is not a
 practical solution. In addition, it is not even of interest to know
 the actual magnitude of the thermal noise for each single
 measurement, we are instead interested in the statistical
 characteristics of the thermal noise.  The special nature of the
 thermal noise has the consequence that this term is treated
 differently than the other variables. Instead of providing weighting
 functions, the forward model gives the covariance matrix for the
 thermal noise.
 
 Thermal noise is introduced in two ways, by the observation of the
 atmosphere, and by the calibration process. The first part is here
 denoted as measurement thermal noise, while the latter is denoted as
 calibration thermal noise. In many cases, there is no practical
 difference between the two terms and they can together be treated as
 measurement thermal noise. However, if a single calibration
 measurement is used for a number of atmospheric spectra that are
 inverted jointly, as is the normal case for limb sounding, the error
 introduced by the calibration is totally correlated between the
 different viewing angles and it could be of importance to consider
 this fact.
 

 \levelc{Measurement thermal noise}
 \label{sec:measerr:mtn}
 
 As mentioned above, measurement thermal noise is here defined to be
 totally uncorrelated between the different viewing angles.  The
 magnitude of the thermal noise, expressed in brightness temperatures,
 is described by the radiometer noise formula
 \begin{equation}
   \sigma_{tn}^i = \frac{q\left(T_{rec}+T_a^i\right)}{\sqrt{\Delta \f^i \tau}}
  \label{eq:measerr:tn}
 \end{equation}
 where $\sigma_{tn}^i$ is the standard deviation of the thermal noise
 for channel $i$, $q$ a compensation factor $T_{rec}$ the receiver
 noise temperature, $T_a$ the antenna temperature, $\Delta \f$ the
 channel bandwidth and $\tau$ the integration time.
 
 The factor $q$ is used to compensate for extra noise introduced by
 the calibration, losses in the spectrometer etc. It is important to
 define $q$ and $\tau$ consistently. Let us take an ordinary load
 switching instrument as example, where one half of the time is used
 to measure the atmosphere and the other half is used to observe a
 reference load. If then $\tau$ gives the total integration time, $q$
 should be (about) a factor $\sqrt{2}$ higher than when $\tau$ gives
 only the integration time for the atmospheric observations.

 The thermal noise is often assumed to be uncorrelated between the
 measurement channels, and the corresponding covariance matrix,
 $\mat{S}$ is then diagonal, where the diagonal elements are
 \begin{equation}
   \mat{S}_{tn}^{ii} = \left( \sigma_{tn}^i \right)^2
  \label{eq:measerr:Stn_diag}
 \end{equation}
 where $\mat{S}^{ii}$ is element $(i,i)$ of the matrix.
 
 However, for most spectrometer types there exist in fact some
 correlation of the noise between the channels as there is an overlap
 of the channel frequency responses.  The inter-channel correlation of
 the thermal noise can be treated in the forward model by three
 different correlation functions: (1) gaussian
 \begin{equation}
  c^{ij} = exp\left(-\left(\frac{\f_i-\f_j}{f_c}\right)^2\right)
 \end{equation}
 (2) exponential
 \begin{equation}
  c^{ij} = exp\left(-\frac{|\f_i-\f_j|}{f_c}\right)
 \end{equation}
 and (3) tenth
 \begin{eqnarray}
  c^{ij} &=& 1-\frac{|\f_i-\f_j|(1-e^{-1})}{\f_c}, \quad 
            |\f_i-\f_j| < \frac{\f_c}{(1-e^{-1})} \nonumber \\
  c^{ij} &=& 0, \quad |\f_i-\f_j| \geq \frac{\f_c}{(1-e^{-1})}
 \end{eqnarray}
 where $\f_c$ is the frequency distance where the correlation has
 declined to $e^{-1}$, the frequency correlation length, and $\f_i$
 the middle frequency of channel $i$ (Fig. \ref{fig:measerr:cfuns}).
 It is also possible to apply a threshold for the correlation, where
 all $c^{ij}$ below the threshold value are set to 0.

 \begin{figure}
  \begin{center}
   \begin{minipage}[c]{0.65\textwidth}
    \centering
    \includegraphics*[width=0.99\hsize]{Figs/fig_corrfuns}
   \end{minipage}%
   \hspace{0.03\textwidth}%
   \begin{minipage}[c]{0.30\textwidth}
    \centering
    \caption{The frequency correlation functions. The frequency is scaled to
             the correlation length as $(\f_i-\f_j)/\f_c$.}
    \label{fig:measerr:cfuns}
   \end{minipage}
  \end{center}
 \end{figure}           
 
 The covariance matrix for one viewing angle with inter-channel
 correlation is
 \begin{equation}
   \mat{S}_{tn}^{ij} = c^{ij} \sigma_{tn}^i \sigma_{tn}^j
  \label{eq:measerr:Stn}
 \end{equation}
 The correlation between different viewing angles is set to 0.

 To include the effect of data reduction, the covariance matrix is
 multiplicated with \Hd\ as
 \begin{equation}
   \mat{S}_{tn} = \Hd \mat{S}_{tn}' \Hm_d^T 
   \label{eq:measerr:HSH}
 \end{equation}
 where $\mat{S}_{tn}'$ is the covariance matrix before data reduction.


 \levelc{Calibration thermal noise}
 \label{sec:measerr:ctn}
 
 In contrast to the measurement thermal noise, the calibration thermal
 noise is assumed to be totally correlated between the different
 viewing angles. This latter noise as assumed to be identical between the
 channels and a simplified expression is used:
 \begin{equation}
   \sigma_{tn}^i = \frac{T_{cal}}{\sqrt{\Delta \f^i \tau_{cal}}}
  \label{eq:measerr:tn2}
 \end{equation}
 where $T_{cal}$ is an effective noise temperature covering all relevant
 effects and $\tau_{cal}$ the calibration integration time.

 The correlation functions used for the measurement thermal noise can
 also be applied for the calibration thermal noise.  
 
 Data reduction is considered by Equation \ref{eq:measerr:HSH}.
 


\levelb{Sinusoidal baseline ripple}
 \label{sec:measerr:sin}
 
 Reflections inside the receiver give theoretically rise to a
 sinusoidal baseline ripple. The relationship between the period
 length in the spectrum, $\Delta \f_{2\pi}$, and the physical distance
 between the reflecting objects, $l$, is \citep{rohlfs:86} 
 \begin{equation}
   \Delta \f_{2\pi} = \frac{c}{2l}
 \end{equation}
 where $c$ is the speed of light.
 
 This type of baseline ripple is retrieved by expressing the sine
 functions, with unknown amplitude and phase, as a sum of sine and
 cosine functions \citep{kuntz:97}
 \begin{equation}
   \merr_{sin} = \sum_{i=1}^n \left( 
          x_i sin\left(2\pi\frac{\f-\bar{\f}}{\Delta \f_{2\pi}^i}\right) +
          x_{i+n} cos\left(2\pi\frac{\f-\bar{\f}}{\Delta \f_{2\pi}^i}\right) \right)
  \label{eq:measerr:sines}
 \end{equation}
 where $n$ is the number of ripple terms, $\bar{\f}$ the mean
 frequency, $\f_{2\pi}^i$ the period length of ripple $i$ and $x_i$
 are the amplitude of the sine and cosine functions to be determined.
 The length of the part of \xt\ used to fit sinusoidal baseline
 ripples is accordingly $2n$. The mean frequency is defined below by
 Equation \ref{eq:measerr:fmean}.
 
 Using Equation \ref{eq:measerr:kx}, the
 WFs for the sine and cosine terms can be determined to be
 \begin{equation}
   \K_{\xt}^p = \Hd \mat{a}_p
 \end{equation}
 and
 \begin{equation}
   \K_{\xt}^p = \Hd \mat{b}_p
 \end{equation}
 respectively, where the elements of the vectors $\mat{a}_p$ and
 $\mat{b}_p$ are
 \begin{equation}
   \mat{a}_p^i = sin\left(2\pi\frac{\f^i-\bar{\f}}{\Delta \f_{2\pi}^p}\right) 
 \end{equation}
 and
 \begin{equation}
   \mat{b}_p^i = cos\left(2\pi\frac{\f^i-\bar{\f}}{\Delta \f_{2\pi}^p}\right),
 \end{equation}
 where $\f^i$ is the frequency for channel $i$.
 
 It should be noted that the treatment of baseline ripple neglects the
 effect of the spectrometer and Equation \ref{eq:measerr:sines} assumes
 that the widths of the spectrometer channels are much smaller than
 the period length of the ripple. However, this should be the
 situation found for most practical situations.



\levelb{Polynomial baseline ripple}
 \label{sec:measerr:pol}
 
 \begin{figure}
  \begin{center}
   \begin{minipage}[c]{0.62\textwidth}
    \centering
    \includegraphics*[width=0.99\hsize]{Figs/kpol}
   \end{minipage}%
   \hspace{0.03\textwidth}%
   \begin{minipage}[c]{0.35\textwidth}
    \centering
    \caption{Polynomial WFs of order 0, 1 and 2. The scaled frequency is 
             $f'=(\f-\bar{\f})/\Delta \f$.}
    \label{fig:measerr:kpol}
   \end{minipage}
  \end{center}
 \end{figure}           

 A polynomial representation of the baseline ripple can be suitable
 at many occasions. One example is when a sinusoidal baseline ripple 
 has a period that exceeds significantly the total frequency coverage
 of the receiver and the exact period length is not known. A baseline
 polynomial can also be used to fit continuum absorption for linear
 situations, e.g. to fit the unknown emission from the troposphere
 for ground-based observations.

 The polynomial measurement error is modeled as
 \begin{equation}
   \merr_{pol} = x_0 + \sum_{i=1}^{n_{pol}} x_i \left( 
                      \frac{\f-\bar{\f}}{\Delta \f} \right)^i
 \end{equation}
 where $n_{pol}$ is the polynomial order selected, $x_i$ are the 
 polynomial coefficients to be determined, and $\bar{\f}$ and
 $\Delta \f$ normalization factors. The part of \xt\ corresponding
 to the polynomial fit of the baseline is accordingly
 \begin{equation}
   \xt = \left[ \begin{array}{c} \vdots \\ x_0\\ x_1 \\ \vdots \\ x_{n_{pol}} \\ \vdots \end{array} \right]
 \end{equation}
 The normalization factors are needed to avoid extreme values (without
 the factors the quantity $\f^i$ would have been calculated),
 resulting in that the magnitudes of the coefficients $x_i$ will not
 deviate too strongly. The factors are calculated as
 \begin{eqnarray}
   \bar{\f} &=& \frac{\f_{min}+\f_{max}}{2} \\
   \label{eq:measerr:fmean}
   \Delta \f &=& \frac{\f_{max}-\f_{min}}{2}
 \end{eqnarray}
 where $\f_{min}$ and $\f_{max}$ are the minimum and maximum value,
 respectively, of the frequency grid given by the spectrometer. These
 definitions of the normalization factors give a scaled frequency grid
 extending from -1 to 1.

 The polynomial WFs are
 \begin{equation}
   \K_{\xt}^p = \Hd \mat{a}_p
 \end{equation}
 where the elements of $\mat{a}_p$ are
 \begin{equation}
   \mat{a}_p^i = \left( \frac{\f^i-\bar{\f}}{\Delta \f} \right)^p
  \label{eq:measerr:kpol}
 \end{equation}
 Note that for $p=0$, $\mat{a}_p=1$. 
 
 Examples on polynomial weighting functions are shown is Figure 
 \ref{fig:measerr:kpol}.



\levelb{Piecewise polynomial baseline ripple}
 \label{sec:measerr:ppol}
 
 If the spectrum is recorded with a number of spectrometers (or
 individual spectrometer parts) there could be a difference in the
 level between the different parts of the spectrum. Figure
 \ref{fig:wfuns:baselinefit} shows an example on such a spectrum.
 
 The baseline for such cases can be retrieved by piecewise polynomials
 where an individual polynomial is applied for each part of the
 spectrum. For frequencies inside the part of concern the WFs are
 given by Equation \ref{eq:measerr:kpol}, while for remaining
 frequencies the WFs are 0.  

 \begin{figure}[t]
  \begin{center}
   \includegraphics*[width=0.72\hsize]{Figs/fig_baselinefit}
   \caption{Example on fit of baseline with piecewise polynomials.
     The top figure shows a (poor!) test measurement with the 22.2 GHz
     water vapor radiometer at Onsala Space Observatory, Sweden.  The
     spectrum was recorded by an auto-correlator spectrometer having
     four 20 MHz wide individual parts, clearly seen in the spectrum.
     The middle figure shows the measurement spectrum after a
     correction based on the retrieved baseline variables, and the
     simulated spectrum corresponding to the retrieved profile. The
     baseline is fitted by 3:rd order polynomial over the whole
     frequency range, and a 2:nd oder polynomial inside each 20 MHz
     range. The lower figure shows the difference between the spectra
     in the middle figure, the residual.}
   \label{fig:wfuns:baselinefit}
  \end{center}
 \end{figure}



%%% Local Variables: 
%%% mode: latex 
%%% TeX-master: "main" 
%%% End:

