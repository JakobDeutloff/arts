\chapter{Include ARTS in third-party C++}
%--------------------------------------------------------------------------
\label{sec:cpp_api}

\starthistory
  2020-10-06 & Created and written by Richard Larsson.\\
\stophistory

%
% Introduction
%

It is possible to include ARTS in another C++ program using
the \verb|public_arts_interface| and linking to \verb|autoarts.h|.
This will pull in the ARTS public API defined in \verb|agendas.cc|,
\verb|groups.cc|, \verb|methods.cc|, \verb|workspace.cc| as well
as an automatically generated ARTS namespace to your project.

\FIXME{a private Module that imports only the ARTS namespace
       should be added as soon as soon as C++20 becomes norm}

\section{Linking the public interface}
%--------------------------------------------------------------------------
\label{sec:cpp_api:public_linking}

To link the public interface, you need to \verb|add_subdirectory(arts)|
anywhere in your CMake project, where the \verb|arts| directory should contain
a current version of ARTS.  You also need to manually link OpenMP again.

An example (in your projects \verb|CMakeLists.txt|):
\begin{verbatim}
###########################################################
# OpenMP Library
include (FindOpenMP)
if (NOT ${OPENMP_FOUND})
  message(FATAL_ERROR "No OPEN MP found, ARTS requires it")
endif()

# ARTS Library
add_library (arts_interface STATIC interface.cpp)
target_link_libraries(arts_interface PUBLIC
                      public_arts_interface
                      OpenMP::OpenMP_CXX)
###########################################################
\end{verbatim}

At this point, your \verb|interface.cpp| must have
\verb|#include <autoarts.h>| as one of its headers
and you are good to go.

\section{Using the C++ namespace interface}
%--------------------------------------------------------------------------
\label{sec:cpp_api:usage}

The ARTS namespace contains all the interfaces you will need to perform
all operations supported by ARTS.  The namespace defines only a single
top level function call, the  \verb|init(...)| function and a single
top-level type, the ARTS Workspace.  The function is used to generate the ARTS Workspace
upon which all your function calls are made.  The ARTS namespace has several 
sub-namespaces for various purposes.  These are in short:
\begin{itemize}
 \item[Group] Defines the ARTS public type-system.
 \item[Internal] Defines the ARTS internal type-system.
 \item[Var] Defines the ARTS interface type-system,
                  create variables for the ARTS interface, and
                  access automatic variables that are defined in the ARTS Workspace
                  from the start.
 \item[Method] Defines the ARTS methods.  The ARTS methods can only be called
                     using their generic inputs and outputs.
 \item[AgendaMethod] Defines the ARTS methods that can be used by agendas.
                           They return an internal type and are best used directly,
                           with no manual modification.
 \item[AgendaDefine] Defines methods to set the different ARTS agendas.
                           The agendas are checked and ready to be used after these
                           methods are called.
 \item[AgendaExecute] Executes an agenda.
\end{itemize}

If you are using the public interface, you need not be concerned with the types in the sub-namespaces Group and Internal.
\FIXME{Otherwise, these define the basic types you need to use ARTS easily.}
These sub-namespaces are used the same as any other C++ type-system.  The other
namespaces are domain specific.

An interface will generally start with a call to the initialize function.
This could look like:
\begin{verbatim}
  auto ws = init();
\end{verbatim}
After this the order and set of commands that are placed is up to the user.
For sake of ease, \verb|ws| will be used below to indicate a defined 
Workspace.  Also, the access to each sub-namespace will be written as if
operating in the ARTS namespace itself.

\subsection{Var}
%--------------------------------------------------------------------------
\label{sec:cpp_api:usage:variable}
The Var namespace, short for Variable namespace, have three purposes
\begin{enumerate}
 \item Type the Method and Agenda interface
 \item Access common Workspace variables
 \item Create new variables on the Workspace
\end{enumerate}
The types that are defined correspond to the types in Group.
The purpose of these types is to pass input to the functions of
Method and AgendaMethod.  The main way to generate
instances of these variables is by their corresponding \verb|*Create|
function or by simply accessing them via their common Workspace
space name.  The only constructor that is recommended to use is the
construction from the corresponding Group, as this can simplify
the access to several methods.

It is highly recommended to not discard created variables, as they
will still occupy memory in the Workspace until the end of the program
and they become impossible to access once discarded.

Examples:
\begin{verbatim}
  // Define an index that is not in the workspace
  Var::Index x{Group::Index(1)};
  
  // Access an index that is in the workspace
  Var::Index y = Var::stokes_dim(ws);
  
  // Create a new index on the workspace
  Var::Index z{Var::IndexCreate(ws,
                                            Group::Index{1},
                                            "new_index_name")};
\end{verbatim}
Note that \verb|x| will not work as an input to AgendaMethod functions
but it will work as an input to Method functions.  It does not append
to the Workspace.  The other two will work both as inputs to
AgendaMethod and to Method functions.

\subsection{Method}
%--------------------------------------------------------------------------
\label{sec:cpp_api:usage:method}
The Method namespace contains all but the Agenda manipulating methods
defined in \verb|methods.cc|.  These can all be called using the generic
input and outputs.  The inputs and outputs are not guaranteed to be the
same as in the ARTS methods however, because C++ requires inputs with 
default values be placed last in the calling order.  Note that all 
standard inputs and outputs taken from the Workspace must be set on the
Workspace itself and cannot be passed as inputs.  This creates a few
idiosyncrasies compared to how ARTS is used in python or in a normal
controlfile.  

Examples:
\begin{verbatim}
  // Call yCalc (no GIN/GOUT)
  Method::yCalc(ws);
  
  // Call Touch on the wind field (Pure GOUT)
  Method::Touch(ws, Var::wind_u_field(ws));
  Method::Touch(ws, Var::wind_v_field(ws));
  Method::Touch(ws, Var::wind_w_field(ws));
  
  // Set p_grid by VectorNLogSpace
  Var::nelem(ws) = 51;
  Method::VectorNLogSpace(ws, Var::p_grid(ws), 1e+05, 1e-4);
  
  // Save x, y, z from the Var example
  Var::output_file_format(ws) = Group::String{"ascii"};
  Method::WriteXML(ws, x);
  Method::WriteXML(ws, y, Group::String{"extra.xml"});
  Method::WriteXML(ws, z);
\end{verbatim}
All methods requires a Workspace (\verb|ws|) to work.
The first case of the examples calls a function with neither generic input nor generic output --- it cannot take anything other than the Workspace.
The second triplet case of the examples calls Touch on all wind-field variables.  They will have been default-initialized after this process.
The third example shows the idiosyncrasies to other methods of using ARTS.  The Workspace variable \verb|nelem| has been used instead of a generic
input index to define \verb|VectorNLogSpace| --- thus \verb|nelem| must be set manually by the user of the interface.
The last examples uses the fact that many of \verb|WriteXML|'s inputs are default defined.  Since the default value of filename is empty and since
the interface then infers it has the variable name input, the first call to \verb|WriteXML| will generate a file called \verb|arts.in.xml|, the second
call will generate a file called \verb|extra.xml|, and the last call will generate a file called \verb|arts.new_index_name.xml|.  The first part of
the name can be changed by calling \verb|init()| with the corresponding arguments.

As a last note.  Several inputs above automatically generates inputs from standard C++ types, such as \verb|VectorNLogSpace|
generating two \verb|Var::Numeric| from two doubles.  This is a convenient way to use the methods but the user should
be aware that these methods will end up deleting variables in the end, so some care has to be taken when the scope of such
automatic variables is long.

\subsection{AgendaMethod, AgendaDefine, and AgendaExecute}
%--------------------------------------------------------------------------
\label{sec:cpp_api:usage:agenda}
The Agenda namespaces deals with setting and defining Workspace Agendas.
It is only possible to set Workspace Agendas that have been defined as part
of the Workspace at compilation time.  The AgendaMethod namespace contains
the same functions as the Method namespace but returns an Internal::MRecord.
The user of this interface is expected to never manually construct an MRecord.
Instead, this type is meant to only be used when Agendas are defined in the
AgendaDefine namespace.  The AgendaDefine namespace defines a variadic 
function per Agenda in the common Workspace and expects a list of MRecord
to set this Agenda's methods.  Finally, AgendaExecute exist to execute a single
Agenda.  Normally, this is not preferred since Agendas should generally just
be used inside methods, but the option still exist.

Examples:
\begin{verbatim}
  // Define the basic iy_main_agenda emission agenda
  AgendaDefine::iy_main_agenda(ws,
    AgendaMethod::ppathCalc(ws),
    AgendaMethod::iyEmissionStandard(ws));
  
  // Define an empty geo positioning agenda
  AgendaDefine::geo_pos_agenda(ws,
    AgendaMethod::Ignore(ws, Var::ppath(ws)),
    AgendaMethod::VectorSet(ws, Var::geo_pos(ws),
      Var::VectorCreate(ws, {}, "Default")));
\end{verbatim}
The first example simply sets its agenda by using two functions that
complete the inputs/outputs expected of the agenda.  The second example
does not need or want to use \verb|Var::ppath(ws)| so it ignores it.  It also
has the need of a default empty \verb|Var::Vector|.  This variable
has to first be created onto the Workspace before it can be used as an 
input.  Had a pure vector been used instead of the create-function,
a runtime error would occur.  The create function is only invoked once,
since the internal workings of the Agenda just need to have defined
the variable.

Lastly, each AgendaMethod function that has a default generic input will
create a static const Workspace variable of this type upon first call to
the method.  This means that calling such a method will incur a memory cost
that lasts until the end of the program.
