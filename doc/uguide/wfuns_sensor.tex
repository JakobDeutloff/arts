%
% To start the document, use
%  \chapter{...}
% For lover level, sections use
%  \section{...}
%  \subsection{...}
%
\chapter{Sensor variables weighting functions}
 \label{sec:wfuns_sens}


%
% Document history, format:
%  \starthistory
%    date1 & text .... \\
%    date2 & text .... \\
%    ....
%  \stophistory
%
\starthistory
  000320 & Created and written by Patrick Eriksson.\\
\stophistory


%
% Symbol table, format:
%  \startsymbols
%    ... & \verb|...| & text ... \\
%    ... & \verb|...| & text ... \\
%    ....
%  \stopsymbols
%
%
%\startsymbols
%  -- & -- & -- \\
% \label{symtable:wfuns_sens}     
%\stopsymbols



%
% Introduction
%
This section presents weighting functions for sensor variables, beside the
ones treated as measurement errors. The only covered feature so far is calibration.




\section{Calibration weighting functions}
 \label{sec:wfuns_sens:cal}
 
 \subsection{Proportional calibration errors} 
 This section gives the WF for situations where a calibration
 uncertainty gives an error that is directly proportinal to the noise
 free spectrum.  Such a calibration uncertainty can be encountered for
 e.g. ground-based observations of altitudes above the tropoapuse,
 where a compensation of the tropospheric attenuation must be made, as an
 error of the assumed tropospheric opacity gives rise to a proportional 
 calibration error.

 A measurement with a proportional calibration uncertainty
 can be expressed as
 \begin{equation}
   \y = \Hd\left( (1+x_{cal})\Hs\iv+\merr' \right)
 \end{equation}
 See Equation \ref{eq:formalism:datared} for definition of the variables.
 The WF for this case is easily obtained
 \begin{equation}
   \K_{\xt} = \Hd\Hs\iv = \Hm\iv = \y - \merr
 \end{equation}
 that is, the WF is identical to the (noise free) spectrum given by the forward
 model. 

 
 \subsection{Calibration load temperatures} 
 The calibration of a Dicke switched radiometer is often performed by
 observing two loads with known intensity. The calibration formula is
 then (neglecting data reduction)
 \begin{equation}
   \y^i =  I_1^i + (I_2^i-I_1^i)\frac{V_{atm}^i-V_1^i}{V_2^i-V_1^i} 
  \label{sec:wfuns_sens:loadcal}
 \end{equation}
 where $\y^i$ is the calibrated value for channel $i$, $I_1$ and $I_2$
 are the assumed intensities of the two loads, $V_{atm}$, $V_1$ and
 $V_2$ are the voltage recorded when observing the atmosphere, load 1
 and load 2, respectively.
 
 The load temperature WFs are obatined by differenting Equation
 \ref{sec:wfuns_sens:loadcal}. For example, we have that \citep{eriksson:97a}
 \begin{equation}
   \frac{\partial \y^i}{\partial I_1^i} = 1 - 
     \frac{V_{atm}^i-V_1^i}{V_2^i-V_1^i} = \frac{I_2^i-\y^i}{I_2^i-I_1^i} 
 \end{equation}
 The WF for load temperature 1 is then
 \begin{equation}
   \K_{\xt} = \Hd \mat{a}
 \end{equation}
 where the elements of the vector $\mat{a}$ are
 \begin{equation}
   \mat{a}^i = \frac{I_2^i-\y^i}{I_2^i-I_1^i}
 \end{equation}
 The corresponding expression for load 2 is
 \begin{equation}
   \mat{a}^i = \frac{\y^i-I_1^i}{I_2^i-I_1^i}
 \end{equation}
 Hence, these WFs are easily calculated if the spectrum (before data
 reduction) is at hand.



%\section{Pointing weighting functions}
% \label{sec:wfuns_sens:point}
% 
% No suitable analytical expression for the pointing WFs has been
% found.  The pointing is a sensor variable and the first choice for a
% perturbatation calculation would be to recalculate the sensor
% transfer matrix for a slightly changed pointing pattern. However, to
% set-up the sensor matrix can be a time consuming task, and Equation
% \ref{eq:wfuns:Hpert} was instead selected for the calculation of the
% pointing WFs.
%
% Think and check with practical calculations deciding calculation approach!!
% 
% \begin{eqnarray}
%   \int_{0}^{2\pi} r_a(\view) I(\view) \dd \view \nonumber
% \end{eqnarray}
%
% \begin{eqnarray}
%   \int_{0}^{2\pi} r_a(\view+\Delta \view) I(\view) \dd \view \approx
%   \int_{0}^{2\pi} r_a(\view) I(\view-\Delta \view) \dd \view \nonumber
% \end{eqnarray}
%
%
%\section{Frequency weighting functions}
% \label{sec:wfuns_sens:freq}




%%% Local Variables: 
%%% mode: latex 
%%% TeX-master: "uguide" 
%%% End:

