\chapter{Refractive index}
 \label{sec:rindex_theory}

\starthistory
  130802 & Started based on AUG chapter and ESA-planetary TN1 (Jana Mendrok).\\
\stophistory

Refractive index\index{refractive index} describes several effects of matter on
propagation of electromagnetic waves. Refractive index is basically a complex
quantity. However, in this chapter it is restricted to its real part, neglecting
the imaginary part, which describes absorption. Effects the real part of the refractive
index describes particularly include changes of the propagation speed of
electromagnetic waves, which leads to a delay of the signal, as well as a change
of the propagation direction, a bending of the propagation path. The latter is
commonly called refraction.

Several components in the atmosphere contribute to refraction, hence to the refractive index: the gas mixture(``air''), solid and liquid constituents (clouds, precipitation, aerosols), and electrons.
Refractivity ($N$) describes the deviation of the refractive index of a medium \Rfr  from the vacuum refractive index (\Rfr$_\mathrm{vacuum}=1$): $N=\Rfr-1$. Contributions of the different components to refractivity are additive.
One distinguishes between monochromatic and group refractive index, which
differ in case of dispersion\index{dispersion} leading to diverging propagation
paths at different frequencies. \FIXME{That needs to be more specific.}



\section{Gases}
 \label{sec:rindex:gases}
%
According to \citet{newell65:_absolute_jap}, the refractivity $N$, i.e., 
the deviation of the refractive index $n$ from 1.0 ($N=n-1$), of a gas
can be assumed to be proportional to its density.
\citet{newell65:_absolute_jap} give no validity range for this
assumption, but at least \citet{stratton68:_optical_jas} assumes that
it is valid even for the relatively high densities in the Venusian
atmosphere. We have not investigated whether this assumption may break
down at some point in the Jupiter atmosphere.

If we accept the assumption of refractivity scaling with gas
density, then the problem of parameterizing the refractivity can
be separated into two sub-problems: (a) determining the refractivity
index for reference conditions (reference pressure $p$ and temperature
$T$), and (b) deciding which gas law to use to scale it to other
conditions. The total refractivity is then simply the sum of all
partial refractivities, in other words
\begin{equation}
  \label{eq:N_density}
  N = N_{\mathrm{ref},1} \frac{n_1}{n_\mathrm{ref,1}} + 
      N_{\mathrm{ref},2} \frac{n_2}{n_\mathrm{ref,2}} + \cdots
\end{equation}
where $N$ is the total refractivity, $N_{\mathrm{ref},1}$ is the
partial refractivity for gas $i$ at reference conditions, $n_i$ is
the partial density, and $n_\mathrm{ref,i}$ is the reference density.

Different solutions have been proposed, where approaches specific for Earth
atmosphere commonly are empirical parameterisations.
Below we describe the background and formulas applied for the different methods
implemented in ARTS.

\subsection{Microwave general method (\shortcode{refr\_index\_airMWgeneral)}}
 \label{sec:rindex:mwgeneral}

Apart from presenting a basic approach, \citet{newell65:_absolute_jap} also
provide a thorough study of both refractivity of different gases for reference
conditions and which gas law to use to scale those to other
conditions. They present laboratory
refractivity measurements for dry CO$_2$-free air, argon,
carbon dioxide, helium, hydrogen, nitrogen, and oxygen covering the
most relevant gases (probably apart from water vapor) in planetary atmospheres.
The actual refractivity values of \citet{newell65:_absolute_jap}
are stated in the paper abstract and are not repeated here. The
conditions for the reported refractivity values are $T$ =
0$^\circ$C and $p$ = 760\,Torr. The measurements are for a frequency
of 47.7\,GHz.

\citet{newell65:_absolute_jap} also adress the question which gas law
to use to scale the measurements to other $p$/$T$ conditions, chosing different
approaches depending on the specific gas in question. While the more complicated
gas laws they suggest can be expected to be more
accurate in the relatively narrow range of $p$/$T$ conditions
considered by \citet{newell65:_absolute_jap}, it is not easy to
assess how well they will hold outside the range for which they were
originally derived. We therefore simply use the ideal gas
law, as given in Equation 7 of \citet{newell65:_absolute_jap} for all
gases. This results in the simple parameterization
\begin{equation}
  \label{eq:N_density2}
  N = \frac{273.15\,\mathrm{K}}{760\,\mathrm{Torr}}  
           \left[
           N_{\mathrm{ref},1} \frac{p_1}{T} + 
           N_{\mathrm{ref},2} \frac{p_2}{T} + \cdots
           \right]
\end{equation}
where $ N(T,p)$ is the total refractivity, the
first factor reflectes the reference conditions for the
\citet{newell65:_absolute_jap} data, $N_{\mathrm{ref},1}$ are the
partial refractivities as reported in the abstract of their
article, $p_i$ are the partial pressures, and $T$ is temperature.

Currently, the above formulas can be used with up to five contributing gas
species with reference refractive indices taken from
\citet{newell65:_absolute_jap}, namely \chem{N_2}, \chem{O_2}, \chem{CO_2},
\chem{H_2}, and \chem{He}. To account for contributions from further gases
(i.e., when volume mixing ratios of those five do not add up to 1), the
calculated refractivity from those five gases is normalised to a volume mixing
ratio of 1. By adding reference refractive index data from further species, the
method can easily be extended and made more complete.

\subsection{Microwave refractive index for Earth (\shortcode{refr\_index\_airThayer})}
 \label{sec:rindex:thayer}
The microwave refractive index due to gases in the Earth’s atmosphere is
calculated considering so-called compressibility factors (to cover non-ideal gas behaviour). The refractivity of ``dry air'' and water vapour is summed. All other gases are assumed to have a negligible contribution. \citep{thayer74_improved_rs}
\FIXME{be more specific and provide actual formula?}

\subsection{Infrared refractive index for Earth (\shortcode{refr\_index\_airIR})}
 \label{sec:rindex:IR}
The infrared refractive index due to gases in the Earth's atmosphere is derived,
but only refractivity of ``dry air'' is considered. The formula used is
contributed by Michael Hoepfner, Forschungszentrum Karlsruhe.
\FIXME{be more specific and document the actual formula?}


\section{Free electrons}
 \label{sec:rindex:freee}
%
Free electrons, as exist in the ionosphere, will affect propagating radio waves
in several ways. Free electrons will have an impact of the propagation speed of
radio waves, hence a signal can be delayed and refracted. 

An electromagnetic wave passing through a plasma (such as the ionosphere) will
drive electrons to oscillate and re-radiating the wave frequency. This is the
basic reason of the contribution of electrons to the refractive index. 
An important variable is the plasma frequency, $\Frq_{p}$:
\begin{equation}
  \omega_{p}=\sqrt{\frac{Ne^{2}}{\epsilon_{0}m}},
\end{equation}
where \(\omega_{p}=2\pi\Frq_{p}\), \(N\) is the electron density, \(e\) is the
charge of an electron, \(\epsilon_{0}\) is the permittivity of free space, and
\(m\) is the mass of an electron. For example, for the Earth's ionosphere
\(\Frq_{p}\) \(\approx\) 9\,MHz.
Waves having a frequency below $\Frq_{p}$ are reflected by a plasma.

Neglecting influences of any magnetic field, the refractive index of a plasma
is \citep[e.g.][]{rybicki:radia:79}
\begin{equation}
\label{eq:n:electrons}
\Rfr =\sqrt{1-\frac{\omega_{p}^{2}}{\omega^{2}}}=\sqrt{1-\frac{Ne^{2}}
{\epsilon_{0}m\omega^{2}}}.
\end{equation}
where $\omega$ is the angular frequency ($\omega=2\pi\Frq$). This refractive
index is less than unity (phase velocity is greater than the speed of light),
but is approaching unity with increasing frequency. The group velocity is
\citep{rybicki:radia:79}
\begin{equation}
\aSpd{g}=\speedoflight\sqrt{1-\frac{Ne^{2}}
{\epsilon_{0}m\omega^{2}}}
\end{equation}
which is clearly less than the speed of light.
The energy (or information) of a signal propagating through the ionosphere
travels with the group velocity, and the group speed refractive
index (\(\Rfr_{g}=\frac{\speedoflight}{\aSpd{g}}\)) is
\begin{equation}
\label{eq:ng:electrons}
  \Rfr_{g}=\left(
    1-\frac{Ne^{2}}
    {\epsilon_{0}m\omega^{2}}
  \right)^{-1/2}.
\end{equation}
Equations~\ref{eq:n:electrons} and \ref{eq:ng:electrons} are implemented in
\wsmindex{refr\_index\_airFreeElectrons}. The method demands that the
radiative transfer frequency is at least twice the plasma frequency.


