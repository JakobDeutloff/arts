\chapter{Faraday rotation}
 \label{sec:faraday}

\starthistory
 121216 & Written (PE), partly based on text written originally by 
          Bengt Rydberg.
\stophistory

The polarisation state of an electromagnetic wave propagating through a plasma
with a static magnetic field will be changed, normally denoted as Faraday
rotation. The effect is present for both passive and active signals, but
Faraday rotation is proportional to $\Frq^{-2}$ and can in general be neglected
for emission measurements due to the relatively high frequencies applied. For
Earth, the effect can in general be neglected above
$\sim$5\,GHz. Hence, \textindex{Faraday rotation} is a special consideration
for radio/microwave radiative transfer. A brief theoretical description is
found below in Section~\ref{sec:faraday:theory}. 



\section{Practical usage}
\label{sec:faraday:arts}
%
Faraday rotation is not treated by the DOIT and MC scattering modules.
Beside this, the effect is automatically considered as soon as both the
magnetic field and the free electron number density deviate from zero (this is
checked for each step of the propagation paths). The resulting
rotation is then included in the radiative transfer following
Equation~\ref{eq:faraday:mueller}. This means that Faraday rotation is treated
fully by e.g.\ \wsmindex{iyTransmissionStandard} and
\wsmindex{iyRadioLink} (as these WSMs do not use MC or DOIT), and also for 
\wsmindex{iyEmissionStandard} in the case of ``clear-sky''. 

However, a constrain for a correct treatment of the Faraday effect is that the
Stokes dimensionality (\wsvindex{stokes\_dim}) is set to 3 or 4, which can be
understood by Equation~\ref{eq:faraday:mueller}. The workspace variables
corresponding to free electrons and the magnetic field are introduced in
Section~\ref{sec:atm:ionosp}.

Information on the actual rotation can be obtained as auxiliary data by
\builtindoc{iyTransmissionStandard} and \builtindoc{iyRadioLink}. The total
rotation along the path is selected as ``Faraday rotation''. The unit is [rad].
The rotation per length unit (Eq.~\ref{eq:faraday:speed}) is selected as
``Faraday speed''. The unit is [rad/m].




\section{Theory}
\label{sec:faraday:theory}

A wave propagating through the ionosphere will force free electrons to move in
curved paths. If the incident wave is circularly polarised, the motion of the
electrons will be circular. The refractive index will then not be a single
constant, but depending on polarisation (i.e.\ anisotropic). More precisely,
left and right hand polarised waves will propagate with different speeds.
Moreover, as a plane polarised wave can be thought of as a linear superposition
of a left and a right hand polarised wave with equal amplitudes, but different
phase, the plane of polarisation will then rotate as the wave is propagating
through the media. This is denoted as Faraday rotation.

Birefringance \index{birefringance} is an associated mechanism, but it is not
yet treated by ARTS. This later effect originates also on the fact that right-
and left-hand circular polarisation have different refractive index. This can
result in that the two polarisations obtain different propagation paths. For
frequencies close to the ``plasma frequency'' the birefringance can be a strong
effect, but for higher frequencies it should be secondary to Faraday rotation.
Expressed roughly, a difference in optical path for the two circular
polarisation of a quarter of a wavelength changes the polarisation state
strongly by Faraday rotation, while additional effects coming from a difference
in propagation path (birefringance) should be negligible.

According to \citep{rybicki:radia:79}, using Gaussian (cgs) units, the angle of
rotation (\(\vartheta_{F}\)) of a plane polarised wave can be described as
\begin{displaymath}
\vartheta_{F}=
\frac{e^{3}}{2\pi \speedoflight^2 m^{2}\Frq^{2}}
\int_{0}^{d}n_{e}(s) {\bf B_{geo}} (s) \cdot  {\bf \DiffD s},  
\end{displaymath}
which converted to SI units becomes
\begin{equation}
\vartheta_{F}=\frac{e^{3}}{8\pi^2 \speedoflight\epsilon_{0}m^{2}\Frq^{2}}
\int_{0}^{d}n_{e}(s) {\bf B_{geo}} (s) \cdot  {\bf \DiffD s} \approx
\frac{23648}{\Frq^2} \int_{0}^{d}n_{e}(s) {\bf B_{geo}} (s) \cdot  {\bf \DiffD s},    
\end{equation}
where \(n_{e}(s)\) is the density of electrons at point \(s\),
\(\mathbf{B_{geo}}(s)\) is the geomagnetic field at point \(s\), and \(\cdot\)
denotes the dot (scalar) product. Accordingly, the Faraday rotation is
proportional to the part of the magnetic field along the propagation path,
the field normal to the path gives no effect.

The change in rotation angle along the
propagation path is then given by
\begin{equation}
  \frac{\DiffD \vartheta_{F}}{\DiffD s}=\frac{e^{3}}{8\pi 
    \speedoflight\epsilon_{0}m^{2}\Frq^{2}}n_{e}(s) {\bf B_{geo}} (s) \cdot 
    \hat{\bf s}. 
 \label{eq:faraday:speed}
\end{equation}
The effect on the Stokes vector of the rotation of the electric field can be
described by a Mueller matrix \citep{goldstein:polar:03,meissner:06:polar}:
\begin{equation}
\left[
\begin{array}{c}
\StoI_{F}\\
\StoQ_{F}\\
\StoU_{F}\\
\StoV_{F}\\
\end{array}
\right]
= \left[
\begin{array}{cccc}
1 & 0 & 0 & 0 \\
0 & \cos(2\vartheta_{F}) & -\sin(2\vartheta_{F}) & 0 \\
0 & \sin(2\vartheta_{F}) & \cos(2\vartheta_{F}) & 0\\
0 & 0 & 0 & 1\\
\end{array}
\right]
\left[
\begin{array}{c}
\StoI\\
\StoQ\\
\StoU\\
\StoV\\
\end{array}
\right], 
\label{eq:faraday:mueller}
\end{equation}
and we see that only the plane polarised components of the Stokes vector will
be affected by Faraday rotation.


