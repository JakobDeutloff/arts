\chapter{Faraday rotation}
 \label{sec:faraday}

\starthistory
 121217 & Written (PE), partly based on text written originally by 
          Bengt Rydberg.
\stophistory

The polarisation state of an electromagnetic wave propagating through a plasma
with a static magnetic field will be changed, normally denoted as Faraday
rotation. The effect is present for both passive and active signals, but
Faraday rotation is proportional to $\Frq^{-2}$ and can in general be neglected
for emission measurements due to the relatively high frequencies applied. For
Earth, the effect can in general be neglected above
$\sim$5\,GHz. Hence, \textindex{Faraday rotation} is a special consideration
for radio/microwave radiative transfer. A brief theoretical description is
found below in Section~\ref{sec:faraday:theory}. 



\section{Practical usage}
\label{sec:faraday:arts}
%
The first step is to ensure that the magnetic field and field of free electrons
are non-zero. See Section~\ref{sec:atm:vecfields} for how to introduce a
magnetic field. Free electrons are treated as an ``absorbing species'', and
hence are part of \wsvindex{vmr\_field} (Sec.~\ref{sec:rteq:absspecies}). A
further constrain is that the Stokes dimensionality (\wsvindex{stokes\_dim}) is
set to 3 or 4. This can be understood by Equation~\ref{eq:faraday:mueller},
Faraday rotation affects Stokes elements 2 and 3.

Faraday rotation is treated by \wsaindex{propmat\_clearsky\_agenda}. For the
moment, there exists a single workspace method for the task and that is
\wsmindex{propmat\_clearskyAddFaraday}. That is, this method must be included
in \builtindoc{propmat\_clearsky\_agenda} if you want to include Faraday
rotation in your calculations.

Information on the actual rotation can be obtained as auxiliary data by
\builtindoc{iyTransmissionStandard} and \builtindoc{iyRadioLink}. The total
rotation along the path is selected as ``Faraday rotation''. The unit is [deg].
The rotation per length unit (Eq.~\ref{eq:faraday:speed}) is selected as
``Faraday speed''. The unit is [deg/m].




\section{Theory}
\label{sec:faraday:theory}

A wave propagating through the ionosphere will force free electrons to move in
curved paths. If the incident wave is circularly polarised, the motion of the
electrons will be circular. The refractive index will then not be a single
constant, but depending on polarisation (i.e.\ anisotropic). More precisely,
left and right hand polarised waves will propagate with different speeds.
Moreover, as a plane polarised wave can be thought of as a linear superposition
of a left and a right hand polarised wave with equal amplitudes, but different
phase, the plane of polarisation will then rotate as the wave is propagating
through the media. This is denoted as Faraday rotation.

Birefringance \index{birefringance} is an associated mechanism, but it is not
yet treated by ARTS. This later effect originates also on the fact that right-
and left-hand circular polarisation have different refractive index. This can
result in that the two polarisations obtain different propagation paths. For
frequencies close to the ``plasma frequency'' the birefringance can be a strong
effect, but for higher frequencies it should be secondary to Faraday rotation.
Expressed roughly, a difference in optical path for the two circular
polarisation of a quarter of a wavelength changes the polarisation state
strongly by Faraday rotation, while additional effects coming from a difference
in propagation path (birefringance) should be negligible.

According to \citet{rybicki:radia:79}, using Gaussian (cgs) units, the angle of
rotation (\(\vartheta_{F}\)) of a plane polarised wave can be described as
\begin{displaymath}
\vartheta_{F}=
\frac{e^{3}}{2\pi \speedoflight^2 m^{2}\Frq^{2}}
\int_{0}^{d}n_{e}(s) {\bf B_{geo}} (s) \cdot  {\bf \DiffD s},  
\end{displaymath}
which converted to SI units becomes \citep{kraus:66}\footnote{The two versions
  of this equation are also found in Wikipedia (under \emph{Faraday effect}),
  but expressed using \Wvl\ and an error for the numerical factor
  (2012-12-17). The value $2.365\cdot10^4$, derived from the fundamental
  constants, is confirmed by \citet{wright:03}.}
\begin{equation}
\vartheta_{F}=\frac{e^{3}}{8\pi^2 \speedoflight\epsilon_{0}m^{2}\Frq^{2}}
\int_{0}^{d}n_{e}(s) {\bf B_{geo}} (s) \cdot  {\bf \DiffD s} \approx
\frac{23648}{\Frq^2} \int_{0}^{d}n_{e}(s) {\bf B_{geo}} (s) \cdot  {\bf \DiffD s},
\end{equation}
where $e$ is the charge of an electron, $\epsilon_{0}$ is the permittivity of
vaccum, $m$ is the electron mass, \(n_{e}(s)\) is the density of electrons at point \(s\),
\(\mathbf{B_{geo}}(s)\) is the geomagnetic field at point \(s\), and \(\cdot\)
denotes the dot (scalar) product. Accordingly, the Faraday rotation is
proportional to the part of the magnetic field along the propagation path,
the field normal to the path gives no effect.

The change in rotation angle along the propagation path, $r$ is then given by
\begin{equation}
  r=\frac{\DiffD \vartheta_{F}}{\DiffD s}=\frac{e^{3}}{8\pi^2
    \speedoflight\epsilon_{0}m^{2}\Frq^{2}}n_{e}(s) {\bf B_{geo}} (s) \cdot 
    \hat{\bf s}. 
 \label{eq:faraday:speed}
\end{equation}
An ``magneto-optical'' effect (Sec.~\ref{sec:rteq:propmat}) of this type is
mapped to a propagation matrix as 
\begin{equation}
 \ExtMat = \left[
\begin{array}{cccc}
0 & 0 & 0 & 0 \\
0 & 0 & 2r & 0 \\
0 & -2r & 0  & 0\\
0 & 0 & 0 & 0\\
\end{array}
\right].
\label{eq:faraday:propmat}
\end{equation}
If no other effects are present, the effect on the Stokes vector for some part
of the propgation path can be expressed as a Mueller rotation matrix
\citep{goldstein:polar:03,meissner:06:polar}:
\begin{equation}
\left[
\begin{array}{c}
\StoI_{F}\\
\StoQ_{F}\\
\StoU_{F}\\
\StoV_{F}\\
\end{array}
\right]
= \left[
\begin{array}{cccc}
1 & 0 & 0 & 0 \\
0 & \cos(2\vartheta_{F}) & -\sin(2\vartheta_{F}) & 0 \\
0 & \sin(2\vartheta_{F}) & \cos(2\vartheta_{F}) & 0\\
0 & 0 & 0 & 1\\
\end{array}
\right]
\left[
\begin{array}{c}
\StoI\\
\StoQ\\
\StoU\\
\StoV\\
\end{array}
\right].
\label{eq:faraday:mueller}
\end{equation}


