\chapter{Description of the atmosphere}
 \label{sec:atmosphere}

\starthistory
 050913 & Created by Claudia Emde. Included GField3 description.\\ 
\stophistory

\FIXME{Patrick: Move parts from chapter 3.1  here}

\section{Atmospheric fields} 
\label{sec:atmosphere:atmospheric_fields}

\subsection{Gridded Fields} 

In order to store three-dimensional atmospheric fields along with the
atmospheric grids, the class \typeindex{GField3} was
implemented. It resides in the files \fileindex{gridded\_fields.h} and
\fileindex{gridded\_fields.cc}.  The reading routine
\wsmindex{AtmRawRead} requires the volume mixing ratio profiles of all
gas species and the altitude and temperature profiles in
\typeindex{GField3} format. The reading routines
\wsmindex{ParticleTypeAdd} and \wsmindex{ParticleTypeAddAll} require
the particle number density fields in \typeindex{GField3} format.

The \typeindex{GField3} consists of the following fields:
\begin{itemize}
\item {\sl Vector} \artsstyle{p\_grid}: Pressure grid [Unit: Pa].
\item {\sl Vector} \artsstyle{lat\_grid}: Latitude grid [Unit:
  \degree].
\item {\sl Vector} \artsstyle{lon\_grid}: Longitude grid [Unit:
  \degree].
\item {\sl Tensor3} \artsstyle{data}: Data of the atmospheric
  field. The dimensions of the {\sl Tensor3} are:
  \artsstyle{ [pressure latitude longitude]}\\
  The unit is chosen according to the atmospheric field. 
\end{itemize} 




%%% Local Variables: 
%%% mode: latex
%%% TeX-master: "uguide"
%%% End: 
