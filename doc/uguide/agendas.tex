
\levela{Agendas}
%-------------------------------
\label{sec:agendas}

\starthistory
  020605 & Created by Stefan Buehler.\\
\stophistory

{\Large \bf FIXME: This is a construction site. Please don't read!}

This chapter deals with the main components of ARTS: \emph{Workspace
  variables}\index{workspace variables} (\textindex{WSVs}) and
\emph{workspace methods}\index{workspace methods} (\textindex{WSMs}).
Furthermore, it explains the use of \textindex{agendas}, a special
group of WSVs.

%To make the text easier to read, we will often use \emph{variables}
%as a synonym for workspace variables, and \emph{methods} as a synonmy
%for workspace methods. So:

%\begin{quote}
%\begin{tabular}{lclcl}
%  workspace variable &=& WSV &=& variable,\\
%  workspace method   &=& WSM &=& method
%\end{tabular}
%\end{quote}

\levelb{Implementation files}
%----------------------------
\label{sec:agendas:files}

The two most important files are:
\begin{itemize}
\item \fileindex{workspace.cc}:\\
  Definition and documentation of WSVs.
\item\fileindex{methods.cc}:\\
  Definition and documentation of WSMs. The
  implementations of WSMs reside in files named
  \artsstyle{m\_something.cc}.
\item \fileindex{agendas.cc}:\\
  Definition and documentation of agendas.
\end{itemize}
It is very likely that you will have to edit these. Less likely, but
possibly, you also have to edit:
\begin{itemize}
\item \fileindex{groups.cc}:\\
  Definition of WSV groups.
\end{itemize}

\vspace{2ex}
When ARTS is built, a number of source code files are generated
automatically. They are listed here in the order in which they are
generated: 
\begin{itemize}
\item \fileindex{auto\_wsv\_groups.h}:\\
  Generated from \artsstyle{groups.cc}.
\item \fileindex{auto\_wsv.h}, \fileindex{auto\_wsv\_pointers.cc}:\\
  Generated from \artsstyle{auto\_wsv\_groups.h} and \artsstyle{workspace.cc}. 
\item \fileindex{auto\_md.h}, \fileindex{auto\_md.cc}:\\
  Generated from \artsstyle{auto\_wsv\_groups.h},
  \artsstyle{auto\_wsv.h}, and \artsstyle{methods.cc}.
\end{itemize}
This is achieved by a set of simple C++ programs:
\begin{itemize}
\item \fileindex{make\_auto\_wsv\_groups\_h.cc}
\item \fileindex{make\_auto\_wsv\_h.cc}
\item \fileindex{make\_auto\_wsv\_pointers\_cc.cc}
\item \fileindex{make\_auto\_md\_h.cc}
\item \fileindex{make\_auto\_md\_cc.cc}
\end{itemize}
The meaning of the names should be self-explanatory. There is one program
for each file to be generated.  The generation of the
\artsstyle{auto\_} files happens automatically when you do a
\artsstyle{make}. Therefore, never edit any of these files.

Next, there are some files that contain the internal implementation
of WSVs and WSMs. These are:
\begin{itemize}
\item \fileindex{wsv\_aux.h}, \fileindex{wsv\_aux.cc},
  \fileindex{workspace\_aux.cc}:\\
  Implementation of class
  \typeindex{WsvRecord}, which stores the lookup information for one
  WSV, plus auxiliary stuff for the workspace.
\item \fileindex{methods.h}, \fileindex{methods\_aux.cc}:\\
  Implementation of class \typeindex{MdRecord}, which stores the
  lookup information for one WSM.
\end{itemize}

Finally, there are some files that contain the internal implementation
of agendas. These are:
\begin{itemize}
\item \fileindex{agenda\_class.h}, \fileindex{agenda\_class.cc}:\\
  Implementation of class \typeindex{MRecord}, which stores runtime
  information for one WSM, and class \typeindex{Agenda}, which stores
  an agenda.
\item \fileindex{agenda\_record.h}, \fileindex{agenda\_record.cc}:\\
  Implementation of class \typeindex{AgRecord}, which is used to store
  agenda lookup information.
\end{itemize}
  

\vspace{2ex} As mentioned above, you will not have to modify any of
the implementation files, they are listed here just for reference.
Normally, you only have to modify \artsstyle{workspace.cc},
\artsstyle{methods.cc}, and \artsstyle{agendas.cc}.

\levelb{Workspace Variables or WSVs}
%--------------------------
\label{sec:agendas:wsvs}

All important variables in ARTS are WSVs. This means that they can be
manipulated by a list of WSMs, which is specified in the ARTS
controlfile. There exists a predefined list of possible WSVs. This
list defines the \emph{workspace}. One can think of each WSV as a
`slot' in the workspace: The WSV can be either \emph{set}, or
\emph{unset}. Set means that the WSV has a well-defined content, unset
means that it has no well-defined content. At the start of an ARTS job
all WSVs are unset.

WSVs are defined in the file \fileindex{workspace.cc}. A typical
definition looks like this:

{\small
\begin{verbatim}
wsv_data.push_back
  (WsvRecord
   ( NAME( "f_grid" ),
     DESCRIPTION
     (
      "The frequency grid for monochromatic pencil beam\n"
      "calculations.\n"
      "\n"
      "Usage:      Set by the user.\n"
      "\n"
      "Unit:       Hz"
      ),
     GROUP( Vector_ )));
\end{verbatim}
}

\noindent
All WSV definitions have the same three elements:
\begin{enumerate}
\item The \emph{name}, exactly the
  same name has to be used in the code.
\item The \emph{description}, which is normally much longer than in
  the example here. It must fully describe the WSV, its purpose, and
  its normal usage. See file \artsstyle{workspace.cc} for instructions
  how to write the documentation.
\item The \emph{group} to which the WSV belongs. You can think of a
  group as something similar to a C++ data type. The WSV in the
  example belongs to the group \artsstyle{Vector}. The allowed groups
  are defined in file \fileindex{groups.cc}. Note that you have to add
  an underscore to the group name. A list of all groups can be found
  in Appendix \ref{app:groups}.
\end{enumerate}

\noindent
See Section \ref{sec:development:extending} for explicit
instructions how to add a new WSV to ARTS.

\levelb{Workspace Methods or WSMs}
%---------------------------------
\label{sec:agendas:wsms}

WSMs manipulate WSVs to produce other WSVs. There are three kinds of
WSMs:
\begin{enumerate}
\item Specific WSMs.
\item Generic WSMs.
\item Agenda WSMs.
\end{enumerate}
As in the case of WSVs, there is a central place in ARTS where
information on the available WSMs is stored. This place is the file
\fileindex{methods.cc}. It contains a record for each WSM. Here is an
example:

{\small
\begin{verbatim}
md_data.push_back
  ( MdRecord
    ( NAME("elsLorentz"),
      DESCRIPTION
      (
       "The Lorentz lineshape.\n"
       "\n"
       "This computes the simple Lorentz lineshape as:\n"
       "\n"
       "els[i] = 1/PI * ls_gamma /\n"
       "         ( (els_f_grid[i])^2 + ls_gamma^2 )\n"
       "\n"
       "Note that the frequency grid els_f_grid must hold\n"
       "offset frequencies from line center. Hence, the\n"
       "line center frequency is not needed as input.\n"
       "\n"
       "Output:\n"
       "   els        : The lineshape function [1/Hz]\n"
       "\n"
       "Input:\n"
       "   ls_gamma   : Line width [Hz].\n"
       "   els_f_grid : Frequency grid [Hz]."
      ),
      OUTPUT( els_ ),
      INPUT(  ls_gamma_, els_f_grid_ ),
      GOUTPUT(),
      GINPUT(),
      KEYWORDS(),
      TYPES()));
\end{verbatim}
}

\noindent
All WSM definitions have the same elements:
\begin{enumerate}
\item The \emph{name}, exactly as in the code.
\item The \emph{description}. This must fully describe the WSM, its
  purpose, and its normal usage. See file \artsstyle{methods.cc} for
  instructions how to write the documentation.
\item The \emph{output}. This is a list of WSV names (with underscores
  at the end). All these WSVs are set by this WSM.
\item The \emph{input}. This is a list of WSV names (with underscores
  at the end). All these WSVs are required as input by this WSM. This
  means they must have been set before.
\item The \emph{generic output}. This is a list of WSV group names (with underscores
  at the end). This defines the group to which output arguments must
  belong (see below).
\item The \emph{generic input}. This is a list of WSV group names (with underscores
  at the end). This defines the group to which input arguments must
  belong.
\item The \emph{keywords}, a list of keyword paramters that the user
  has to specify in the controlfile.
\item The \emph{types}, the data type associated with each keyword.
\end{enumerate}

\levelc{Specific WSMs}
%---------------------

For this type of WSM the output and input is fixed. Fields
\artsstyle{GINPUT} and \artsstyle{GOUTPUT} are empty. The example
above belongs in this category. It sets the WSV \artsstyle{els}, using
the WSVs \artsstyle{ls\_gamma} and \artsstyle{els\_f\_grid} as
inputs. What the function actually does is to compute a Lorentzian
lineshape function with width \artsstyle{ls\_gamma}, for the
frequencies given in the grid \artsstyle{els\_f\_grid}. (The line center
is at frequency 0.) The result is then stored in the output WSV
\artsstyle{els}. 

To call this method in the controlfile, you just have to write
\verb|elsLorentz{}|. 
% We have to use verb here, because otherwise the curly braces get lost.
The curly braces are empty, because the
method takes no keyword arguments. For an example with keyword
arguments see below.

\levelc{Generic WSMs}
%--------------------

This class of WSM is more powerful, because it can be applied to any
WSV that belongs to the right group. A good example is:

{\small
\begin{verbatim}
md_data.push_back
  ( MdRecord
    ( NAME("VectorCopy"),
      DESCRIPTION
      (
       "Creates a copy of a vector. \n"
       "\n"
       "Generic output: \n"
       "   Vector : The vector to be created. \n"
       "\n"
       "Generic input: \n"
       "   Vector : The vector to be copied. "
      ),
      OUTPUT(),
      INPUT(),
      GOUTPUT( Vector_ ),
      GINPUT( Vector_ ),
      KEYWORDS(),
      TYPES()));
\end{verbatim}
}

\noindent
As you probably have guessed, this WSM can make a copy of a WSV of
group \artsstyle{Vector} to another WSV of the same group. You would
use it as follows: 

{\small
\begin{verbatim}
VectorCopy(p_grid,f_grid){} 
\end{verbatim}
}

\noindent
This would copy the WSV \artsstyle{f\_grid} to the WSV
\artsstyle{p\_grid}.  Note that output arguments always come first,
input arguments last.

This example would copy the content of the WSV for the frequency grid
to the WSV for the pressure grid. Definitely not something you would
normally want to do. See the example controlfiles in the
\artsstyle{doc/examples} directory for more realistic examples. 

Here is a usage example of a generic WSM with keyword arguments:

{\small
\begin{verbatim}
VectorSet(f_grid){
  length = 100
  value  = 0
}
\end{verbatim}
}

\noindent
It sets \artsstyle{f\_grid}, adjusts the size to 100 elements, and
initializes all elements with 0. Try \artsstyle{arts -d VectorSet} to
get more information on this method. (See section
\ref{sec:online-docu} for information on command line switches.)

\levelc{Agenda WSMs}
%---------------------------------


\levelb{Agendas}
%--------------------------
\label{sec:agendas:agendas}


\levelb{ARTS online documentation}
%---------------------------------
\label{sec:online-docu}


%%% Local Variables: 
%%% mode: latex
%%% TeX-master: "uguide"
%%% End: 

