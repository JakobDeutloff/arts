\chapter{The Python interface}

This section is devoted to describing various techniques and code used to enhance the Python-ARTS interface: \mintinline{python}{pyarts}.
It will not be exhaustive but will go over the basic ideas of how to add and modify classes and functions.

The basic section will explain where the files are located in an ARTS directory and what the general build steps involved. 
 The classes section described how new classes are added and modified.  
 The workspace and agenda classes are special and so they get their own section.  
 The functions section will be short and go over how functions are added.

\section{Basics}
A \mintinline{python}{pyarts} installation consists of two parts: the pure Python code available in the main directory \verb|python/pyarts/| folder and the C++ interface.
The C++ interface relies on the \mintinline{cmake}{pybind11} Python bindings package available via \url{https://github.com/pybind/pybind11}.
We keep a local copy of the latest release of \mintinline{cmake}{pybind11} around in \verb|3rdparty/pybind11/|.
\mintinline{cmake}{pybind11} is a pure headers only library and these headers can be found at \verb|3rdparty/pybind11/includes/pybind11|.
The \verb|CMake| settings ensure that these can be included by any file compiled as part of the \mintinline{cmake}{pyarts_cpp} target using
\mintinline{cpp}{#include <pybind11/*.h>}.
The ARTS side of C++ linkage is available in the \verb|src/python_interface| folder.  There are two parts: the auto-generated files
that are only available after compiling the target, and the other static files.

The \verb|py_module.cpp| file links everything together.
It is responsible for passing a module reference object along to all other functions.
To keep things simple, we avoid creating a \verb|py_module.h| file and simply declare
that the functions from the other files are available to the linker down-the-line.

The steps to add a new file with new functionality are as follows (for \verb|py_my_file.cpp| as the file and \mintinline{cpp}{my_new_functionality} as function):
\begin{enumerate}
\item Create the \verb|py_my_file.cpp| file
\item Add \verb|py_my_file.cpp| to the \mintinline{cmake}{pyarts_cpp} target in the \verb|src/python_interface/CMakeLists.txt|.
\item Fill the file with the relevant includes (see below).
\item Within \mintinline{cpp}{namespace Python} setup \mintinline{cpp}{void my_new_functionality(py::module_& m) {*}}.
\item Replace \mintinline{cpp}{*} with your new functionality (and run the automatic formatter)
\item Add \mintinline{cpp}{void my_new_functionality(py::module_& m);} directly within the \mintinline{cpp}{Python} namespace of  \verb|py_module.cpp|.
\item Add \mintinline{cpp}{my_new_functionality(m);} within the \mintinline{cpp}{PYBIND11_MODULE(pyarts_cpp, m)} function definition of  \verb|py_module.cpp|.
\item Compile.  The new functionality is available via \mintinline{python}{pyarts.pyarts_cpp.*}.
\item Add a continuous integration test that tests the new functionality in a limited fashion.
\end{enumerate}

\subsection{Includes}
To keep a consistent experience across the code, it is recommended to put either \mintinline{cpp}{#include "python_interface.h"} or \mintinline{cpp}{#include <py_auto_interface.h>}
at the top of the files.  This ensures that you have access to almost all of \mintinline{cmake}{pybind11} and \mintinline{cmake}{artscore}.  If you need any of the headers that are
not included from \mintinline{cmake}{artscore}, the \verb|src/| folder is also linked to the \mintinline{cmake}{pyarts_cpp} target so you can include it without using relative paths.

Note that the documentation of the Python functions and classes that are generated depends on the order of calls within \verb|py_module.cpp|.  If your
class depends on  \mintinline{cmake}{matpack} and on gridded fields, place the call after these.  If in turn the Jacobian calculations depends on your type,
place the call above it.  This only applies to the call to your functionality (not the declaration above the module creation macro).

\subsection{Limitations}
There are many limitations.  Most of these are weeded out by keeping a constantly up-to-date set of tests available to be run as part of the continuous
integration.  Any feature not tested that way will be considered unwanted as part of the Python interface.  Add new tests as you add new functionality!
This is very important as \mintinline{cmake}{pybind11} links directly to Python, meaning that new features might work on your computer but not on others.  Even worse,
changes in the future might break your feature.

One limitation we cannot work around is that there are no references to \mintinline{python}{int} and \mintinline{python}{float} in Python.  The 
ARTS \mintinline{cpp}{Index} and \mintinline{cpp}{Numeric} map directly to these Python types.  As we need references to these variables,
wrappers known as \mintinline{cpp}{Index_} and \mintinline{cpp}{Numeric_} have been created.  If you need references, ensure that you are returning
these types instead of the pure types.  Note that there exist a helper function \mintinline{cpp}{as_ref} that can deal with this problem
for you.  Only use this if you know that you have a reference that will persist to a base-type.

A limitation that we can work around is that Python and \mintinline{cmake}{pybind11} does not have the concept of constants and
references in general.  If the output of a function is a reference, we need to mark that that variable is not allowed to
be converted.  More about this below.

Lastly, we often compile ARTS without debug information, turning off several sanity checks inside the classes.
These might cause segmentation fault if encountered in Python.  Please ensure that you do proper input/output
error handling manually when mapping to these functionalities.

\section{Classes, structs, and enums}
These are the core data holders in ARTS and the Python interface aims to wrap them in the way that their use in
Python mostly mimics their use in ARTS.  Of course, you might sometimes need to modify ARTS classes to get the
desired Python behaviour.

It is important that all classes that can be returned or used as arguments in the Python interface are defined as such.
The only exceptions to this rule are inputs that can be automatically converted from standard types to Python types
and back by the \mintinline{cmake}{pybind11} automatic conversion functions.

\subsection{Adding new class or struct \label{subsec: new class}}
A new class and struct is registered as
\begin{minted}{cpp}
auto c = py::class_<NewClassOrStruct>(m, "NameInPython");
\end{minted}

This new class will be registered as the type \mintinline{python}{pyarts.pyarts_cpp.NameInPython}.  It is also mapped via \mintinline{python}{pyarts.classes.NameInPython}.
Here \mintinline{python}{NewClassOrStruct} is a class or struct that the compiler is aware of from C++, \mintinline{cpp}{m} is the common module name (see above), and
\mintinline{cpp}{c} variable can accessed to modify the functionality of the class as seen in Python.  Most of these modifications returns \mintinline{cpp}{*this} as 
a reference.  It is thus often more convenient to write
\begin{minted}{cpp}
py::class_<NewClassOrStruct>(m, "NameInPython")
    .def(...)
    .def_property(...)
    .def_readwrite(...)
    ...
\end{minted}
Below the use of these \mintinline{cpp}{def*} will be discussed.

The main reason to use the former rather than the latter technique to define classes and structs is if you need the name
registered in the module early on.  As mentioned above, this changes how the Python documentation looks, so it is important
to register names in a timely fashion.

\subsection{Adding new enum}
There are two types of enumeration in C++.  Enum classes and C-style enums.  It is strongly recommended to make use of the
ARTS \mintinline{cpp}{ENUMCLASS} macro when creating new enum classes in the future.  These behave very much like normal classes,
providing both stream operators and conversions from and to string views.

If you want to add an \mintinline{cpp}{enum class} created by this macro, the following
should help (from \verb|py_jac.cpp|):
\begin{minted}{cpp}
py::class_<Jacobian::Atm>(m, "JacobianAtm")
    .def(py::init([]() { return new Jacobian::Atm{}; }))
    .def(py::init([](const std::string& c) { 
          return Jacobian::toAtmOrThrow(c); 
        }), py::arg("str").none(false))
    .PythonInterfaceCopyValue(Jacobian::Atm)
    .PythonInterfaceBasicRepresentation(Jacobian::Atm)
    .def(py::pickle(
        [](const Jacobian::Atm& t) {
          return py::make_tuple(std::string(Jacobian::toString(t)));
        },
        [](const py::tuple& t) {
          ARTS_USER_ERROR_IF(t.size() != 1, "Invalid state!")
          return new Jacobian::Atm{
              Jacobian::toAtm(t[0].cast<std::string>())};
        }));
py::implicitly_convertible<std::string, Jacobian::Atm>();
\end{minted}
This will allow you to read and write to the \mintinline{cpp}{enum} using Python \mintinline{python}{str}.

If you have an \mintinline{cpp}{enum class} that does not derive from \mintinline{cpp}{ENUMCLASS} or a pure C-style \mintinline{cpp}{enum}, the following example should help:
\begin{minted}{cpp}
py::enum_<PType>(m, "PType")
    .value("PTYPE_GENERAL", PType::PTYPE_GENERAL)
    .value("PTYPE_AZIMUTH_RND", PType::PTYPE_AZIMUTH_RND)
    .value("PTYPE_TOTAL_RND", PType::PTYPE_TOTAL_RND)
    .PythonInterfaceCopyValue(PType)
    .def(py::pickle(
        [](const PType& self) {
          return py::make_tuple(static_cast<Index>(self));
        },
        [](const py::tuple& t) {
          ARTS_USER_ERROR_IF(t.size() != 1, "Invalid state!")
          
          return static_cast<PType>(t[0].cast<Index>());
        }));
\end{minted}
This allows values to be accessed using, e.g., \mintinline{python}{pyarts.pyarts_cpp.PTYPE_GENERAL}.  The \mintinline{cpp}{.value} register a name of a value of the enum.

\subsection{Conversion from Python to ARTS types}
As said about the \mintinline{cpp}{"JacobianAtm"} example above, it is possible and quite easy to initialize ARTS types from Python types.  This works by
making use of the \mintinline{cpp}{def(py::init(*))} method.  The \mintinline{cpp}{*} should be replaced by a lambda that takes any number of arguments and 
returns a raw pointer or a value of the type that is being initialized.  (Note that it does not work to use smart pointers here as
the Python interface needs to take ownership.)

The \mintinline{cpp}{def(py::init([](const std::string& c) {*})} call from the example above, for instance, makes it possible to call
\begin{minted}{python}
import pyarts.pyarts_cpp as cxx
x = cxx.JacobianAtm("Temperature")
\end{minted}
in Python.  A \mintinline{cpp}{std::string} is not a Python type, but \mintinline{cmake}{pybind11} provides several automatic conversions between
Python and some standard C++ types.  These conversions generally depend on the right files being included.
Likewise, we can define our own automatic conversions using \mintinline{cpp}{py::implicitly_convertible<From, To>();}.  Since we registered the 
conversion from a C++ string to a \mintinline{cpp}{Jacobian::Atm} above, almost any functionality that takes this type (e.g., a call to a function)
also accepts a Python string.  The one exception here is if the function argument has been explicitly marked to not allow any conversions.

The lambda function initializing a class can take any number of variables. These can also have default values.  More about that later.

Feel free to add as many implicit conversions to a class as you see fit, but remember to always have the accompanying
initialization function defined when you do so.  The more implicit conversions we can guarantee to work, the easier it
will be to use ARTS from Python.

\subsection{Modifying an existing class}
We can add pretty much any functionality to a Python class.  This section will list the ones we use.
For more details on how function arguments work, see the relevant section further down.
This section assumes you have \mintinline{cpp}{c} defined as in subsection~\ref{subsec: new class}.  For short
we will say that the C++ class \mintinline{cpp}{c} represents is \mintinline{cpp}{MyClass}.

Note that this subsection will only deal with non-static modifications.  These must always have
the class itself as the first argument, just as Python.  All of these options have static
versions as well, where the only difference is that there is no first property.  The exception here
is the \mintinline{cpp}{py::init} function, which does not require an object already but must return it.

Several of these methods can be modified for documentation and other purposes.
These are detailed in the function subsection below.  Also in this subsection,
there will be more advanced variable usage demonstrated.

\subsubsection{Adding a Property}
Properties are added as
\begin{minted}{cpp}
c.def_property("property_name", reading, writing);
\end{minted}
This property is now available as a normal Python property in all instances of \mintinline{cpp}{c} in Python.
Here \mintinline{cpp}{reading} and \mintinline{cpp}{writing} are instances of \mintinline{cpp}{py::cpp_function}.  This class
can be constructed from a simple lambda function or be allowed more arguments.  Note that if \mintinline{cpp}{reading}
is constructed from a lambda function, it will return a copy of the object.  It might be better to allow it to return
a reference to the object.  This can be done  by changing the return value policy as
\begin{minted}{cpp}
c.def_property("property_name", py::cpp_function(reading, 
    py::return_value_policy::reference_internal), writing);
\end{minted}
Here, \mintinline{cpp}{reading} is again just a lambda function that returns a reference. The 
\mintinline{cpp}{writing} function should not return anything.

Syntax for reading and returning a value:
\begin{minted}{cpp}
// lambda:
reading := [](MyClass& x) {return x.my_property_value();}

// pointer:
reading := &MyClass::my_property_value
\end{minted}

Syntax for reading and returning a reference to a value:
\begin{minted}{cpp}
// lambda:
reading := py::cpp_function([](MyClass& x) -> 
  std::remove_cv_t<std::add_lvalue_reference_t<
  	decltype(x.my_property_value())>> {
    return x.my_property_value();
  }, py::return_value_policy::reference_internal)

// pointer (if x.my_property_value() is a non-constant reference):
reading := py::cpp_function(&MyClass::my_property_value,
    py::return_value_policy::reference_internal)
\end{minted}
Note that the long term to the right of the right arrow (\mintinline{cpp}{->}) is just C++ type-trait deduction to
enforce that the return type has to be a non-constant reference.  This is mostly
a compile time test in case at any point the implementation of \mintinline{cpp}{MyClass::my_property_value}
changes.  It does not need to be there.  Unless you have an identically named \mintinline{cpp}{x.my_property_value()}
that also returns a constant reference.

The writing lambda should be defined as
\begin{minted}{cpp}
// lambda (if x.my_property_value() is a reference):
writing := [](MyClass& x, decltype(x.my_property_value()) y) {
    x.set_my_property_value() = y;
}

// lambda (if x.set_my_property_value(y) sets the value):
writing := [](MyClass& x, decltype(x.my_property_value()) y) {
    x.set_my_property_value(y);
}

// pointer (if x.set_my_property_value(y) sets the value):
writing := &MyClass::set_my_property_value
\end{minted}
It might be worth here to play around with the input type.  Some input
is better as copies with a corresponding \mintinline{cpp}{std::move}, others might need
a constant reference instead.

Note that properties are sometimes weird workarounds around for C++ classes that provides get- and set-functions
instead of exposing member variables.  If you really want to have properties on your class, it might be worth
the effort to modify the original ARTS class to expose its members variables in public instead.  Why would you
allow direct manipulation of data this way in Python but not in C++ if the setter does not perform any additional
logic or checks on the set value?

\subsubsection{Adding a modifiable value}
Any publically exposed variable of a class can be read from or written to in Python using

\begin{minted}{cpp}
c.def_readwrite("readwrite_name", &MyClass::value)
\end{minted}
As for with properties, you can also add a return value policy if you wish the reading
to not return a copy.  This is done as:
\begin{minted}{cpp}
c.def_readwrite("readwrite_name", &MyClass::value,
    py::return_value_policy::reference_internal)
\end{minted}
Note that from Python, this will look exactly like a standard property.

\subsubsection{Adding a function}
A function is defined as
\begin{minted}{cpp}
// lambda:
c.def("function_name", [](MyClass& f, ...) {
    return f.function(...); 
});

// pointer:
c.def("function_name", &MyClass::function);
\end{minted}
where the ellipsis indicates any number of arguments in the first example.  Of course,
you have to match this to the number of arguments that are actually required.

\subsubsection{Helper macros}
The file \verb|py_macros.h| contains several helper macros.  They all look like \mintinline{cpp}{PythonInterface*}.
Their use is not required but it is recommended to make use of them if you want to reproduce functionality.  This document will not go
through these macros at all.  It is advised to copy their usage from already defined classes.

\subsection{Using previously declared options}
It is possible to use previously defined functions by passing the class itself into C++ as a \mintinline{cpp}{py::object&}.
For instance:
\begin{minted}{cpp}
c.def("call_function_name_with_readwrite_name",\
    [](py::object& self) {
    return self.attr("function_name")(
        self.attr("readwrite_name"));
})
\end{minted}
will call the function defined above with the read-write property also defined above.

\subsection{Monkey patching}
Monkey patching is when a functionality is added to a class at runtime.  We need to make use of this to simplify features such as
conversions to \mintinline{python}{xarray} or to make use of \mintinline{python}{numpy} or \mintinline{python}{scipy} when deemed necessary.  If possible, this should be
avoided as far as possible because it has several drawbacks, one of which is that it is relatively slow.

Anyways, the solution for monkey patching we have gone with in ARTS is to retain static function pointers that are overwritten
when \mintinline{python}{pyarts} is imported and, crucially, reset when \mintinline{python}{pyarts} is destroyed.  The latter is important to not create
segmentation faults.

In C++, the following code is required as one of the functions defined for the \mintinline{cpp}{MyClass} interface:
\begin{minted}{cpp}
c.def("monkey_function", [](py::object& self, ...) {
    return details::MyClass::monkey_function(...);
})
\end{minted}
This will call a function that must have been defined statically inside a struct called \mintinline{cpp}{MyClass} that's part of the \mintinline{cpp}{details} namespace.
Outside the function that is declaring the \mintinline{cpp}{MyClass} interface but still within the Python namespace, this struct may be defined as:
\begin{minted}{cpp}
namespace details {
struct MyClass {
    inline static std::function<py::object(py::object&, ...)>
      monkey_function{
        [](py::object&, ...) {
            throw std::logic_error("Not implemented");
            return py::none();
        }
    };
};
}  // namespace details
\end{minted}
Note that we provide the \verb|details.h| file to help this, so in case the function takes two arguments, you can write
\begin{minted}{cpp}
namespace details {
struct MyClass {
    inline static auto monkey_function{two_args};
};
}  // namespace details
\end{minted}
Now again inside the function we need to define a way to overwrite this from Python and to cleanly destroy it.
To overwrite the function, we consider it best practice to hide the class from normal users by defining it as
\begin{minted}{cpp}
py::class_<details::MyClass>(m, "MyClass::details")
    .def_readwrite_static("monkey_function",
                        &details::MyClass::monkey_function);
\end{minted}
This can only be overwritten in Python using the \mintinline{python}{getattr} built-in function such as
\begin{minted}{python}
import pyarts.pyarts_cpp as cxx
def print_two(a, b): print(a, b)
getattr(cxx, "MyClass::details").monkey_function = print_two
\end{minted}
The importance to actually destroy the variable upon deleting \mintinline{python}{pyarts} also exist.  We can do this as
\begin{minted}{cpp}
m.add_object("_cleanupMyClass", py::capsule([]() {
                details::MyClass::monkey_function =
                  details::two_args;
            }));
\end{minted}
The trick here is that there will exist a \mintinline{cpp}{"_cleanupMyClass"} on the \mintinline{python}{pyarts.pyarts_cpp} namespace.
This is ugly and technically users can overwrite it.  However, it is guaranteed that when a module (like \mintinline{python}{pyarts})
is destroyed, it will first destroy all variables beginning with a single underscore.  Thus the first thing that will
happen is that the \mintinline{cpp}{details::MyClass} struct will be restored to its original state and it will not have dangling
pointers when the Python function \mintinline{cpp}{print_two} is destroyed.

Note: do not define any monkey patch Python function with a name containing a leading underscore.  This might lead to
segmentation faults when closing down \mintinline{python}{pyarts}.

\section{Workspace and Agenda}
The workspace and agenda are special classes in ARTS.  An agenda can only exist on a workspace but the linkage to that workspace 
is implicit and not explicit.

Normally, there should be no problems adding new types and functions to the workspace within the scope of the current implementations.  The file \verb|gen_auto_py.cpp| will generate the interface for you.

Note that each new workspace group has to define a large set of functionality defined by the \verb|TestClassesBasic.py| continuous integration test.

\section{Functions}
Functions can be defined on the Python module in C++.
Given the module \mintinline{cpp}{m} as used above, this is done with
\begin{minted}{cpp}
// lambda:
m.def("module_level_function",
    [](...) {return module_level_function(...);});

// pointer:
m.def("module_level_function", &module_level_function);
\end{minted}
Here again the ellipsis means any number of arguments.
This is pretty much identical to how functions are defined for classes, but the first variable does not have to be a reference to \mintinline{python}{self} or \mintinline{cpp}{this}.

Note again that all the classes and types that this function accept must have been defined either as automatically convertible from standard Python types, or as classes or enums in the way detailed above.

The main point of this section is to go over some of the additional options that can be given to the function definition to change and modify their behaviour.

\subsection{Keep alive}
This is defined as
\begin{minted}{cpp}
// lambda:
m.def("module_level_function",
    [](...) {return module_level_function(...);},
    py::keep_alive<Nurse, Patient>());

pointer:
m.def("module_level_function", &module_level_function,
    py::keep_alive<Nurse, Patient>());
\end{minted}
where the nurse and the patient are indices indicating the value which keeps another value alive.  (The nurse keeps the patient alive.)
The indices here are important to keep track of.  An index of 0 means the function's return value.  An index of 1 means the first input of
the function (for classes, this means the object itself). 2, 3, 4, and so on is thus the number of the argument as passed to the function.

The main effect of keeping an object alive with the help of another is that it stops you from having dangling references.
Most of the time you do not want pure functions returning references, but it is common practice for classes.  For instance,
the \mintinline{cmake}{matpack} types can all return a non-owning \mintinline{python}{numpy} array using the \mintinline{cpp}{value} property.  This array is not
a reference to any value inside the type, yet the \mintinline{python}{numpy} array needs the \mintinline{cmake}{matpack} type to remain alive for as long
as it exists.  It thus sets \mintinline{cpp}{py::keep_alive<0, 1>()} so that the return value of the \mintinline{cpp}{value} getter keeps the main
object alive for as long as it exists.

\subsection{Return value policy}
This is defined as 
\begin{minted}{cpp}
// lambda:
m.def("module_level_function",
    [](...) {return module_level_function(...);},
    py::return_value_policy::*);

// pointer:
m.def("module_level_function", &module_level_function,
    py::return_value_policy::*);
\end{minted}
where the \mintinline{cpp}{*} is replaced by the policy.  This determines how Python views the ownership of the returned object.  For instance,
the policy: \mintinline{cpp}{reference_internal} is used by all array access getters. This ensures that the returned Python object keeps the
array alive and that also that the Python object is not the owner of itself.  The difference with pure keep alive is small but still there.

\subsection{Function arguments}
This is defined as:
\begin{minted}{cpp}
// lambda:
m.def("module_level_function",
    [](...) {return module_level_function(...);},
    py::arg("val"),
    py::arg("foo")=1,
    py::arg_v("bar", nullptr, "None"),
    py::arg_v("foobar", 5, "5").noconvert(),
    py::arg_v("valfoo", 5.0, "4").none(false),
    py::arg_v("valbar", Vector(5),
      "JustAVector").noconvert().none(false)
);

// pointer:
m.def("module_level_function", &module_level_function,
    py::arg("val"),
    py::arg("foo")=1,
    py::arg_v("bar", nullptr, "None"),
    py::arg_v("foobar", 5, "5").noconvert(),
    py::arg_v("valfoo", 5.0, "4").none(false),
    py::arg_v("valbar", Vector(5),
      "JustAVector").noconvert().none(false)
);
\end{minted}
for a function that takes 6 arguments.
\begin{itemize}
\item The first argument will be named \mintinline{cpp}{"val"}.  It has have no default value.  The documentation will give no default.
\item The second argument will be named \mintinline{cpp}{"foo"}.  It has the default value of 1.  The documentation will give an automatic default format.
\item The third argument will be named \mintinline{cpp}{"bar"}.  It has the default value of \mintinline{python}{None} and the documentation text will say so.
\item The fourth argument will be named \mintinline{cpp}{"foobar"}.  It has the default value of 5 and the documentation text will say so.  It cannot be implicitly converted to the type it has but must have an exactly matching type to whatever \mintinline{cpp}{module_level_function} takes as a fourth argument.
\item The fifth argument will be names \mintinline{cpp}{"valfoo"}.  It has the default value 5.0 but the documentation will say it is 4.  It cannot be set from \mintinline{python}{None}.
\item The sixth argument will be names \mintinline{cpp}{"valbar"}.  It has the default value of \mintinline{cpp}{Vector(5)} but the documentation will say it is \mintinline{python}{JustAVector}.  It cannot be \mintinline{python}{None}.  It is also not possible to set the sixth argument from any other type than whatever \mintinline{cpp}{module_level_function} takes as a sixth argument.
\end{itemize}

\subsubsection{Variant}
The C++ interface allows using \mintinline{cpp}{std::variant} to represent that a function can have more than one type as input.
You are able to do this:
\begin{minted}{cpp}
m.def("varfun", [](std::variant<Numeric_,
                                Index_,
                                Vector> x) {...} );
\end{minted}
or this
\begin{minted}{cpp}
m.def("varfun", [](Numeric_ x) {...} );
m.def("varfun", [](Index_   x) {...} );
m.def("varfun", [](Vector   x) {...} );
\end{minted}
to state this in C++.  The main difference is that the latter will generate a lot more documentation.

Note that there is a small problem with the functions above.  This is what will happen in both cases
\begin{minted}{python}
import pyarts.pyarts_cpp as cxx
cxx.varfun(1)  # calls m.def("varfun", [](Numeric_ x) ...
cxx.varfun(cxx.Index(1))  # calls m.def("varfun", [](Index_   x) ...
\end{minted}
The reason is that the constant 1 is a Python \mintinline{python}{int}.  Since we allow the numerical
class of ARTS to be created from integers, and since it is declared before the integer version of the function, the 
numerical call to the function has precedence.  To call the integer version of the function, you
have to place it above or match its type exactly.  The order of resolution for function calls with overloads are:
\begin{enumerate}
\item Check if an exact overload for this type exists and use it.
\item Try to convert the input to one of the overloads starting from the first declared overload and working through the rest until any such conversion works.  Call that function.
\end{enumerate}

\subsubsection{Optional}
Values can be made optional using
\begin{minted}{cpp}
m.def("optfun", [](std::optional<Numeric> x) {...} );
\end{minted}
This allows calling the function with \mintinline{python}{None} as input, which will simply leave \mintinline{cpp}{x} valueless.
\subsubsection{Documentation}
To add documentation to a function do this
\begin{minted}{cpp}
// lambda:
m.def("module_level_function",
    [](...) {return module_level_function(...);},
    py::doc("..."));

// pointer:
m.def("module_level_function", &module_level_function,
    py::doc("..."));
\end{minted}
The function will now have the ellipsis as its Python documentation.  Please follow best practices for Python documentation while writing these strings.  Also
note that it is very convenient to just
\begin{minted}{cpp}
m.def("module_level_function",
    [](...) {return module_level_function(...);},
    py::doc(R"--(Write multi-line comments

using a raw string.  Even " will appear correct!
)--"));
\end{minted}
