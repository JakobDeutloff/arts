\chapter{Refractive index}
 \label{sec:rindex}

\starthistory
  120918 & Started (Patrick Eriksson).\\
\stophistory


Refractive index\index{refractive index} \dots

\FIXME{Write a proper introduction. Comment on that this is not the complex
refractive index, also covering absorption.} 


\section{Monochromatic and group refractive index}
%
The corresponding workspace variables are \wsvindex{refr\_index} and
\wsvindex{refr\_index\_group}. If they differ, there is
dispersion\index{dispersion}. The monochromatic refractive index is below
denoted as \Rfr, while the group one is written as $\Rfr_{g}$.

\FIXME{Describe where each variable is used. Expand \dots} 



\section{Gases}
%
The following workspace methods are at hand for calculating the summed
refractive index of atmospheric gases: \wsmindex{refr\_indexThayer} and
\wsmindex{refr\_indexIR}. \FIXME{Expand \dots} 


\section{Free electrons}
%
Free electrons, as exist in the ionosphere, will affect propagating radio waves
in several ways. Free electrons will have an impact of the propagation speed of
radio waves, hence a signal can be delayed and refracted. This section assumes
that the magnetic field is zero, for Faraday rotation see Section \FIXME{add
  reference when this section exists}.

An electromagnetic wave passing through a plasma (such as the ionosphere) will
drive electrons to oscillate and re-radiating the wave frequency. This is the
basic reason of the contribution of electrons to the refractive index. 
An important variable is the plasma frequency, $\Frq_{p}$:
\begin{equation}
  \omega_{p}=\sqrt{\frac{Ne^{2}}{\epsilon_{0}m}},
\end{equation}
where \(\omega_{p}=2\pi\Frq_{p}\), \(N\) is the electron density, \(e\) is the
charge of an electron, \(\epsilon_{0}\) is the permittivity of free space, and
\(m\) is the mass of an electron. For example, for the Earth's ionosphere
\(\Frq_{p}\) \(\approx\) 9\,MHz.
Waves having a frequency below $\Frq_{p}$ are reflected by a plasma.

Neglecting influences of any magnetic field, the refractive index of a plasma
is \citep[e.g.][]{rybicki:radia:79}
\begin{equation}
\label{eq:n:electrons}
\Rfr =\sqrt{1-\frac{\omega_{p}^{2}}{\omega^{2}}}=\sqrt{1-\frac{Ne^{2}}
{\epsilon_{0}m\omega^{2}}}.
\end{equation}
where $\omega$ is the angular frequency ($\omega=2\pi\Frq$). This refractive
index is less than unity (phase velocity is greater than the speed of light),
but is approaching unity with increasing frequency. The group velocity is
\citep{rybicki:radia:79}
\begin{equation}
\aSpd{g}=\speedoflight\sqrt{1-\frac{Ne^{2}}
{\epsilon_{0}m\omega^{2}}}
\end{equation}
which is clearly less than the speed of light.
The energy (or information) of a signal propagating through the ionosphere
travels with the group velocity, and the group speed refractive
index (\(\Rfr_{g}=\frac{\speedoflight}{\aSpd{g}}\)) is
\begin{equation}
\label{eq:ng:electrons}
  \Rfr_{g}=\left(
    1-\frac{Ne^{2}}
    {\epsilon_{0}m\omega^{2}}
  \right)^{-1/2}.
\end{equation}
Equations~\ref{eq:n:electrons} and \ref{eq:ng:electrons} are implemented in
\wsmindex{refr\_indexFreeElectrons}. The method demans that the radative
transfer frequency is at least twice the plasma frequency.


