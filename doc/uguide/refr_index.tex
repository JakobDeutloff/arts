\chapter{Refractive index}
 \label{sec:rindex}

\starthistory
  120918 & Started (Patrick Eriksson).\\
\stophistory

\FIXME{Write a proper introduction. Comment on that this is not the complex
refractive index, also covering absorption.} \dots

Refractive index\index{refractive index} (here restricted to the real part of
the refractive index, which is basically a complex quantity with the imaginary
part expressing absorption) describes several effects of matter on propagation
of electromagnetic waves. This particularly includes changes of the propagation
speed of electromagnetic waves, which leads to a delay of the signal as well as
a change of the propagation direction, a bending of the propagation path. The
latter is commonly called refraction.

Several components in the atmosphere contribute to refraction, hence to the
refractive index: the gas mixture(``air''), solid and liquid constituents
(clouds, precipitation, aerosols), and electrons. ARTS includes mechanisms for
deriving the contributions from gases and electrons with the available methods
described in the Sections below. ARTS does not consider
refraction by solid and liquid particle as their refractive index is not known
to ARTS (when scattering is considered, ARTS gets their single scattering
properties as input, see Chapter~\ref{sec:clouds}). However, the contribution of
solid and liquid constituents to the refractive index is commonly neglected in
radiative transfer models and is expected to have only small effects.

Refractivity ($N$) describes the deviation of the refractive index of a medium
\Rfr from the vacuum refractive index (\Rfr$_\mathrm{vacuum}=1$): $N=\Rfr-1$.
Contributions of the different components to refractivity are additive.
Therefore, all ARTS methods that provide refractive index, calculate
refractivity and sum it up with the input refractive index.

Within ARTS, refractive index is required for calculations of refracted
propagation paths and related parameters (e.g., deriving viewing angle for a
given tangent altitude). \FIXME{for more?}


%\section{Monochromatic and group refractive index}

Whenever refractive index is required, e.g., at each point along a propagation
path, it is evaluated according to the mechanism specified by
\wsaindex{refr\_index\_air\_agenda}. \wsaindex{refr\_index\_air\_agenda}
provides both the monochromatic refractive index \wsvindex{refr\_index\_air},
in the following denoted as \Rfr, as well as the group refractive index
\wsvindex{refr\_index\_air\_group}, denoted as \Rfr$_{g}$.
\wsvindex{refr\_index\_air} differs from \wsvindex{refr\_index\_air\_group}
in case of dispersion\index{dispersion}, which e.g. leads to diverging
propagation paths at different frequencies.

\FIXME{Describe where each variable is used. Expand \dots}

\section{Gases}
%
For calculating the contribution to the refractive index from atmospheric
gases, the following workspace methods are currently available in ARTS:
\wsmindex{refr\_index\_airMWgeneral}, \wsmindex{refr\_index\_airMicrowaves} and
\wsmindex{refr\_index\_airIR}. All of them are non-dispersive, i.e.,
monochromatic and group refractive index are identical. They are supposed to be
applied as alternatives, not in addition to each other.

\wsmindex{refr\_index\_airMWgeneral} provides refractivity due to different gas
mixtures as occuring in planetary atmospheres and is valid in the microwave
spectral region. It uses the methodology introduced by
\citet{newell65:_absolute_jap} for calculating refractivity of the gas mixture
at actual pressure and temperature conditions based on the refractivity of the
individual gases at reference conditions. Reference refractivities from
\citet{newell65:_absolute_jap} are available for N$_2$, O$_2$, CO$_2$, H$_2$,
and He.  Additionally, reference refractivity for H$_2$O has been derived from
H$_2$O contribution as described by \wsmindex{refr\_index\_airMicrowaves} (see
below) for a reference temperature of $T_0$=273.15\,K\footnote{Reducing the
\wsmindex{refr\_index\_airMicrowaves} approach to inverse temperature
proportionality as applied by \wsmindex{refr\_index\_airMWgeneral} causes
significant deviations from H$_2$O refractivity from
\wsmindex{refr\_index\_airMicrowaves}. However, they are smaller than when
refraction by H$_2$O is not accounted for.}.
Any mixture of these gases can be taken into account. The missing
contribution from further gases is roughly accounted for by
normalising the calculated refractivity from the six reference gases to a
volume mixing ratio of 1. More details on the applied formulas are given in
Section~\ref{T-sec:rindex:mwgeneral} of \theory.
%Chapter~\ref{T-sec:ppaththeory} of \theory

\wsmindex{refr\_index\_airMicrowaves} calculates the microwave ``air''
refractivity in the Earth's atmosphere taking into account refractivity of
``dry air'' and water vapour. All other gases are assumed to have a negligible
contribution. 

\wsmindex{refr\_index\_airIR} derives the infrared ``air'' refractivity in the
Earth's atmosphere considering only refractivity of ``dry air''. 

\section{Free electrons}
 \label{sec:rindex:freee}
%
Free electrons, as exist in the ionosphere, affect propagating radio waves
in several ways. Free electrons will have an impact of the propagation speed of
radio waves, hence a signal can be delayed and refracted. This section
consideres only the refraction effect (neglecting influences of any magnetic
field). For effects on polarisation state of the
waves in presence of a static magnetic field, i.e., Faraday rotation, see
Section~\ref{sec:faraday}.

\wsmindex{refr\_index\_airFreeElectrons} derives this contribution of free
electrons to the refractive index. The method is only valid when the radiative
transfer frequency is large enough (at least twice the plasma frequency).
Information on theoretical background and details on the applied formulas are
provided in Section~\ref{T-sec:rindex:freee} of \theory.


