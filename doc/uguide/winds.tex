\chapter{Doppler effects and winds}
 \label{sec:winds}


 \starthistory
 121218 & Written by Patrick Eriksson.\\
 \stophistory

%\graphicspath{{Figs/rte/}}

 The default assumption in ARTS can be expressed as the atmosphere is assumed
 to be static and the observation platform is at a constant geographical during
 each merasurement, while any relative movement between the atmosphere and the
 sensor will cause Doppler effects. For the moment ARTS is handling two sources
 to Doppler shifts: winds and the planets rotation.

 Atmospheric transport is not considered by ARTS, but winds can still be of
 importance due to the \textindex{Doppler effect} they can cause. This effect
 is most significant at high altitudes where the line shape is narrow, and a
 frequency shift of absorption and emission is most easily discerned. The
 effect depends also on the angle between the wind vector and the
 line-of-sight.

 ARTS applies an Earth-Centred, Earth-Fixed, (ECEF) coordinate system. This
 implies that the receiver follows the planet's rotation. If, for example, the
 sensor is placed on another planet, or is in a transit orbit, the rotation of
 the observed planet will cause Doppler effects. This is treated in ARTS by
 short-cut, the rotational movement can be translated to an imaginary wind by
 the method \wsmindex{wind\_u\_fieldIncludePlanetRotation}.


\section{Definition of winds}
%==============================================================================
\label{sec:winds:defs}

\subsection{Needed input}
%
The workspace variables to specify winds are \wsvindex{wind\_u\_field},
\wsvindex{wind\_v\_field} and \wsvindex{wind\_w\_field}, below denoted as
\WindWE, \WindSN\ and \WindVe, respectively. The user need to set all these
three variables. It is allowed for all three wind components to set the
variable to be empty, which is shorthand for saying that the wind is zero
throughout the atmosphere. Otherwise, the size of the variable is required to
match the atmospheric grids. 

No further input is required, a Doppler shift is added as soon as any of the
winds is non-zero (with exceptions discussed in
Section~\ref{sec:winds:limitations}). For clarity, even though a setting of
\builtindoc{wind\_u\_field} always is demanded, this wind component has no
effect on Doppler shifts for 1D and 2D calculations, as the wind moves at an
angle of 90$^\circ$ from the observation plane.

The definition of the wind components is
\begin{itemize}
\item[\WindWE] The zonal wind, where a positive wind is defined as air moving
  from west to east (i.e.\ towards higher longitudes).
\item[\WindSN] The meridional wind, where a positive wind is defined as air
  moving from south to north (i.e.\ towards higher latitudes).
\item[\WindVe] The vertical wind, where a positive wind is defined as air
  moving upwards.
\end{itemize}



\section{Limitations}
\label{sec:winds:limitations}
%
For efficiency reasons, a single Doppler shift is applied for all frequencies.
That is, broadband calculations could be treated in an approximative manner. A
general (but slow) way to overcome this simplification is to use the same
mechanism as for including dispersion, as discussed in
Section~\ref{sec:fm_defs:dispersion}. If \wsmindex{iyLoopFrequencies} is
applied, the Doppler shift will be calculated for every frequency seperately.

Winds are neglected inside the DOIT and MC scattering modules.



\section{Equations}
%==============================================================================
\label{sec:winds:eqs}

The main equations for deriving the Doppler shift from the winds are given in
this section. The total wind, \Wind, is
\begin{equation}
  \Wind = \sqrt{\WindWE^2+\WindSN^2+\WindVe^2}.
  \label{eq:winds:total}
\end{equation}
The zenith angle of the wind direction is 
\begin{equation}
  \aZntAng{\Wind} = \arccos(\WindVe/\Wind),
  \label{eq:winds:za}
\end{equation}
and the azimuth angle is 
\begin{equation}
  \aAzmAng{\Wind} = \arctan(\WindWE/\WindSN). \qquad 
                           \mathrm{(implemented\ by\ the\ atan2\ function)}
  \label{eq:winds:aa}
\end{equation}
The cosine of the angle between the wind vector and the line-of-sight is
\begin{equation}
  \cos\gamma = \cos\aZntAng{\Wind}\cos\aZntAng{l} + 
               \sin\aZntAng{\Wind}\sin\aZntAng{l}
               \cos(\aAzmAng{\Wind}-\aAzmAng{\l}),
  \label{eq:winds:dang}
\end{equation}
where \aZntAng{l}\ and \aAzmAng{l}\ are the angles of the line-of-sight. 

Finally, as the winds do not reach relativistic values, the Doppler shift can
be calculated as
\begin{equation}
  \Delta\Frq = \frac{-\Wind\aFrq{0}\cos\gamma}{\speedoflight},
\end{equation}
where \aFrq{0}\ is the rest frequency and \speedoflight\ is the speed of light.
The core part of these calculations is implemented in the genral internal
function \funcindex{dotprod\_with\_los}.

In practise, \aFrq{0}\ is taken as the mean of the end values of
\builtindoc{f\_grid}. The sign of the expression can be understood by the fact
that if the wind and line-of-sight are aligned and going in the same direction,
the air movement is directly away from the sensor, that gives a shift towards
lower frequencies. As workspace variable $\Delta\Frq$ is denoted as
\wsvindex{rte\_doppler}, and is applied on \wsvindex{f\_grid} when extracting
the (non-polarised) absorption (where a single $\Delta\Frq$ is applied on the
complete \wsvindex{f\_grid}, as also mentioned in
Section~\ref{sec:winds:limitations}).

