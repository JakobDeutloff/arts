\chapter{Atmospheric winds}
 \label{sec:winds}


 \starthistory
 110621 & Created by Patrick Eriksson.\\
 \stophistory

%\graphicspath{{Figs/rte/}}

 Atmospheric transport is not considered by ARTS, but winds can still be of
 importance due to the Doppler effect they can cause. This effect is most
 significant at high altitudes where the line shape is narrow, and a frequency
 shift of absorption and emission is most easily discerned. The effect depends
 also on the angle between the wind vector and the line-of-sight.



\section{Definitions}
%==============================================================================
\label{sec:winds:defs}

The workspace variables to specify winds are \wsvindex{wind\_u\_field},
\wsvindex{wind\_v\_field} and \wsvindex{wind\_w\_field}, here denoted as
\WindWE, \WindSN\ and \WindVe, respectively. The definition of these winds
should be customary for 3D atmospheres, while the \WindSN-wind is treated
differenntly for 1D and 2D atmospheres.

\subsection{3D}
%
\begin{itemize}
\item[\WindWE] The zonal wind, where a positive wind is defined as air moving
  from west to east (i.e.\ towards higher longitudes).
\item[\WindSN] The meridional wind, where a positive wind is defined as air
  moving from south to north (i.e.\ towards higher latitudes).
\item[\WindVe] The vertical wind, where a positive wind is defined as air
  moving upwards.
\end{itemize}


\subsection{1D and 2D}
%
The \WindSN-component is assumed to be aligned horisontally with the
line-of-sight. 

 The \WindWE-component is ignored.
