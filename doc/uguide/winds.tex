\chapter{Atmospheric winds}
 \label{sec:winds}


 \starthistory
 110622 & Created by Patrick Eriksson.\\
 \stophistory

%\graphicspath{{Figs/rte/}}

 Atmospheric transport is not considered by ARTS, but winds can still be of
 importance due to the Doppler effect they can cause. This effect is most
 significant at high altitudes where the line shape is narrow, and a frequency
 shift of absorption and emission is most easily discerned. The effect depends
 also on the angle between the wind vector and the line-of-sight.



\section{Definitions and practicalities}
%==============================================================================
\label{sec:winds:defs}

\subsection{Needed input}
%
The workspace variables to specify winds are \wsvindex{wind\_u\_field},
\wsvindex{wind\_v\_field} and \wsvindex{wind\_w\_field}, below denoted as
\WindWE, \WindSN\ and \WindVe, respectively. The user need to set all these
three variables. It is allowed for all three wind components to set the
variable to be empty, which is shorthand for saying that the wind is zero
throughout the atmosphere. Otherwise, the size of the variable is required to
match the atmospheric grids. The variable \builtindoc{wind\_u\_field} is not
considered for 1D and 2D.

No further input is required, a Doppler shift is
added as soon as any of the winds is non-zero (with exceptions discussed in
Section~\ref{sec:winds:limitations}).


\subsection{Definition for 3D atmospheres}
%
The definition of the winds should be the standard one for 3D atmospheres:
\begin{itemize}
\item[\WindWE] The zonal wind, where a positive wind is defined as air moving
  from west to east (i.e.\ towards higher longitudes).
\item[\WindSN] The meridional wind, where a positive wind is defined as air
  moving from south to north (i.e.\ towards higher latitudes).
\item[\WindVe] The vertical wind, where a positive wind is defined as air
  moving upwards.
\end{itemize}


\subsection{Definition for 1D and 2D atmospheres}
%
As these atmospheric dimensionalities do not have an azimuthal dimensions, it
only makes sense to operate with a single horizontal wind component and
\builtindoc{wind\_v\_field} is used for this purpose, \builtindoc{wind\_u\_field}
is ignored. The \WindSN-component is assumed to be totally aligned azimuthally
with the line-of-sight. That is, the horizontal wind is radial, centred around
the sensor. For 1D a positive wind signifies air moving away from the sensor.
For 2D, a positive wind is air moving against higher latitudes (as for 3D).
The vertical component is defined as for 3D.
 

\subsection{Limitations}
\label{sec:winds:limitations}
%
A single Doppler shift is applied for all frequencies. That is, broadband
calculations are treated in an approximative manner. So far, non-zero winds
require that the absorption is calculated ``on-the-fly''
(Section~\ref{sec:absorption}). The scattering algorithms do not apply a
Doppler shift, i.e.\ the winds are ignored.



\section{Equations}
%==============================================================================
\label{sec:winds:eqs}

The main equations for deriving the Doppler shift from the winds are given in
this section. The total wind, \Wind, is
\begin{equation}
  \Wind = \sqrt{\WindWE^2+\WindSN^2+\WindVe^2}.
  \label{eq:winds:total}
\end{equation}
The zenith angle of the wind direction is 
\begin{equation}
  \aZntAng{\Wind} = \arccos(\WindVe/\Wind),
  \label{eq:winds:za}
\end{equation}
and the azimuth angle (for 3D only) is 
\begin{equation}
  \aAzmAng{\Wind} = \arctan(\WindWE/\WindSN). \qquad 
                           \mathrm{(implemented\ by\ the\ atan2\ function)}
  \label{eq:winds:aa}
\end{equation}
For 1D and 2D, \aAzmAng{\Wind}\ is set to 0\degree\ if $\WindSN \geq 0$,
otherwise to 180\degree. 

The cosine of the angle between the wind vector and the line-of-sight is
\begin{equation}
  \cos\gamma = \cos\aZntAng{\Wind}\cos\aZntAng{l} + 
               \sin\aZntAng{\Wind}\sin\aZntAng{l}
               \cos(\aAzmAng{\Wind}-\aAzmAng{\l}),
  \label{eq:winds:dang}
\end{equation}
where \aZntAng{l}\ and \aAzmAng{l}\ are the angles of the line-of-sight. In
this equation, the azimuth angle \aAzmAng{l}\ is treated to be 0\degree\ for 1D
and 2D with zenith angles above 0, and 180\degree for 2D with negative zenith
angles.

Finally, as the winds do not reach relativistic values, the Doppler shift can
be calculated as
\begin{equation}
  \Delta\Frq = \frac{-\Wind\aFrq{0}\cos\gamma}{\speedoflight},
\end{equation}
where \aFrq{0}\ is the rest frequency and \speedoflight\ is the speed of light.
In practise, \aFrq{0}\ is taken as the mean of the end values of
\builtindoc{f\_grid}. The sign of the expression can be understood by the fact
that if the wind and line-of-sight are aligned and going in the same direction,
the air movement is directly away from the sensor, that gives a shift towards
lower frequencies. As workspace variable $\Delta\Frq$ is denoted as
\wsvindex{doppler\_shift}, and is applied on \wsvindex{f\_grid} when extracting
the (non-polarised) absorption (where a single $\Delta\Frq$ is applied on the
complete \wsvindex{f\_grid}, as also mentioned in
Section~\ref{sec:winds:limitations}).

