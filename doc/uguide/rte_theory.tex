\graphicspath{{Figs/rte_theory/}}

%Inverse Wave Impendance
\newcommand{\ColVctTwo}[2]{\left[
    \begin{array}{c} #1\\ #2
    \end{array} \right] }

\newcommand{\ColVctFour}[4]{\left[
    \begin{array}{c} #1\\ #2 \\ #3 \\ #4
    \end{array} \right] }

\chapter{Basic radiative transfer theory}
 \label{sec:rte_theory}


 \starthistory 
 211122 & Add Section about direct sources. \\
 120924 & Added discussion of the n2-law, mainly using text
          originally written for an ESA report by Bengt Rydberg (PE). \\
 110615 & Revised, and moved parts about surface from AUG (PE). \\
 050224 & Most text replaced by chapter 1 from Claudia Emde's phd-thesis. \\
 030305 & Copied from a compendium written by Patrick Eriksson. \stophistory


 When dealing with atmospheric radiation a division can be made
 between two different wavelength ranges where the limit is found
 around 5 $\mu$m, i.e.\ one range consists of the near IR, visible and UV
 regions while the second range covers thermal and far IR and
 microwaves. The first reason to this division is the principal
 sources to the radiation in the two ranges, for wavelengths shorter
 than 5 $\mu$m the solar radiation is dominating while at longer
 wavelengths the thermal emission from the surface and the atmosphere
 is more important. A second reason is the importance of scattering
 but here it is impossible to give a fixed limit. Clouds are important
 scattering objects for most frequencies but at cloud free conditions
 scattering can in many cases be neglected for wavelengths $>$ 5 $\mu$m. If
 the atmosphere can be assumed to be in local thermodynamic
 equilibrium the radiative transfer can be simplified considerably,
 and this is a valid assumption for the IR region and microwaves but
 not for e.g. UV frequencies.
 
 The radiative transfer in the atmosphere must be adequately described
 in many situations, as when estimating rates of photochemical
 reactions, calculating radiative forcing in the atmosphere or
 evaluating a remote sensing observation. It is not totally
 straightforward to quantify the radiative transfer with good accuracy
 because the calculations can be very computationally demanding and
 many of the parameters needed are hard to determine. For example,
 situations when a great number of transitions or multiple scattering
 must be considered will cause long calculations while as a rule
 scattering is difficult to model because the shape and size
 distribution of the scattering particles are highly variable
 quantities.  

 This chapter introduces the theoretical background which is essential
 to develop a radiative transfer model including scattering. The theory
 is based on concepts of electrodynamics, starting from the Maxwell
 equations.  An elementary book for electrodynamics is written by
 \citet{jackson98:_class}.  For optics and scattering of radiation by
 small particles the reader may refer for instance to
 \citet{hulst57:_light_scatt_small} and \citet{bohren:98}. The notation
 used in this chapter is mostly adapted from the book ``Scattering,
 Absorption, and Emission of Light by Small Particles'' by
 \citet{Mishchenko:02}. Several lengthy derivations of formulas, which
 are not shown in detail here, can also be found in this book. The
 purpose of this chapter is to provide definitions and give ideas, how
 these definitions can be derived using principles of electromagnetic
 theory. For the derivation of the radiative transfer equation an
 outline of the traditional phenomenological approach is given.
 
 \section{Basic definitions}
 \label{sec:rtetheory:theory_basics}
 
 From the Maxwell equations one can derive the formula for the
electromagnetic field vector $\VctStl{E}$ of a plane electromagnetic
wave propagating in a homogeneous medium without sources:
\begin{equation}
  \VctStl{E}(\VctStl{r}, t) =
  \VctStl{E}_0
  \exp\left(-\frac{\omega}{c} m_{\mathrm{I}} \hat{\PDir} \cdot \VctStl{r}\right)
  \exp\left(i \frac{\omega}{c} m_{\mathrm{R}} \hat{\PDir} \cdot \VctStl{r}  - i \omega t \right),
\end{equation}
where $\VctStl{E}_0$ is the amplitude of the electromagnetic wave in
vacuum, $c$ is the speed of light in vacuum, $\omega$ is the angular
frequency, $\VctStl{r}$ is the position vector and $\hat{\PDir}$
is a real unit vector in the direction of propagation. The complex
refractive index $m$ is
\begin{equation}
  m = m_{\mathrm{R}}+ i m_{\mathrm{I}} = c \sqrt{\epsilon \mu},
\end{equation}
where $m_{\mathrm{R}}$ is the non-negative real part and $m_{\mathrm{I}}$ is the non-negative imaginary part. Furthermore $\mu$ is the
permeability of the medium and $\epsilon$ the permittivity.  For a
vacuum, $m = m_{\mathrm{R}} = 1$.  The imaginary part of the refractive
index, if it is non-zero, determines the decay of the amplitude of the
wave as it propagates through the medium, which is thus absorbing.
The real part determines the phase velocity $v = c/m_{\mathrm{R}}$.
The time-averaged Poynting vector $\VctStl{P}(\VctStl{r})$, which describes the
flow of electromagnetic energy, is defined as
\begin{equation}
   \VctStl{P}(\VctStl{r}) =
     \frac{1}{2} \Re \bigl(\EnsAvr{\VctStl{E}(\VctStl{r})} \times
     \EnsAvr{\VctStl{H}^*(\VctStl{r})}\bigr),
\end{equation}
where $\VctStl{H}$ is the magnetic field vector and the $*$ denotes the
complex conjugate. The Poynting vector for a homogeneous wave is given
by
\begin{equation}
\label{eq:rtetheory:Poynting_vec}
  \EnsAvr{\VctStl{P}(\VctStl{r})} =
   \frac{1}{2}\Re \left( \sqrt{\frac{\epsilon}{\mu}} \right)
   \Abs{\VctStl{E}_0}^2
   \exp\left(-2\frac{\omega}{c} m_{\mathrm{I}} \hat{\PDir} \cdot
     \VctStl{r}\right) \hat{\PDir}.
\end{equation}
Equation \ref{eq:rtetheory:Poynting_vec} shows that the energy flows in the
direction of propagation and its absolute value $I(\VctStl{r}) =
\Abs{\EnsAvr{\VctStl{P}(\VctStl{r})}}$, which is usually called
intensity (or irradiance), is exponentially attenuated. Rewriting
Equation \ref{eq:rtetheory:Poynting_vec} gives
\begin{equation}
  I(\VctStl{r}) = I_0 \exp(-\alpha^p \hat{\PDir} \cdot \VctStl{r}),
\end{equation}
where $I_0$ is the intensity for $\VctStl{r} = \VctStl{0}$. The absorption
coefficient $\alpha^p$ is
\begin{equation}
  \alpha^p =
  2 \frac{\omega}{c} m_{\mathrm{I}} =
    \frac{4\pi m_{\mathrm{I}}}{\lambda} =
    \frac{4\pi m_{\mathrm{I}}\nu}{c},
\end{equation}
where $\lambda$ is the free-space wavelength and $\nu$ the frequency.
Intensity has the dimension of monochromatic flux
[$\mathrm{energy}/(\mathrm{area}\times\mathrm{time})$].
  

% \section[The Stokes parameters]{The Stokes parameters}
% \label{sec:rtetheory:Stokes_components}
% Sensors usually do not measure directly the electric and the magnetic
% fields associated with a beam of radiation. They measure quantities
% that are time averages of real-valued linear combinations of products
% of field vector components and have the dimension of intensity.
% Examples of such observable quantities are the Stokes parameters.
% Figure \ref{fig:RT_theory_coordinates} shows the coordinate system used
% to describe the direction of propagation $\hat{\PDir}$ and the
% polarization state of a plane electromagnetic wave.
% The unit vector $\hat{\PDir}$ can equivalently be described by a
% couplet $(\ZntAng, \AzmAng)$, where ${\ZntAng \in [0,\pi]}$ is the polar
% (zenith) angle and $\AzmAng \in [0,2\pi)$ is the azimuth angle. The
% electric field at the observation point is given by $\VctStl{E} =
% \VctStl{E}_\ZntAng + \VctStl{E}_\AzmAng$, where $ \VctStl{E}_\ZntAng$ and
% $\VctStl{E}_\AzmAng$ are the $\ZntAng$- and $\AzmAng$-components of the
% electric field vector.  $\VctStl{E}_\ZntAng$ lies in the meridional
% plane, which is the plane through $\hat{\PDir}$ and the $z$-axis, and
% $\VctStl{E}_\AzmAng$ is perpendicular to this plane.
% The Stokes parameters are defined as linear combinations of products
% of the amplitudes $\VctStl{E}_\ZntAng$ and  $\VctStl{E}_\AzmAng$ which
% form the $4\times1$ column vector \StoVec,
% which is known as the Stokes vector.  Since the Stokes parameters are
% real-valued and have the dimension of intensity, they can be measured
% directly with suitable instruments. The Stokes parameters are a
% complete set of quantities needed to characterize a plane
% electromagnetic wave. They carry information of the complex amplitudes
% and the phase difference.  The first Stokes parameter $I$ is the
% intensity and the other components $Q$, $U$ and $V$ describe the
% polarization state of the wave. For a detailed definition of the
% Stokes parameters and how they can be measured refer to
% Section \ref{sec:polarization}.


\section[Single particle scattering]
{Scattering, absorption and thermal emission by a single particle}
\label{sec:rtetheory:theory_single_part}

A parallel monochromatic beam of electromagnetic radiation propagates
in vacuum without any change in its intensity or polarization state. A
small particle, which is interposed into the beam, can cause several
effects:
\begin{description}
\item[Absorption:] The particle converts some of the energy
  contained in the beam into other forms of energy.
\item[Elastic scattering:] Part of the incident energy is
  extracted from the beam and scattered into all spatial directions at
  the frequency of the incident beam. Scattering can change the
  polarization state of the radiation.
\item[Inelastic scattering:] As above, but the frequency is changed by the
  scattering. This process is neglected below.
\item[Extinction:] The energy of the incident beam is reduced by
  an amount equal to the sum of absorption and scattering.
\item[Dichroism:] The change of the polarization state of the beam
  as it passes a particle.
\item[Thermal emission:] If the temperature of the particle is
  non-zero, the particle emits radiation in all directions over a
  large frequency range.
\end{description}
The beam is an oscillating plane magnetic wave, whereas the particle
can be described as an aggregation of a large number of discrete
elementary electric charges. The incident wave excites the charges to
oscillate with the same frequency and thereby radiate secondary
electromagnetic waves. The superposition of these waves gives the
total elastically scattered field.

One can also describe the particle as an object with a refractive
index different from that of the surrounding medium. The presence of
such an object changes the electromagnetic field that would otherwise
exist in an unbounded homogeneous space. The difference of the total
field in the presence of the object can be thought of as the field
\emph{scattered} by the object. The angular distribution and the
polarization of the scattered field depend on the characteristics of
the incident field as well as on the properties of the object as its
size relative to the wavelength and its shape, composition and
orientation.

\subsection{Definition of the amplitude matrix}
\label{sec:rtetheory:amp_mat}

For the derivation of a relation between the incident and the
scattered electric field we consider a finite scattering object in the
form of a single body or a fixed aggregate embedded in an infinite
homogeneous, isotropic and non-absorbing medium. We assume that the
individual bodies forming the scattering object are sufficiently large
that they can be characterized by optical constants appropriate to
bulk matter, not to optical constants appropriate for single atoms or
molecules. Solving the Maxwell equations for the internal volume,
which is the interior of the scattering object, and the external
volume one can derive a formula, which expresses the total electric
field everywhere in space in terms of the incident field and the field
inside the scattering object. Applying the far field approximation
gives a relation between incident and scattered field, which is that
of a spherical wave.  The amplitude matrix
$\MtrStl{S}(\hat{\PDir}^{\sca},\hat{\PDir}^{\inc})$ includes this relation:
\begin{equation}
  \label{eq:rtetheory:amplitude_matrix}
  \ColVctTwo{E_\ZntAng^{\sca}(r\hat{\PDir}^{\sca})}%
         {E_\AzmAng^{\sca}(r\hat{\PDir}^{\sca})}
         = \frac{e^{ikr}}{r}\MtrStl{S}(\hat{\PDir}^{\sca},\hat{\PDir}^{\inc}) 
 \ColVctTwo{E_{0\ZntAng}^{\inc}}%
         {E_{0\AzmAng}^{\inc}}.
\end{equation}
The amplitude matrix depends on the directions of incident
$\hat{\PDir}^{\inc}$ and scattering $\hat{\PDir}^{\sca}$ as well as on size,
morphology, composition, and orientation of the scattering object with
respect to the coordinate system. The distance between the origin and
the observation point is denoted by $r$ and the wave number of the
external volume is denoted by $k$.

The amplitude matrix provides a complete description of the scattering
pattern in the far field zone. The amplitude matrix explicitly depends
on $\AzmAng^{\inc}$ and $\AzmAng^{\sca}$ even when $\ZntAng^{\inc}$ and/or
$\ZntAng^{\sca}$ equal 0 or $\pi$.

\subsection{Phase matrix}
\label{sec:rtetheory:pha_mat}

The phase matrix \PhaMat\ describes the transformation of the Stokes
vector of the incident wave into that of the scattered wave for
scattering directions away from the incidence direction
$(\hat{\PDir}^{\sca}\neq\hat{\PDir}^{\inc})$,
\begin{equation}
  \label{eq:rtetheory:phase_matrix}
  \StoVec^{\sca}(r \hat{\PDir}^{\sca}) =
    \frac{1}{r^2}\PhaMat(\hat{\PDir}^{\sca},\hat{\PDir}^{\inc}) \StoVec^{\inc}.
\end{equation}
The $4\times4$ phase matrix can be written in terms of the amplitude
matrix elements for single particles \citep{Mishchenko:02}. All
elements of the phase matrix have the dimension of area and are real.
As the amplitude matrix, the phase matrix depends on $\AzmAng^{\inc}$
and $\AzmAng^{\sca}$ even when $\ZntAng^{\inc}$ and/or $\ZntAng^{\sca}$ equal
0 or $\pi$.  In general, all 16 elements of the phase matrix are
non-zero, but they can be expressed in terms of only seven independent
real numbers. Four elements result from the moduli $\Abs{S_{ij}}$ ($i,j =
1,2$) and three from the phase-differences between $S_{ij}$.  If the
incident beam is unpolarized, i.e., $\StoVec^{\inc} = (I^{\inc},
0, 0, 0)^T$, the scattered light generally has at least one
non-zero Stokes parameter other than intensity:
\begin{eqnarray}
  I^{\sca} &= Z_{11}I^{\inc},\\
  Q^{\sca} &= Z_{21}I^{\inc},\\
  U^{\sca} &= Z_{31}I^{\inc},\\
  V^{\sca} &= Z_{41}I^{\inc}.
\end{eqnarray}
This is the phenomena is traditionally called ``polarization''. The
non-zero degree of polarization Equation \ref{eq:polarization:pol_degree}
can be written in terms of the phase matrix elements
\begin{equation}
  p = \frac{\sqrt{Z_{21}^2+Z_{31}^2+Z_{41}^2}}{Z_{11}}.
\end{equation}


\subsection{Extinction matrix}
\label{sec:rtetheory:ext_mat}
In the special case of the exact forward direction
$(\hat{\PDir}^{\sca}=\hat{\PDir}^{\inc})$ the attenuation of the
incoming radiation is described by the extinction matrix \ExtMat. In
terms of the Stokes vector we get
\begin{equation}
  \StoVec(r \hat{\PDir}^{\inc}) \Delta S =
    \StoVec^{\inc} \Delta S - \ExtMat(\hat{\PDir}^{\inc}) \StoVec^{\inc} + O(r^{-2}).
\end{equation}
Here $\Delta S$ is a surface element normal to
$\hat{\PDir}^{\inc}$.  The extinction matrix can also be expressed
explicitly in terms of the amplitude matrix. It has only seven
independent elements. Again the elements depend on $\AzmAng^{\inc}$ and
$\AzmAng^{\sca}$ even when the incident wave propagates along the
$z$-axis.

\subsection{Absorption vector}
\label{sec:rtetheory:abs_vec}
The particle also emits radiation if its temperature $T$ is above zero
Kelvin. According to Kirchhoff's law of radiation the emissivity
equals the absorptivity of a medium under thermodynamic equilibrium.
The energetic and polarization characteristics of the emitted
radiation are described by a four-component Stokes emission column
vector $\AbsVec(\hat{\VctStl{r}}, T, \omega)$. The emission vector is
defined in such a way that the net rate, at which the emitted energy
crosses a surface element $\Delta S$ normal to $\hat{\VctStl{r}}$ at
distance $r$ from the particle at frequencies from $\omega$ to $\omega
+ \Delta\omega$, is
\begin{equation}
  W^e = \frac{1}{r^2}\AbsVec(\hat{\VctStl{r}}, T, \omega)
  \Planck(T,\omega) \Delta S \Delta\omega,
\end{equation}
where $W^e$ is the power of the emitted radiation and \Planck\ is the
Planck function.  In order to calculate \AbsVec\ we assume that the
particle is placed inside an opaque cavity of dimensions large
compared to the particle and any wavelengths under consideration. We
have thermodynamic equilibrium if the cavity and the particle is
maintained at the constant temperature $T$. The emitted radiation
inside the cavity is isotropic, homogeneous, and unpolarized. We can
represent this radiation as a collection of quasi-monochromatic,
unpolarized, incoherent beams propagating in all directions
characterized by the Planck blackbody radiation
\begin{equation}
  \label{eq:rtetheory:Planck}
  \Planck(T,\omega)\Delta S \Delta\Omega =
    \frac{\hbar\omega^3}{2\pi^2 v^2\left[\exp\left(\frac{\hbar\omega}{k_B T}
   \right) -1\right]}
     \Delta S \, \Delta\Omega,
\end{equation}
where $\Delta \Omega$ is a small solid angle about any direction, $\hbar$ is
the Planck constant divided by $2\pi$, and $k_B$ is the Boltzmann constant.
With respect to the n$^2$-law discussed below, it could be noticed that the
Planck law is governed by the local phase velocity, $v$, \citep[see
e.g.][]{thomas2002radiative}, and not the vacuum speed.

The blackbody Stokes vector is
\begin{equation}
  \StoVec_b(T, \omega) = \ColVctFour{\Planck(T,\omega)}{0}{0}{0}.
\end{equation}
For the Stokes emission vector, which we also call particle absorption
vector, we can derive
\begin{equation}
  \label{eq:rtetheory:abs_vec}
  a_i^p(\hat{\VctStl{r}}, T, \omega) =
  K_{i1}(\hat{\VctStl{r}},\omega) - \int_{4\pi} \DiffD\hat{\VctStl{r}}' Z_{i1}(\hat{\VctStl{r}}, \hat{\VctStl{r}}', \omega) ,
 \quad i = 1,\ldots,4. 
\end{equation}
This relation is a property of the particle only, and it is valid for
any particle, in thermodynamic equilibrium or non-equilibrium.


\subsection{Optical cross sections}
\label{sec:rtetheory:cross_sects}

The optical cross-sections are defined as follows: The product of the
scattering cross section $C_{\sca}$ and the incident monochromatic
energy flux gives the total monochromatic power removed from the
incident wave as a result of scattering into all directions. The
product of the absorption cross section $C_{\mathrm{abs}}$ and the incident
monochromatic energy flux gives the power which is removed from the
incident wave by absorption. The extinction cross section $C_{\mathrm{ext}}$ is
the sum of scattering and absorption cross section.  One can express
the extinction cross sections in terms of extinction matrix elements
\begin{eqnarray}\label{eq:rtetheory:ext_cross_sec}
    C_{\mathrm{ext}} =
      \frac{1}{I^{\inc}} \, \bigl( &
                K_{11}(\hat{\VctStl{n}}^{\inc})I^{\inc}
             +  K_{12}(\hat{\VctStl{n}}^{\inc})Q^{\inc} + \\
         &      K_{13}(\hat{\VctStl{n}}^{\inc})U^{\inc}
             +  K_{14}(\hat{\VctStl{n}}^{\inc})V^{\inc} \bigr),  
\end{eqnarray}
and the scattering cross section in terms of phase matrix elements
\begin{eqnarray}\label{eq:rtetheory:sca_cross_sec}
  C_{\sca} =
      \frac{1}{I^{\inc}} \int_{4\pi} \DiffD\hat{\VctStl{r}}
    \bigl(&
          Z_{11}(\hat{\VctStl{r}}, \hat{\VctStl{n}}^{\inc})I^{\inc}
        + Z_{12}(\hat{\VctStl{r}}, \hat{\VctStl{n}}^{\inc})Q^{\inc} + \\
     &    Z_{13}(\hat{\VctStl{r}}, \hat{\VctStl{n}}^{\inc})U^{\inc} +
              Z_{14}({\hat{\VctStl{r}}, \hat{\VctStl{n}}^{\inc}})V^{\inc} \bigr).
\end{eqnarray}
The absorption cross section is the difference between extinction and
scattering cross section:
\begin{equation}
  \label{eq:rtetheory:abs_cross_sec}
  C_{\mathrm{abs}} = C_{\mathrm{ext}} - C_{\sca}. 
\end{equation}
The single scattering albedo $\omega_0$, which is a commonly used
quantity in radiative transfer theory, is defined as the ratio of the
scattering and the extinction cross section:
\begin{equation}
  \omega_0 = \frac{C_{\sca}}{C_{\mathrm{ext}}} \le 1.
\end{equation}
All cross sections are real-valued positive quantities and have the
dimension of area.

The phase function is generally defined as
\begin{eqnarray}\label{eq:rtetheory:phase_function}
  p(\hat{\VctStl{r}}, \hat{\VctStl{n}}^{\inc}) =
    \frac{4\pi}{C_{\sca} I^{\inc}} \, \bigl( &
         Z_{11}(\hat{\VctStl{r}}, \hat{\VctStl{n}}^{\inc})I^{\inc}
       + Z_{12}(\hat{\VctStl{r}}, \hat{\VctStl{n}}^{\inc})Q^{\inc} \rlap{${}+$} \\
    &    Z_{13}(\hat{\VctStl{r}}, \hat{\VctStl{n}}^{\inc})U^{\inc}
       + Z_{14}(\hat{\VctStl{r}}, \hat{\VctStl{n}}^{\inc})V^{\inc} \rlap{$\bigr)$.}
 \end{eqnarray}
The phase function is dimensionless and normalized:
\begin{equation}
  \frac{1}{4\pi}\int_{4\pi} 
  p(\hat{\VctStl{r}}, \hat{\VctStl{n}}^{\inc}) \,\DiffD\hat{\VctStl{r}} = 1.
\end{equation}


\subsection[Coordinate systems]{\textindex{Coordinate systems}: The
  \textindex{laboratory frame} and the \textindex{scattering frame} }
\label{sec:rtetheory:particle_coordinate_sytems}

For radiative transfer calculations we need a coordinate system to
describe the direction of propagation. For this purpose we use the
laboratory frame, which is shown in
Figure \ref{fig:RT_theory_coordinates_and_frames}, right panel. The z-axis
corresponds to the local zenith direction and the x-axis points towards the
north-pole. The propagation direction is described by the local zenith
angle $\theta$ and the local azimuth angle $\phi$. This coordinate
system is the most appropriate frame to describe the propagation
direction and the polarization state of the radiation. However, in
order to describe scattering of radiation by a particle or a particle
ensemble, it makes sense to define another coordinate system taking
into consideration the symmetries of the particle or the scattering
medium, as one gets much simpler expressions for the single scattering
properties. For macroscopically isotropic and mirror-symmetric
scattering media it is convenient to use the scattering frame, in
which the incidence direction is parallel to the z-axis and the x-axis
coincides with the scattering plane, that is, the plane through the
unit vectors \VctStl{\hat{n}^{\inc}}\ and \VctStl{\hat{n}^{\sca}}. The
scattering frame is illustrated in
Figure \ref{fig:RT_theory_coordinates_and_frames}, left panel. For symmetry
reasons the single scattering properties defined with respect to the scattering
frame can only depend on the scattering angle $\Theta$,
\begin{equation}
  \label{eq:scat_angle}
  \Theta=\arccos(\hat{\PDir}^{\inc} \cdot {\hat{\PDir}^{\sca}}),
\end{equation}
between the incident and the scattering direction.


\begin{figure}[t]
 \centering
 \begin{minipage}{.5\textwidth}
  \centering
   \includegraphics*[width=\textwidth]{coordinate_system}
 \end{minipage}%
 \hfill
 \begin{minipage}{.4\textwidth}
  \centering
   \includegraphics*[width=\textwidth]{part_frame}
 \end{minipage}%
   \caption{
Right: Coordinate system to describe the direction of propagation and the
polarization state of a plane electromagnetic wave (adapted from Mishchenko).
Left: Illustration of the scattering frame. The z-axis coincides with the
incident direction $\hat{\PDir}^{\inc}$. The scattering angle $\Theta$ is the
angle between  $\hat{\PDir}^{\inc}$ and $\hat{\PDir}^{\sca}$.
           }
  \label{fig:RT_theory_coordinates_and_frames}  
\end{figure}


\section[Particle Ensembles]
{Scattering, absorption and emission by ensembles of independent particles}
\label{sec:rtetheory:part_ensembles}

The formalism described in the previous chapter applies only for
radiation scattered by a single body or a fixed cluster consisting of
a limited number of components. In reality, one normally finds
situations, where radiation is scattered by a very large group of
particles forming a constantly varying spatial configuration. Clouds
of ice crystals or water droplets are a good example for such a
situation. A particle collection can be treated at each given moment
as a fixed cluster, but as a measurement takes a finite amount of
time, one measures a statistical average over a large number of
different cluster realizations.

Solving the Maxwell equations for a whole cluster, like a collection
of particles in a cloud, is computationally too expensive.
Fortunately, particles forming a random group can often be considered
as independent scatterers. This approximation is valid under the
following assumptions:
\begin{enumerate}
\item Each particle is in the far-field zone of all other particles.
\item Scattering by the individual particles is incoherent.
\end{enumerate}

As a consequence of assumption 2, the Stokes parameters of the partial
waves can be added without regard to the phase. If the particle number
density is sufficiently small, the single scattering approximation can
be applied. The scattered field in this approach is obtained by
summing up the fields generated by the individual particles in
response to the external field in isolation from all other particles.
If the particle positions are random, one can show, that the phase
matrix, the extinction matrix and the absorption vector are obtained
by summing up the respective characteristics of all constituent
particles.

\subsection{Single scattering approximation}

We consider a volume element containing $N$ particles. We assume that
$N$ is sufficiently small, so that the mean distance between the
particles is much larger than the incident wavelength and the average
particle size. Furthermore we assume that the contribution of the
total scattered signal of radiation scattered more than once is
negligibly small.  This is equivalent to the requirement
\begin{equation}
  \frac{N\EnsAvr{C_{\sca}}}{l^2} \ll 1,
\end{equation}
where $\EnsAvr{C_{\sca}}$ is the average scattering cross section per
particle and $l$ is the linear dimension of the volume element.  The
electric field scattered by the volume element can be written as the
vector sum of the partial scattered fields scattered by the individual
particles:
\begin{equation}
  \VctStl{E}^{\sca}(\VctStl{r}) = \sum_{n=1}^{N}\VctStl{E_n}^{\sca}(\VctStl{r}).
\end{equation}
As we assume single scattering the partial scattered fields are given
according to Equation \ref{eq:rtetheory:amplitude_matrix}:
\begin{equation}
  \ColVctTwo{\left[ E_n^{\sca}(\VctStl{r}) \right]_\ZntAng}%
         {\left[ E_n^{\sca}(\VctStl{r})\right]_\AzmAng}
         = \frac{e^{ikr}}{r}{\MtrStl{S}(\hat{\VctStl{r}},\hat{\VctStl{n}}^{\inc})} 
 \ColVctTwo{E_{0\ZntAng}^{\inc}}%
         {E_{0\AzmAng}^{\inc}},
\end{equation}
where $\MtrStl{S}$ is the total amplitude scattering matrix given by:
\begin{equation}
  \MtrStl{S}(\hat{\VctStl{r}},\hat{\VctStl{n}}^{\inc}) = \sum_{n=1}^N e^{i\Delta_n}  {\MtrStl{S}_n(\hat{\VctStl{r}},\hat{\VctStl{n}}^{\inc})}.
\end{equation}
$\MtrStl{S}_n(\hat{\VctStl{r}},\hat{\VctStl{n}}^{\inc})$ are the individual amplitude
matrices and the phase $\Delta_n$ is given by
\begin{equation}
  \Delta_n = k \VctStl{r}_\mathrm{On} \cdot(\hat{\VctStl{n}}^{\inc} - \hat{\VctStl{r}}),
\end{equation}
where the vector $\VctStl{r}_\mathrm{On}$ connects the origin of the volume
element $O$ with the $n$th particle origin (see
Figure \ref{fig:part_ensembles}).  Since $\Delta_n$ vanishes in forward
direction and the individual extinction matrices can be written in
terms of the individual amplitude matrix elements, the total
extinction matrix is given by
\begin{equation}
  \label{eq:rtetheory:ext_mat_tot}
  \ExtMat = \sum_{n=1}^N  \MtrStl{K}_n = N \EnsAvr{\ExtMat},
\end{equation}
where $\EnsAvr{\ExtMat}$ is the average extinction matrix per
particle. One can derive the analog equation for the phase matrix
 \begin{equation}
  \label{eq:rtetheory:pha_mat_tot}
  \PhaMat = \sum_{n=1}^N  \MtrStl{Z}_n = N \EnsAvr{\PhaMat},
\end{equation}
where $\EnsAvr{\PhaMat}$ is the average phase matrix per particle.  In
almost all practical situations, radiation scattered by a collection
of independent particles is incoherent, as a minimal displacement of a
particle or a slight change in the scattering geometry changes the
phase differences entirely.  It is important to note, that the
ensemble averaged phase matrix and the ensemble averaged extinction
matrix have in general 16 independent elements. The relations between
the matrix elements, which can be derived for single particles, do not
hold for particle ensembles.

\begin{figure}[t]
 \begin{center}
   \includegraphics*[width=0.6\hsize]{part_ensemble}
   \caption{A volume element of a scattering medium consisting of a particle
ensemble. $O$ is the origin of the volume element, $r_{O1}$ connects the origin
with particle $1$ and $r_{O2}$ with particle $2$. The observation point is
assumed to be in the far-field zone of the volume element.}
  \label{fig:part_ensembles}  
 \end{center}
\end{figure}


\section[Radiative transfer equation]
{Phenomenological derivation of the radiative transfer equation}
\label{sec:rtetheory:theory_rte}

When the scattering medium contains a very large number of particles
the single scattering approximation is no longer valid. In this case
we have to take into account that each particle scatters radiation
that has already been scattered by another particle. This means that
the radiation leaving the medium has a significant multiple scattered
component. The observation point is assumed to be in the far-field
zone of each particle, but it is not necessarily in the far-field zone
of the scattering medium as a whole. A traditional method in this case
is to solve the radiative transfer equation.  This approach still
assumes, that the particles forming the scattering medium are randomly
positioned and widely separated and that the extinction and the phase
matrices of each volume element can be obtained by incoherently adding
the respective characteristics of the constituent particles. In other
words the scattering media is assumed to consist of a large number of
discrete, sparsely and randomly distributed particles and is treated
as continuous and locally homogeneous.  Radiative transfer theory is
originally a phenomenological approach based on considering the
transport of energy through a medium filled with a large number of
particles and ensuring energy conservation.
\citet{mishchenko02:_vector} has demonstrated that it can be derived
from electromagnetic theory of multiple wave scattering in discrete
random media under certain simplifying assumptions.

In the phenomenological radiative transfer theory, the concept of
single scattering by individual particles is replaced by the
assumption of scattering by a small homogeneous volume element. It is
furthermore assumed that the result of scattering is not the
transformation of a plane incident wave into a spherical scattered
wave, but the transformation of the specific intensity vector, which
includes the Stokes vectors from all waves contributing to the
electromagnetic radiation field.

The vector radiative transfer equation (VRTE) is
 \begin{eqnarray}
  \label{eq:rtetheory:VRTE}
  \frac {\DiffD\StoVec(\Frq,\PPos,\PDir)}{\DiffD s} =
    &{}-\ExtMat(\Frq,\PPos,\PDir) \StoVec(\Frq,\PPos,\PDir) +
    \AbsVec(\Frq,\PPos,\PDir) \rlap{$B(\Frq,\PPos)$} \\ \nonumber
    &{}+ \int_{4\pi} \DiffD\PDir' \PhaMat(\Frq,\PPos,\PDir,\PDir')
    \StoVec(\Frq,\PPos,\PDir'),
\end{eqnarray}
where $\StoVec$ is the specific intensity vector, $\ExtMat$ is the total
extinction matrix, $\AbsVec$ is the total absorption vector, $B$ is the Planck
function and $\PhaMat$ is the total phase matrix. Furthermore $\Frq$ is the
frequency of the radiation, $\DiffD s$ is a
path-length-element of the propagation path, \PPos\ represents the atmospheric
position and $\PDir$ the propagation direction. Equation
\ref{eq:rtetheory:VRTE} is valid for monochromatic or quasi-monochromatic
radiative transfer. We can use this equation for simulating microwave radiative
transfer through the atmosphere, as the scattering events do not change the
frequency of the radiation.

The four-component specific intensity vector $\StoVec = (I,Q,U,V)^T$
fully describes the radiation and it can directly be associated with
the measurements carried out by a radiometer used for remote sensing.
For the definition of the components of the specific intensity vector
refer to Section \ref{sec:polarization}, where the Stokes
components are described. 

The three terms on the right hand side of Equation \ref{eq:rtetheory:VRTE}
describe physical processes in an atmosphere containing different particle
types and different trace gases. The first term represents the extinction of
radiation traveling through the scattering medium, \ExtMat. For microwave
radiation in cloudy atmospheres, extinction is caused by gaseous absorption,
particle absorption and particle scattering. Therefore \ExtMat\ can be written
as a sum of two matrices, the particle extinction matrix $\ExtMat^{p}$ and the
gaseous extinction matrix $\ExtMat^{g}$:
\begin{equation}
  \ExtMat(\Frq,\PPos,\PDir) = \ExtMat^{p}(\Frq,\PPos,\PDir)+
  \ExtMat^{g}(\Frq,\PPos,\PDir).
\end{equation}
The particle extinction matrix is the sum over the individual specific
extinction matrices $\ExtMat_i^p$ of the $N$ different particles types
contained in the scattering medium weighted by their particle number densities
$n^p_i$:
\begin{equation}
  \ExtMat^{p}(\Frq,\PPos,\PDir) =
  \sum_{i=1}^N n^p_i \ExtMat_i^p(\Frq,\PPos,\PDir).
\label{eq:rtetheory:ext_mat_p}
\end{equation}
The gaseous extinction matrix can
normally be derived from the scalar gas absorption. This as there is no
polarization due to gas absorption at cloud altitudes, and the off-diagonal
elements of the gaseous extinction matrix are zero. On the other hand, at very
high altitudes above approximately 40\,km there is polarization due to the
Zeeman effect, mainly due to oxygen molecules. In addition, in the toposphere
and stratosphere molecular scattering can be neglected in the microwave
frequency range. Hence the coefficients on the diagonal correspond to the gas
absorption coefficient:
\begin{eqnarray}
\ExtMat^{g}_{l,m}(\Frq,\PPos) = \alpha^g(\Frq,\PPos) & 
    \mathrm{if }\ l = m \nonumber \\
    0 & \mathrm{if }\ l \neq m.
\label{eq:rtetheory:Kgas}
\end{eqnarray}
where $\alpha^g$ is the total scalar gas absorption coefficient, which is
calculated from the individual absorption coefficients of all $M$
trace gases $\alpha_i^g$ and their volume mixing ratios
$n^g_i$ as:
\begin{equation}
   \alpha^g(\Frq,\PPos) =  \sum_{i=1}^M  n^g_i \alpha_i^g(\Frq,\PPos).
\end{equation}
The second term in Equation \ref{eq:rtetheory:VRTE} is the thermal source term.
It describes thermal emission by gases and particles in the atmosphere. The
absorption vector $\AbsVec$ is
\begin{equation}
  \AbsVec(\Frq,\PPos,\PDir)  =
  \AbsVec^{p}(\Frq,\PPos,\PDir) +
  \AbsVec^{g}(\Frq,\PPos,\PDir),
\end{equation}
where $\AbsVec^{p}$ and $\AbsVec^{g}$
are the particle absorption vector and the gas absorption vector,
respectively.  The particle absorption vector is a sum over the
individual absorption vectors $\AbsVec_i^p$, again
weighted with $n^p_i$:
\begin{equation}
  \AbsVec^{p}(\Frq,\PPos,\PDir) = \sum_{i=1}^N n^p_i \AbsVec_i^p(\Frq,\PPos,\PDir).
\label{eq:rtetheory:abs_vec_p}
\end{equation}
The gas absorption vector is simply (if no Zeeman splitting)
\begin{equation}
 \AbsVec^{g}  = [\alpha^p, 0, 0, 0]^T.
\end{equation}
The last term in Equation \ref{eq:rtetheory:VRTE} is the scattering source
term. It adds the amount of radiation which is scattered from all directions
$\PDir'$ into the propagation direction \PDir. The phase matrix \PhaMat\
is the sum of the individual phase matrices $\PhaMat_i$ weighted with
$n^p_i$:
\begin{equation}
  \PhaMat(\Frq,\PPos,\PDir) = 
  \sum_{i=1}^N n^p_i \PhaMat_i(\Frq,\PPos,\PDir).
  \label{eq:rtetheory:pha_mat}
\end{equation}
The scalar radiative transfer equation (SRTE)
\begin{eqnarray}
  \label{eq:rtetheory:SRTE}
  \frac{\DiffD I(\Frq,\PPos,\PDir)}{\DiffD s}& = 
  -K_{11}(\Frq,\PPos,\PDir) I(\Frq,\PPos,\PDir) +
  a_1(\Frq,\PPos,\PDir) B(\Frq,\PPos)   \nonumber\\
  &+\int_{4\pi} \DiffD \PDir' Z_{11}(\Frq,\PPos,\PDir,\PDir') I(\Frq,\PPos,\PDir')
\end{eqnarray}
can be used presuming that the radiation field is unpolarized. This
approximation is reasonable if the scattering medium consists of spherical or
completely randomly oriented particles, where $\ExtMat^p$ is diagonal and only
the first element of $\AbsVec^p$ is non-zero.


 
% \section{Blackbody radiation}
%  \label{sec:rtetheory:rtetheory:planck}
 
%  All natural bodies with a temperature $>$ 0 K emit thermal radiation.
%  The thermal motion in the matter is translated by collisions to
%  excitations in the molecules. The transition from the excited state
%  to a lower state causes emission of electromagnetic radiation.
%  Depending on the temperature the distribution of the emission will
%  change. An ideal body that absorbs all incoming radiation is called a
%  blackbody and its emittance follows Planck's formula: 

%  \begin{equation}
%    B(\lambda,T) = \frac{2\pi hc^2}{\lambda^5} \frac{1}{e^{hc/\lambda k_bT}-1}
%   \label{eq:rtetheory:planck}
%  \end{equation}
%  where $B$ is the emitted radiation, $\lambda$ the wavelength, $T$ the 
%  temperature, $h$ the Planck constant, $c$ the speed of light and $k_b$
%  the Boltzmann constant. Equation \ref{eq:rtetheory:planck} is shown in Figure 
%  \ref{fig:rtetheory:planck} for some temperatures.

%  \begin{figure}
%   \begin{center}
%    \includegraphics*[width=0.8\hsize]{fig_planck}
%     \caption{The blackbody radiation as a function of wavelength for
%              some temperatures.}
%     \label{fig:rtetheory:planck}
%   \end{center}
%  \end{figure}   
                                       
%  The maximum of the Planck formula is a function temperature and is
%  given by the Wien's displacement law. The maximum of
%  Equation \ref{eq:rtetheory:planck} is found at: 

%  \begin{equation}
%    \lambda_{max} = \frac{K}{T}
%   \label{eq:rtetheory:wien}
%  \end{equation}
%  where is $\lambda_{max}$ the wavelength of maximum emittance and $K
%  =$~2.898\topowerten{-3} Km. Consequently the maximum is encountered
%  at a shorter wavelength for a higher temperature as can be seen in
%  Figure \ref{fig:rtetheory:planck}.
                                     
%  No natural object can be said to be a perfect blackbody but the Sun
%  and the Earth can approximately be treated as blackbodies with
%  temperature of about 6000 and 290 K, respectively, to explain some
%  basic conditions. Wien's displacement law gives maximum for the solar
%  radiation at 0.55 $\mu$m while the Earth thermal radiation is maximal
%  around 10 $\mu$m. The solar maximum coincides with a region of high
%  atmospheric transmission and this explains why the
%  evolution has placed the vision between about 400 - 700 nm, the
%  amount of energy at surface level is maximal in this range.  
 
%  If the radiation from the Sun is scaled to compensate for the
%  attenuation due to the distance it will be found that the radiation
%  from the Earth itself will dominate for wavelengths longer than about
%  5 $\mu$m. This means, for example, that remote sensing techniques in
%  the thermal IR and microwave region primarily detect thermal emission
%  from the surface or the atmosphere while observations in the optical
%  and UV regions use absorbed, scattered or reflected solar radiation.
%  This also explains why gases that exhibit absorption between 5 - 50
%  $\mu$m are called greenhouse gases, they absorb partly some of the
%  outgoing radiation from the surface and the sea.


\section{The n$^2$-law of radiance}
\label{sec:n2law}


\subsection{Introduction}

The radiance, \StoVec, is unchanged for propagation in ``free space''. The term
free space implies a refractive index of unity and that extinction is zero.
However, it is possible to define a slightly different quantity that is
conserved also for propagation with a varying refractive index. This quantity
is here denoted as, $\StoVec_{n2}$, and is defined as
\citep{mobley:light:94,matlzer:radia:06}:
\begin{equation}
\label{eq:mobley}
\StoVec_{n2}\equiv\frac{\StoVec}{\Rfr^{2}}.
\end{equation}
That is, for radiation propagating without extinction or any sources,
$\StoVec_{n2}$ is constant along the propagation path. This is denoted as the
\(\Rfr^{2}\)-law for radiance. This impact of \Rfr\ can, for different reasons,
normally be neglected. As a consequency and to keep the nomenclature simple,
the n2-law is in general ignored in the ARTS documentation. 


\subsection{Treatment in ARTS}

As mentioned, the quantity defined by Equation~\ref{eq:mobley} is constant for
propagation without attenuation. Further, it can be shown that the radiance
corresponding to some emission is independent on the refractive index along the
propagation path, only the refractive indexes at the emission and measurement
points matter. This is also valid with attenuation along the propagation path
\citep[][Eq.~4.23]{mobley:light:94}:
\begin{equation}
  \label{eq:mobley423}
  \frac{\aStoI{m}}{\aRfr{m}^2} = e^{-\Oth}\frac{\aStoI{e}}{\aRfr{e}^2},
\end{equation}
where \aStoI{m}\ is measured radiance, \aRfr{m}\ the refractive index where the
measurement is performed, \aStoI{e}\ emitted radiance, \aRfr{e}\ the refractive
index at the emission point, and \Oth\ is the optical thickness between the two
points. 

As long as LTE applies, the emission is proportional to the Planck function,
\Planck\ (Eq.~\ref{eq:rtetheory:Planck}). Hence, using an emissivity,
$\varepsilon$, we have
\begin{equation}
  \frac{\aStoI{e}}{\aRfr{e}^2} = \frac{\varepsilon B(\aTmp{e})}{\aRfr{e}^2}=
   \varepsilon B_{n2}(\aTmp{e}),
\end{equation}
where \aTmp{e}\ is the temperature of the emitting substance, and
\begin{equation}
  \label{eq:planckconst}
  \Planck_{n2}(\BT)\equiv
  \frac{\Planck(\BT)}{\Rfr^{2}} = \frac{2\planckCns\Frq^3}
  {\speedoflight^{2}(exp(\planckCns\Frq/\boltzmannCns\BT)-1)}.
\end{equation}
That is, it turns out that by consistently apply $\speedoflight$ in the Planck
function (instead of $v$), the dependency of \aRfr{e}\ is removed. What
remains to obtain the correct radiance to output, \StoI, is to consider the
impact of \aRfr{m}:
\begin{equation}
  \label{eq:n2finalstep}
  \StoI = \aRfr{m}^2\StoI',
\end{equation}
where $\StoI'$ is the radiance calculated ignoring the n$^2$-law.

As discussed by \citet{matlzer:radia:06}, it can be deduced from basic
principles that the brightness temperature must be a preserved quantity, even
in light of the n$^2$-law. This statement can also be understood from
Equation~\ref{eq:planckconst}. In simple terms, the brightness temperature is
defined with respect to the local Planck function and the impact of refractive
index variations vanishes if the radiance is measured in terms of brightness
temperature. 



\section{Simple solution without scattering and polarization}

 If scattering can be neglected and the atmosphere is assumed to be in
 local thermodynamic equilibrium, the radiative transfer equation gets
 unusually simple. These assumptions will be made below and
 they are normally valid for the infrared region and longer
 wavelengths as in the microwave region. For these conditions the
 atmospheric absorption and emission are linked and the basic problem
 to determine the radiative transfer is to calculate the absorption.
 At the wavelengths considered rotational and vibrational transitions
 are the dominating absorbing processes.

 The basic equation describing radiative transfer along a specific 
 direction is
 \begin{equation}
   \frac{dI(\nu)}{dl} = k(l,\nu)(B(l,\nu)-I(\nu))
  \label{eq:rtetheory:chand}
 \end{equation} 
 where $I$ is the intensity per unit area, $\nu$ the frequency, $l$
 the distance along the propagation path, $k$ the total absorption
 coefficient (summed over all species and transitions) and $B$ the
 Planck function. This differential equation can be solved:
 \begin{equation}
   I(\nu) = I_0(\nu)e^{-\int^h_0{k(l',\nu)dl'}} + 
     \int^h_0{k(l,\nu)B(T(l),\nu) e^{-\int^l_0{k(l',\nu)dl'}} dl}
  \label{eq:rtetheory:rte}
 \end{equation}  
 where the receiver is assumed to be placed at $l$~=~0 and $h$ is the
 distance along the path to the limit of the media. $I_0$ is the
 intensity at the point $h$ which can represent thermal emission from
 the surface, solar radiation at top of the atmosphere or cosmic
 background radiation depending on the observation geometry. When
 discussing radiative transfer the quantity optical depth, $\tau$, is
 commonly used and it is defined as

 \begin{equation}
   \tau(l,\nu) = \int^l_0{k(l',\nu)dl'} 
  \label{eq:rtetheory:tau}
 \end{equation}  
 and Equation \ref{eq:rtetheory:rte} can be written as
 
 \begin{equation}
   I(\nu) = I_0(\nu)e^{-\tau(h,\nu)dl'} + 
     \int^h_0{k(l,\nu)B(T(l),\nu) e^{-\tau(l',\nu)} dl}
  \label{eq:rtetheory:rte2}
 \end{equation}  
 The terms inside the integral found in this equation have a simple
 physical meaning, the radiation emitted at one point is $kBdl$ and this
 quantity is attenuated by the factor $e^{-\tau}$ before it reaches the
 observation point.

 \section{Clear sky simulations with star like sources}
 \label{sec:rtetheory:ClearskyDirectSources}

 In this section we describe the radiative transfer in clearsky with star 
 like sources.
 In this context clear sky means that there is no scattering by liquid or 
 solid matter but scattering by atoms and molecules (gas scattering, see also ....).
 Fig. XXX  shows the geometry for a reciever outside of the atmosphere.

 The star is assumed to be far away from the planet or to be more precise
 \begin{equation}
  $r_{\astrosun}\ll d_{\astrosun p}$
 \end{equation} with $r_{\astrosun}$ the star radius,
 and $d_{\astrosun p}$ the distance between star and TOA of the planet. 
 This means that the angular variability of the direction of the  radiation 
 from the star can be neglected (parallel beam or collimated  beam approximation) 
 and the incoming radiation at top of  the atmosphere (TOA) can be described as 
 \begin{equation}
 I_{\astrosun,TOA}\left(\Omega\right)=F_{\astrosun}\delta\left(\Omega,\Omega'\right)\label{eq:rtetheory:directsource}
 \end{equation}
 with $\Omega'$ the direction of the star. We neglect polarization here and in the follwoing part of this section for 
 the sake of simplicity. 

 The flux of the incoming radiation from the star at the top of the atmosphere
 $F_{\astrosun}$ is 
\begin{align}
F_{\astrosun} & =\stackrel[0]{2\pi}{\int}\stackrel[0]{\alpha}{\int}I_{\astrosun}\cos\theta\sin\theta\text{d}\theta\text{d}\phi\\
 & =\pi I_{\astrosun}\sin^{2}\alpha
\end{align}
with $I_{\astrosun}$ the star spectral radiance, which is assumed
to be isotropic, and $\alpha$ the angular radius of the star seen
from TOA of the planet. This is 
\begin{equation}
\alpha=\arctan\frac{r_{\astrosun}}{d_{\astrosun p}}\,.
\end{equation}

As gas scattering is rather weak, multi scattering can be neglected and only
single scattering of the incoming radiation from the star needs to be considered. 
With this and Eq. \ref{eq:rtetheory:directsource}, Eq. \ref{eq:rtetheory:VRTE} 
neglecting polarization simplifies to  
\begin{eqnarray}
  \label{eq:rtetheory:ClearskyRTE}
  \frac{\mathrm{d}I\left(s,\nu\right)}{\mathrm{d}s} & = & -kI\left(s,\nu\right)
  +a\Planck\left(T\left(s\right)\right)+\underset{4\pi}{\int}P\left(\nu,\Omega_{i},\Omega_{s}\right)
  F_{\astrosun}^{*}\left(\nu,s\right)\delta\left(\Omega_{i},\Omega'\right)\mathrm{d}\Omega_{i}\\
  & = & -kI\left(s,\nu\right)+aB\left(T\left(s\right)\right)+P\left(\nu,\Omega',\Omega_{s}\right)F_{\astrosun}^{*}\left(\nu,s\right)
\end{eqnarray}
with $k$ and $a$ the extinction coefficients, $\Planck$ the Planck function
 


\section{Special solutions}
 \label{sec:rtetheory:special}
 
 If the total emission along the propagation path can be neglected
 compared to the transmitted part of the incoming radiation, the
 radiative transfer equation is simplified to the well known Beer-Lambert law:
 
 \begin{equation}
   I(\nu) = I_0(\nu)e^{-\tau(h,\nu)}
  \label{eq:rtetheory:beer}
 \end{equation}  
 This equation can for example be used when evaluating solar
 occultation observations.  
 
 If the temperature is constant through the medium studied (Fig.
 \ref{fig:rtetheory:layer}) the integral in Equation \ref{eq:rtetheory:rte} can
 be solved analytically:

 \begin{figure}
  \begin{center}
   \begin{minipage}[c]{0.4\textwidth}
    \centering
    \caption{Schematic picture of the radiative transfer through a medium with
             constant temperature.}
    \label{fig:rtetheory:layer}
   \end{minipage}%
   \hspace{0.05\textwidth}%
   \begin{minipage}[c]{0.50\textwidth}
    \centering
    \includegraphics*[width=0.99\hsize]{fig_layer}
   \end{minipage}
  \end{center}
 \end{figure}   
  
 \begin{equation}
   I^{out} = I^{in}e^{-\tau} + B(T,\nu)(1-e^{-\tau})
  \label{eq:rtetheory:layer}
 \end{equation}  
 where is $\tau$ the total optical thickness of the medium. Two
 special cases can be distinguished. If the layer is totally optically
 thick ($\tau \to \infty$) then $I^{in}$ is totally absorbed and
 $I^{out} = B$, the medium emits as a blackbody. If the layer has no
 absorption ($\tau=0$) then Equation \ref{eq:rtetheory:layer} gives
 $I^{out} = I^{in}$ as expected.
 
 In microwave radiometry the measured intensity is normally presented
 by means of the brightness temperature, $T_b$. This quantity is
 derived from the Rayleigh-Jeans approximation of the Planck function:

 \begin{equation}
   B(T,\nu) \approx \frac{2\nu^2k_bT}{c^2} = \frac{2k_bT}{\lambda^2}
  \label{eq:rtetheory:rayjean}
 \end{equation}  
 This equation is valid when $h\nu \ll kT$ which is the case in the
 microwave region due to the relatively low frequencies. If the
 temperature is 50 K, $hv$ equals $kT$ at 1.04 THz. The important
 aspect of Equation \ref{eq:rtetheory:rayjean} is the linear relationship
 between the intensity and the physical temperature. The natural
 definition of brightness temperature, $T_b$, is then

 \begin{equation}
   T_b(\nu) = \frac{\lambda^2}{2k_bT} I(\nu)
  \label{eq:rtetheory:tb}
 \end{equation}  
 The difference between the brightness temperature and the physical
 temperature (corresponding to the actual intensity) increases with
 frequency which is exemplified in Figure \ref{fig:rtetheory:rayjean}. The
 differences for higher frequencies are certainly not negligible and
 the brightness temperature shall not be mistaken for the physical
 temperature. The important fact is that the brightness temperature
 has a linear relationship to the intensity and gives a more intuitive
 understanding of the magnitude of the emission. In the Rayleigh-Jeans
 limit Equation \ref{eq:rtetheory:rte} can be written as

 \begin{equation}
   T_b(\nu) = T_{b0}(\nu)e^{-\tau(h,\nu)} + 
     \int^h_0{k(l,\nu)T(l) e^{-\tau(l',\nu)} dl}
  \label{eq:rtetheory:rte_tb}
 \end{equation}  

 \begin{figure}
  \begin{center}
   \begin{minipage}[c]{0.35\textwidth}
    \centering
    \caption{The difference between the physical temperarature of a 
             blackbody and the equivalent brightness temperature
             calculated using the Rayleight-Jeans approximation.}
    \label{fig:rtetheory:rayjean}
   \end{minipage}%
   \hspace{0.05\textwidth}%
   \begin{minipage}[c]{0.55\textwidth}
    \centering
    \includegraphics*[width=0.99\hsize]{fig_rayjean}
   \end{minipage}
  \end{center}
 \end{figure}   





\section{Surface emission and reflection}
\label{sec:surface}

\subsection{The dielectric constant and the refractive index}
%===================
The properties of a material can be reported either as the relative
dielectric constant, $\epsilon$, or the refractive index, $n$. Both
these quantities can be complex and are related as
\begin{equation}
  \label{eq:surface_eps2n}
  n = \sqrt{\epsilon}.
\end{equation}


\subsection{Relating reflectivity and emissivity}
 %===================
 Thermodynamic equilibrium can be assumed for natural surfaces, as
 long as there exist no strong temperature gradients. The Kirchoff law
 can then be used to relate the reflectivity and emissivity of a
 surface. For rough surfaces the scattering properties must be
 integrated to determine the
 emissivity (Equation~\ref{eq:surface:brdf:r}). For specular
 reflections (defined below) and scalar radiative transfer
 calculations, the emissivity $e$ is
 \begin{equation}
  \label{eq:e=1-r}
   e = 1 - r,
 \end{equation}
 where $r$ is the reflectivity (power reflection coefficient) of the
 surface.  Equation \ref{eq:e=1-r} is valid for each polarisation state
 individually \citep[Eq.\ 4.190a]{ulaby:81}.

 We have then that
 \begin{equation}
  \Mpi^\mathrm{up} = \Mpi^\mathrm{down}r + (1-r)B,
 \end{equation}
 where $\Mpi^\mathrm{up}$ is upwelling radiation, $\Mpi^\mathrm{down}$
 is downwelling radiation and $B$ is the magnitude of blackbody
 radiation. As expected, if $\Mpi^\mathrm{down}=B$, also
 $\Mpi^\mathrm{up}$ equals $B$.  Expressing the last observation using
 vector nomenclature gives
 \begin{equation}
   \left[\begin{array}{c} B \\ 0 \\0\\0 \end{array}\right] =
  \MtrStl{R} \left[\begin{array}{c} B \\ 0 \\0\\0 \end{array}\right] + 
  \VctStl{b},
 \end{equation}
 where $\MtrStl{R}$ is the matrix (4\,x\,4) correspondence to the
 scalar reflectivity, describing the properties of the surface
 reflection. The vector \VctStl{b} is the surface emission, that
 can be expressed as
 \begin{equation}
  \label{eq:surface:bvector} 
  \VctStl{b} = (\MtrStl{1}-\MtrStl{R})
      \left[\begin{array}{c} B \\ 0 \\0\\0 \end{array}\right],
 \end{equation}
 where \MtrStl{1} is the identity matrix. 



\subsection{Specular reflections}
 %===================
 
 If the surface is sufficiently smooth, radiation will be
 reflected/scattered only in the complementary angle, specular
 reflection. Required smoothness for assuming specular reflection is
 normally estimated by the Rayleigh criterion:
 \begin{equation}
   \label{eq:surface:rayleigh}
   \Delta h < \frac{\Wvl}{8\cos\theta_1}
 \end{equation}
 where $\Delta h$ is the root mean square variation of the surface
 height, \Wvl\ the wavelength and $\theta_1$ the angle between the
 surface normal and the incident direction of the radiation. The
 criterion can also be defined with the factor 8 replaced with a 
 higher number.
 
 The complex reflection coefficient for the amplitude of the
 electromagnetic wave for vertical ($R_v$) and horizontal ($R_v$)
 polarisation is for a flat surface (if the relative magnetic
 permeability ($\mu_r$) of both media is 1) given by the Fresnel equations:
 \begin{eqnarray}
   \label{eq:surface_fresnel}
   R_v &=& \frac{n_2\cos\theta_1-n_1\cos\theta_2}
                                           {n_2\cos\theta_1+n_1\cos\theta_2} \\
   R_h &=& \frac{n_1\cos\theta_1-n_2\cos\theta_2}
                                           {n_1\cos\theta_1+n_2\cos\theta_2} 
 \end{eqnarray}
 where $n_1$ is refractive index for the medium where the reflected radiation
 is propagating, $\theta_1$ is the incident angle (measured from the local
 surface normal) and $n_2$ is the refractive index of the reflecting medium.
 The angle $\theta_2$ is the propagation direction for the transmitted part,
 and is (approximately) given by Snell's law:
 \begin{equation}
   \label{eq:surface:snell}
   \Re(n_1)\sin\theta_1 = \Re(n_2)\sin\theta_2,
 \end{equation}
 where $\Re(\cdot)$ denotes the real real part. Equation~\ref{eq:surface:snell}
 is theoretically correct only if both $n_1$ and $n_2$ have no imaginary part. 
 For cases where medium 1 is air, $n_1$ can (in this context) be set to 1, and 
an expression allowing $n_2$ to be complex is found in Section 5.4.1.3 of
\citet{liou:02}. We are not aware of any expression for the case when both
$n_1$ and $n_2$ are complex.

The power reflection coefficients are converted to an intensity
 reflection coefficient as
 \begin{equation}
   \label{eq:surface:R2r}
   r = |R|^2,
 \end{equation}
 where $|\!\cdot\!|$ denotes the absolute value. Note that $R$ can be
 complex, while $r$ is always real.

The surface reflection can be seen as a scattering event and
Section~\ref{sec:polarization:ampmatrix} can be used to derive the
reflection matrix values. The scattering amplitude functions of
Equation~\ref{eq:polarisation:ampmatrix1} are simply
\begin{eqnarray}
  S_2 &=& R_v, \\
  S_1 &=& R_h, \\
  S_3 = S_4 &=& =0.
\end{eqnarray}
This leads to that the transformation matrix for a specular surface
reflection is (compare to \citet[Sec.\ 5.4.3]{liou:02})
\begin{equation}
  \label{eq:surface:specular_matrix}
  \MtrStl{R} =
     \left[\begin{array}{cccc}
       \frac{r_v+r_h}{2}&\frac{r_v-r_h}{2}&0&0\\
       \frac{r_v-r_h}{2}&\frac{r_v+r_h}{2}&0&0\\
    0&0&\frac{R_hR_v^\ast+R_vR_h^\ast}{2}&i\frac{R_hR_v^\ast-R_vR_h^\ast}{2}\\
    0&0&i\frac{R_vR_h^\ast-R_hR_v^\ast}{2}&\frac{R_hR_v^\ast+R_vR_h^\ast}{2}\\
     \end{array}
     \right].
\end{equation}
In some cases just $r_v$ and $r_h$ are at hand (as assumed by
\wsmindex{surfaceFlatRvRh}), the matrix \MtrStl{R}\ is then
set as:
\begin{equation}
  \label{eq:surface:rvrh}
  \MtrStl{R} =
     \left[\begin{array}{cccc}
       \frac{r_v+r_h}{2}&\frac{r_v-r_h}{2}&0&0\\
       \frac{r_v-r_h}{2}&\frac{r_v+r_h}{2}&0&0\\
    0&0&\frac{r_v+r_h}{2}&0\\
    0&0&0&\frac{r_v+r_h}{2}\\
     \end{array}
     \right].
\end{equation}
For example, the sea surface parameterization TESSEM provides the emissivity
for vertical and horizontal polarisation ($\epsilon_v$ and $\epsilon_h$). The
reflectivities are then set as $r_v=1-\epsilon_v$ and $r_h=1-\epsilon_h$,
respectively, and Eq.~\ref{eq:surface:rvrh} is applied to set \MtrStl{R}
(WSM is \wsmindex{surfaceTessem}).

For the case of $R_v=R_h$ (as assumed by
\wsmindex{surfaceFlatScalarReflectivity}) the matrix in
Equation~\ref{eq:surface:specular_matrix} is strictly diagonal and all the
diagonal elements have the same value, $(r_v+r_h)/2=r$.

If the downwelling radiation is unpolarised, the reflected part of the
upwelling radiation is
\begin{equation}
  \MtrStl{R}
  \left[ \begin{array}{c} I\\0\\0\\0 \end{array} \right] =
  \left[ \begin{array}{c} I(r_v+r_h)/2\\I(r_v-r_h)/2\\0\\0 
  \end{array} \right].
\end{equation}
as expected.


If \MtrStl{R} is given by Equation \ref{eq:surface:specular_matrix},
Equation \ref{eq:surface:bvector} gives that the surface emission  is
\begin{equation}
  \label{eq:surface:specular_emission}
   \VctStl{b} = \left[\begin{array}{c}
     B\left(1-\frac{r_v+r_h}{2}\right) \\
     B\frac{r_h-r_v}{2} \\
     0\\0
   \end{array}\right].
\end{equation}


\subsection{Rough surfaces}
%---
The scattering of rough surfaces is normally described by the bidirectional
reflectance distribution function, BRDF. With the BRDF,
$f(\theta_0,\phi_0,\theta_1,\phi_1)$, the scattered radiance in the
direction $(\theta_1,\phi_1)$ can be written as (see e.g.\ 
\citet{rees:01} or \citet{petty:06})
\begin{equation}
  \label{eq:surface:brdf1}
  I'(\theta_1,\phi_1) = \int_0^{\pi/2} \! \int_0^{2\pi} I(\theta,\phi) 
  \cos(\theta) f(\theta,\phi,\theta_1,\phi_1)
  \sin(\theta) \, \DiffD\phi \, \DiffD\theta,
\end{equation}
where $I(\theta,\phi)$ is the downwelling radiance for incidance angle $\theta$
and azimuth angle $\phi$. One important property of the BRDF is
\begin{equation}
  f(\theta_0,\phi_0,\theta_1,\phi_1) = f(\theta_1,\phi_1,\theta_0,\phi_0).
\end{equation}
The reflectivity is the half-sphere integral of the BRDF
\begin{equation}
  \label{eq:surface:brdf:r}
  r(\theta_1,\phi_1) = \int_0^{\pi/2} \! \int_0^{2\pi} 
  f(\theta_1,\phi_1,\theta,\phi) \cos(\theta)
  \sin(\theta) \, \DiffD\phi \, \DiffD\theta.
\end{equation}
An ideally rough surface is denoted as Lambertian. The BRDF is for this case
constant, and normally expressed using the diffuse reflectivity, $r_d$ 
 \citep[e.g.][]{petty:06}:
\begin{equation}
  \label{eq:surface:lambertian1}
  f = \frac{r_d}{\pi}.
\end{equation}
From Eq.~\ref{eq:surface:brdf:r} it follows that $r = r_d$.





  
%%% Local Variables: 
%%% mode: latex
%%% TeX-master: "uguide"
%%% End: 
