\graphicspath{{Figs/retrievals/}}

\chapter{Optimal estimation method}
 \label{sec:oem}

\starthistory
 190903 & Created and written by Simon Pfreundschuh\\ 
\stophistory

ARTS provides functionality to perform retrievals of remote sensing
observations using the optimal estimation method (OEM). This section 
describes the basic usage of the OEM implementation available inside ARTS
and aims to complement the example controlfiles distributed with the
ARTS source code. For a more detailed presentation of the method itself,
the reader may refer to the book by \citet{rodgers:00}.

\section{Formulation}

The optimal estimation method solves the retrieval problem of finding an
atmospheric state $\vec{x} \in \mathrm{R}^n$ that best explains a given observation
vector $\vec{y} \in \mathrm{R}^m$. This is done by fitting a forward model
$\mathbf{F}: \vec{x} \mapsto \vec{y}_f$ to the observations $\vec{y}$ in a way
consistent with a-priori-assumed properties of the retrieval quantities
within $\vec{x}$.

\subsection{Fundamental assumptions}

The OEM is based on a Bayesian formulation of the retrieval problem.
The three fundamental assumptions of this formulation are:
\begin{enumerate}
\item That the a priori assumed properties of all quantities in $\vec{x}$ can be described by
 a multivariate Gaussian distribution with mean vector $\vec{x}_a$ and covariance
 matrix $\mathbf{S}_x$.
\item That the forward model $\mathbf{F}$ is exact up to a zero-mean, Gaussian measurement
      error with covariance matrix $\mathbf{S}_e$.
\item That the forward model is linear or at most moderately non-linear.
\end{enumerate}

\subsection{The retrieval as optimization problem}

Under the assumptions listed above, the a posteriori distribution $p(\vec{x} | \vec{y})$
of the atmospheric state $\vec{x}$ given the observations in $\vec{y}$
is Gaussian as well. The negative log-likelihood of the a posteriori distribution
is found to be

\begin{eqnarray}\label{eq:oem_cost}
 \mathcal{L}(\vec{x}) &= \frac{1}{2} \left (
	(\vec{x} - \vec{x}_a)^T \mathbf{S}^{-1}_x (\vec{x} - \vec{x}_a) +
        (\vec{y} - \mathbf{F}(\vec{x}))^T \mathbf{S}^{-1}_e (\vec{y} - \mathbf{F}(\vec{x}))
       	\right ).
\end{eqnarray}

The most likely vector $\vec{x}$ given the observations $\vec{y}$, also referred to
as \textit{the maximum a posteriori estimator} of $\vec{x}$, can then be
 found by minimizing $\mathcal{L}(\vec{x})$. The ARTS OEM implementation provides
serveral methods to minimize this cost function using ARTS itself as the forward model
$\mathbf{F}$.

\section{Overview}

The main OEM functionality within ARTS is implemented by the OEM workspace method.
A diagram of the data flow of the OEM WSM is given in Fig.~\ref{fig:oem_flow}.

The main input arguments to the WSM call are the workspace variables 
\shortcode{xa, covmat\_sx, y} and \shortcode{covmat\_se} together with the agenda
\shortcode{inversion\_iterate\_agenda}. As their names imply, the variables
 \shortcode{xa} and {covmat\_sx} descripbe the a priori distribution given
by mean vector $\vec{x}_a$ (\shortcode{xa}) and covariance matrix $\mathbf{S}_x$
(\shortcode{covmat\_sx}). Similarly, the variables \shortcode{y} and
\shortcode{covmat\_se} represent the observation vector $\vec{y}$ and the
 covariance matrix $\mathbf{S}_e$
(\shortcode{covmat\_se}). Finally, the \shortcode{jacobian\_quantities} input
 variable contains a list of the quantities that should be retrieved from the
given observations.

In addition to the variables described above, OEM takes as additional input argument
the \shortcode{inversion\_iterate\_agenda} agenda. The role of this agenda is to define
the foward model used by the OEM method. This agenda is executed repeatedly during the
optimization procedure to simulate the measurement vector and its Jacobian corresponding to
the current atmospheric state.

\begin{figure}[!h]
  \begin{center}
   \includegraphics*[width=0.75\hsize, angle=0]{oem_flow}
  \end{center}
  \caption[OEM]{Data flow related to the OEM WSM.}
 \label{fig:oem_flow}
\end{figure}

\section{Usage}

There are three main steps involved in running the OEM within ARTS:
\begin{enumerate}
 \item Setup: Defining the measurement space and forward model
 \item Running the OEM WSM
 \item Calculating relevant diagnostic quantities
\end{enumerate}

Arguably the most complex of these is the setting up of the input variables.
Running the OEM requires only a single call to the OEM WSM. Also the calculation
of diagnistic quantities then can be performed with a small number of WSM calls.

\subsection{Setup}

Setting up an OEM calculation in ARTS involves three main steps:
\begin{enumerate}
  \item Defining the state space and a priori distribution (\shortcode{jacobian\_quantities, xa, covmat\_sx})
  \item Defining the observations and measurement errors (\shortcode{y, covmat\_se})
  \item Defining the forward model (\shortcode{inversion\_iterate\_agenda})
\end{enumerate}

\subsubsection{Defining the state space}

In the OEM formulation, the state vector $\vec{x}$ holds the values of all variables
that are retrieved.  ARTS therefore somehow needs to know which variables from
the workspace should be included in  $\vec{x}$. This information is contained in the
\shortcode{jacobian\_quantities} WSV. Each entry in \shortcode{jacobian\_quantities}
representes a potentially multi-dimensional retrieval variable that is retrieved
using the OEM. The state vector $\vec{x}$ is formed from these variables by concatenation
of the flattened values of each retrieval variable.

The setup of \shortcode{jacobian\_quantities} follows the principle of the setup for
regular Jacobian calculations. Before retrieval quantities can be added to
\shortcode{jacobian\_quantities}, the \shortcode{retrievalDefInit} WSM must be called.
This enusers that \shortcode{jacobian\_quantities} is empty and initializes required
internal variables. After the call to \shortcode{retrievalDefInit} retrieval
quantities can be registered by calling any of the \shortcode{retrievalAdd...}
 methods.

After the retrieval definition block can be finalized by calling 
\shortcode{retrievalDefClose}, also the a priori covariance matrix
 \shortcode{covmat\_sx} must be set. This is done using the
\shortcode{covmat\_sxAddBlock} WSMs. More details on the handling of
covariance matrix and the special requirements on \shortcode{covmat\_sx}
can be found in the group documentation for covariance matrices and
the WVS documentation for \shortcode{covmat\_sx}.

An example of how to add ozone as a retrieval quantity for a simple ozone
retrieval is given below:

\begin{code}
retrievalDefInit

# Add retrieval species
retrievalAddAbsSpecies(species = "O3", unit = "vmr")

# Create diagonal block and add to covariance matrix
nelemGet(nelem, p_ret_grid)
VectorSetConstant(vars, nelem, 0.5)
DiagonalMatrix(sparse_block, vars)
covmat_sxAddBlock(block = sparse_block )

retrievalDefClose
\end{code}

Instead of retrieving a quantity directly, it is not uncommon
to instead perform the retrieval in a transformed state space. ARTS
provides support for arbitrary affine and linear transforms, such
as for example retrieving principal components, as well as functional
transforms, such as retrieving the logarithm of a quantity. More
information on the usage of this functionality can be found in the
WSM documentation of the \shortcode{jacobianSetAffineTransformation}
and \shortcode{jacobianSetFuncTrasformation} WSMs.

An important point for the user to keep in mind is that for the OEM to
work the units of the vectors \shortcode{x} and \shortcode{xa} and
 the covariance matrix \shortcode{covmat\_sx} must be the same. To be
able to keep track of how the different retrieval variables are mapped
to the \shortcode{x} vector the user should refer to corresponding
\shortcode{retrievalAdd...} WSM.

\subsubsection{Defining observations and measurement error}

The observations to fit should be copied into the \shortcode{y} workspace
variable and the observation error covariance $\mathbf{S}_e$ into \shortcode{covmat\_se}.
The most important point to keep in mind here is that the dimensions of
\shortcode{y}  and \shortcode{covmat\_se} must match.

\subsubsection{Inversion iteration agenda}

The \shortcode{inversion\_iterate\_agenda} constitutes the forward model used by
the OEM WSM. This agenda is executed every time Jacobians and observations need
to be computed from the forward model during the OEM iterations.

Within  \shortcode{inversion\_iterate\_agenda} the following steps must be
performed:

\begin{enumerate}
\item Transform values in \shortcode{x} to the corresponding
ARTS WSVs using one of the \shortcode{x2Arts...} WSMs.
\item Perform the radiative transfer simulation (for example using
\shortcode{yCalc})
\item  Store resulting $\mathbf{y}$ vector in \shortcode{yf}
\item If any transformation are applied or one of the retrieval quantities
is retrieved in relative units, call
 \shortcode{jacobianAdjustAndTransform} for the computed Jacobian
to be transformed accordingly
\end{enumerate}

As an example, the definition of the \shortcode{inversion\_iterate\_agenda}
for the ozone retrieval is given below.

\begin{code}
AgendaSet( inversion_iterate_agenda ){

  Ignore(inversion_iteration_counter)
    
  # Map x to ARTS' variables
  x2artsAtmAndSurf
  x2artsSensor   # No need to call this WSM if no sensor variables retrieved

  # Calculate yf and Jacobian matching x.
  yCalc( y=yf )

  #Not needed in this simple case.
  #jacobianAdjustAndTransform
}
\end{code}

\subsection{Running OEM}

If the ARTS workspace has been setup properly, running an OEM calculation
simply requires calling the OEM WSM. 

The ARTS OEM calculation can be run in linear or non-linear mode. In linear
mode, only a single optimization step is performed, because in this case 
the Gauss-Newton iteration yields an exact solution already after the
first step. If the forward model is non-linear, however,
 multiple minimization steps are performed. The most important available
configuration options relating to the optimization procedure are described
below.

\subsubsection{Gauss-Newton optimization}
\label{ref:oem:gn}

In the standard formulation, the Gauss-Newton iteration for the OEM problem
takes the following form:

\begin{eqnarray}\label{eq:oem:gn}
\vec{x}_{i + 1} &= \vec{x}_i - 
\left ( \underbrace{\mathbf{K}_i^T \mathbf{S}_e^{-1} \mathbf{K} + \mathbf{S}^{-1}_x)^{-1} }_{ \mathbf{H}}\right )
\left(\underbrace{\mathbf{K}^T\mathbf{S}^{-1}_e (\mathbf{F}(\vec{x}_i) - \vec{y}) + \mathbf{S}_x^{-1}(\vec{x}_i - \vec{x}_a)}_{\vec{g}} \right)
\end{eqnarray}
This means that each optimization step involves solving a linear system of equations
 of size $n \times n$, where $n$ is the dimensionality of the measurement space.
The linear system is defined by the matrix $\mathbf{H}$ which approximates the
the Hessian of the cost function $\mathcal{L}$. The ARTS OEM method provides two methods
to solve these linear systems: By default, a QR solver is used which requires to
explicitly compute $\mathbf{H}$. If $n$ is very large, however, calculating
 $\mathbf{H}$ can become computationally very expensive. In this cases it may be
 advantageous to use a conjugate gradient (CG) solver, which does not require explicitly
computing $\mathbf{H}$. More information on how to use the CG solver is given in
 Sec.\ref{sec:oem:cg} below.

Additionally, two further forms of the GN iteration can be derived, the so called
\textit{m- and n-form}:

\begin{eqnarray}
\vec{x}_{i + 1} &= \vec{x}_a - 
\left (\mathbf{K}_i^T \mathbf{S}_e^{-1} \mathbf{K}_i + \mathbf{S}_x^{-1} \right )^{-1} \mathbf{K}_i^T\mathbf{S}_e^{-1}
\left(\vec{y} - \mathbf{F}(\vec{x}) - K(\vec{x} - \vec{x}_a) \right ) \\
&= \vec{x}_a  + \mathbf{S}_x \mathbf{K}_i ^ T
 \left (\mathbf{K}_i \mathbf{S}_x \mathbf{K}_i^T + \mathbf{S}_e) \right )^{-1}
\left ( \mathbf{F}(\vec{X}_i) - \vec{y} - \mathbf{K}_i(\vec{x}_i - \vec{x}_a) \right)
\end{eqnarray}
The names of these two forms derive from the size of the linear system of equations
that must be solved in each iteration step, which is $n \times n$ for the first
form and $m \times m$ for the seconds form. Moreover, the forms differ in whether
or not they involve the covariance only as its inverse or only directly.

\subsubsection{Levenberg-Marquardt}
In the Levenberg-Marquardt method, an additional damping term is added to the
matrix $\mathbf{H}$ in Eq. (\ref{eq:oem:gn}):


\begin{eqnarray}
\mathbf{H} = \mathbf{K}^T\mathbf{S}_e^{-1}\mathbf{K} + \mathbf{S}_x^{-1} +\gamma\mathbf{D}
\end{eqnarray}

The additional term $\gamma \mathbf{D}$ may be viewed as an adaptive regularization that
can help the optimization convergence in non-linear problems. The ARTS implementation
uses the diagonal of the inverse of the covariance matrix $\mathbf{S}_x^{-1}$.

The value of $\gamma$ is decreased when the decrease in $\mathcal{L}$ from an iteration
step matches that expected from a quadratic fit to the loss function. If that is not
the case but the value of $\mathcal{L}$ at $\vec{x}_{i + 1}$ is still lower than at
$\vec{x}_i$ the step is accepted and $\gamma$ is kept constant. If, contrarily, 
$\mathcal{L}(\vec{x}_{i+1})$ is higher than  $\mathcal{L}(\vec{x}_i)$
 $\gamma$ is increased and $\vec{x}_{i+1}$  recomputed with the new $\gamma$.


\subsubsection{Conjugate gradient solver}
\label{sec:oem:cg}
As mentioned above, the ARTS OEM implementation also allows using a CG solver
to solve the linear systems occuring in each iteration step of the OEM. Using
a CG solver is advantageous for retrieval problems with very high-dimensional
state and measurement spaces. The CG solver can be used for any of the optimization
methods described above and is enable simply by appending the \shortcode{\_cg} suffix
to the \shortcode{method} GIN of the OEM WSM.

\subsubsection{Convergence}

In the non-linear case the optimization iterations are continued until 
a convergence criterion is met. This implementation uses the approximation
of the $\chi^2$ value from  Eq. (5.28) in \citet{rodgers:00}, which is
 described in Eq. (5.31) on the same page:

\begin{eqnarray}
\chi^2 \approx (\vec{x}_{i + 1} - \vec{x}_i)^T 
(\mathbf{K}_i^T \mathbf{S}_e^{-1} (\vec{y} - \mathbf{F}(\vec{x}_i))
 - \mathbf{S}_x^{-1}(\vec{x} - \vec{x}_a)
\end{eqnarray}
.

\subsection{Diagnostic quantities}

One of the advantages of the OEM is that it provides several diagnostic
quantities that allow a thorough characterization of the retrieval. After
a successful iteration the OEM returns the Jacobian in
\shortcode{jacobian} as well as the  gain matrix 
%
\begin{eqnarray}
\mathbf{G} = (\mathbf{K}^T \mathbf{S}_e\mathbf{K} + \mathbf{S}_a^{-1})^{-1}
\mathbf{K}^T\mathbf{S}_e^{-1}
\end{eqnarray}
%
in \shortcode{dxdy}. A successful OEM calculation is a precondition for
the calculation of any of the diagnostic quantities described below.

\subsubsection{Averaging kernel matrix}
The averaging kernel matrix is defined as
%
\begin{eqnarray}
\mathbf{A} &= \mathbf{G}\mathbf{K}.
\end{eqnarray}
%
and provides information on the contribution of the measurement and 
a priori information on the retrieved state.  In arts the averaging
kernel matrix can be computed using the \shortcode{avkCalc}.

\subsubsection{Smoothing error}

The covariance matrix of the smoothing error, i.e. the contribution
 to the overall error caused by the finite resolution of the observation
 system is given by
%
\begin{eqnarray}
\mathbf{S}_s &= (\mathbf{A} - \mathbf{I})\mathbf{S}_a(\mathbf{A} -\mathbf{I})
\end{eqnarray}
%
In ARTS, the smoothin error can be computed using the \shortcode{covmat\_ssCalc}
workspace method.

\subsubsection{Retrieval noise}

The second component that contributes to the overall retrieval error is
the error caused by the random errors in the measurement vector $\vec{y}$.
  The covariance of this retrieval noise is given by
%
\begin{eqnarray}
\mathbf{S}_o &= \mathbf{G}\mathbf{S}_e\mathbf{G}^T.
\end{eqnarray}
%
In ARTS, the retrieval error covariance matrix can be computed using the
 \shortcode{covmat\_soCalc} workspace method. For this, the user should keep
in mind that for $\mathbf{S}_o$ to provide a realistic description of the
error caused by measurement errors $\mathbf{S}_e$ must take into account all
forward model errors that cause deviations between the simulated and true
observations.  The covariance of the total retrieval error can be obtained
 by adding the covariance matrices of the smoothing error and the retrieval
 noise covariance matrix.
