%
% === General definitions and packages to use
%
\documentclass[11pt,twoside,a4paper,fleqn,draft]{book} 
%
%\includeonly{development}
%
\usepackage[dvips]{graphicx}    % includegraphics
\usepackage[]{fancyhdr}   
\usepackage[]{natbib}
\usepackage{alltt}
%
\sloppy


%
% === Margins
%
\voffset-2.0cm
\headheight16pt
\headsep1.1cm
\textheight22cm
\hoffset-1.3cm
\oddsidemargin2.2cm
\textwidth14.0cm


%
% === Headings
%
\pagestyle{fancy}
\renewcommand{\chaptermark}[1]{\markboth{#1}{}}
\renewcommand{\sectionmark}[1]{\markright{\thesection\ #1}}
\fancyhf{}
\fancyhead[LE,RO]{\small{\sc\thepage}}
\fancyhead[LO]{\small{\scshape\rightmark}}
\fancyhead[RE]{\small{\scshape\leftmark}}
\renewcommand{\headrulewidth}{0.5pt}
\renewcommand{\footrulewidth}{0pt}
\fancypagestyle{plain}{%
  \fancyhead{}
  \renewcommand{\headrulewidth}{0pt}
}  


%
% === Page layout definitions
%
% Chapters and sections.
\newcommand{\levela}[1]    {\chapter{#1}}
\newcommand{\levelb}[1]    {\section{#1}}
\newcommand{\levelc}[1]    {\subsection{#1}}
%
% Document history
\newcommand{\starthistory} {\begin{table}[b]  \begin{tabular}{l p{11cm}} 
                             \hline {\bf History} & \\ }
\newcommand{\stophistory}  {\end{tabular} \end{table} }
%
% Symbol table
\newcommand{\startsymbols} {\begin{table}[t]  \begin{tabular}{l l l}
                            {\bf Here} & {\bf In ARTS} & {\bf Description} 
                            \\ \hline \\ } 
\newcommand{\stopsymbols}  {\\ \hline \end{tabular} 
                            \caption{Symbols used in this chapter and the
                            corresponding ARTS notation. }
                            \end{table} }      



%
% === Symbol definitions
%   Only the most general symbols are defined here. Other variables
%   are local for each part.
%
% == Scalars
% Monochromatic pencil beam intensity
\newcommand{\mpbi}       {\ensuremath{I}}  
% Frequency  
\newcommand{\f}          {\ensuremath{\nu}}  
% Viewing angle from zenith
\newcommand{\view}       {\ensuremath{\phi}}  
%
% == Forward, inverse and transfer models
% Total forward model
\newcommand{\fm}         {\ensuremath{\mathcal{F}}}  
% Atmospgeric forward model  
\newcommand{\fma}        {\ensuremath{\mathcal{F}_a}}  
% Sensor forward model  
\newcommand{\fms}        {\ensuremath{\mathcal{F}_s}}  
% Inverse model  
\newcommand{\im}         {\ensuremath{\mathcal{I}}}  
% Transfer model  
\newcommand{\tm}         {\ensuremath{\mathcal{T}}}  
%
% == Vectors and matrices of Rodgers formalism
% Measurement vector
\newcommand{\y}          {\ensuremath{\mathbf{y}}}  
% Vector of monochromatic pencil beam intensities
\newcommand{\iv}         {\ensuremath{\mathbf{i}}}  
% Vector of monochromatic pencil beam intensities
\newcommand{\merr}       {\ensuremath{\varepsilon}}  
% Total state vector
\newcommand{\xt}         {\ensuremath{\mathbf{x}}}  
% Retrieved state vector
\newcommand{\xret}       {\ensuremath{\hat{\mathbf{x}}}}
% Total forward model parameter vector
\newcommand{\bt}         {\ensuremath{\mathbf{b}}}  
% Inverse model parameters
\newcommand{\ct}         {\ensuremath{\mathbf{c}}}  
% Weighting function matrix
\newcommand{\K}          {\ensuremath{\mathbf{K}}}  
% State vector weighting function matrix
\newcommand{\Kx}         {\ensuremath{\mathbf{K_x}}}  
% Model parameter weighting function matrix
\newcommand{\Kb}         {\ensuremath{\mathbf{K_b}}}  
% Contribution function matrix
\newcommand{\Dy}         {\ensuremath{\mathbf{D_y}}}  
% Averaging kernel matrix
\newcommand{\A}          {\ensuremath{\mathbf{A}}}  
% Total sensor and data reduction matrix
\newcommand{\Hm}         {\ensuremath{\mathbf{H}}}  
% Sensor matrix
\newcommand{\Hs}         {\ensuremath{\mathbf{H}_s}}  
% Data reduction matrix
\newcommand{\Hd}         {\ensuremath{\mathbf{H}_d}}  
% Transformation between vector spaces
\newcommand{\B}          {\ensuremath{\mathbf{B}}}  
%
% == Gemeral math
% Make matrix
\newcommand{\mat}[1]     {\ensuremath{\mathbf{#1}}}
% Identity matrix
\newcommand{\Id}         {\ensuremath{\mathbf{I}}}  
% Matrix sizes
\newcommand{\msize}[2]   {{\bf R}^{#1 {\mathrm x} #2}}
% Differential d
\newcommand{\dd}         {\ensuremath{\mathrm{d}}}  
%
% == Plotting curves
\def \lsolid     {\mbox{------}}
\def \ldashed    {\mbox{--~--~--}}
\def \ldashdot   {\mbox{--~$\cdot$~--}}
\def \ldotted    {\mbox{$\cdot~\cdot~\cdot$}}

  



%===   Start of report   ===================================================
\begin{document}
%
\bibliographystyle{agu}


%
% === Title page
%
\thispagestyle{plain}
\begin{center}
  \vspace*{2cm}
  {\Huge \verb|ARTS User Guide|\\}
  \vspace*{1cm}
  {\large by \\}
  \vspace*{1cm}
  {\large \bf Stefan B\"uhler\\Institute of Remote Sensing\\University of Bremen, Germany\\}
  \vspace*{3mm}
  {\large and\\}
  \vspace*{3mm}
  {\large \bf Patrick Eriksson\\Department of Radio and Space Science\\Chalmers University of Technology, Sweden\\}
  \vspace*{2cm}
  {\large \today\\
    ARTS Version \input{version.tex}}
\end{center}
  \vspace*{4cm}
{\large \bf
  \noindent
  This is a working document. The implementation approaches and the
  algorithms are preliminary and can be subject to changes. In addition,
  not all features described in this document are implemented in ARTS.
  
  We welcome gladly comments and reports on errors in the document.
  Send then an e-mail to: \verb|patrick@rss.chalmers.se| or 
  \verb|sbuehler@uni-bremen.de|.
}

\newpage                          
\thispagestyle{empty}
\cleardoublepage


\pagenumbering{arabic}
     


%
% === The chapters
%
%
% To start the document, use
%  \levela{...}
% For lover level, sections use
%  \levelb{...}
%  \levelc{...}
%
\levela{Theoretical formalism}
 \label{sec:formalism}

%
% Document history, format:
%  \starthistory
%    date1 & text .... \\
%    date2 & text .... \\
%    ....
%  \stophistory
%
\starthistory
  000306 & Written by Patrick Eriksson, partly 
           based on \citet{eriksson:99} and \citet{eriksson:00a}. \\
\stophistory



%
% Symbol table, format:
%  \startsymbols
%    ... & \verb|...| & text ... \\
%    ... & \verb|...| & text ... \\
%    ....
%  \stopsymbols
%
%
%\startsymbols
%  \mpbi   & \verb|y|      & monochromatic pencil beam intensity      \\
%  \f      & \verb|f_mono| & monochromatic frequency                  \\
%  \view   & \verb|za_pencil| & pencil beam zenith angle              \\
%  \iv     & \verb|y|      & vector of monochromatic pencil beam intensities \\
%  \y      & \verb|y|      & spectrum recorded by a sensor            \\
%  \fm     & \verb|-|      & forward model                            \\
%  \fma    & \verb|-|      & atmospheric part of \fm                  \\
%  \fms    & \verb|-|      & sensor part of \fm                       \\
%  \xt     & \verb|-|      & state vector (variables to be retrieved) \\
%  $\xt_r$ & \verb|-|      & atmospheric part of \xt                  \\
%  $\xt_s$ & \verb|-|      & sensor part of \xt                       \\
%  $\xt_\merr$& \verb|-|   & part of \xt\ describing measurement errors   \\
%  \bt     & \verb|-|      & forward model parameter vector           \\
%  $\bt_r$ & \verb|-|      & atmospheric part of \bt                  \\
%  $\bt_s$ & \verb|-|      & sensor part of \bt                       \\
%  $\bt_\merr$& \verb|-|   & part of \bt\ describing measurement errors   \\
%  \Kx     & \verb|k|      & state weighting function matrix          \\
%  \Kb     & \verb|k|      & model parameter weighting function matrix\\  
% \label{symtable:formalism}     
%\stopsymbols



%
% Introduction
%
In this section a theoretical framework for the forward model is
presented. The presentation follows \citet{rodgers:90}, but some
extensions are made, for example, the distinction between the
atmospheric and sensor parts of the forward model is also discussed.
After this chapter was written, C.D. Rodgers published a textbook
\citep{rodgers:00} presenting the formalism in more detail than
\citet{rodgers:90}. Modelling of sensor characteristics is not yet
included in ARTS (this part is so far covered by AMI, see Section
\ref{sec:concept:scope}), but treatment of the sensor is here included
for completness.



\levelb{The forward model}
 \label{sec:formalism:fm}
 
 The radiative intensity, \mpbi, at a point in the atmosphere, $r$, for
 frequency \f\ and traversing in the direction, \view, is dependent
 on a variety of physical processes and continuous variables such as
 the temperature profile, $T$:

 \begin{equation}
   \mpbi = F(r,\f,\view,T,\dots)
 \end{equation} 
 To detect the spectral radiation some kind of sensor, having a finite
 spatial and frequency resolution, is needed, and the observed
 spectrum becomes a vector, \y, instead of a continuous function.
 The atmospheric radiative transfer is simulated by a computer model
 using a limited number of parameters as input, and the forward model,
 \fm, used in practice can be expressed as
 
 \begin{equation}
   \y = \fm(\xt_\fm,\bt_\fm) + \merr(\xt_\merr,\bt_\merr)
  \label{eq:formalism:fm}
 \end{equation}
 where $(\xt_\fm,\bt_\fm)$ and $(\xt_\merr,\bt_\merr)$ together give a
 total description of both the atmospheric and sensor states, and
 \merr\ is the measurement errors. The parameters are divided in such
 way that \xt, the state vector, contains the parameters to be
 retrieved, and the remainder is given by \bt, the model parameter
 vector. The total state vector is
 \begin{equation}
   \xt = \left[ \begin{array}{c} \xt_\fm \\ \xt_\merr \end{array} \right]
 \end{equation}
 and the total model parameter vector is
 \begin{equation}
   \bt = \left[ \begin{array}{c} \bt_\fm \\ \bt_\merr \end{array} \right]
 \end{equation}
 The actual forward model consists of either empirically determined
 relationships, or numerical counterparts of the physical
 relationships needed to describe the radiative transfer and sensor
 effects. The forward model described here is mainly of the latter
 type, but some parts are more based on empirical investigations, such
 as the parameterisations of continuum absorption. 
  
 Both for the theoretical formalism and the practical implementation,
 it is suitable to make a separation of the forward model into two
 main sections, a first part describing the atmospheric radiative
 transfer for pencil beam (infinite spatial resolution) monochromatic
 (infinite frequency resolution) signals \citep{eriksson:99},

 \begin{equation}
   \iv = \fma(\xt_r,\bt_r)
  \label{eq:formalism:fma}
 \end{equation}
 and a second part modelling sensor characteristics,
 \begin{equation}
   \y = \fms(\iv,\xt_s,\bt_s) + \merr(\xt_\merr,\bt_\merr)
  \label{eq:formalism:fms}
 \end{equation}
 where \iv\ is the vector holding the spectral values for the
 considered set of frequencies and viewing angles
 ($\iv^i=I(\f^i,\view^i)$, where $i$ is the vector index), and
 $\xt_\fm$ and $\bt_\fm$ are separated correspondingly, that is,
 $\xt_\fm^T= [\xt_r^T,\xt_s^T]$ and $\bt_\fm^T= [\bt_r^T,\bt_s^T]$.
 The vectors \xt\ and \bt\ can now be expressed as
 \begin{equation}
   \xt = \left[ \begin{array}{c} \xt_r\\ \xt_s \\ \xt_\merr \end{array} \right]
 \end{equation}
 and
 \begin{equation}
   \bt = \left[ \begin{array}{c} \bt_r\\ \bt_s \\ \bt_\merr \end{array}\right],
 \end{equation}
 respectively.

 The subscripts of \xt\ and \bt\ are below omitted as the distinction should be clear by the context. 



\levelb{The sensor transfer matrix} 
 \label{sec:formalism:sensor}
  
 The modelling of the different sensor parts can be described by a
 number of of analytical expressions (see \citet{eriksson:97a}) that
 together makes the basis for the sensor model. These expressions are
 throughout linear operations and it possible, as suggested in
 \citet{eriksson:00a}, to implement the sensor model as a
 straightforward matrix multiplication:
 \begin{equation}
   \y = \Hm \iv + \merr
  \label{eq:formalism:H}
 \end{equation}
 where \Hm\ is here denoted as the sensor transfer matrix.  The matrix
 \Hm\ further incorporate effects of a data reduction and
 the total transfer matrix is then
 \begin{equation}
   \Hm = \Hd \Hs
  \label{eq:formalism:Hs}
 \end{equation}
 as
 \begin{equation}
   \y = \Hd \y' = \Hd (\Hs \iv + \merr') = \Hm \iv + \merr
  \label{eq:formalism:datared}
 \end{equation}
 where \Hd\ is the reduction matrix, \Hs\ the sensor matrix, and $\y'$
 and $\merr'$ are the measurement vector and the measurement errors,
 respectively, before data reduction. The matrices \Hd\ and \Hs\ are
 described in Section \ref{sec:sensor} and \ref{sec:red}, respectively.



\levelb{Weighting functions} 
 \label{sec:formalism:wfuns}

 \levelc{Basics} 
 A weighting function is the partial derivative of the spectrum vector
 \y\ with respect to some variable used by the forward model.  As the
 input of the forward model is divided between \xt\ or \bt, the
 weighting functions are divided correspondingly between two matrices,
 the state weighting function matrix

 \begin{equation}
   \Kx = \frac{\partial \y}{\partial \xt}
  \label{eq:formalism:kx}
 \end{equation}
 and the model parameter weighting function matrix
 \begin{equation}
   \Kb = \frac{\partial \y}{\partial \bt}
  \label{eq:formalism:kb}
 \end{equation}
 For the practical calculations of the weighting functions, it is
 important to note that the atmospheric and sensor parts can be
 seperated. For example, if \xt\ only hold atmospheric and
 spectroscopic variables, \Kx\ can be expressed as

 \begin{equation}
   \Kx = \frac{\partial \y}{\partial \iv}\frac{\partial\iv}{\partial \xt} =
    \Hm\frac{\partial\iv}{\partial \xt}
  \label{eq:formalism:kx2}
 \end{equation}
 This equation shows that the new parts needed to calculate
 atmospheric weighting functions, are functions giving $\partial\iv /
 \partial \xt$ where \xt\ can represent the vertical profile of a
 species, atmospheric temperature, spectroscopic data etc.
% The practical calculation of weighting functions is discussed in
% detail in Sections \ref{sec:wfuns} and \ref{sec:wfuns_sens}.


 \levelc{Transformation between vector spaces}
 
 It could be of interest to transform a weighting function matrix from
 one vector space to another\footnote{This subject is also discussed
   in \citet{rodgers:00} (published after writing this).}. The new
 vector, $\xt'$, is here assumed to be of length $n$ $(\xt' \in
 \msize{n}{1})$, while the original vector, \xt\ is of length $p$
 $(\xt \in \msize{p}{1})$.  The relationship between the two vector
 spaces is described by a transformation matrix $\B$:
  \begin{equation}
    \xt = \B\xt'
  \end{equation}
  where $\B \in \msize{p}{n}$. For example, if $\xt'$ is assumed to be
  piecewise linear, then the columns of $\B$ contain tenth functions,
  that is, a function that are 1 at the point of interest and decreases
  linearly down to zero at the neighbouring points.  The matrix can
  also hold a reduced set of eigenvectors.
    
  The weighting function matrix corresponding to $\xt'$ is
  \begin{equation}
    \K_{\xt'} = \frac{\partial \y}{\partial \xt'}
  \end{equation}
  This matrix is related to the weighting function matrix of \xt\ (Eq.
  \ref{eq:formalism:kx}) as
  \begin{equation}
    \K_{\xt'}
      = \frac{\partial \y}{\partial \xt} \frac{\partial \xt}{\partial \xt'}
      = \frac{\partial \y}{\partial \xt} \B 
      = \Kx \B
  \end{equation}
  Note that
  \begin{equation}
    \K_{\xt'}\xt' = \Kx\B\xt' =  \Kx\xt
  \end{equation}
  However, it should be noted that this relationship only holds for
  those \xt\ that can be represented perfectly by some $\xt'$ (or vice
  versa), that is, $\xt=\B\xt'$, and not for all combinations of \xt\ 
  and $\xt'$.

  If $\xt'$ is the vector to be retrieved, we have that \citep{rodgers:90}
  \begin{equation}
    \xret' = \im(\y,\ct) = \tm(\xt,\bt,\ct)
  \end{equation}
  where \im\ and \tm\ are the inverse and transfer model, respectively.

  The contribution function matrix is accordingly
  \begin{equation}
    \Dy =  \frac{\partial \xret'}{\partial \y}
  \end{equation}
  that is, \Dy\ corresponds to $\K_{\xt'}$, not \Kx.
  
  We have now two possible averaging kernel matrices
  \begin{equation}
    \A_{\xt} 
      = \frac{\partial \xret'}{\partial \xt} 
      = \frac{\partial \xret'}{\partial \y} \frac{\partial \y}{\partial \xt}
      = \Dy \Kx
  \end{equation}
  \begin{equation}
    \A_{\xt'} 
      = \frac{\partial \xret'}{\partial \xt'} 
      = \frac{\partial\xret'}{\partial\y}\frac{\partial\y}{\partial\xt}
      \frac{\partial\xt}{\partial\xt'}
      = \Dy \K_{\xt'}
      = \A_{\xt} \B
  \end{equation}
  where $\A_{\xt} \in \msize{p}{n}$ and $\A_{\xt'} \in \msize{p}{p}$,
  that is, only $\A_{\xt'}$ is square. If $p>n$, $\A_{\xt}$ gives more
  detailed information about the shape of the averaging kernels than
  the standard matrix ($\A_{\xt'}$). If the retrieval grid used is
  coarse, it could be the case that $\A_{\xt'}$ will not resolve all
  the oscillations of the averaging kernels, as shown in
  \citet[][Figure 11]{eriksson:99}.

  
  

    




%%% Local Variables: 
%%% mode: latex
%%% TeX-master: "uguide"
%%% End: 

\chapter{Clear-sky radiative transfer}
 \label{sec:rte}


 \starthistory
 130220 & Revised after parts moved to a new chapter (Patrick Eriksson).\\
 120831 & Added flowchart and sections on polarised absorption, 
       \builtindoc{iyCalc}, auxiliary data and dispersion (Patrick Eriksson).\\
 110611 & Extended and general revision (Patrick Eriksson).\\
 050613 & First complete version by Patrick Eriksson.\\
 \stophistory

\graphicspath{{Figs/rte/}}

This section discusses variables and the approach used to handle the actual
radiative transfer calculations. This includes how effects caused by the sensor
and surface are incorporated. Measurements of thermal emission in absence of
particle scattering are used as example, and the basic theory for such
simulations is also covered. The first ARTS version was developed for emission
measurements, and such observations remain the standard case in ARTS.

A basic assumption for this chapter is thus that there is no particle
scattering. This is denoted as clear-sky calculations. 
Scattering is restricted to the ``cloud box''
(Sec.~\ref{sec:fm_defs:cloudbox}). In short, the more demanding calculations
are restricted to a smaller domain of the model atmosphere, and the radiative
transfer in that domain is mainly treated by dedicated workspace methods. For
pure transmission measurements (where scattering into the line-of-sight is
neglected), see Chapter~\ref{sec:trans}. This chapter discusses only the direct
radiative transfer, partial derivatives (i.e.\ the
Jacobian or weighting functions) are discussed in Section~\ref{sec:wfuns}. 

Absorption by atmospheric gases does normally not depend on polarisation but
exceptions exist, where Zeeman splitting is one example. Both polarised and
unpolarised absorption is handled. Even if the gaseous absorption in itself is
unpolarised, the expressions to apply must allow that polarisation signals
from the surface and the cloud box are correctly propagated to the sensor.

For an introduction to a complete radiative transfer calculations, see
Chapter~\ref{sec:complcalcs}. For example, the content of this chapter
corresponds roughly to the flowchart displayed in Figure~\ref{fig:ycalc_flow},
outlining a standard radiative transfer emission calculation. In fact, this
chapter can be seen as a direct continuation of Chapter~\ref{sec:complcalcs}.



\section{Overall calculation procedure}
%===================
\label{sec:fm_defs:calcproc}

The structure handling complete radiative transfer calculations is fixed, where
the main workspace method is denoted as \wsmindex{yCalc}
(Fig.~\ref{fig:ycalc_flow}). That is, most ARTS control files include a call of
\builtindoc{yCalc} and this section outlines this method and the associated
main variables.

The calculation approach fits with the formalism presented in Sections
\ref{T-sec:formalism:fm}-\ref{T-sec:formalism:sensor} of \theory, where the
separation between atmospheric radiative transfer and inclusion of sensor
effects shall be noted especially, and a similar nomenclature is used here:
\begin{description}
\item[\MsrVct]: Complete measurement vector. In addition to atmospheric
  radiative transfer, the vector can include effects by sensor characteristics
  and data reduction operations. The corresponding workspace variable is 
  \wsvindex{y}.
\item[\aMpiVct{b}]: Monochromatic pencil beam data for a measurement block. The
  definition of a measurement block is found in
  Section~\ref{sec:fm_defs:seqsandblocks}. This vector is only affected by
  atmospheric radiative transfer. As workspace variable denoted as
  \wsvindex{iyb}, but can be considered as a pure internal variable and should
  not be of concern for the user.
\item[\aMpiVct{y}]: Monochromatic data for one line-of-sight, i.e.\ a single
  pencil beam calculation. The corresponding workspace variable is
  \wsvindex{iy}. (\aMpiVct{b} consists of one or several \aMpiVct{y} appended.)
\item[\aSnsMtr{b}]: The complete sensor response matrix, for a measurement
  block. Can include data reduction. The corresponding workspace variable is
  \wsvindex{sensor\_response}.
\end{description}
\begin{algorithm}[t]
 \begin{algorithmic}
  \STATE{allocate memory for the matrix $\MsrVct$}
  \COMMENT{Equation \ref{eq:fm_defs:measseq}}
  \STATE{allocate memory for the matrix \aMpiVct{b}}
  \COMMENT{Equation \ref{eq:fm_defs:freqs_of_ib}}
  \FORALL{sensor positions}
   \FORALL[Section \ref{sec:fm_defs:seqsandblocks}]
                                    {pencil beam directions of the block}
    \STATE{call \builtindoc{iy\_main\_agenda}, giving \aMpiVct{y}}
    \COMMENT{Algorithm \ref{alg:fm_defs:iyCSagenda}}
    \STATE{copy \aMpiVct{y} to correct part of \aMpiVct{b}}
   \ENDFOR
   \STATE{put the product \aSnsMtr{b}\aMpiVct{b} in correct part of 
          $\MsrVct$}
  \ENDFOR
 \end{algorithmic}
 \caption{Outline of the overall clear sky radiative transfer calculations
   (\builtindoc{yCalc}).}
 \label{alg:fm_defs:yCalc}
\end{algorithm}
\begin{algorithm}[t]
 \begin{algorithmic}
   \STATE{determine the propagation path by \builtindoc{ppath\_agenda}}
   \COMMENT{Section \ref{sec:fm_defs:ppaths}}
   \STATE{determine the radiation at the start of the propagation path}
   \COMMENT{Section \ref{sec:fm_defs:rad_bkgr}}
   \STATE{perform radiative transfer along the propagation path}
   \COMMENT{Section \ref{sec:fm_defs:rte}}
   \STATE{unit conversion of \aMpiVct{y} following \wsvindex{iy\_unit}}
   \COMMENT{Section \ref{sec:fm_defs:unit}}
 \end{algorithmic}
 \caption{The main operations for methods to be part of
   \wsaindex{iy\_main\_agenda}.}
 \label{alg:fm_defs:iyCSagenda}
\end{algorithm}
The \builtindoc{yCalc} method is outlined in Algorithm~\ref{alg:fm_defs:yCalc}.
For further details of each calculation step, see the indicated equation or
section. In summary, \builtindoc{yCalc} appends data from different pencil beam
calculations and applies the sensor response matrix (\aSnsMtr{b}). The actual
radiative transfer calculations are not part of \builtindoc{yCalc}.

Atmospheric radiative transfer is solved for each pencil beam direction
(line-of-sight) separately. It is the task of \wsaindex{iy\_main\_agenda}
(Algorithm~\ref{alg:fm_defs:iyCSagenda}) to perform a single clear sky
radiative transfer calculation. This agenda, in its turn, makes us of other
agendas, such as \builtindoc{ppath\_agenda}. All methods developed for
\builtindoc{iy\_main\_agenda} adapt automatically to the value of
\wsvindex{stokes\_dim}.

That is, \builtindoc{yCalc} is a common method, independent of the details of
the radiative transfer. For example, \builtindoc{yCalc} is used both if emission
measurements or pure transmission data are simulated, that choice is made
inside \builtindoc{iy\_main\_agenda}. 
The three following sections describes the main calculation steps of 
\builtindoc{iy\_main\_agenda}, in the order they are executed.


\section{Propagation paths}
%===================
\label{sec:fm_defs:ppaths}

A pencil beam path through the atmosphere to reach a position along a specific
line-of-sight is denoted as the \textindex{propagation path}. Propagation paths
are described by a set of points on the path, and the distance along the path
between the points. These quantities, and a number of auxiliary variables, are
stored together in a structure described in Section~\ref{sec:ppath:Ppath}. The
path points are primarily placed at the crossings of the path with the
atmospheric grids (\builtindoc{p\_grid}, \builtindoc{lat\_grid} and
\builtindoc{lon\_grid}). A path point is also placed at the sensor if it is
placed inside the atmosphere. Points of surface reflections
are also included if such exist. More points can also be added to the
propagation path, for example, by setting an upper limit for the distance along
the path between the points. This is achieved by the variable
\builtindoc{ppath\_lmax}, see further Sections \ref{sec:fm_defs:accuracy} and
\ref{sec:ppath:usage}.

\begin{figure}[p]
 \begin{center}
  \includegraphics*[width=0.99\hsize]{ppath_cases2}
  \caption{Examples on allowed propagation paths for a 2D atmosphere. The
    atmosphere is plotted as in Figure~\ref{fig:fm_defs:2d} beside that the
    points for the atmospheric fields are not emphasised. The position of the
    sensor is indicated by an asterisk $(\ast)$, the points defining the paths
    are plotted as circles $(\circ)$, joined by a solid line. The part of the
    path outside the atmosphere, not included in the path structure, is shown
    by a dashed line. Path points corresponding to a tangent point are marked
    by an extra plus sign $(\oplus)$; but note that these no longer are
    explicitly included as path point (in contrast to ARTS-2.0 and earlier).
    The shown paths include the minimum set of definition points. There exists
    also the possibility to add points inside the grid cells, for example, to
    ensure that the distance between the path points does not exceed a
    specified limit.}
  \label{fig:fm_defs:ppath_cases2}
 \end{center}
\end{figure}
% This figure was produced by the Matlab function mkfigs_ppath_cases.

\begin{figure}[p]
 \begin{center}
  \includegraphics*[width=0.99\hsize]{ppath_cases1}
  \caption{Examples on allowed propagation paths for a 1D atmosphere
    with an activated cloud box. Plotting symbols as in
    Figure~\ref{fig:fm_defs:ppath_cases2}. When the sensor is placed 
    inside the cloud box, the path is defined with a single point, 
    to know for which position and line-of-sight the intensity field of
    the cloud box shall be interpolated. }
  \label{fig:fm_defs:ppath_cases1}
 \end{center}
\end{figure}
% This figure was produced by the Matlab function mkfigs_ppath_cases.

%\begin{figure}
% \begin{center}
%  \includegraphics*[width=0.95\hsize]{ppath_badcases}
%  \caption{Examples on \emph{not} allowed propagation paths for a 2D 
%    atmosphere. The constraints for allowed paths are discussed in the
%    text.}
%  \label{fig:fm_defs:ppath_badcases}
% \end{center}
%\end{figure}
%% This figure was produced by the Matlab function mkfigs_ppath_cases.


The propagation paths are determined basically by starting at the
sensor and following the path backwards by some \textindex{ray
  tracing} technique. If the sensor is placed above the model
atmosphere, geometrical calculations are used (as there is no
refraction in space) to find the crossing between the path and the top
of the atmosphere where the ray tracing then starts. Paths are tracked
backwards until the top of the atmosphere or to an
intersection with the cloud box or the surface. The propagation path
(or paths) before a surface reflection is calculated when determining
the up-welling radiation from the surface
(Section~\ref{sec:fm_defs:surface}). Example on propagation
paths are shown in Figures~\ref{fig:fm_defs:ppath_cases2} and 
\ref{fig:fm_defs:ppath_cases1}.
 
Not all propagation paths are allowed for 2D and 3D. In short, the paths can
only enter and leave the model atmosphere at the top of the atmosphere, as the
atmospheric fields are treated to be undefined outside the covered latitude and
longitude ranges.

Controlled by \builtindoc{ppath\_step\_agenda}, propagation paths can be 
calculated purely geometrically or considering refraction. When considering 
refraction, the refractive index is determined at each point along the path 
according to \builtindoc{refr\_index\_agenda}. Details about different methods 
applicable within \builtindoc{refr\_index\_agenda} are given in 
Chapter~\ref{sec:rindex}.

If nothing else is stated, it assumed that all frequency components share a
single propagation path. Another way to express this assumption is that
\textindex{dispersion} is neglected. See Section~\ref{sec:fm_defs:dispersion}
for how to consider dispersion. In the non-dispersive case, the propagation
path is valid for average of the first and last element in \wsvindex{f\_grid},
as this is the frequency given to \wsaindex{refr\_index\_agenda}.

Propagation paths can be calculated separately by the method
\wsmindex{ppathCalc}, but for standard calculations the propagation paths are
calculated internally by \builtindoc{yCalc}. Methods and variables to control
the path calculations are discussed in Section~\ref{sec:ppath:usage}. 



\section{The radiative background}
%===================
\label{sec:fm_defs:rad_bkgr}

The radiative intensity at the starting point of the path, and in the
direction of the line-of-sight at that point, is denoted as the
\textindex{radiative background}. Four possible radiative backgrounds
exist:
\begin{description}
\item[Space] When the propagation path starts at the top of the
  atmosphere, space is the radiative background. The normal case
  should be to set the radiation at the top of the atmosphere to be
  cosmic background radiation. An exception is when the sensor is
  directed towards the sun. The radiative background at the top of the
  atmosphere is determined by \wsaindex{iy\_space\_agenda}. If a
  propagation path is totally outside the model atmosphere, the
  observed monochromatic pencil beam intensity (\aMpiVct{y}\ in
  Algorithm~\ref{alg:fm_defs:yCalc}) equals the output of
  \builtindoc{iy\_space\_agenda}.
\item[The surface] The sum of surface emission and radiation reflected by the
  surface is the radiative background when the propagation path intersects with
  the surface. It is the task of \wsaindex{iy\_surface\_agenda} to return this
  up-welling radiation from the surface, see further
  Chapter~\ref{sec:surface}.
\item[Surface of cloud box] For cases when the propagation path enters
  the cloud box the radiative background is the intensities leaving
  the cloud box. This radiation is obtained by
  \wsaindex{iy\_cloudbox\_agenda}. 
\item[Interior of cloud box] If the sensor is situated inside the
  cloud box, there is basically no propagation path. The radiative
  background, and also the final spectrum, equals the internal
  intensity field of the cloud box at the position of the sensor, in
  the direction of the sensor line-of-sight. This case is also handled
  by \builtindoc{iy\_cloudbox\_agenda}.
\end{description}
It should be noted that except for the first case above, the determination of
the radiative background involves further radiative transfer calculations. For
example, in the case of surface reflection, the down-welling radiation could be
determined by a new call of \builtindoc{iy\_main\_agenda} and the radiative
background for that calculation is then space or the cloud box. The intensity
field entering the cloud box is in some cases calculated by calls of
\builtindoc{iy\_main\_agenda} (with cloud box deactivated) and the radiative
background for these calculations is then space or the surface. This results
in that space is normally the ultimate radiative background for the
calculations. The exception is for propagation paths that intersects with the
surface, and the surface is treated to act as a blackbody. For such cases, the
propagation path effectively starts at the surface.




\section{Basic radiative transfer variables and expressions}
%---
\label{sec:fm_defs:rte}

This section describes how the core radiative transfer equation is solved
practically in ARTS. As mentioned, in this chapter focus is put on emission
measurements. Local thermodynamic equilibrium (LTE) is throughout assumed.
The equation to solve is Equation~\ref{eq:VRTE1}:
\begin{equation}
  \frac{\DiffD\StoVec}{\DiffD \PpathLng} = \aAbsMat{a}\left[ \EmsVec- \StoVec
  \right] = -\aAbsMat{a}\StoVec + B\aAbsVec{a},\nonumber
\end{equation}
where the involved quantities are defined and discussed in
Section~\ref{sec:rteq}.


\subsection{Unpolarised absorption}

Let's start with the simpler case of non-polarised absorption (that is, the
absorption is independent of polarisation state). For unpolarised absorption the
matrix \\aAbsMat{a}\ is diagonal, with all diagonal elements equal, and only the
first of the elements of \aAbsVec{a})\ is non-zero.

The radiative transfer equation above can be solved in many ways, and with
different level of refinement. The standard approach in ARTS is to solve the
radiative transfer from one point of the propagation path to next. For the
first Stokes element the following expression is applied (compare \theory,
Equation~\ref{T-eq:rtetheory:layer})
\begin{equation}
  \label{eq:fm_defs:rte_step}
  \aStoI{i+1} = \aStoI{i}ie^{-\aOth{i}} + \bar{\Planck}_i(1-e^{-\aOth{i}}),
\end{equation}
with
\begin{eqnarray}
  \bar{\Planck}_i &=& (\Planck(\aTmp{i})+\Planck(\aTmp{i+1}))/2, \\
  \aOth{i} &=& \Delta\aPpathLng{i}(\aAbsCoef{i}+\aAbsCoef{i+1})/2,  
  \label{eq:taustep}
\end{eqnarray}
where \aStoI{i}, \aTmp{i}\ and \aAbsCoef{i}\ are the radiance, temperature and
absorption coefficient, respectively, at point $i$ of the propagation path, and
$\Delta\aPpathLng{i}$ is the distance along the path between point $i$ and
$i+1$. That is, $\bar{B}_i$ is an average of the Planck function at the path
step end points, and the absorption is assumed to vary linearly between the two
points. The start value of \StoI\ is governed by the radiative background
(Section \ref{sec:fm_defs:rad_bkgr}).

A consequence of unpolarised absorption is that also the emission is
unpolarised, and the emission term vanishes for higher Stokes elements.
Accordingly, the expression for the second Stokes component is
\begin{equation}
  \aStoQ{i+1}(\Frq) = \aStoQ{i}(\Frq)e^{-\aOth{i}}.
  \label{eq:fm_defs:rte_step2}
\end{equation}
The third and forth Stokes component are handled likewise. The expressions
above are implemented in the workspace method \wsmindex{iyEmissionStandard},
intended to be part of \builtindoc{iy\_main\_agenda}.

An alternative way to perform the calculations for the first Stokes element
would be
\begin{equation}
  \label{eq:fm_defs:rte_alt}
  \StoI = \sum_i \aTrnMat{i+1} \bar{\Planck}_i(1-e^{-\aOth{i}}),
\end{equation}
where \StoI\ is the final intensity and \aTrnMat{i}\ is the transmission
between the sensor and point $i$. This calculation approach is not used as it
fits poorer with the calculation of weighting functions (\aStoI{i} must be
known, Section~\ref{sec:wfuns}). However, the calculation of weighting
functions is simplified if \aTrnMat{i}\ is at hand, and this quantity is
also tracked by \builtindoc{iyEmissionStandard}.


\subsection{Polarised absorption}

The overall calculation procedure is the same with polarised absorption, the
only difference is the radiative transfer expression applied. The calculations
for the different Stokes components can here not be separated, and
matrix-vector notation is required: 
\begin{equation}
  \label{eq:fm_defs:vrte_step}
  \aStoVec{i+1} = e^{-\Delta\aPpathLng{i}\bar{\ExtMat}} \aStoVec{i} + 
                  (\IdnMtr-e^{-\bar{\ExtMat}\Delta\aPpathLng{i}})\bar{\EmsVec},
\end{equation}
where \IdnMtr\ is the identity matrix. The extinction matrices and \VctStl{b}
at point $i$ and $i+1$ are averaged (element-wise) to give $\bar{\ExtMat}$ and
$\bar{\EmsVec}$, respectively, in line with Equation~\ref{eq:taustep}. The
calculation of the transmission matrix,
\begin{equation}
  \TrnMat = e^{-\Delta\aPpathLng{i}\bar{\ExtMat}},
  \label{eq:rte:transmat}
\end{equation}
involves a matrix exponential, that is calculated by the Pad\'e approximations,
as described in \developer, Section \ref{D-sec:lin_alg:mat_exp}. Only the first
element of $\bar{\EmsVec}$ is non-zero, and only the first column of the matrix
corresponding to the term $(\IdnMtr-e^{-\Delta\aPpathLng{i}\bar{\ExtMat}})$ is
of interest.



\subsection{Blackbody and cosmic background radiation}

The term \Planck\ is set by \wsaindex{blackbody\_radiation\_agenda}. The
setting of this agenda basically determines the unit of the final outcome of
Eq.~\ref{eq:fm_defs:rte_step}, see further Sec.~\ref{sec:fm_defs:unit}. For
radiance calculations, the standard workspace method to use inside
\builtindoc{blackbody\_radiation\_agenda} is
\wsmindex{blackbody\_radiationPlanck}. The Planck function is in this method,
and in ARTS generally, defined as
\begin{equation}
  \label{eq:planck}
  \Planck(\Tmp) = \frac{2\planckCns\Frq^3}
                  {\speedoflight^2(exp(\planckCns\Frq/\boltzmannCns\Tmp)-1)}
\end{equation}
where \planckCns\ is the Planck constant, \speedoflight\ the speed of light and
\boltzmannCns\ the Boltzmann constant. This expression gives the total power,
per unit frequency per unit area per solid angle. (The Planck function can also
be defined as a function of wavelength.) The expression in
Equation~\ref{eq:planck} deviates from the exact definition (see
Eq.~\ref{T-eq:rtetheory:Planck} in \theory) as it includes \speedoflight\
instead of the local propagation speed ($v$). The reason for this is the
n$^2$-law of radiance, discussed in the section below.

As long as cosmic background radiation is the only type of non-telluric
radiation that has to be considered, the standard method for inclusion in
\wsaindex{iy\_space\_agenda} is \wsmindex{MatrixCBR} (together with some calls
of \builtindoc{Ignore}). Please, note that
\builtindoc{blackbody\_radiation\_agenda} and \builtindoc{iy\_space\_agenda},
as well as \wsaindex{iy\_surface\_agenda}, must be defined in a consistent
manner (that they use the same unit for \Planck).




\section{Output unit and the n$^2$-law}
%==============================================================================
\label{sec:fm_defs:unit}

First of all, it should be noticed that ARTS does not enforce any fixed unit for
calculated spectra (\wsvindex{y}), it depends on the calculation set-up. For
example, if emission is considered, or if just transmissions are calculated.

The primary unit for emission data (radiances) is [W/(Hz$\cdot$m$^2\cdot$sr)].
The emission intensity corresponds directly with the definition of the Planck
function (Eq.~\ref{eq:planck}). Conversion to other units is selected by the
\wsvindex{iy\_unit} workspace variable. The standard manner is to apply the
unit conversion as part of the calculations performed inside
\builtindoc{yCalc}. See the built-in documentation of the workspace method you
have selected for \builtindoc{iy\_main\_agenda} for comments on practical
aspects and available output units. The most extensive support for conversion
to other units is provided by \builtindoc{iyEmissionStandard}, while other
methods have no support at all (ie.\ they ignore \builtindoc{iy\_unit}). It is
also possible to change the unit as a post-processing step by
\wsmindex{yApplyUnit} (or \wsmindex{iyApplyUnit}), but some restrictions apply
and there are no automatic checks if the input data have correct unit. Further
considerations and expressions for the unit conversion are discussed in the
ARTS-2 journal paper \citep[][Sec.~5.7]{eriksson:arts2:11}.

The n$^2$-law of radiance\index{n2-law of radiance} is introduced in
Section~\ref{T-sec:n2law} of \theory. As shown in that section, the main impact
of the law is handled by consistently using the vacuum speed in the definition
of the Planck radiation law, as done inside ARTS (Eq.~\ref{eq:planck}). This
suffices if the sensor is placed in space (where the refractive index is 1), or
if you use brightness temperatures. Remaining cases are also handled exactly if
\builtindoc{iyEmissionStandard} is used. For those remaining cases, radiance
data shall be scaled with the refractive index squared at the observation
position. For Earth, the maximum value of this factor is about 0.1\,\%, and can
anyhow normally be neglected.

In summary, there is normally no need for you as an user to consider the
n$^2$-law. The exception is if you extract radiance data for a point inside an
atmosphere, and the refractive index deviates significantly from 1 at this
point.




\section{Single pencil beam calculations}
%==============================================================================
\label{sec:fm_defs:single_beams}

The text above assumes that \wsmindex{yCalc} is used. This method can always be
used, but \builtindoc{yCalc} is not mandatory if the simulations only deal with
monochromatic data for a single line-of-sight. In this case, it could be more
handy to use \wsmindex{iyCalc}, which basically is a direct call of
\wsaindex{iy\_main\_agenda}. A reason for selecting \builtindoc{iyCalc} is that
a larger set of auxiliary quantities can be extracted
(Sec.~\ref{sec:fm_defs:aux}). 

On the input side, the main difference when using \builtindoc{iyCalc} is that
the observation position and line-of-sight are specified by \wsvindex{rte\_pos}
and \wsvindex{rte\_los} (instead of \builtindoc{sensor\_pos} and
\builtindoc{sensor\_los}). The calculated radiances are returned as the matrix
\wsvindex{iy} (instead of the vector \builtindoc{y}). 
No automatic unit conversion is made inside \builtindoc{iyCalc}. This is
instead handled separately by \wsmindex{iyApplyUnit}.




\section{Dispersion}
%==============================================================================
\label{sec:fm_defs:dispersion}

The clear-sky radiative transfer methods handle all frequencies in
\builtindoc{f\_grid} in parallel, for efficiency reasons. One consequence of
this feature is that only a single propagation path is calculated, that is
assumed to be common for all frequencies. With other words,
\textindex{dispersion} is not considered. This is in general an acceptable
simplification, but exceptions exist where one example is radiative
transfer through the ionosphere at frequencies approaching the ``plasma
frequency''.

When dispersion is expected to give a significant impact on the results, ARTS
offers a general solution. Dispersion can be handled by setting
\wsaindex{iy\_main\_agenda} as:
\begin{code}
AgendaSet( iy_main_agenda ){
  iyLoopFrequencies
}
\end{code}
The radiative transfer method you put in \builtindoc{iy\_main\_agenda} for
non-dispersive calculations are now moved to \wsaindex{iy\_sub\_agenda}. For
example, if \builtindoc{iyEmissionStandard} is the method of your choice:
\begin{code}
AgendaSet( iy_sub_agenda ){
  iyEmissionStandard
}
\end{code}
The approach is simple, \wsmindex{iyLoopFrequencies} calls
\builtindoc{iy\_sub\_agenda} for each single frequency in \builtindoc{f\_grid}
and appends the output. With some details, \builtindoc{iyLoopFrequencies}
performs a loop over the \builtindoc{f\_grid}, creates an internal
\builtindoc{f\_grid} of length 1 holding the frequency of concern and calls
\builtindoc{iy\_sub\_agenda} with this length-1 frequency grid. This has the
result that a propagation path is calculated for each frequency component.

Some more steps are required to correctly include dispersion. A basic demand is
that \wsaindex{ppath\_agenda} considers refraction. Further,
\wsaindex{refr\_index\_agenda} must provide a dispersive refractive index. Most
methods aimed for \builtindoc{refr\_index\_agenda} give a refractive index that
does not varies with frequency. An example on the opposite is
\wsmindex{refr\_indexFreeElectrons}. If a method with dispersive refractive
index is used for non-dispersive calculations, it receives the mean of the
first and last element in \builtindoc{f\_grid} (as already commented above).

A limitation of \builtindoc{iyLoopFrequencies} is that it can not be combined
with auxiliary data of along-the-path character (Sec.~\ref{sec:fm_defs:aux})






\section{Auxiliary data}
%==============================================================================
\label{sec:fm_defs:aux}

The core output of the radiative calculations is \builtindoc{y}
(\builtindoc{iy} if \builtindoc{iyCalc} is used, jacobians discussed in
Sec.~\ref{sec:wfuns}), but different auxiliary data can be extracted. First of
all, \wsmindex{yCalc} outputs automatically \wsvindex{y\_f}, \wsvindex{y\_pol},
\wsvindex{y\_pos} and \wsvindex{y\_los}. These data give information about the
frequency, polarisation, sensor position and sensor bore-sight, respectively,
corresponding to each value in \builtindoc{y}. The content of the variables are
governed by the sensor settings and the order calculated radiances are stored
(discussed in Sec.~\ref{sec:fm_defs:seqsandblocks}).

A more general mechanism for extracting auxiliary data is controlled by the
\wsvindex{iy\_aux\_vars} workspace variable. This mechanism is most useful
together with \builtindoc{iyCalc}, and for the moment we assume that this
method is used (limitations for \builtindoc{yCalc} are discussed below). The
quantities that can be extracted differ, see the built-in documentation for
the options for each workspace method of concern, e.g.:
\begin{code}
arts -d iyEmissionStandard
\end{code}
The options for this particular method  (\wsmindex{iyEmissionStandard}) can be
divided into different groups (more variables will/can be added):
\begin{description}
\item[Atmosphere, along-the-path] The pressure, temperature and volume mixing
  rations along the propagation path.
\item[Attenuation, along-the-path] Total and species specific absorption
  coefficients along the propagation path.
\item[Radiative properties, along-the-path] The radiance at each propagation
  path point.
\item[Overall radiative properties] The total (clear-sky) optical depth along
  the path and flag giving the radiative background.
\end{description}
``Along-the-path'' means that data are provided for each point of the
propagation path. The path is described by \wsvindex{ppath}, that is also
returned by \wsaindex{iy\_main\_agenda}. The \builtindoc{ppath} variable
contains the information needed to geo-position, for example, ``along-the path
temperatures''. 

Example on setting of \wsvindex{iy\_aux\_vars} (again valid for
\builtindoc{iyEmissionStandard}):
\begin{code}
ArrayOfStringSet( iy_aux_vars,  
    [ "Temperature", 
      "VMR, species 0",
      "Absorption, summed", 
      "Absorption, species 0",
      "Absorption, species 2",
      "iy", 
      "Optical depth" ] )
\end{code}
The data are outputted in a single variable, \wsvindex{iy\_aux}. This variable
is an array of Tensor4. All dimensions are used when storing e.g.\ the
propagation matrix along the path (for all frequencies of
\builtindoc{f\_grid}). For other types of quantities, one or several dimensions
are set to have length 1. See the built-in documentation for further details,
such as the order of the data dimensions.

Storage of quantities of ``along-the-path'' type assumes that there exists a
common propagation path. This is necessarily not the case for calculations by
\builtindoc{yCalc}. This is the case as a calculation considering an antenna
response includes radiative transfer along several propagation paths. The
points of these paths do not end up on common altitude grid, neither are at a
fixed distance from the sensor. In fact, the number of points of the paths will
likely differ. For this reason, \wsmindex{yCalc} will issue an error if you
in \wsvindex{iy\_aux\_vars} include a quantity of ``along-the-path'' character.

The same applies to dispersion calculations (here the propagation path differs
already between the frequencies), and also \wsmindex{iyLoopFrequencies} gives
also an error if ``along-the-path'' auxiliary data are selected.

To simplify the practical usage this mechanism to extract auxiliary data,
\builtindoc{iyEmissionStandard} also accepts some other variable, related to
particle properties, but do not trigger any calculations. The corresponding
elements of \builtindoc{iy\_aux} can at a later stage be filled with
\builtindoc{iy\_auxFillParticleVariables}. This feature can be useful for
checking if any particles are found along the propagation path, to determine if
scattering calculations are required.



\section{Calculation accuracy}
%===================
\label{sec:fm_defs:accuracy}

The accuracy of the calculations depends on many factors. For many
factors, such as spectroscopic parameters, there is nothing else to do
than using best available data. On the other hand, for other factors
there is a trade-off between accuracy and speed. More accurate
calculations require normally also more computer memory. All
different grids and the propagation path step length fall into this
category of accuracy factors. It could be worth discussing the
selection of atmospheric grids and the path step length as there can
be some confusion about how that affects the accuracy.

The main purpose of the atmospheric grids (\builtindoc{p\_grid},
\builtindoc{lat\_grid} and \builtindoc{lon\_grid}) is to build up the
mesh on which the atmospheric fields are defined. This means that the
spacing of these grids shall be selected having the representation of
the atmospheric fields in mind. That is, the spacing shall be fine
enough that the atmospheric field is sufficiently well approximated by
the piece-wise (multi-)linear representation between the grid
crossings. The result is that a finer spacing must be used to
represent correctly atmospheric fields with a lot of structure, while
the grids can have fewer points when the atmospheric fields are
smooth. 

The accuracy when performing the actual radiative transfer calculations depends
on the refinement of the expressions used and the discretisation of the
propagation path. If Equation \ref{eq:fm_defs:rte_step} is used, the
underlying assumption is that the Planck function and the absorption vary
linearly along the propagation path step. These assumptions are of course less
violated if the path step length is made small. An upper limit of the path step
length is set by \wsvindex{ppath\_lmax}. In many cases it should suffice to
just include path points at the crossings of the atmospheric grids
(\builtindoc{ppath\_lmax}$\leq0$). An exception can be limb sounding where the
path step length can be very long around the tangent point, but a limit of
about 25~km should suffice normally. See also Section~\ref{sec:ppath:lmax}.



%
% To start the document, use
%  \levela{...}
% For lover level, sections use
%  \levelb{...}
%  \levelc{...}
%
\levela{Line of sight, 1D}
 \label{sec:los}


%
% Document history, format:
%  \starthistory
%    date1 & text .... \\
%    date2 & text .... \\
%    ....
%  \stophistory
%
\starthistory
  000307 & Started by Patrick Eriksson. \\
  010219 & First version finished by Patrick Eriksson.\\
\stophistory


%
% Symbol table, format:
%  \startsymbols
%    ... & \verb|...| & text ... \\
%    ... & \verb|...| & text ... \\
%    ....
%  \stopsymbols
%
%
%\startsymbols
%  \view     & \verb|view|     & zenith angle from zenith                  \\
%  $z$       & \verb|z|       & vertical altitude                          \\
%  $z_p$     & \verb|z_plat|  & platform altitude                          \\
%  $z_t$     & \verb|z_tan|   & tangent altitude                           \\
%  $z_g$     & \verb|z_ground|& altitude of the ground                     \\
%  $z_{lim}$ & \verb|z_abs_max|& practical upper limit of the atmosphere   \\
%  $l$       & \verb|l|       & distance along LOS                         \\
%  $e$       & \verb|gr_emiss|& ground emissivity                          \\
%  $\Delta l$& \verb|l_step|  & step length along LOS                      \\
%  $l_{lim}$ & \verb|llim|    & distance from lowest LOS point to $z_{lim}$\\
%  $l_p$     & \verb|l1|      & distance used for downward observation     \\
%  $i_p$     & -              & index for platform altitude for downward obs.\\
% \label{symtable:los}     
%\stopsymbols



%
% Introduction
%
This section describes how the line of sight (LOS) is determined
for situations where the atmosphere is assumed to be horizontally
stratified, a 1D atmosphere. Expressions are given both for pure
geometrical calculations and when considering refraction.



\levelb{Definitions}
 \label{sec:los:defs}
 
 Vertical (geometrical) altitudes are denoted as $z$, pressures as $p$
 and distances along the LOS are denoted as $l$. Vertical distances
 are measured from the geoid and $l$ is the distance from the lowest
 point of the LOS. 
 
 As a 1D atmosphere is assumed here, the conditions are symmetrical
 around tangent points and points of ground reflection, and, for such
 cases, only one half of the LOS is stored for efficiency reasons.
 The points of the LOS are stored by increasing vertical altitude
 point. Index 1 corresponds accordingly to either the platform, the
 tangent point or the ground.  The internal description of the LOS is
 further described in the file \verb|los.h|.
  
 The line of sight is defined by two variables, the platform altitude,
 $z_p$, and the zenith angle, $\view$, (see Fig. \ref{fig:los1d:geoms}):

 \begin{description}
  \item[The platform altitude] is the altitude above the geoid of the
       sensor used to detect the spectrum simulated.
  \item[The zenith angle] is the angle between the zenith
       direction and the direction of observation. As an 1D atmosphere is
       assumed, there is no difference between positive and negative
       zenith angles.
  \end{description}

  \noindent
  The lower limit of the atmosphere is given by the ground altitude,
  $z_g$. The practical upper limit of the atmosphere is denoted
  $z_{lim}$ and is in the forward model determined by the highest
  point of the absorption grid. The absorption grid can
  extend below $z_g$. On the other hand, it is not allowed that any
  part of the LOS is between the lowest absorption altitude and the ground.
 
  If $\view>90^{\circ}$ the lowest point of the LOS is not the platform
  altitude, and this point is denoted as the tangent point, $z_t$. The
  angle between the LOS and the vector to the Earth center is at the
  tangent point $90^\circ$. If the tangent point is below ground
  level, $z_t$ is determined by an imaginary geometric prolonging of
  the LOS inside the Earth.

  \begin{figure}[tb]
   \begin{center}
    \includegraphics*[width=0.95\hsize]{Figs/geoms}
    \caption{Schematic description of the main variables of the 
             observation geometry and the LOS. $R_e$ is the Earth
             radius. Other variables are defined in the text.}  
    \label{fig:los1d:geoms}  
   \end{center}
  \end{figure}
  
  The forward model uses internally three main observation geometries:

  \begin{description}
  \item[Limb sounding] covers here all observations from a point
    outside the atmosphere ($z_p \geq z_{lim}$). All zenith angles are
    covered, and, for example, nadir looking observations
    ($\view=180$) are treated as limb sounding in the forward model.
    If the LOS does not pass the atmosphere ($z_{tan} \geq z_{lim}$),
    cosmic background radiation, or correspondingly, is returned.
  \item[Upward looking] signifies observation from
    within the atmosphere in an upward direction ($z_p<z_{lim}$ and
    $\view\leq90^{\circ}$). 
  \item[Downward looking] is observation from within the atmosphere in
    a downward direction ($z_p<z_{lim}$ and $\view>90^{\circ}$).
  \end{description}
 

\levelb{Outlook towards 2D}
 \label{sec:los:2d}
 So far ARTS is only capable of calculating spectra for 1D cases.
 It is planned to also handle satellite measurements with atmospheric 
 horizontal variations, but limited to observations in the orbit
 plane, here denoted as 2D observations. 
 
 For 2D observations there is no symmetry to be used, each point of
 the LOS is unique. This is also the case for 1D upward looking
 observations, and it is planned that 2D and 1D upward calculations of
 radiative transfer and weighting functions shall be performed with
 the same general functions.  The 2D case exhibits however one
 difference compared to the 1D upward case. For 2D cases there could
 be a ground reflection along the LOS, which is never the case for 1D
 upward looking observations by definition.  Note that if the 1D
 upward functions are used for 2D simulations, the point of LOS
 closest to the sensor will throughout have index 1.

 As a first preparation for the 2D calculations, the angular distances
 between the sensor and the points of the LOS, $\psi$, are stored
 beside the pressure and vertical altitudes of the points. The
 variable $\psi$ is defined to be the angle between the vectors going
 from the Earth's center to the sensor and the LOS point,
 respectively.  For cases with symmetry, the angles are valid for the
 part of the LOS furthest away from the sensor.
 


\levelb{The step length}
 \label{sec:rte:stepl}
 
 As described in Section \ref{sec:rte}, the LOS is divided into equal
 long geometrical steps, $\Delta l$. The user gives an upper limit for
 this step length. A point of the LOS is always placed at the sensor
 (if inside the atmosphere), tangent points and points of ground
 reflection, but no adjustment to the upper atmospheric limit is made.
 This gives a single fixed point for limb sounding and upward looking
 observation and $\Delta l$ is set to the value given by the user if
 the LOS has at least two definition points. If the LOS gets only one
 point with the user defined value, for example when the tangent point
 is just below the atmospheric limit, the step length is adjusted to
 the length from the fixed point of the LOS (the sensor or the tangent 
 point) and the atmospheric limit.
 
 In contrast to upward and limb sounding observations, for downward
 observations there are two fixed points inside the atmosphere (the
 platform and the tangent point, or the point of ground reflection)
 and $\Delta l$ is here adjusted according to the the distance between
 these two points. See further Section \ref{sec:los:down}.



\levelb{Geometrical calculations}
 
 \levelc{General expressions}
  \label{sec:los:general}

  The relationship between vertical altitude ($z$) and distance along
  LOS ($l$) can be found be the law of cosines, giving
  \begin{equation}
    (R_e+z)^2 = (R_e+z_0)^2 + l^2 + 2l(R_e+z_0)\cos(\view)
  \end{equation}
  where $z_0$ is the lowest point of the LOS (where $l=0$) and \view\
  is the angle between the LOS and zenith at $z_0$. This equation 
  gives
  \begin{equation}
    z = \sqrt{ (R_e+z_0)^2 + l^2 + 2l(R_e+z_0)\cos(\view) } - R_e
   \label{eq:los:geom:z}
  \end{equation}
  The distance between the sensor and the limit of the atmosphere is
  \begin{equation}
      l_{lim} = \sqrt{ (R_e+z_{lim})^2 - (R_e+z_0)^2\sin^2(\view) } - 
                                       (R_e+z_0)\cos(\view)
   \label{eq:los:geom:llim}
  \end{equation}
  The angle $\psi$ between the point corresponding to $z_0$ and
  some altitude $z$ is
  \begin{equation}
      \psi = \cos^{-1}\left( \frac{(R_e+z_0)^2 + (R_e+z)^2 - l^2}
                                                   {2(R_e+z_0)(R_e+z)} \right) 
   \label{eq:los:geom:psi}
  \end{equation} 



 \levelc{Limb sounding}
  \label{sec:los:limb}
  
  For limb sounding the lowest point of the LOS is (by definition) the
  tangent point, and it is given by the expression
  \begin{equation}
    z_t = (R_e+z_p)\sin(\view) - R_e \qquad  \view\geq90^\circ
   \label{eq:los:ztan}
  \end{equation}
  This relationship holds even if $z_t<z_g$. Note that
  $\sin(180^\circ-\view)=\sin(\view)$ and it must be checked that
  $\view\geq90^\circ$. Zenith angles $<90^\circ$ correspond to an
  imaginary tangent point behind the sensor, and are treated as
  observations into the space.

  The LOS starting at the tangent point is then calculated by 
  Equations \ref{eq:los:geom:z} -- \ref{eq:los:geom:psi} with $z_0 = z_t$
  and $\view = 90^\circ$. The angle between the vectors going from the
  Earth's center and the sensor and the tangent point, respectively,
  is
  \begin{equation}
    \psi_0 = \view - 90^\circ
  \end{equation}
  The value of $\psi_0$ is added to the angles given by Equation 
  \ref{eq:los:geom:psi} as the equation in this case gives the angles 
  from the tangent point instead from the sensor.

  If the tangent point is below ground, $z_0$ is set to $z_g$ and \view\
  to $\view_g$ where
  \begin{equation}
    \view_g = \sin^{-1} \left( \frac{R_e+z_t}{R_e+z_g} \right)
  \end{equation}
  The correction term for $\psi$ is here
  \begin{equation}
    \psi_0 =  \view + \view_g - 180^\circ
  \end{equation}



 \levelc{Upward looking}   
  \label{sec:los:up}

  The LOS for upward looking observations is given by Equations 
  \ref{eq:los:geom:z} -- \ref{eq:los:geom:psi} where $z_0$ is set to
  the platform altitude and \view\ to the observation zenith angle.



 \levelc{Downward looking}
  \label{sec:los:down}

  The altitude of the tangent point is given by Equation \ref
  {eq:los:ztan}. As both the sensor and the tangent point (or the
  ground) are treated to be fixed points of the LOS, the step length
  must be adjusted. The distance between the sensor and a tangent point is
  \begin{equation}
    l_p = \sqrt{ (R_e+z_p)^2 - (R_e+z_t)^2 } \qquad  z_t \geq z_g
  \end{equation}
  and the distance between the sensor and a point of ground
  reflection is
  \begin{equation}
    l_p = \sqrt{ (R_e+z_p)^2 - (R_e+z_t)^2 } - \sqrt{ (R_e+z_g)^2-(R_e+z_t)^2}
            \qquad z_t < z_g
  \end{equation}
  The part of the LOS between the sensor and the tangent or ground
  point gets the following number of points:
  \begin{equation}
     m = 1 + \mathbf{ ceil}(l_{lim}/\Delta l_{max})
   \label{eq:los:m}
  \end{equation}
  where $\Delta l_{max}$ is the upper limit for $\Delta l$ specified by
  the user, and $\mathbf{ ceil}$ is a function giving the first integer
  larger than the argument. The step length is accordingly
  \begin{equation}
     \Delta l = \frac{l_{lim}}{m-1}
   \label{eq:los:dl}
  \end{equation} 
  The LOS is determined in the same manner as for limb sounding
  described above, but with the adjusted value for $\Delta l$.
  The angular distance between the the tangent point, or the ground.
  and the sensor $(\psi_0)$ is value $m$ of the angle vector given by 
  Equation \ref{eq:los:geom:psi}.

  

 \levelb{With refraction}
  \label{sec:los:refraction}
  
  Refraction affects the radiative transfer in several ways. The
  distance through a layer of a fixed vertical thickness will be
  changed, and for a limb sounding observation the tangent point is
  moved both vertically and horizontally. If the atmosphere is assumed
  to be horizontally stratified, as done here (1D), a horizontal
  displacement is of no importance but for 2D calculations this effect
  must be considered. For limb sounding and a fixed zenith angle, the
  tangent point is moved downwards compared to the pure geometrical
  case, resulting in that inclusion of refraction in general gives
  higher intensities. However, the LOS is still symmetric around
  tangent and ground points.

 \levelc{General theory}
  \label{sec:los:reftheory}

   When determining the LOS through the atmosphere geometrical optics 
   can be applied because the change of the refractive index over a
   wavelength can be neglected. Applying Snell's law to the geometry 
   shown in Figure \ref{fig:los:snell} gives
   \begin{equation}
     n_i \sin (\theta_i) = n_{i+1} \sin (\theta_i')
   \end{equation}
   \begin{figure}
    \begin{center}
      \includegraphics*{Figs/snell}
      \caption{Geometry to derive Snell's law for a spherical atmosphere. 
               The Earth radius is $R_e$, the vertical
               altitude $z$, the refractive index $n$ and the angle
               between the LOS and the vector to the Earth center $\theta$.}
      \label{fig:los:snell} 
    \end{center} 
  \end{figure}
  Using the same figure, the law of sines gives the relationship
  \begin{equation}
    \frac{\sin(\theta_{i+1})}{R_e+z_i} = 
    \frac{\sin(180^\circ-\theta_{i+1}')}{R_e+z_{i+1}} =
    \frac{\sin(\theta_i')}{R_e+z_{i+1}} 
  \end{equation}
  By combining the two equations above, the Snell's law for a spherical
  atmosphere (i.e. 1D) is derived \citep[e.g.][]{kyle:91,balluch:97}:
  \begin{equation}
    c = (R_e+z_i) n_i \sin(\theta_i) = (R_e+z_{i+1}) n_{i+1}\sin(\theta_{i+1}) 
   \label{eq:los:snellspherical}
  \end{equation}
  where $c$ is a constant. With other words, the Snell's law for spherical
  atmospheres states that the product of $n$, $(R_e+z)$ and $\sin(\theta)$ is
  constant along the LOS.

  The radiative transfer is evaluated along the LOS, while Equation 
  \ref{eq:los:snellspherical} is expressed for vertical altitudes.
  The relationship between a change in vertical altitude and the
  corresponding change along the LOS is here denoted as the geometrical term
  and it is
  \begin{equation}
    g(z) = \frac{1}{\cos(\theta)}
  \end{equation}
  which can be rewritten using trigonometric identities and Equation
  \ref{eq:los:snellspherical}:
  \begin{equation}
    g(z) = \frac {(R_e+z)n(z)} {\sqrt{ (R_e+z)^2n^2(z) - c^2 }}
   \label{eq:los:gterm}
  \end{equation}


 \levelc{Practical solution}
  A possible solution for calculating the LOS with refraction would be to 
  integrate numerically the geometrical term \citep{eriksson:00a} but 
  this approach is problematic for limb sounding as the geometric factor is
  singular at the tangent point (Figure \ref{fig:los:gfac}).
  \begin{figure}
   \begin{center}
    \includegraphics*[width=0.7\hsize]{Figs/fig_geomfac}
     \caption{The geometrical factor, as a function of altitude, for limb 
              sounding and three tangent altitudes. Taken from
              \citet{eriksson:97a}}.
    \label{fig:los:gfac}
   \end{center} 
  \end{figure}
  Further, Equation \ref{eq:los:gterm} cannot be solved analytically 
  for the simple reason that no general analytical expression for 
  $n$ exists. A possible solution would be to assume that $n$ is a
  piecewise linear function but the solution of Equation \ref{eq:los:gterm}
  is then unfortunately a very lengthy expression (at least the one provided 
  by Mathematica!). However, for a piecewise constant $n$ it is very simple
  to derive a solution of the integral, and thus avoiding the problem
  with singularities:
  \begin{equation}
    \Delta l = \sqrt{(R_e+z_2)^2 - \left( \frac{c}{\bar{n}} \right)^2} -
                     \sqrt{(R_e+z_1)^2 - \left( \frac{c}{\bar{n}} \right)^2}
  \end{equation}
  where $z_1$ and $z_2$ are two vertical altitudes, $\Delta l$ the length
  along the LOS between these two altitudes and $\bar{n}$ a mean value of
  the refractive index between $z_1$ and $z_2$. The calculations are 
  performed along the LOS and the follwing expression is used in practice
  \begin{equation}
     z_2 = \sqrt{ \left( \Delta l + 
           \sqrt{(R_e+z_1)^2 - \left( \frac{c}{\bar{n}} \right)^2}\, \right)^2
                                  + \left( \frac{c}{\bar{n}} \right)^2 } - R_e 
   \label{eq:los:refr:deltal}
  \end{equation}
  The angular distance between the points corresponding to $z_1$ and $z_2$
  is
  \begin{equation}
   \Delta \psi = \cos^{-1}\left( \frac{(R_e+z_1)^2 + (R_e+z_2)^2 - l^2}
                                                 {2(R_e+z_1)(R_e+z_2)} \right) 
   \label{eq:los:refr:deltapsi}
  \end{equation}
  The practical calculations are performed as follows:
  \begin{enumerate}
    \item The lowest point of the LOS is determined and the ``zenith
          angle'' at this point. 
    \item The ray tracing step length is set to the LOS step length divided
          by the factor given by the user (\verb|refr_lfac|).
    \item The ray tracing is performed from the lowest altitude of the LOS
          until the upper limit of the atmosphere is reached.
  \end{enumerate}

  \noindent
  Each ray tracing step is performed as
  \begin{enumerate}
    \item The refractive index $(\bar{n})$ is set to the value at $z_1$.
    \item The altitude of the other end of the ray tracing step is calculated
          by Equation \ref{eq:los:refr:deltal}.
    \item The refractive index at $z_2$ is determined by an interpolation
          and $(\bar{n})$ is set to the mean value of the refractive index at
          $z_1$ and $z_2$.
    \item Step 2 and 3 are repeated two times.
    \item The change in the angle $\psi$ is calculated by Equation
          \ref{eq:los:refr:deltapsi}.
  \end{enumerate}
  The number of iterations of step 2 and 3 is hard coded. A practical
  test showed a clear improvement when going from 1 to 2 iterations, a
  small improvement when going from 2 to 3 iterations and no practical
  improvement when going from 3 to 4 iterations. Accordingly, 3
  iterations are needed to reach convergence, but as the estimated
  accuracy for 2 iterations was judged to be sufficient 2 iterations
  was selected as a compremise between speed and accuracy. However,
  if the best accuracy possible is wanted, the number of iterations
  can easily be changed in the code.

  This calculation scheme has the advantage of always starting from the 
  lowest point of the LOS which should be beneficial for the calculation
  accuracy. How the tangent altitude is determined for limb sounding is 
  described below. 
  

 \levelc{Limb sounding}
    
   \begin{figure}
    \begin{center}
      \includegraphics*[width=0.8\hsize]{Figs/fig_bendingangle}
      \caption{Bending angle as a function of tangent altitude. The bending
        angle is the angle between the line from the tangent point and
        the sensor and the LOS tangent at the tangent point (see also
        \citet[][Figure 1]{kursinski:97}). Calculated for the FASCODE
        mid-latitude summer atmosphere. The figure can be compared to
        \citet[][Figure 3]{kursinski:97} and the agreeemnt is as good as
        expected.}
      \label{fig:los:bendingangle} 
    \end{center} 
  \end{figure}

  The most important factor for limb sounding is to get a correct
  tangent altitude. Fortunately, there is a way to determine the
  tangent altitude directly for 1D cases, without following the LOS
  from the top of the atmosphere.

  The tangent altitude is given by the relationship
  \begin{equation}
    (R_e+z_t)n(z_t) = (R_e+z_p)\sin(\view) = c
   \label{eq:los:ztan_ref}
  \end{equation}
  as $\sin(\theta)=1$ at tangent points, the refractive index in space
  is 1 and $sin(180^{\circ}-\view)=sin(\view)$. The tangent altitude
  is practically determined by finding the highest altitude where
  $(R_e+z)n(z)$ exceeds the value of $c$, followed by an interpolation
  of the product $(R_e+z)n(z)$ between the found altitude and the
  altitude above to find the altitude fulfilling Equation
  \ref{eq:los:ztan_ref}.
 
  For cases with ground reflections, a similar relationship,
  \begin{equation}
    (R_e+z_g)n(z_g)sin(\theta_g) = (R_e+z_p)\sin(\view) = c,
  \end{equation}
  gives the angle between the LOS and the ground normal.

  The angular distance between the tangent point and the sensor $(\psi_0)$
  is calculated as
  \begin{equation}
    \psi_0 = \theta_{z_{max}} + \view  - 180^\circ + \psi_{z_{max}}
  \end{equation}
  where $\theta_{z_{max}}$ and $\psi_{z_{max}}$ are the angles for the
  highest point of the LOS ($\theta$ defined in Figure \ref{fig:los:snell}).
  
  Figure \ref{fig:los:bendingangle}) gives a good confirmation of the
  implemented refraction ray tracing scheme.

  
  
 \levelb{Ground intersections}
  \label{sec:los:ground}
  
  Ground reflections are indicated by a special flag. This flag is
  zero when there is no ground intersection or gives the index of the
  LOS point corresponding to the ground, $i_g$ (for 1-based indexing).
  For 1D calculations, $i_g$ is either 0 or 1, as index 1 is here
  defined to always be the lowest altitude of the LOS. However, to
  pave the way for 2D calculations, cases where the ground is placed
  at other positions than index 1 are handled.
  
  For 1D cases, where only half of the total LOS is stored and the
  ground can only have index 1 ($i_g=1$), the effect of a ground
  reflection (Eq. \ref{eq:rte:ground}) is put in when reversing the
  loop order. Accordingly, the calculation order is: ... step2, step
  1, ground, step 1, step 2, ... Ground reflections for 1D cases are
  treated internally in ARTS by the limb sounding functions.
  
  \begin{figure}
   \begin{center}
     \includegraphics*[width=0.75\hsize]{Figs/ground}
    \caption{Schematic of ground reflections for 2D cases. The index 
             of the point corresponding to the ground is $i_g$. Point 1
             of the LOS is the point closest to the sensor. }  
    \label{fig:los1d:ground}  
   \end{center}
  \end{figure}
 


\levelb{Control file examples}
 \label{sec:los:cfe}
 
 The practical calculations are performed by a set of functions
 \verb|refrCalc|, \verb|losCalc|, \verb|sourceCalc|, \verb|transCalc|
 and \verb|yCalc|. All these functions have no global input/output or
 keyword arguments, and the main task is to define the input for the
 functions. The sequence {\footnotesize \begin{verbatim}
refrCalc{}
losCalc{}
sourceCalc{}
transCalc{}
yCalc{} 
\end{verbatim} 
} 
\noindent 
must always be used as, for example, the variable \verb|refr_index|
must be set when calling \verb|losCalc| and \verb|source|
must be set when calculating spectra by \verb|yCalc|. If refraction
is not considered, \verb|refrCalc| sets \verb|refr_index| to be empty
and \verb|sourceCalc| does the same with \verb|source| for transmission
calculations (\verb|emission| = 0).


 \levelc{Ground-based observation}

 The following control file excerpt shows a typical example for
 simulating a ground-based observation:

 {\footnotesize
 \begin{verbatim}
# Set the radius of the geoid to a standard value
r_geoidStd{
}

# Set the platform altitude to 50 m
NumericSet( z_plat ) { 
   value = 50 
}

# Measurement in the zenith direction
VectorSet( za_pencil ) {
   length = 1
   value = 0
} 

# A step length for LOS of 500 m
NumericSet( l_step ) { 
   500 
}

# Here we don't need to care about the ground and refraction
groundOff{
}
refrOff{
}

# Cosmic radiation
y_spaceStd{ "cbgr" }

# An emission measurement
emissionOn {}

# Do the actual calculations 
refrCalc{
}
losCalc{
}
sourceCalc{
}
transCalc{
}
yCalc{
}

# Convert to Rayleigh-Jean temperature
yTRJ{}

# Save the spectra
VectorWriteBinary( y ) { 
   ""
}

 \end{verbatim}
 }



 \levelc{Limb sounding}

 The following control file excerpt shows a typical example for
 limb sounding:

 {\footnotesize
 \begin{verbatim}
# Set the geoid radius for observation in the S-N direction
# at latitude 45 degrees 
r_geoidWGS84{
   latitude     = 45
   obsdirection = 0
}

# Set the platform altitude to 620 km
NumericSet( z_plat ) { 
   value = 620e3 
}

# Five zenith angles between 113.5 and 114.0
VectorNLinSpace (za_pencil) {
   start = 113.5
   stop  = 114.0
   n     = 5 
}

# A step length for LOS of 10 km
NumericSet( l_step ) { 
   10e3
}

# A blackbody ground at 200 m
groundSet{
   z = 200
   e = 1
}

# Turn on refraction, select parameterization for refractive
# index and set ray tracing step length to 2.5 km
refrSet{
  on    = 1
  model = "Boudouris"
  lfac  = 4
}

# An emission measurement
emissionOn {}

# Cosmic radiation
y_spaceStd{ "cbgr" }

# Do the actual calculations 
refrCalc{
}
losCalc{
}
sourceCalc{
}
transCalc{
}
yCalc{
}

# Convert to Rayleigh-Jean temperature
yTRJ{}

# Save the spectra
VectorWriteBinary( y ) { 
   ""
}
 \end{verbatim}
 }



 \levelc{Limb transmission calculations}

 Simulation of transmission measurements is performed in the same way
 as emission observations. Compared to the example above, beside that
 converion to brightness temperatures shall not be done, the only
 changes are:
 {\footnotesize
 \begin{verbatim}
...
# Turn off emission
emissionOff {}

# We don't need y_space here, set to be empty
VectorSet( y_space ) {
   length = 0
   value = 0
}
...
 \end{verbatim}
 }
\noindent


%%% Local Variables: 
%%% mode: latex
%%% TeX-master: "uguide"
%%% End: 

%
% To start the document, use
%  \levela{...}
% For lover level, sections use
%  \levelb{...}
%  \levelc{...}
%
\levela{Sensor modeling}
 \label{sec:sensor}


%
% Document history, format:
%  \starthistory
%    date1 & text .... \\
%    date2 & text .... \\
%    ....
%  \stophistory
%
\starthistory
  000321 & Started by Patrick Eriksson.\\
  000826 & First version finished by Patrick Eriksson.\\
\stophistory


%
% Symbol table, format:
%  \startsymbols
%    ... & \verb|...| & text ... \\
%    ... & \verb|...| & text ... \\
%    ....
%  \stopsymbols
%
%
%\startsymbols
%  -- & -- & -- \\
% \label{symtable:sensor}     
%\stopsymbols



%
% Introduction
%
{\it Modeling of the sensor is not yet part of ARTS. Sensor modeling is so
far covered by Qpack} but this chapter is included here for
completness. On the other hand, conversion of radiances to brightness
temperatures is part ARTS and this issue is also discussed here.

A sensor model is needed because a practical instrument gives
consistently spectra deviating from the hypothetical monochromatic
pencil beam spectra provided by the atmospheric part of the forward
model (that is $\y\neq\iv$ always). For a radio (heterodyne)
instrument, the most influential sensor parts are the antenna, the
mixer, the sideband filter and the spectrometer. Limb sounding
observations are also affected by Doppler shifts, but this effect is
not considered here, it is assumed to be treated separately.
Conversion of radiances to brightness temperatures is also treated
here.




\levelb{Implementation strategy}
 \label{sec:sensor:strategy}

\levelc{The sensor transfer matrix}
 \label{sec:sensor:strategy:h}
 
 The modeling of a sensor part is either a summation of different
 frequency components (mixer), or a weighting of the spectra as a
 function of frequency (spectrometer) or viewing direction (antenna)
 with the instrument response of concern. In all cases it is
 possible to describe the sensor influence by an analytical
 expression. See for example \citet{eriksson:97a} for more details.
 These analytical expressions can be implemented and solved for each
 run of the sensor model, but this would be relatively computationally
 demanding for cases when the settings are kept constant, as the
 calculations are duplicated in an unnecessary manner, and we want to
 find a better implementation strategy.
 
 Summation and weighting of the spectral components are both linear
 operations, and thus it is possible to model the effect of the
 different sensor parts as subsequent matrix multiplications of the
 monochromatic pencil beam spectrum, as suggested in \citet{eriksson:00a}:
 \begin{eqnarray}
   \y = \Hm_n\dots\Hm_2\Hm_1\iv + \merr
 \end{eqnarray}
 where $n$ is the number of sensor parts to consider, and this results
 in that the sensor model can be expressed as a single matrix
 multiplication (Eq. \ref{eq:formalism:H})
 \begin{eqnarray}
   \y = \Hm\iv + \merr                     \nonumber
 \end{eqnarray}
 Applying Equation \ref{eq:formalism:H} for the sensor model will
 clearly give very rapid calculations, and we must find ways to
 calculate $\Hm$.


\levelc{Normalization of \Hm}
 \label{sec:sensor:strategy:norm}
 
 It is important that the transfer matrix for
 each sensor part is normalized in such way that a unit response is
 obtained. A unit response signifies here that a constant intensity
 (as a function of frequency or zenith angle) is preserved, that is
 \begin{equation}
   \mat{u}_2 = \Hm\mat{u}_1
 \end{equation}
 where $\mat{u}_1$ and $\mat{u}_2$ are vectors of appropriate length
 where each element is $1$. This criterion equals that the sum of 
 the elements of each row of \Hm\ is 1.


\levelb{Integration as vector multiplication} 
 \label{sec:sensor:integr}
  
 The effect of both the antenna and the spectrometer can be expressed
 as an integral \citep[e.g.][Eq. 86 and 94]{eriksson:97a}, and the
 question is how to transform these integrals into matrix operations.
  
 The problem at hand is that the antenna and spectrometer responses
 and the zenith angle and frequency grids are known, while the spectral
 values are unknown. This problem corresponds to determine a (row)
 vector $\mat{h}$ that multiplied with an unknown (column) vector,
 $\mat{g}$, approximates the integral of the product between the
 functions $g$ and $f$:
 \begin{equation} 
   \mat{hg} = \int{f(x)g(x) \dd x}
   \label{eq:sensor:integral_problem}
 \end{equation}
 where $\mat{g}$ contains values of $g$ at some discrete points. The
 functions $f$ is here the response for some sensor part, and $g$
 holds the spectral values. The shape of $f$ and $g$ between the grid
 points must be known to solve this problem.


\levelc{Piecewise linear functions} 
 \label{sec:sensor:integr:lins}
 
 In this section the problem of
 Equation~\ref{eq:sensor:integral_problem} is solved analytically when
 both functions are piecewise linear. The practical solution used
 Qpack is discussed in next section.
  
 Following Figure \ref{fig:sensor:vecintegr}, the function $g$ can between
 the points $x_1$ and $x_4$ be expressed as a sum of the two unknown
 values $g_1$ and $g_2$:
 \begin{equation}
   g(x) = g_1 + (g_2-g_1)\frac{x-x_1}{x_4-x_1} =
           g_1 \frac{x_4-x}{x_4-x_1} + g_2\frac{x-x_1}{x_4-x_1}
 \end{equation}
 which can be rewritten as
 \begin{equation}
   g(x) = g_1(a+bx)+g_2(c-bx), \qquad x_1 \leq x \leq x_4
   \label{eq:sensor:glin}
 \end{equation}
 where
 \begin{eqnarray}
    a=\frac{x_4}{x_4-x_1}, \qquad b=\frac{-1}{x_4-x_1}, \qquad 
    c=\frac{-x_1}{x_4-x_1}   \nonumber
 \end{eqnarray} 
 A shorter expression can be obtained for the function $f$ as the
 values $f_1$ and $f_2$ are known:
 \begin{equation}
   f(x) = (d+ex), \qquad x_2 \leq x \leq x_3
 \end{equation}
 where 
 \begin{eqnarray}
    d=f_1-x_2\frac{f_2-f1}{x_3-x_2} \qquad e=\frac{f_2-f_1}{x_3-x_2} \nonumber
 \end{eqnarray}
 \begin{figure}[tb]
    \begin{center}
      \includegraphics*{Figs/vecintegr}
      \caption{The quantities used in Section \ref{sec:sensor:integr}.}  
      \label{fig:sensor:vecintegr} 
    \end{center} 
 \end{figure}
 The integral in Equation \ref{eq:sensor:integral_problem} can now for
 ranges between $x_2$ and $x_3$ be calculated analytically in a
 straightforward manner:
 \begin{eqnarray}
    \int_{x_a}^{x_b}{f(x)g(x) \dd x} =
    \int_{x_a}^{x_b}{\big(d+ex\big)\big(g_1(a+bx)+g_2(c-bx)\big) \dd x}  
    =\dots= \nonumber\\
    \bigg[ g_1x\Big(ad+\frac{x}{2}(bd+ae)+\frac{x^2}{3}be\Big) + 
           g_2x\Big(cd+\frac{x}{2}(ce-bd)-\frac{x^2}{3}be \Big)
           \bigg]_{x_a}^{x_b}
    \label{eq:sensor:integr_weights}
 \end{eqnarray}
 For the practical calculations, the integral is solved from one grid
 point to next, of either $\mat{f}$ or $\mat{g}$. The functions are 
 assumed to be zero outside their defined ranges (for example, $f=0$ 
 for $x<x_2$).
 For the case
 shown in Figure \ref{fig:sensor:vecintegr}, the integration order would be
 $(x_a,x_b)=(x_2,x_3)$, $(x_a,x_b)=(x_3,x_4)$, $(x_a,x_b)=(x_4,x_5)$
 \ldots\
  
 Using Equation \ref{eq:sensor:integr_weights}, we can now determine how to
 calculate $\mat{h}$. For each integration step, $\mat{h}_i$ and
 $\mat{h_{i+1}}$ are increased as
 \begin{eqnarray}
    \mat{h}_i \!\! &=& \!\! \mat{h}_i +    
              x_b\Big(ad+\frac{x_b}{2}(bd+ae)+\frac{x_b^2}{3}be\Big) - 
              x_a\Big(ad+\frac{x_a}{2}(bd+ae)+\frac{x_a^2}{3}be\Big) 
    \nonumber \\
    \mat{h}_{i+1} \!\! &=& \!\! \mat{h}_{i+1} +
              x_b\Big(cd+\frac{x_b}{2}(ce-bd)-\frac{x_b^2}{3}be\Big) - 
              x_a\Big(cd+\frac{x_a}{2}(ce-bd)-\frac{x_a^2}{3}be\Big) 
    \nonumber
 \end{eqnarray}
 where $i$ is the index for which $\mat{x}^i \leq x_a$ and $x_b \leq
 \mat{x}^{i+1}$. The vector $\mat{h}$ is initialized with
 zeros before the calculation starts.


\levelc{Practical solution} 
 \label{sec:sensor:integr:practical}
 
 The functions $f$ and $g$ can in Qpack be treated to be piecewice
 linear or cubic functions. The polynomial order of the two functions
 is set individually. When a function is assumed to be piecewise
 cubic, two points on each side of the range of interest (that is, in
 total 4 points) are used to determine the polynomial. For the end
 ranges, a quadratic polynomial is used as there exists only a single
 point on one of the sides. 
 
 Accordingly, Equation~\ref{eq:sensor:integral_problem} must be
 handled in Qpack for combinations of piecewise linear, quadratic and
 cubic functions. Instead of repeating the calculations in Section
 \ref{sec:sensor:integr:lins} for all possible polynomial
 combinations, a more general solution was implemented. The polynomial
 coefficents for $f$ are simply obtained by doing a polynomial fit to
 the considered points (by the Matlab function \verb|polyfit|). The
 polynomial basis for $g$ ($a$, $b$ and $c$ in Equation
 \ref{eq:sensor:glin}) is obtained by Lagrange's formula (Equation
 \ref{eq:wfuns:lagrange}), which expresses the polynomial that passes
 a fixed set of points. The Lagrange's formula can be written as:
 \begin{eqnarray}
  g(x) &=& (a_{11}+a_{12}x+\dots+a_{1N}x^N)*g_1 + \nonumber \\
       & & (a_{21}+a_{22}x+\dots+a_{2N}x^N)*g_2 + \nonumber \\
       & & \dots \nonumber \\
       & & (a_{N1}+a_{N2}x+\dots+a_{NN}x^N)*g_N 
  \label{eq:sensor:pbasis}
 \end{eqnarray}
 With the obtained coefficients for $f$ and $g$, Equation
 \ref{eq:sensor:integr_weights} can easily be solved analytically in a
 general manner. The polynomial pasis is determined by the AMI
 function \verb|pbasis|, the both set of coefficients are
 multiplicated in the function \verb|pbasis_x_pol| and the integral is
 solved by the function \verb|pbasis_integrate|.


\levelb{Summation as vector multiplication}
 \label{sec:sensor:mixer}
  
 The influence of the mixer and sideband filter of the sensor
 correspond to a summation of pairs of frequency components. The two
 frequencies of the pair are related as
 \begin{equation}
    \f' = 2\f_{LO}-\f
 \end{equation}
 where $\f_{LO}$ is the frequence of the local oscillator signal, and
 $\f'$ is denoted as the image frequency.

 \begin{figure}[tb]
  \begin{center}
    \includegraphics*[width=0.8\hsize]{Figs/sideband}
    \caption{Schematic description of image frequency and sideband filtering.}
   \label{fig:sensor:sideband} 
  \end{center} 
 \end{figure}
 
 The intensity correspondence after the mixer and the sideband filter
 can be written as
 \begin{equation}
   I_{IF}(\f) = \frac{f_s(\f)I(\f)+f_s(\f')I(\f')}{f_s(\f)+f_s(\f')}
  \label{eq:sensor:sband}
 \end{equation}
 where $I(\f)$ is the intensity for frequency $\f$ and $f_s$ the response
 of the sideband filter as a function of frequency.

 The frequency grid after the mixer consists of the frequencies inside
 the primary band of the grid before the mixer. To include frequencies
 from the image band (mirrored to the primary band) would need an 
 interpolation in the primary band that could cause unexpected effects.  


\levelc{Piecewise linear functions} 
 \label{sec:sensor:mixer:lins}

 If the intensity is assumed to vary linearly between the points of the
 frequency grid, Equation \ref{eq:sensor:sband} can be written as
 \begin{eqnarray}
   I_{IF}(\f^i) &=& \frac{1}{f_s(\f_i)+f_s(\f_i')} \bigg[ f_s(\f_i)I(\f_i)+ \nonumber \\ 
      & & + \frac{f_s(\f_i')}{\f_{j+1}-\f_j} \Big( I(\f_j)(\f_{j+1}-\f_i')
           + I(\f_{j+1})(\f_i'-\f_j) \Big)  \bigg]
  \label{eq:sensor:mixer}
 \end{eqnarray}
 where $f_s$ for the different frequencies is obtained by linear
 interpolation, and $\f_j$ and $\f_{j+1}$ are the two
 points of the frequency grid surrounding the image frequency,
 $\f_i'$. The row of the $\Hm$ matrix corresponding to $\f^i$ is then
 \begin{eqnarray}
    \label{eq:sensor:mixer:hi}
    \mat{h}^i &=& \frac{f_s(\f_i)}{f_s(\f_i)+f_s(\f_i')}  \\
    \mat{h}^j &=& \frac{f_s(\f_i')}{f_s(\f_i)+f_s(\f_i')}
                  \frac{\f_{j+1}-\f_i'}{\f_{j+1}-\f_j}     \nonumber \\
    \mat{h}^{j+1} &=& \frac{f_s(\f_i')}{f_s(\f_i)+f_s(\f_i')}
                  \frac{\f_i'-\f_j}{\f_{j+1}-\f_j}     \nonumber
 \end{eqnarray}
 where $\mat{h}^i$ is the value of $\mat{h}$ for frequency $\f_i$ etc.
 Remaining values of $\Hm$ are zero.

 For the special case when the image frequency matches perfectly a frequency
 grid point, the equations above can be simplified to give
 \begin{eqnarray}
    \mat{h}^i &=& \frac{f_s(\f_i)}{f_s(\f_i)+f_s(\f_i')}    \nonumber \\
    \mat{h}^j &=& \frac{f_s(\f_i')}{f_s(\f_i)+f_s(\f_i')}    \nonumber
 \end{eqnarray}


\levelc{Practical solution} 
 \label{sec:sensor:mixer:practical}
 
 The responses of the sideband filter is determined by linear or cubic
 interpolation, dependent on the selected order.
 As the frequency in the primary band always equals one of the points
 of the monochromatic frequency grid, Equation
 \ref{eq:sensor:mixer:hi} can be used throughout. The weights for the
 image band are found by evaluating the polynomial basis from Equation
 \ref{eq:sensor:pbasis} at $\f_i'$ and multiplicate with 
 $f_s(\f_i') / (f_s(\f_i)+f_s(\f_i'))$. These calculations are 
 performed in the AMI function \verb|h_matrix|.

 
\levelb{Brightness temperature} 
 \label{sec:sensor:tb}

 Some kind of calibration process, either in absolute or relative units,
 is always needed. For mm and sub-mm receivers, the calibration normally
 presents the measured intensity in some temperature scale, and conversion
 to brightness and Rayleigh-Jeans temperatures is also treated in this section.


\levelc{Conversion to Planck brightness temperature} 
 \label{sec:sensor:tb_planck}

 The brightness temperature is defined as the temperature a blackbody 
 shall have to give the same intensity magnitude as observed. The 
 brightness temperature is thus calculated as
 \begin{equation}
   T_b = \frac{h\f}{k_B} \frac{1}{\ln{ \left( \frac{2h\f^3}{c^2\mpbi}+1 \right)}}
   \label{eq:sensor:cal:tb}
 \end{equation}
 where \mpbi\ is the radiative intensity.
 
 It should be noted that the conversion from intensity to brightness
 temperature is non-linear. This non-linearity has (at least) two important
 consequences:
 \begin{itemize}
  \item The conversion from intensity to brightness temperature cannot be
        included in \Hm.
  \item \bf{Brightness temperature cannot be used for retrievals.}
 \end{itemize}
 Accordingly, the main reason to convert a spectrum to brightness
 temperatures is to display the spectrum in an unit that gives a more
 intuitive understanding of the emission magnitude.


\levelc{Conversion to Rayleigh-Jean temperature} 
 \label{sec:sensor:tb:rj}

 For lower frequencies where $h\f \ll k_BT$ the Planck function can
 be approximated by the Rayleigh-Jean (RJ) formula:
 \begin{equation}
   B \approx \frac{2\f^2k_BT}{c^2}
 \end{equation}
 This relationship holds rather well in the microwave region. For example,
 for $T=50$~K, $h\f = k_BT$ at 1.04~THz. The RJ approximation of the Planck
 function gives a natural definition on a ``brightness temperature'' with
 that has a linear relationship to the intensity:
 \begin{equation}
   T_{rj} = \frac{c^2}{2\f^2k_B} \mpbi
   \label{eq:sensor:cal:rj}
 \end{equation}
 This intensity unit is often referred to as the brightness temperature but
 to avoid confusion it is here denoted as the RJ temperature.
 
 As the intensity from intensity to RJ temperature is linear, this
 conversion can be included in \Hm\ and weighting functions can be
 converted using \ref{eq:sensor:cal:rj}, that is, retrievals are
 possible using RJ temperatures.
 On the other hand, the RJ temperature shall not be mistaken for the
 ``physical'' brightness temperature $(T_b)$ as the deviation between
 $T_b$ and $T_{rj}$ is not negligible \citep{eriksson:97a}.


\levelb{Control file examples}
 \label{sec:sensor:cfe}

 The following sequence of ARTS functions can be used to store the
 spectra in both brightness temperature units:

 {\footnotesize
 \begin{verbatim}
VectorCopy( y0, y ) {
}
yTRJ{
}
VectorWriteAscii( y ) {
   "ytb_rj.aa"
}
VectorCopy( y, y0 ) {
}
yTB{
}
VectorWriteAscii( y ) {
   "ytb_planck.aa"
}
 \end{verbatim}
 }
 \noindent
 A weighting function matrix is converted to Rayleigh-Jean temperature
 as:

 {\footnotesize
 \begin{verbatim}
MatrixTRJ( kx, kx ) {
}

 \end{verbatim}
 }
 


%%% Local Variables: 
%%% mode: latex 
%%% TeX-master: "uguide" 
%%% End:


%
% To start the document, use
%  \chapter{...}
% For lover level, sections use
%  \section{...}
%  \subsection{...}
%
\chapter{Data reduction}
 \label{sec:red}


%
% Document history, format:
%  \starthistory
%    date1 & text .... \\
%    date2 & text .... \\
%    ....
%  \stophistory
%
\starthistory
  000321 & Created and written by Patrick Eriksson.\\
\stophistory


%
% Symbol table, format:
%  \startsymbols
%    ... & \verb|...| & text ... \\
%    ... & \verb|...| & text ... \\
%    ....
%  \stopsymbols
%
%
%\startsymbols
%  -- & -- & -- \\
% \label{symtable:red}     
%\stopsymbols



%
% Introduction
%
Many observation scenarios give rise to very large measurement
vectors, larger than can be handled practically during the inversions,
and some kind of reduction of the data size is needed. This data
reduction can be made part of the sensor transfer matrix. In fact, the
data reduction can be viewed upon as an imaginary second spectrometer.
The transfer matrix to use is then (Eq. \ref{eq:formalism:Hs})
\begin{eqnarray}
  \Hm = \Hd \Hs  \nonumber
\end{eqnarray}
where \Hd\ is the data reduction matrix and \Hs\ the sensor matrix.
{\it Data reduction can so far only be performed in Qpack.}


\section{Averaging of viewing angles}
 \label{sec:red:view}
 
 In some cases the spectra from different viewing angles are combined,
 either as a pure data reduction or internally in the spectrometer.
 The rows of \Hd\ for this case have the structure
 \begin{equation}
   \mat{h} = \big[ 0,\dots,0,\frac{1}{n_v},0,\dots,0,\frac{1}{n_v},0,\dots,0,\frac{1}{n_v},0,\dots,0\big]
 \end{equation}
 where $n_v$ is the number of viewing angles to combine.


\section{Data binning}
 \label{sec:red:binning}
 
 Data binning means that neighboring channels are combined by
 weighted averaging. If channels $i_1$ to $i_2$ of $\y'$ are combined to
 give element $j$ of $\y$, the binning can be expressed as
 \begin{equation}
   \y^j = \frac{1}{\sum_{i=i_1}^{i_2}{\Delta \f^i}} \sum_{i=i_1}^{i_2}{\Delta \f^i (\y')^i}
 \end{equation}
 Row $j$ of \Hd\ is accordingly
 \begin{equation}
   \mat{h}^i = \frac{\Delta \f^i}{\sum_{i=i_1}^{i_2}{\Delta \f^i}}, \qquad
    i_1\leq i \leq i_2
 \end{equation}
 Other values of $\mat{h}$ are zeros. The matrix \Hd\ is for data
 binning highly sparse.



\section{Reduction by eigenvectors}
 \label{sec:red:eig}
 
 A commonly used approach for reducing data sizes is to base the
 reduction of the eigenvectors of the covariance matrix expressing the
 variability of the measurements. These empirical eigenvectors
 fulfills the relationships
 \begin{equation}
   \mat{S}_\y = \mat{E}\Lambda\mat{E}^T
 \end{equation}
 where $\Lambda$ is a diagonal matrix holding the eigenvalues
 corresponding to the eigenvectors, the columns of $\mat{E}$. The
 eigenvectors form an orthogonal basis:
 \begin{equation}
   \Id = \mat{E}^T_j\mat{E}_j
 \end{equation}
 where $\mat{E}_j$ signifies the $j$ first columns of the matrix.

 The data reduction for this case is performed as
 \begin{equation}
   \y = \mat{E}^T_j \y'
 \end{equation}
 that is
 \begin{equation}
   \Hd = \mat{E}^T_j 
 \end{equation}
 By basing the data reduction on the covariance matrix eigenvectors,
 the reduction maintaining the maximum possible fraction of the
 variability of the spectra, for a given $j$, is achieved.

 Different versions of this scheme are described in \citet{eriksson:01c}. 
 The existing options in Qpack are described in the file \verb|README|.





%%% Local Variables: 
%%% mode: latex
%%% TeX-master: "uguide"
%%% End: 

\graphicspath{{Figs/wfuns_atm/}}

%
% To start the document, use
%  \chapter{...}
% For lover level, sections use
%  \section{...}
%  \subsection{...}
%
\chapter{Atmospheric weighting functions}
 \label{sec:wfuns}


%
% Document history, format:
%  \starthistory
%    date1 & text .... \\
%    date2 & text .... \\
%    ....
%  \stophistory
%
\starthistory
  000310 & Started by Patrick Eriksson.\\
  000911 & First version finished by Patrick Eriksson.\\
\stophistory


%
% Symbol table, format:
%  \startsymbols
%    ... & \verb|...| & text ... \\
%    ... & \verb|...| & text ... \\
%    ....
%  \stopsymbols
%
%
%\startsymbols
%  -- & -- & -- \\
% \label{symtable:wfuns}     
%\stopsymbols



%
% Introduction
%
This section describes how the calculation of the atmospheric weighting
functions (WFs) matrices is performed in the forward model. For
several types of variables (such as species profiles and fit of
absorption continuum) WFs are obtained by semi-analytical expressions,
while for other quantities the WFs are obtained by straightforward
perturbation calculations.



\section{Calculation approaches}
 \label{sec:wfuns:approaches}

  \subsection{Pure numerical calculation} 
  The most straightforward method to determine WFs is by perturbing
  one parameter at a time. For example, the WF for the state variable
  $p$ can always be calculated as
  \begin{equation}
    \K_{\xt}^p = \frac{\fm(\xt+\Delta\xt^p\mat{e}^p,\bt)-\fm(\xt,\bt)}
                                     {\Delta\xt^p}
   \label{eq:wfuns:perturb}
  \end{equation}
  where $\K_{\xt}^p$ is column $p$ of \Kx, $(\xt,\bt)$ is the
  linearization state, $\mat{e}^p$ is a vector of zeros except for the
  component $p$ that is unity, and $\Delta\xt^p$ is a small disturbance
  (but sufficiently large to avoid numerical instabilities).
  
  However, it is normally not needed to make a recalculation using the
  total forward model as the variables are in general either part of the
  atmospheric or the sensor state, but not both. If $\xt^p$ is an atmospheric
  variable, the calculation can be performed as (Eq. \ref{eq:formalism:kx2})
  \begin{equation}
    \K_{\xt}^p = \Hm \bigg[
    \frac{\fm_r(\xt_r+\Delta\xt^p\mat{e}^p,\bt_r)-
           \fm_r(\xt_r,\bt_r)}  {\Delta\xt^p} \bigg]
   \label{eq:wfuns:Hpert}
  \end{equation}
  where $\xt_r$ is the atmospheric part of the state vector etc (see
  further Sec. \ref{sec:formalism}).
 

  \subsection{Analytical expressions} 
  \label{sec:wfuns:approaches:anal}
  For some atmospheric variables, such as species abundance, it is
  possible to derive a semi-analytical expression for the WFs. This is
  advantageous because it results in faster and more accurate
  calculations. By Equation \ref{eq:formalism:kx2},
  \begin{eqnarray*}
    \Kx = \Hm\frac{\partial\iv}{\partial \xt},
  \end{eqnarray*}
  it can be seen that the core problem of finding these analytical
  expressions is to determine $\partial\iv / \partial \xt$. 
  If $\xt^p$ influences only the conditions at one altitude, the
  problem can be simplified as \citep[][Eq. 43]{eriksson:00a}
  \begin{equation}
    \K_{\xt}^p = \Hm \frac{\partial\iv}{\partial \xt^p} = 
      \Hm \Bigg[ \frac{\partial\iv}{\partial \mat{S}^p}
                 \frac{\partial \mat{S}^p}{\partial \xt^p} +
                 \frac{\partial\iv}{\partial \mat{k}^p}
                 \frac{\partial \mat{k}^p}{\partial \xt^p} \Bigg]
   \label{eq:wfuns:taylor}
  \end{equation}
  where $\mat{S}^p$ and $\mat{k}^p$ are the source function and the
  absorption at the (vertical) altitude $p$, respectively.
  
  It is very important to note that the analytical expressions are
  derived with the assumption that $\xt^p$ influences only the local
  conditions. For species it is further assumed that the absorption
  can be expressed as (see Section \ref{sec:wfuns:species} for
  definitions and details)
  \begin{equation}
   \mat{k}^p = \mat{\bar{k}}^p_s \xt^p + \sum_{i\ne s} \mat{k}^p_i
  \end{equation}
  These assumption should be of general validity
  for species above the tropopause. Two examples on when the
  analytical expressions will be approximative are
  \begin{itemize}
  \item The variable of interest can change the line-of-sight (by the
    refractive index). This is an example of a non-local effect. This
    is always valid for temperature.
  \item The amount of different species must be considered when
    calculating the pressure broadening, and not only the total
    absorption.
  \end{itemize}
  If the analytical expressions can be used for such cases must be
  tested numerically. When it is found that the analytical approach
  cannot be used, the WFs must be calculated by perturbations to
  include the neglected effects (such a function for species is not
  yet implemented in ARTS). An important example when these questions
  must be considered is limb sounding of water vapor in the
  troposphere where both points above are true. The abundance of water
  in the troposphere is sufficient high to have a significant
  influence on both the refractive index and the pressure broadening.
  These questions are discussed somewhat further in \citet{eriksson:01d}.

  The absorption and source function in Equation \ref{eq:wfuns:taylor}
  are defined in vertical coordinates (as we retrieve atmospheric
  variables as functions of altitude). For different reasons it is
  more practical to work with these quantities defined along the LOS.
  For example, the source function and transmission along the LOS are
  already determined when calculating the spectra. To solve this
  problem, Equation \ref{eq:wfuns:taylor} is expanded one step further
  \begin{equation}
    \K_{\xt}^p = \Hm \Bigg[ \frac{\partial\iv}{\partial \sigma}
                 \frac{\partial \sigma}{\partial \mat{S}^p} 
                 \frac{\partial \mat{S}^p}{\partial \xt^p} +
                 \frac{\partial\iv}{\partial \kappa}
                 \frac{\partial \kappa}{\partial \mat{k}^p}
                 \frac{\partial \mat{k}^p}{\partial \xt^p} \Bigg]
   \label{eq:wfuns:taylor2}
  \end{equation}
  where $\sigma$ and $\kappa$ are the source function and the absorption 
  along the LOS, respectively.
  
  The term $\partial\iv / \partial \sigma$ is here denoted as source
  line of sight weighting functions (source LOS WFs) and is discussed
  in Section \ref{sec:wfuns:sourceloswfs}. The term $\partial\iv/
  \partial \kappa$ is denoted as absorption LOS WFs and is discussed
  in Sections \ref{sec:wfuns:absloswfs} and
  \ref{sec:wfuns:absloswfs2}. These terms are treated separately as
  they are common for all variables influencing the source function or
  the absorption.

  The term $\partial \mat{S}^p/\partial \xt^p$ can often be neglected.
  When scattering is neglected and local thermodynamic equilibrium is
  assumed, the only variable of interest affecting the source function
  is the temperature.  See further Section \ref{sec:wfuns:temp}. For
  other variables, such as species abundance, $\partial
  \mat{S}^p/\partial \xt^p=0$.
  
  It was decided to allow that the retrieval grids differ between
  species, temperature etc. This results in that the terms $\partial
  \sigma/ \partial \mat{S}^p$ and $\partial \kappa/ \partial
  \mat{k}^p$ are not constant, they change according to the selected
  retrieval grid. Accordingly, it is not suitable to include these terms
  in the corresponding LOS WFs, they must be treated separately.
  
  

\section{Absorption LOS WFs with emission}
 \label{sec:wfuns:absloswfs}

 The absorption line of sight weighting functions are defined as
 \begin{equation}
   \K_{\kappa}^q =  \frac{\partial\iv}{\partial \kappa^q}
  \label{eq:wfuns:loswfs}
 \end{equation}
 These weighting functions express how the intensity is
 affected by changes of the absorption at the points of the line of
 sight. Note that $\kappa$ is the total absorption, not the
 absorption of a single species. 
 
 For simplicity, the absorption LOS WFs are below derived without
 using vector notation. The notation used here is identical to the one
 used in Section \ref{sec:rte}. The calculation approach used for the
 LOS WFs is ``inspired'' by the corresponding work in
 \citet{master00}.


 \subsection{Single pass}
 \label{sec:wfuns:single}
 
 This section derives the absorption LOS WFs for cases when each
 individual part of the atmosphere is passed only once, as for upward
 looking measurements, or when each point in the atmosphere is treated
 separately (2D simulations). With other words, the conditions are not
 assumed to be symmetrical around some point. Accordingly, 1D limb
 sounding and 1D downward observations are not treated here, and are
 instead discussed in Section \ref{sec:wfuns:limb} and
 \ref{sec:wfuns:down}, respectively.

 \begin{figure}[t]
  \begin{center}
   \includegraphics*[width=0.95\hsize]{wf1}
   \caption{The terms used for the derivation of line of sight weighting
            functions when the individual atmospheric parts are passed a
            single time. The variables are defined in Figure 
            \ref{fig:rte:los}.}
   \label{fig:wfuns:single}  
  \end{center}
 \end{figure}

 By rewriting Equation \ref{eq:rte:rteprod}, the monochromatic pencil beam
 intensity can be expressed in the following ways (see Fig. 
 \ref{fig:wfuns:single})\footnote{The indexing used here is 1-based 
  (starts at 1), while inside ARTS 0-based indexing is used.}
 \begin{eqnarray}
   I &=& I_2\zeta_1+\psi_1(1-\zeta_1) \quad (q=1) 
     \nonumber \\
   I &=&\Big[I_{q+1}\zeta_q\zeta_{q-1}+\psi_q(1-\zeta_q)\zeta_{q-1} +
            \psi_{q-1}(1-\zeta_{q-1}) \Big] \Theta^{q-1}_1, \quad 1<q<n 
    \label{eq:wfuns:mpbi} \\
   I &=& \Big[I_n\zeta_{n-1}+\psi_{n-1}(1-\zeta_{n-1})\Big]\Theta^{n-1}_{1}
     \quad (q=n)
     \nonumber
 \end{eqnarray}
 where it assumed that the LOS has $n$ points, index 1 is the point
 closest to the sensor,
 \begin{equation}
   I_q = I_n \Theta^{n}_{q} + \sum_{i=q}^{n-1}\psi_i(1-\zeta_i) 
             \Theta_{q}^{i}, \quad 1 \leq q < n
  \label{eq:wfuns:iq}
 \end{equation}
 is the intensity reaching point $q$ along the LOS, $I_n$ is the radiation at
 point n (the radiation entering the atmosphere), and
 \begin{equation}
   \Theta_q^p = \prod_{i=q}^{p-1}\zeta_i\quad \mathrm{for} \quad p>q, 
     \quad \mathrm{and} \quad \Theta_p^p = 1
  \label{eq:wfuns:Theta}
 \end{equation}
 the transmission from point $q$ and $p$. It should be noted that
 $I_q$ and $\Theta_q^p$ not are calculated as indicated by the
 equations above. These quantities are instead updated when going from
 one step of the LOS to the next, as described below. It should also be
 noted that ground reflections are here neglected and are discussed 
 separately below.

 The transmissions $\zeta_{q-1}$ and $\zeta_q$ are separated in Equation
 \ref{eq:wfuns:mpbi} as they are the only terms including the absorption
 at point $q$. For example
 \begin{equation}
   \zeta_{q-1} = e^{-\Delta l(\kappa_{q-1}+\kappa_q)/2}
 \end{equation}
 and we have that
 \begin{equation}
   \frac{\partial \zeta_q}{\partial \kappa_q} = -\frac{\Delta l}{2}\zeta_q
  \label{eq:wfuns:dzeta1}
 \end{equation}
 \begin{equation}
   \frac{\partial\zeta_{q-1}}{\partial \kappa_q}=-\frac{\Delta l}{2}\zeta_{q-1}
 \end{equation}
 \begin{equation}
   \frac{\partial \zeta_{q-1}\zeta_q}{\partial \kappa_q} = 
          -\Delta l \zeta_{q-1}\zeta_q
  \label{eq:wfuns:dzeta2}
 \end{equation}
 The derivate of transmission values beside $\zeta_q$ and
 $\zeta_{q-1}$ with respect to $\kappa_q$ is zero.

 The LOS WFs are now easily determined, using the case $1<q<n$ as example
 \begin{equation}
   \K_{\kappa}^q = -\frac{\Delta l}{2} \Big[ 2I_{q+1}\zeta_q\zeta_{q-1}+
     \psi_q(1-2\zeta_q)\zeta_{q-1} - \psi_{q-1}\zeta_{q-1} \Big] 
     \Theta^{q-1}_1, \, 1<q<n
 \end{equation}
 which can be rewritten as
 \begin{eqnarray}
   \K_{\kappa}^1 &=& -\frac{\Delta l}{2} \big[ I_2-\psi_1 \big] \Theta^2_1 \nonumber \\
   \K_{\kappa}^q &=& -\frac{\Delta l}{2} \big[ 2(I_{q+1}-\psi_q)\zeta_q+
           \psi_q-\psi_{q-1} \big] \Theta^q_1, \quad 1<q<n 
  \label{eq:wfuns:loswfsxx} \\
   \K_{\kappa}^n &=& -\frac{\Delta l}{2} \big[ I_n-\psi_{n-1} \big] \Theta^n_1 \nonumber
 \end{eqnarray}
 Note that one $\zeta_q$ is incorporated in $\Theta^q_q$, and that 
 $\Theta^2_1=\zeta_1$.
 
 These equations are used for the practical calculations, but it could
 be of interest to note that Equation \ref{eq:wfuns:loswfsxx} can be
 written
 \begin{equation}
   \K_{\kappa}^q = -\frac{\Delta l}{2} \big[ (I_{q+1}-\psi_q)\zeta_q+
           I_q-\psi_{q-1} \big] \Theta^q_1, \quad 1<q<n ,
 \end{equation}
 showing that the expressions for $q=1$ and $q=n$ are special cases of
 the general expression where the terms connected to $q-1$ and $q$,
 are neglected, respectively.
 
 The iteration starts here at the end closest to the
 sensor, that is, at index 1 (reversed order to the RTE part).  The
 iteration is started by setting $I_1$ to the already calculated
 spectrum and $\Theta^1_1$ to 1.  These two variables are updated as
 \begin{eqnarray}
   I_{q+1} = \frac{I_q - \psi_q(1-\zeta_q)}{\zeta_q} 
 \end{eqnarray}
 \begin{eqnarray}
   \Theta_1^{q+1} =  \Theta_1^q \zeta_q
 \end{eqnarray}
 For 2D calculations possible ground reflections inside the LOS must
 be handled. The ground cannot be found at any of the end points of
 the LOS, and the correspondence to Equation \ref{eq:wfuns:mpbi} for a
 ground point is (c.f. Equations \ref{eq:rte:ground} and
 \ref{eq:rte:tground})
 \begin{eqnarray}
   I &=&\Big[I_{q+1}\zeta_q(1-e)\zeta_{q-1}+\psi_q(1-\zeta_q)(1-e)\zeta_{q-1}
          +eB\zeta_{q-1}+ \nonumber \\
     & & + \psi_{q-1}(1-\zeta_{q-1}) \Big] \Theta^{q-1}_1, \quad 1<q<n 
    \label{eq:wfuns:mpbi_ground}
 \end{eqnarray}
 and the corresponding absorption LOS WF for this point is (cf. Eq.
 \ref{eq:wfuns:loswfsxx})
 \begin{eqnarray}
   \K_{\kappa}^q &=& -\frac{\Delta l}{2} \big[ 2(I_{q+1}-\psi_q)\zeta_q(1-e)+
           \psi_q(1-e)+eB-\psi_{q-1} \big] \Theta^q_1 
 \end{eqnarray}
 The intensity and the transmission are here updated as
 \begin{eqnarray}
   I_{q+1} &=& \frac{I_q-\psi_q(1-\zeta_q)(1-e)-eB}{\zeta_q(1-e)}  \nonumber \\
   \Theta_1^{q+1} &=& \Theta_1^{q}\zeta_q(1-e) \nonumber
 \end{eqnarray}
 It is noteworthy that the effect of a ground intersection is included
 in $I_1$ when the iteration starts.  


 
 \subsection{1D limb sounding}
 \label{sec:wfuns:limb}
    
 For limb sounding and when the atmosphere is assumed to be consist of
 homogenous layers (horizontally stratified), there is a perfect
 symmetry around the tangent point. This covers also the case with a
 ground reflection. For these cases the distance from the sensor is
 neglected, the important factor is the vertical altitude.  All
 altitudes above the tangent point are passed twice (Fig.
 \ref{fig:wfuns:limb}) and both crossings of an atmospheric layer are
 treated to be identical for the retrievals, and this fact must also
 be reflected by the WFs.

 Using a nomenclature similar to the one used for Equation
 \ref{eq:wfuns:mpbi}, the intensity of a limb sounding
 observations can be expressed as (Fig. \ref{fig:wfuns:limb})
 \begin{eqnarray}
   I & = & \Big(I_2 \Big( \zeta_1\Theta^1_1 \Big)^2 + \psi_1(1-\zeta_1)
            \Big( \Theta^1_1 \Big)^2 \zeta_1+
            I_1^1\zeta_1+\psi_1(1-\zeta_1) \Big)\Theta^n_{2} \quad (q=1) 
        \nonumber \\
   I & = & \Big[\Big(I_{q+1}\zeta_q\zeta_{q-1} +\psi_q(1-\zeta_q)\zeta_{q-1} + 
           \psi_{q-1}(1-\zeta_{q-1})\Big)\Big(\Theta^{q-1}_{1}\Big)^2
           \zeta_{q-1}\zeta_q + \nonumber \\
      & & + I_{q-1}^{q-1}\zeta_{q-1}\zeta_q + \psi_{q-1}(1-\zeta_{q-1})
           \zeta_q + \psi_q(1-\zeta_q) \Big] \Theta^n_{q+1}, \, 1<q<n
  \label{eq:wfuns:limb1}  \\
   I & = & \Big(I_n\zeta_{n-1}+\psi_{n-1}(1-\zeta_{n-1})\Big)\Big
           (\Theta^{n-1}_{1}\Big)^2\zeta_{n-1} + I_{n-1}^{n-1}\zeta_{n-1} +
             \nonumber \\
      & &  +   \psi_{n-1}(1-\zeta_{n-1}) \quad (q=n) \nonumber
 \end{eqnarray}
 where the expression for $q=1$ is commented below, index 1 of the LOS
 is the tangent (or the ground) point, index $n$ corresponds to the
 highest altitude,
 \begin{equation}
   I_q = I_n \Theta^{n}_{q} + \sum_{i=q}^{n-1}\psi_i(1-\zeta_i) 
             \Theta_{q}^{i-1}
  \label{eq:wfuns:iqq}
 \end{equation}
 is the intensity reaching point $q$ from the part of the
 atmosphere furthest away from the sensor, $I_n$ the intensity at point $n$,
 \begin{equation}
   I_q^q = \Big[ \sum_{i=1}^{q-1}(\psi_i(1-\zeta_i)\Theta_{1}^{i-1}\Big]
             \Theta_{1}^{q} + \sum_{i=1}^{q-1}\psi_i(1-\zeta_i)
            \Theta_{i+1}^{q}, \qquad q>1
 \end{equation}
 is the intensity generated along the LOS (towards the sensor) between
 the two crossing with altitude $q$, $I_1^1=0$, $\Theta$ is defined by
 Equation \ref{eq:wfuns:Theta}. The equations defining $I_q$, $I_q^q$
 and $\Theta$ neglect ground reflections, but could easily be extended
 to cover also such cases. However, $I_1^1$ and $\Theta_1^1$ are
 included for $q=1$ to make Equation \ref{eq:wfuns:limb1} valid for
 cases with ground reflections. The treatment of ground reflections
 are discussed separately last in the section.

 \begin{figure}[tb]
  \begin{center}
   \includegraphics*[width=0.95\hsize]{wf2}
   \caption{The terms used for the derivation of line of sight weighting
            functions for 1D limb sounding.}
   \label{fig:wfuns:limb}  
  \end{center}
 \end{figure}
 
 If the different combinations of $\zeta_{q-1}$ and $\zeta_q$ are 
 grouped, for example, Equation \ref{eq:wfuns:limb1} becomes
 \begin{eqnarray}
   I & = & \Big[\Big((I_{q+1}-\psi_q)\zeta_{q-1}^2\zeta_q^2+(\psi_q-\psi_{q-1})
            \zeta_{q-1}^2\zeta_q + \psi_{q-1}\zeta_{q-1}\zeta_q
            \Big)\Big(\Theta^{q-1}_{1}\Big)^2 + \nonumber \\
    &  &     + (I_{q-1}^{q-1}-\psi_{q-1})\zeta_{q-1}\zeta_q + 
            (\psi_{q-1}-\psi_q)\zeta_q + \psi_q \Big] \Theta^n_{q+1} 
 \end{eqnarray}
 This equation has some higher products between
 $\zeta_{q-1}$ and $\zeta_q$ than Equation \ref{eq:wfuns:mpbi}, and
 the derivatives, with respect to $\kappa_q$, of these product are
 \begin{equation}
   \frac{\partial \zeta_{q-1}^2\zeta_q}{\partial \kappa_q} = 
         -\frac{3\Delta l}{2} \zeta_{q-1}^2\zeta_q
  \label{eq:wfuns:dzeta3}
 \end{equation}
 \begin{equation}
   \frac{\partial \zeta_{q-1}^2\zeta_q^2}{\partial \kappa_q} = 
          -2\Delta l \zeta_{q-1}^2\zeta_q^2
  \label{eq:wfuns:dzeta4}
 \end{equation}
 Using Equations \ref{eq:wfuns:dzeta1}, \ref{eq:wfuns:dzeta2},
 \ref{eq:wfuns:dzeta3} and \ref{eq:wfuns:dzeta4}, the LOS WFs for 1D
 limb sounding can be determined to be
 \begin{eqnarray}
   \K_{\kappa}^1& = & -\frac{\Delta l}{2}\Big[ \Big( 2I_2\zeta_1+\psi_1(1-
       2\zeta_1)\Big) \Big(\Theta^1_1\Big)^2 +I_1^1-\psi_1 \Big]\Theta^n_1
          \nonumber \\
   \K_{\kappa}^q& = & -\frac{\Delta l}{2}\Big[\Big(4(I_{q+1}-\psi_q)
           \zeta_{q-1}\zeta_q+
            3(\psi_q-\psi_{q-1})\zeta_{q-1} + 2 \psi_{q-1}
            \Big) \Big(\Theta^{q-1}_{1}\Big)^2\zeta_{q-1}  \nonumber \\
       &  & + 2(I_{q-1}^{q-1}-\psi_{q-1})\zeta_{q-1} + 
            \psi_{q-1}-\psi_q \Big] \Theta^n_{q}, \quad 1<q<n
  \label{eq:wfuns:loswfs2} \\
   \K_{\kappa}^n& = & -\frac{\Delta l}{2}\Big[\Big( 2I_n\zeta_{n-1}+
         \psi_{n-1}(1-2\zeta_{n-1}) \Big)\Big(\Theta^{n-1}_{1}\Big)^2\zeta_{n-1}+ 
             \nonumber \\   
       & &  + I_{n-1}^{n-1}-\psi_{n-1}\Big] \zeta_{n-1} \nonumber
 \end{eqnarray}
 The function calculating these LOS WFs takes the total spectrum as
 input (that is, $I_n^n$) and it is then most suitable to iterate
 downwards, starting with point $n$. For each iteration, the
 quantities are updated as
 \begin{eqnarray}
   I_q = I_{q+1}\zeta_q + \psi_q(1-\zeta_q) \nonumber
 \end{eqnarray}
 \begin{eqnarray}
   \Theta_{1}^{q-1} =  \frac{\Theta_{1}^{q}}{\zeta_{q-1}} \nonumber
 \end{eqnarray}
 \begin{eqnarray}
   I_{q-1}^{q-1} = \frac{I_q - \psi_{q-1}(1-\zeta_{q-1})
       (1+\big(\Theta^{q-1}_{1}\big)^2\zeta_{q-1})}{\zeta_{q-1}} \nonumber
 \end{eqnarray}
 The iteration is started by setting $I_n$ to cosmic
 background radiation, or correspondingly, and setting
 $\Theta^n_1$ to the square root of the total transmission. As
 mentioned above, $I_n^n$ is an input to the function.
 
 No special attention needs to be given here to possible ground
 reflections.  This as the effects of a ground reflection are already
 included in $I_n^n$ and $\Theta^n_1$ when starting the iteration. The
 procedure of setting $\Theta^n_1$ to the square root of the total
 transmission maintains the symmetry and makes it possible to treat
 the ground as an imaginary altitude ``below'' point 1. If there is a
 ground reflection, $\Theta^1_1$ and $I_1^1$ equal $\sqrt{1-e}$ and
 $eB$, respectively, at the end of the iteration.


 

 \subsection{1D downward looking observations}
  \label{sec:wfuns:down}
  Downward observation from an aircraft or a balloon can mainly be
  treated as a combination of limb sounding and upward looking
  observations.  The altitudes below the platform altitude are covered
  by the limb sounding expressions with a suitable choice of $I_q$ for
  the highest point. The altitudes above the platform altitude are
  treated by the upward looking equations, but also considering the
  transmission through the lower altitudes. 
  
  If $q$ is the index for platform altitude, the intensity can be
  expressed as
  \begin{eqnarray}
   I &=& \Big(I_{q+1}\zeta_q\zeta_{q-1} +\psi_q(1-\zeta_q)\zeta_{q-1} + 
           \psi_{q-1}(1-\zeta_{q-1})\Big)\Big(\Theta^{q-1}_{1}\Big)^2
           \zeta_{q-1} + \nonumber \\
      & & + I_{q-1}^{q-1}\zeta_{q-1} + \psi_{q-1}(1-\zeta_{q-1})
    \label{eq:wfuns:idown}
  \end{eqnarray}
  and the corresponding WF is
  \begin{eqnarray}
   \K_{\kappa}^q& = & -\frac{\Delta l}{2}\Big[\Big(3(I_{q+1}-\psi_q)
           \zeta_{q-1}\zeta_q+ 2(\psi_q-\psi_{q-1})\zeta_{q-1} + \psi_{q-1} \Big)
           \Big(\Theta^{q-1}_{1}\Big)^2 + \nonumber \\
      & &  + I_{q-1}^{q-1}-\psi_{q-1}\Big]\zeta_{q-1}
  \end{eqnarray}



\section{Absorption LOS WFs for optical thicknesses}
 \label{sec:wfuns:absloswfs2}

 This section treats the absorption LOS WFs for cases when emission
 can neglected. For such pure absorption calculations the output
 of ARTS is optical thicknesses (instead of e.g. transmissions) and
 for these conditions the absorption LOS WFs get very simple.

 \subsection{Single pass}
 The optical thickness $(\tau)$ is for single pass cases (cf. Eq. 
 \ref{eq:rte:tau})
 \begin{equation}
   \tau = \Delta l \left( \frac{\kappa_1+\kappa_2}{2} +
                          \frac{\kappa_2+\kappa_3}{2} + \dots +
                          \frac{\kappa_{n-2}+\kappa_{n-1}}{2} +
                          \frac{\kappa_{n-1}+\kappa_n}{2} \right)
 \end{equation}
 and we have that
 \begin{eqnarray}
   \K_{\kappa}^1& = & \Delta l / 2 \nonumber \\
   \K_{\kappa}^q& = & \Delta l, \quad 1<q<n  \\
   \K_{\kappa}^n& = & \Delta l / 2 \nonumber
 \end{eqnarray}


 \subsection{1D limb sounding}
 For limb sounding each altitude is passed twice and the total optical 
 thickness is double the optical thickness from the tangent point to the
 atmospheric limit. This fact results in that the absorption LOS WFs
 for 1D limb sounding are just the single pass ones multiplicated by two:
 \begin{eqnarray}
   \K_{\kappa}^1& = & \Delta l  \nonumber \\
   \K_{\kappa}^q& = & 2 \Delta l, \quad 1<q<n  \\
   \K_{\kappa}^n& = & \Delta l  \nonumber
 \end{eqnarray}
 

 \subsection{1D downward looking observations}
 If $q$ is the point where the sensor is placed, the optical thickness
 is
 \begin{equation}
   \tau = \Delta l \left( \frac{\kappa_q+\kappa_{q-1}}{2} + \dots+
                          \frac{\kappa_2+\kappa_1}{2} + \dots +
                          \frac{\kappa_{q-1}+\kappa_q}{2} +
                          \frac{\kappa_q+\kappa_{q+1}}{2} \dots \right)
 \end{equation}
 and the absorption LOS WF for this altitude is accordingly
 \begin{equation}
   \K_{\kappa}^q =  \frac{3}{2} \Delta l
 \end{equation}





\section{Source line of sight weighting functions}
 \label{sec:wfuns:sourceloswfs}

 The source line of sight weighting functions are defined as
 \begin{equation}
   \K_{\sigma}^q =  \frac{\partial\iv}{\partial \sigma^q}
  \label{eq:wfuns:sloswfs}
 \end{equation}
 These weighting functions express how the intensity is affected by
 changes of the source function at the points of the line of sight.
 The source and absorption LOS WFs are tightly related and this
 section follows closely Section \ref{sec:wfuns:absloswfs}.


 \subsection{Single pass}
  \label{sec:wfuns:single2}
  As, for example,
  \begin{equation}
    \psi_{q} = \frac{\sigma_q+\sigma_{q+1}}{2}
  \end{equation}
  the derivate of the mean source function values with respect to 
  $\sigma_q$ is
  \begin{equation}
    \frac{\partial \psi_{q-1}}{\partial \sigma_q} = 
    \frac{\partial \psi_q}{\partial \sigma_q} = \frac{1}{2}
   \label{eq:wfuns:dpsi}
  \end{equation}
  This derivate for other $\psi$ terms is zero.
 
  Using \ref{eq:wfuns:mpbi}, the source LOS WFs for upward looking
  observations can be determined to be
  \begin{eqnarray}
    \K_{\sigma}^q &=& \frac{1-\zeta_1}{2}, \quad q=1 
     \nonumber \\
    \K_{\sigma}^q &=& \frac{1-\zeta_{q-1}\zeta_q}{2} 
                                            \Theta^{q-1}_1, \quad 1<q<n \\
    \K_{\sigma}^q &=& \frac{1-\zeta_{n-1}}{2}\Theta^{n-1}_{1}, \quad q=n
     \nonumber
  \end{eqnarray}
  For ground points in 2D calculations, the WFs are (cf. Eq. 
  \ref{eq:wfuns:mpbi_ground})
  \begin{equation}
    \K_{\sigma}^q = \frac{(1-\zeta_q)(1-e)\zeta_{q-1}+1-\zeta_{q-1}}{2} 
                                            \Theta^{q-1}_1, \quad 1<q<n \\
  \end{equation}
  The practical calculations, such as the updating of $\Theta$, follow the
  absorption LOS WFs (Sec. \ref{sec:wfuns:single}).


 \subsection{1D limb sounding}
  \label{sec:wfuns:limb2}
  The 1D limb sounding source LOS WFs are (derived using Eq.
  \ref{eq:wfuns:limb1})
  \begin{eqnarray}
    \K_{\sigma}^q & = & \frac{1}{2} \Big(1-\zeta_1\Big) \Big(1+
        \Big( \Theta^1_1 \Big)^2\zeta_1 \Big) \Theta^n_2, 
                                                    \quad q=1  \nonumber \\
    \K_{\sigma}^q & = & \frac{1}{2} \Big[ (1-\zeta_{q-1}\zeta_q)
           \Big(\Theta^{q-1}_{1}\Big)^2\zeta_{q-1}\zeta_q + 
           (1-\zeta_{q-1})\zeta_q + \nonumber \\
      & & + 1-\zeta_q \Big] \Theta^n_{q+1}, \quad 1<q<n \\
    \K_{\sigma}^q & = & \frac{1}{2} \Big( (1-\zeta_{n-1}) \Big(
           \Theta^{n-1}_{1}\Big)^2\zeta_{n-1} + 1-\zeta_{n-1} \Big), \quad q=n \nonumber
  \end{eqnarray}
  The practical calculations follow the absorption LOS WFs (Sec.
  \ref{sec:wfuns:limb}).


 \subsection{1D downward looking observations}
  \label{sec:wfuns:down2}
  The source LOS WFs for downward looking observations are determined
  by the upward and the limb sounding expressions in the same manner
  as for the absorption LOS WFs (Sec. \ref{sec:wfuns:down}).

  The LOS WF for the index corresponding to the platform altitude is
  (cf. Eq. \ref{eq:wfuns:idown})
  observations can be determined to be
  \begin{equation}
   \K_{\sigma}^q = \frac{1}{2}\Big[ (1-\zeta_{q-1}\zeta_q) 
           \Big(\Theta^{q-1}_{1}\Big)^2
           \zeta_{q-1} + 1-\zeta_{q-1} \Big]\
  \end{equation}



\section{Transformation from vertical altitudes to distances along LOS}
 \label{sec:wfuns:bases}
 
 \subsection{Basis functions} 
 The source function and the absorption, both
 as a function of vertical altitude $(\mat{k})$ and along the LOS
 $(\kappa)$, are assumed to vary linear between the points of the grid
 of concern. The functions to express the quantities between grid
 points are denoted as basis functions. For piecewise linear functions,
 the basis functions decline, from the point of interest, linearly
 down to zero at neighboring points. Such functions are here denoted
 as tenth functions (Fig. \ref{fig:wfuns:zbasis}).
 
 
 \subsection{Transformation from $z$ to $l$} 
 The forward model uses
 internally a grid along the line of sight (Sec. \ref{sec:los}), while
 the atmospheric WF matrices are calculated for some user specified
 vertical grid, and a transformation between these two grids must be
 performed. This transformation is achieved by the terms,
 $\partial\kappa/ \partial \mat{k}^p$ and $\partial \sigma / \partial
 S^\mat{p}$. As the source function and the absorption are assumed to
 have the same functional behaviour (piecewise linear), these two
 terms are identical if the retrieval grid is the same for both quantities:
 \begin{equation}
   \frac{\partial \kappa}{\partial \mat{k}^p} =
   \frac{\partial \sigma}{\partial S^\mat{p}}
  \label{eq:wfuns:sandk}
 \end{equation}
 \begin{figure}[t]
  \begin{center}
   \includegraphics*[width=0.7\hsize]{fig_absbasis_z}
   \caption{Examples on basis functions for a vertical grid with a 1 km
            spacing: \lsolid~30~km, \ldashed~31~km and \ldashdot~32~km.}
   \label{fig:wfuns:zbasis}  
  \end{center}
 \end{figure}
 \begin{figure}[t]
  \begin{center}
   \includegraphics*[width=0.7\hsize]{fig_absbasis_l}
   \caption{The basis functions of Figure \ref{fig:wfuns:zbasis} shown
            as a function of the distance from the tangent point, where
            $z_{tan}=30$ km.}
   \label{fig:wfuns:lbasis}  
  \end{center}
 \end{figure}
 For example, the term $\partial\kappa/ \partial \mat{k}^p$ gives the
 relationship between the absorption along the LOS and a change of the
 absorption at one altitude.  Figure \ref{fig:wfuns:lbasis}
 exemplifies $\partial\kappa/ \partial \mat{k}^p$ for three altitudes.
 Ideally, the following relationship should be fulfilled for all $z$
 \begin{equation}
   \sum_i\mat{k}^i\phi^i_\mat{k}(z(l)) = \sum_j \kappa^j\phi^j_{\kappa}(l)
  \label{eq:wfuns:bases}
 \end{equation}
 where $\phi_\mat{k}$ and $\phi_{\kappa}$ are the basis functions for
 $\mat{k}$ and $\kappa$, respectively. However, as can be seen in
 Figure \ref{fig:wfuns:lbasis}, $\phi^i_\mat{k}$ expressed along the
 LOS is not a piecewise linear function and cannot be fitted perfectly
 by the basis $\phi_{\kappa}$. Hence, some approximation is needed,
 and the most natural choice for this approximation is to fulfill
 Equation \ref{eq:wfuns:bases} only for the grid points along the LOS:
 \begin{equation}
   \kappa^q = \sum_i\mat{k}^i\phi^i_\mat{k}(z(\mat{l}^q))
 \end{equation}
 where $\mat{l}^q$ is the distance along the LOS for the corresponding to
 $\kappa^q$. Note that at $\mat{l}^q$ all $\phi_{\kappa}^j$ are zero except
 for $\phi_{\kappa}^q$, that is unity.

 We have now that
 \begin{equation}
   \frac{\partial \kappa^q}{\partial \mat{k}^p} = \phi^p_\mat{k}(z(\mat{l}^q))
  \label{eq:wfuns:kappak}
 \end{equation}
 Hence, term $\partial\kappa/ \partial \mat{k}^p$ is determined by the
 values of $\phi^p_\mat{k}$ at the altitudes corresponding to the grid
 points of the LOS.
 
 Assuming that the LOS altitude $q$, $z_{\kappa^q}$, is found between
 retrieval points $p-1$ and $p$, at the altitudes $z_{\mat{k}^{p-1}}$ and 
 $z_{\mat{k}^p}$, respectively, we have that
 \begin{equation}
   \frac{\partial \kappa^q}{\partial \mat{k}^p} =
   \frac{z_{\kappa^q}-z_{\mat{k}^{p-1}}}{z_{\mat{k}^{p}}-z_{\mat{k}^{p-1}}}
  \label{eq:wfuns:zz}
 \end{equation}
 If $z_{\kappa^q}$ is further away from $z_{\mat{k}^p}$ than the neighboring
 retrieval points, the derivative is zero. The derivative is also treated to 
 be zero if $z_{\kappa^q}$ is outside the retrieval grid (that is, below
 or above all retrieval altitudes).

 The basis functions for $\mat{k}$ change if the retrieval grid is
 changed, and as the retrieval grid is individual for the species, 
 temperature etc., the term $\partial\kappa/ \partial \mat{k}^p$ 
 must be determined for each calculation of a WF matrix.


\section{Species WFs}
 \label{sec:wfuns:species}
 
 As it is assumed here that the species have no influence on
 the source function, species WFs are calculated as (cf. Eq.
 \ref{eq:wfuns:taylor2})
 \begin{equation}
    \K_{\xt}^p = \Hm
                 \frac{\partial\iv}{\partial \kappa}
                 \frac{\partial \kappa}{\partial \mat{k}^p}
                 \frac{\partial \mat{k}^p}{\partial \xt^p}
  \label{eq:wfuns:species}
 \end{equation}
 The term $\partial\iv / \partial \kappa$ is described in Section
 \ref{sec:wfuns:absloswfs}, while the term $\partial \kappa /\partial
 \mat{k}^p$ is treated in Section \ref{sec:wfuns:bases}, and it
 remains to determine $\partial \mat{k}^p / \partial \xt^p$. It is
 assumed below in this section that \xt\ only represents a single 
 species and that the species absorption can be written as
 \begin{equation}
   \mat{k}^p = \mat{\bar{k}}^p_s \xt^p + \sum_{i\ne s} \mat{k}^p_i
  \label{eq:wfuns:kspecies}
 \end{equation}
 where $p$ is the altitude of concern, $\mat{\bar{k}}_s$ is the
 absorption of the species of interest, normalized to the units of the
 corresponding values of \xt\ (or \bt) and $\mat{k}_i$ the total
 absorption for other species. Equation \ref{eq:wfuns:kspecies}
 assumes that a change for one species does not influence the
 absorption of other species, and that the shape of the absorption for
 one species does not change with the abundance of that species.  This
 assumption is not valid, for example, when the amount of different
 species must be considered when calculating the pressure broadening,
 and not only the total absorption. The validity of the analytical
 expressions for the WFs is discussed in Section
 \ref{sec:wfuns:approaches:anal}.

 If Equation \ref{eq:wfuns:kspecies} is valid, we have then that
 \begin{equation}
   \frac{\partial \mat{k}^p}{\partial \xt^p} = \mat{\bar{k}}^p_s
  \label{eq:wfuns:dkspecies}
 \end{equation}
 Different units for species retrievals are allowed. The possible units are
 \begin{enumerate}
    \item Fractions of linearization state [-], i.e. $\xt/\xt_0$ where
          $\xt_0$ is the linearization state 
    \item Volume mixing ratio [-] (no dimension)
    \item Number density [molecules/m$^3$)
 \end{enumerate}
 Accordingly, for the practical calculations, the absorption of the
 species of interest is needed, and a possibility to scale to the
 absorption from the unit used by the forward model to the other two
 units considered.
 
 It is advantageous for the retrieval that the values of \xt\ are of
 similar magnitudes \citep{schimpf:97,eriksson:99} as the numerical
 precision is limited. This fact makes WFs
 in fractions of the linearization state (or rather, the a priori
 state) interesting as the values of \xt\ are then all around 1. In 
 addition, Equation \ref{eq:wfuns:dkspecies} is especially simple
 for this case:
 \begin{equation}
   \frac{\partial \mat{k}^p}{\partial \xt^p} = \mat{k}^p_s
 \end{equation}
 as $\xt^p=1$.


\section{Continuum absorption WFs}
 \label{sec:wfuns:cont}

 These WFs are used to fit unknown absorption that varies smoothly inside
 the frequency range covered. This absorption
 is added to the species absorption:
 \begin{equation}
   \mat{k}^p = \mat{k}^p_s + \mat{k}^p_c
 \end{equation}
 where $\mat{k}^p_s$ is the summed species absorption and $\mat{k}^p_s$
 the continuum absorption.
 
 The continuum absorption is represented by a polynomial for each
 altitude. The polynomials are characterized by the magnitude of the
 absorption at a number of points inside the frequency range covered
 (Fig. \ref{fig:wfuns:cont}). This approach was selected as it gives
 the possibility to impose positive constraints in a straightforward
 manner. A direct polynomial representation ($k=k_0+k_1\f+k_2\f^2...$) 
 is less favorable regarding this aspect.
 
 \begin{figure}[t]
  \begin{center}
   \includegraphics*[width=0.95\hsize]{contfit}
   \caption{Fit of continuum absorption with off-sets at three 
            positions ($n_{cont}=2$). The outermost frequencies, here 
            $\f_1$ and $\f_3$, are placed at the end points of the 
            range covered ($\f_{min}$ and $\f_{max}$, respectively).}
   \label{fig:wfuns:cont}  
  \end{center}
 \end{figure}

 
 The number of points is $n_{cont}+1$ where $n_{cont}$ is the
 polynomial order selected.  The points are equally spaced between the
 lowest and highest frequency, $\f_{min}$ and $\f_{max}$, considered.
 Figure \ref{fig:wfuns:cont} exemplifies this for $n_{cont}=2$.  The
 points are accordingly placed at the following frequencies
 \begin{equation}
   \f_i = \f_{min} + \frac{(\f_{max}-\f_{min})(i-1)}{n_{cont}}, \
          \quad 1 \leq i \leq (n_{cont}+1)
  \label{eq:wfuns:cont:f}
 \end{equation}
 This equation results in that the single point for $n_{cont}=0$ is
 placed at $\f_{min}$, but the position of the frequency point is
 for this case of no importance as the corresponding WF is constant
 (as a function of frequency). With other words, 
 if $n_{cont}=0$, the WFs are simply 
 \begin{equation}
   \frac{\partial \mat{k}^p}{\partial \xt^p_1} = 1
 \end{equation}
 To determine the frequency dependency of the WFs for higher values of
 $n_{cont}$, the Lagrange's formula can be used. This formula gives
 the polynomial of order $N-1$ that passes through $N$ fixed points
 \citep[][Eq. 3.1.1]{press:92}:
 \begin{eqnarray}
   k(\f) &=& \frac{(\f-\f_2)(\f-\f_3)\dots(\f-\f_N)}
                  {(\f_1-\f_2)(\f_1-\f_3)\dots(\f_1-\f_N)}
           x_1 + \nonumber \\ 
       & & +\frac{(\f-\f_1)(\f-\f_3)\dots(\f-\f_N)}
                 {(\f_2-\f_1)(\f_2-\f_3)\dots(\f_2-\f_N)}
           x_2 + \cdots + \nonumber \\
       & & +\frac{(\f-\f_1)(\f-\f_2)\dots(\f-\f_{N-1})}
                 {(\f_N-\f_1)(\f_N-\f_2)\dots(\f_N-\f_{N-1})} x_N
  \label{eq:wfuns:lagrange}
 \end{eqnarray}
 where $x_i$ is the absorption at the selected frequency points, $\f_i$,
 that are given by Equation \ref{eq:wfuns:cont:f}, and $N=n_{cont}+1$.
 
 The frequency dependency of the continuum WFs can be obtained by
 differentiating Equation \ref{eq:wfuns:lagrange}:
 \begin{equation}
   \frac{\partial \mat{k}^p(\f)}{\partial \xt^p_i} =
   \frac{(\f-\f_1)\dots(\f-\f_{i-1})(\f-\f_{i+1})\dots(\f-\f_N)}{(\f_i-\f_1)\dots(\f_i-\f_{i-1})(\f_i-\f_{i+1})\dots(\f_i-\f_N)}
 \end{equation}
 This equation gives, for example, for $n_{cont}=1$
 \begin{eqnarray}
   \frac{\partial \mat{k}^p(\f)}{\partial \xt^p_1} &=& \frac{\f_{max}-\f}
          {\f_{max}-\f_{min}}, \quad \f_{min}\leq \f \leq \f_{max} \\
   \frac{\partial \mat{k}^p(\f)}{\partial \xt^p_2} &=& \frac{\f-\f_{min}}
          {\f_{max}-\f_{min}}, \quad \f_{min}\leq \f \leq \f_{max}
 \end{eqnarray}
 Note that these WFs have no altitude variation. Or with
 other words, they are identical for all $p$.


\section{Temperature profile WFs}
 \label{sec:wfuns:temp}
 
 A critical factor for the calculation of temperature WFs is if
 hydrostatic equilibrium is assumed or not. If hydrostatic equilibrium
 is neglected, the WFs can be calculated by semi-analytical
 expressions, while if hydrostatic equilibrium is assumed, the WFs are
 obtained by perturbations. The analytical version is so far only
 implemented for emission measurements (and not for transmission
 measurements).


 \subsection{Without hydrostatic equilibrium}
 
 For some measurement situations it can be questionable to assume that
 the pressure, temperature and geometrical altitude, valid for the
 measurement, fulfill the law of hydrostatic equilibrium. One example
 is 1D limb sounding when there is a large horizontal distance between
 the nadir point of the tangent point for the start and end points of
 the scan. This is, for example, the case for the Odin observations
 where the tangent point will move in the latitude direction with a
 speed of about 9 km/s and a scan takes 1 -- 2 minutes.
 
 If the constrain of hydrostatic equilibrium is neglected, WFs for the
 temperature profile can be calculated following Equation
 \ref{eq:wfuns:taylor2}, that is:
 \begin{eqnarray}
    \K_{\xt}^p = \Hm \Bigg[ \frac{\partial\iv}{\partial \sigma}
                 \frac{\partial \sigma}{\partial \mat{S}^p} 
                 \frac{\partial \mat{S}^p}{\partial \mat{t}^p} +
                 \frac{\partial\iv}{\partial \kappa}
                 \frac{\partial \kappa}{\partial \mat{k}^p}
                 \frac{\partial \mat{k}^p}{\partial \mat{t}^p} \Bigg]
 \end{eqnarray}  
 where $\mat{t}$ is the vector describing the vertical temperature profile. 
 
 The term $\partial \iv/\partial \sigma$, the source LOS WFs, are
 derived in Section \ref{sec:wfuns:sourceloswfs}, while the absorption
 LOS WFs ($\partial \iv/\partial \kappa$) are found in Section
 \ref{sec:wfuns:absloswfs}. As a single grid is here of concern,
 Equation \ref{eq:wfuns:sandk} is valid, that is, $\partial\kappa/
 \partial \mat{k}^p$ equals $\partial \sigma / \partial S^\mat{p}$.
 These two terms are discussed in Section \ref{sec:wfuns:bases}.
 
 It is noteworthy that a change of the temperature inside an
 atmospheric layer will change the line-of-sights for beams passing
 this altitude, but this is here neglected. See further Section
 \ref{sec:wfuns:approaches:anal}.

 Here it is assumed that $S$ equals the Planck function, $B$
 (Equation \ref{eq:rte:planck}), and the derivative of the source
 function with respect to the temperature is (see also Equation 44 of
 \citet{eriksson:00a})
 \begin{equation}
   \frac{\partial S}{\partial T} = \frac{h\f}{k_BT^2}
        \Big( e^{h\f/k_BT} - 1  \Big)^{-1}B(\f,T)
   \label{eq:wfuns:dsdt}
 \end{equation}
 The term $\partial \mat{S}^p / \partial \mat{t}^p$ is calculated
 using Equation \ref{eq:wfuns:dsdt} where $T$ is replaced by $\mat{t}^p$.

 The term $\partial \mat{k}^p/\partial \mat{t}^p$ cannot easily be
 determined analytically. Instead, the total absorption is calculated
 for a temperature profile that is 1~K higher at all altitudes than
 the assumed profile. The difference between the two absorption
 matrices are then interpolated to the temperature profile retrieval
 grid, giving an estimation of the derivative of the absorption
 with respect to the temperature at the grid altitudes. Schematically
 \begin{eqnarray}
   \frac{\partial \mat{k}^p}{\partial \mat{t}^p} = \Upsilon(k(T_0+1)-k(T_0))
     \nonumber
 \end{eqnarray}
 where $\Upsilon$ is the interpolating function from the vertical
 absorption grid to the retrieval grid, $k$ the total absorption, and
 $T_0$ the assumed temperature profile.
 

 \subsection{With hydrostatic equilibrium}
 
 The gases in the atmosphere behave like an ideal gas, and the pressure,
 the temperature and the vertical altitudes above one point are
 linked by the fact that hydrostatic equilibrium must be fulfilled
 (see Section \ref{sec:los:hse}). 

 The temperature WFs with hydrostatic equilibrium are calculated by
 perturbations (Eq. \ref{eq:wfuns:perturb}). See further the on-line
 information (type \verb|arts -d kTemp|).


%\section{WF for ground emission factor}
% \label{sec:wfuns:eground}
% 
% This WF is not yet implemented but this can easily be done.

%%% Local Variables: 
%%% mode: latex 
%%% TeX-master: "uguide" 
%%% End:


%
% To start the document, use
%  \levela{...}
% For lover level, sections use
%  \levelb{...}
%  \levelc{...}
%
\levela{Measurement errors}
 \label{sec:measerr}


%
% Document history, format:
%  \starthistory
%    date1 & text .... \\
%    date2 & text .... \\
%    ....
%  \stophistory
%
\starthistory
  000315 & Created and written by Patrick Eriksson.\\
\stophistory


%
% Symbol table, format:
%  \startsymbols
%    ... & \verb|...| & text ... \\
%    ... & \verb|...| & text ... \\
%    ....
%  \stopsymbols
%
%
\startsymbols
  -- & -- & -- \\
 \label{symtable:measerr}     
\stopsymbols



%
% Introduction
%
Following Equation \ref{eq:formalism:fm},
\begin{eqnarray}
   \y = \fm + \merr, \nonumber
\end{eqnarray}
measurement errors, \merr\, are here defined as errors that are
additive to the spectrum, that is, not dependent on the actual spectrum.
Error sources falling into this category are thermal noise and
baseline ripples (there is a small influence of the magnitude of the
spectrum on the thermal noise but this effect is normally totally
negligible).

The term baseline ripple is used here as a common name for all instrumental
imperfections causing a distortion of the spectra, for example,
reflections inside the receiver, adding theoretically a sinusoidal term
to the spectrum.



\levelb{General}
 \label{sec:measerr:general}
 
 The sensor transfer matrix can be neglected when treating measurement
 errors as these errors are assumed to be additive to the spectra. On
 the other hand, a possible data reduction must be considered. This
 fact can also be understood by Equation \ref{eq:formalism:datared}:
 \begin{eqnarray}
   \y = \Hd \y' = \Hd (\Hs \iv + \merr') = \Hm \iv + \merr \nonumber
 \end{eqnarray}
 Using this equation, a measurement error WF can be
 written as
 \begin{equation}
    \K_{\xt}^p = \frac{\partial \y}{\partial \xt^p} 
               = \frac{\partial \merr}{\partial \xt^p}
               =  \Hd \frac{\partial \merr'}{\partial \xt^p}
  \label{eq:measerr:kx}
 \end{equation}
 Accordingly, quantities connected with the measurement errors shall be
 multiplicated with the data reduction matrix \Hd, this in contrast to
 the atmospheric WFs where the total reduction sensor matrix must be applied
 (Eq. \ref{eq:formalism:kx2}).



\levelb{Thermal noise}
 \label{sec:measerr:tn}
 
 The nature of the thermal noise differs from all other variables and
 error sources. The most distinct feature of the thermal noise is the
 low correlation between the measurements channels, in fact, the
 thermal noise is normally assumed to be totally uncorrelated. Such an
 assumption results in that a variable for each channel would be
 needed to model, or to fit, the measurement noise, and this is not a
 practical solution. In addition, it is not even of interest to know
 the actual magnitude of the thermal noise for each single
 measurement, we are instead interested in the statistical
 characteristics of the thermal noise.  The special nature of the
 thermal noise has the consequence that this term is treated
 differently than the other variables. Instead of providing weighting
 functions, the forward model gives the covariance matrix for the
 thermal noise.
 
 Thermal noise is introduced in two ways, by the observation of the
 atmosphere, and by the calibration process. The first part is here
 denoted as measurement thermal noise, while the latter is denoted as
 calibration thermal noise. In many cases, there is no practical
 difference between the two terms and they can together be treated as
 measurement thermal noise. However, if a single calibration
 measurement is used for a number of atmospheric spectra that are
 inverted jointly, as is the normal case for limb sounding, the error
 introduced by the calibration is totally correlated between the
 different viewing angles and it could be of importance to consider
 this fact.
 
 The magnitude of the thermal noise, expressed in brightness
 temperature, is described by the radiometer noise formula
 \begin{equation}
   \sigma_{tn}^i = \frac{q\left(T_{rec}+T_a^i\right)}{\sqrt{\Delta \f^i \tau}}
  \label{eq:measerr:tn}
 \end{equation}
 where $\sigma_{tn}^i$ is the standard deviation of the thermal noise
 for channel $i$, $q$ a compensation factor $T_{rec}$ the receiver
 noise temperature, $T_a$ the antenna temperature, $\Delta \f$ the
 channel bandwidth and $\tau$ the integration time.
 
 The factor $q$ is used to compensate for extra noise introduced by
 the calibration, losses in the spectrometer etc. It is important to
 define $q$ and $\tau$ consistently. Let us take an ordinary load
 switching instrument as example, where one half of the time is used
 to measure the atmosphere and the other half is used to observe a
 reference load. If then $\tau$ gives the total integration time, $q$
 should be (about) a factor $\sqrt{2}$ higher than when $\tau$ gives
 only the integration time for the atmospheric observations.

 
 

 \levelc{Measurement thermal noise}
 \label{sec:measerr:mtn}
 
% As mentioned above, measurement thermal noise is here defined as
% thermal noise that is uncorrelated between the different viewing
% angles. Often it is possible to treat both measurement and calibration
% noise together, e.g. this is the normal case for ground-based
% measurements, and to cover these cases, a noise factor, $q$ is
% included in the radiometer noise formula:
% \begin{equation}
%   \sigma_{tn}^i = \frac{q(T_{rec}+T_a)}{\sqrt{\Delta \f^i \tau}}
%  \label{eq:measerr:qtn}
% \end{equation}
 
 The thermal noise is often assumed to be uncorrelated between the
 measurement channels, and the corresponding covariance matrix,
 $\mat{S}$ is then diagonal, where the diagonal elements are
 \begin{equation}
   \mat{S}_{tn}^{ii} = \left( \sigma_{tn}^i \right)^2
  \label{eq:measerr:Stn_diag}
 \end{equation}
 where $\mat{S}^{ii}$ is element $(i,i)$ of the matrix.
 
 However, for most spectrometer types there exist in fact some
 correlation of the noise between the channels as there is an overlap
 of the channel frequency responses.  The inter-channel correlation of
 the thermal noise can be treated in the forward model by three
 different correlation functions: (1) gaussian
 \begin{equation}
  c^{ij} = exp\left(-\left(\frac{\f_i-\f_j}{f_c}\right)^2\right)
 \end{equation}
 (2) exponential
 \begin{equation}
  c^{ij} = exp\left(-\frac{|\f_i-\f_j|}{f_c}\right)
 \end{equation}
 and (3) tenth
 \begin{eqnarray}
  c^{ij} &=& 1-\frac{|\f_i-\f_j|(1-e^{-1})}{\f_c}, \quad 
            |\f_i-\f_j| < \frac{\f_c}{(1-e^{-1})} \nonumber \\
  c^{ij} &=& 0, \quad |\f_i-\f_j| \geq \frac{\f_c}{(1-e^{-1})}
 \end{eqnarray}
 where $\f_c$ is the frequency distance where the correlation has
 declined to $e^{-1}$, the frequency correlation length, and $\f_i$
 the middle frequency of channel $i$ (Fig. \ref{fig:measerr:cfuns}).
 It is also possible to apply a threshold for the correlation, where
 all $c^{ij}$ below the threshold value are set to 0.

 \begin{figure}
  \begin{center}
   \begin{minipage}[c]{0.65\textwidth}
    \centering
    \includegraphics*[width=0.99\hsize]{Figs/fig_corrfuns.eps}
   \end{minipage}%
   \hspace{0.03\textwidth}%
   \begin{minipage}[c]{0.30\textwidth}
    \centering
    \caption{The frequency correlation functions. The frequency is scaled to
             the correlation length as $(\f_i-\f_j)/\f_c$.}
    \label{fig:measerr:cfuns}
   \end{minipage}
  \end{center}
 \end{figure}           
 
 The covariance matrix for one viewing angle with inter-channel
 correlation is
 \begin{equation}
   \mat{S}_{tn}^{ij} = c^{ij} \sigma_{tn}^i \sigma_{tn}^j
  \label{eq:measerr:Stn}
 \end{equation}
 The correlation between different viewing angles is set to 0.

 To include the effect of data reduction, the covariance matrix is
 multiplicated with \Hd\ as
 \begin{equation}
   \mat{S}_{tn} = \Hd \mat{S}_{tn}' \Hm_d^T 
   \label{eq:measerr:HSH}
 \end{equation}
 where $\mat{S}_{tn}'$ is the covariance matrix before data reduction.


 \levelc{Calibration thermal noise}
 \label{sec:measerr:ctn}
 
 In contrast to the measurement thermal noise, the calibration thermal
 noise is assumed to be totally correlated between the different
 viewing angles.  This latter thermal noise is assumed to be identical
 between the channels.  The final influence of the thermal noise from
 the calibration measurement(s) depends on the actual calibration
 scheme.  To avoid to include all details of all possible calibration
 schemes, the calibration thermal noise is modeled by a simplified
 noise formula:
 \begin{equation}
   \sigma_{tn}^i = \frac{q_{cal}}{\sqrt{\Delta \f^i}}
 \end{equation}
 where $q_{cal}$ is the thermal noise magnitude resulting of the calibration
 process for a bandwidth of 1 MHz.

 The correlation functions used for the measurement thermal noise can
 also be applied for the calibration thermal noise.  
 
 Data reduction is considered by Equation \ref{eq:measerr:HSH}.
 


\levelb{Sinusoidal baseline ripple}
 \label{sec:measerr:sin}
 
 Reflections inside the receiver give theoretically rise to a
 sinusoidal baseline ripple. The relationship between the period
 length in the spectrum, $\Delta \f_{2\pi}$, and the physical distance
 between the reflecting objects, $l$, is \citep{rohlfs:86} 
 \begin{equation}
   \Delta \f_{2\pi} = \frac{c}{2l}
 \end{equation}
 where $c$ is the speed of light.
 
 This type of baseline ripple is retrieved by expressing the sine
 functions, with unknown amplitude and phase, as a sum of sine and
 cosine functions \citep{kuntz:97}
 \begin{equation}
   \merr_{sin} = \sum_{i=1}^n \left( 
              x_i sin\left(2\pi\frac{\f}{\Delta \f_{2\pi}^i}\right) +
              x_{i+n} cos\left(2\pi\frac{\f}{\Delta \f_{2\pi}^i}\right) \right)
  \label{eq:measerr:sines}
 \end{equation}
 where $n$ is the number of ripple terms, $\f_{2\pi}^i$ the period
 length of ripple $i$ and $x_i$ are the amplitude of the sine and
 cosine functions to be determined. The length of the part of \xt\ 
 used to fit sinusoidal baseline ripples is accordingly $2n$.
 
 Using Equation \ref{eq:measerr:kx}, the
 WFs for the sine and cosine terms can be determined to be
 \begin{equation}
   \K_{\xt}^p = \Hd \mat{a}_p
 \end{equation}
 and
 \begin{equation}
   \K_{\xt}^p = \Hd \mat{b}_p
 \end{equation}
 respectively, where the elements of the vectors $\mat{a}_p$ and
 $\mat{b}_p$ are
 \begin{equation}
   \mat{a}_p^i = sin\left(2\pi\frac{\f^i}{\Delta \f_{2\pi}^p}\right) 
 \end{equation}
 and
 \begin{equation}
   \mat{b}_p^i = cos\left(2\pi\frac{\f^i}{\Delta \f_{2\pi}^p}\right),
 \end{equation}
 where $\f^i$ is the frequency for channel $i$.
 
 It should be noted that the treatment of baseline ripple neglects the
 effect of the spectrometer and Equation \ref{eq:measerr:sines} assumes
 that the widths of the spectrometer channels are much smaller than
 the period length of the ripple. However, this should be the
 situation found for most practical situations.



\levelb{Polynomial baseline ripple}
 \label{sec:measerr:pol}
 
 \begin{figure}
  \begin{center}
   \begin{minipage}[c]{0.65\textwidth}
    \centering
    \includegraphics*[width=0.99\hsize]{Figs/kpol.eps}
   \end{minipage}%
   \hspace{0.03\textwidth}%
   \begin{minipage}[c]{0.30\textwidth}
    \centering
    \caption{Polynomial WFs of order 0, 1 and 2. The scaled frequency is
             \mbox{$f'=(\f-\bar{\f})/\Delta \f$.}}
    \label{fig:measerr:kpol}
   \end{minipage}
  \end{center}
 \end{figure}           

 A polynomial representation of the baseline ripple can be suitable
 at many occasions. One example is when a sinusoidal baseline ripple 
 has a period that exceeds significantly the total frequency coverage
 of the receiver and the exact period length is not known. A baseline
 polynomial can also be used to fit continuum absorption for linear
 situations, e.g. to fit the unknown emission from the troposphere
 for ground-based observations.

 The polynomial measurement error is modeled as
 \begin{equation}
   \merr_{pol} = x_0 + \sum_{i=1}^{n_{pol}} x_i \left( 
                      \frac{\f-\bar{\f}}{\Delta \f} \right)^i
 \end{equation}
 where $n_{pol}$ is the polynomial order selected, $x_i$ are the 
 polynomial coefficients to be determined, and $\bar{\f}$ and
 $\Delta \f$ normalization factors. The part of \xt\ corresponding
 to the polynomial fit of the baseline is accordingly
 \begin{equation}
   \xt = \left[ \begin{array}{c} \vdots \\ x_0\\ x_1 \\ \vdots \\ x_{n_{pol}} \\ \vdots \end{array} \right]
 \end{equation}
 The normalization factors are needed to avoid extreme values (without
 the factors the quantity $\f^i$ would have been calculated),
 resulting in that the magnitudes of the coefficients $x_i$ will not
 deviate too strongly. The factors are calculated as
 \begin{eqnarray}
   \bar{\f} &=& \frac{\f_{min}+\f_{max}}{2} \\
   \Delta \f &=& \frac{\f_{max}-\f_{min}}{2}
 \end{eqnarray}
 where $f_{min}$ and $f_{max}$ are the minimum and maximum value,
 respectively, of the frequency grid given by the spectrometer. These
 definitions of the normalization factors give a scaled frequency grid
 extending from -1 to 1.

 The polynomial WFs are
 \begin{equation}
   \K_{\xt}^p = \Hd \mat{a}_p
 \end{equation}
 where the elements of $\mat{a}_p$ are
 \begin{equation}
   \mat{a}_p^i = \left( \frac{\f^i-\bar{\f}}{\Delta \f} \right)^p
  \label{eq:measerr:kpol}
 \end{equation}
 Note that for $p=0$, $\mat{a}_p=1$. 
 
 Examples on polynomial weighting functions are shown is Figure 
 \ref{fig:measerr:kpol}.



\levelb{Piecewise polynomial baseline ripple}
 \label{sec:measerr:ppol}
 
 If the spectrum is recorded with a number of spectrometers (or
 individual spectrometer parts) there could be a difference in the
 level between the different parts of the spectrum. Figure
 \ref{fig:wfuns:baselinefit} shows an example on such a spectrum.
 
 The baseline for such cases can be retrieved by piecewise polynomials
 where an individual polynomial is applied for each part of the
 spectrum. For frequencies inside the part of concern the WFs are
 given by Equation \ref{eq:measerr:kpol}, while for remaining
 frequencies the WFs are 0.  

 \begin{figure}[t]
  \begin{center}
   \includegraphics*[width=0.72\hsize]{Figs/fig_baselinefit.eps}
   \caption{Example on fit of baseline with piecewise polynomials.
     The top figure shows a (poor!) test measurement with the 22.2 GHz
     water vapor radiometer at Onsala Space Observatory, Sweden.  The
     spectrum was recorded by an auto-correlator spectrometer having
     four 20 MHz wide individual parts, clearly seen in the spectrum.
     The middle figure shows the measurement spectrum after a
     correction based on the retrieved baseline variables, and the
     simulated spectrum corresponding to the retrieved profile. The
     baseline is fitted by 3:rd order polynomial over the whole
     frequency range, and a 2:nd oder polynomial inside each 20 MHz
     range. The lower figure shows the difference between the spectra
     in the middle figure, the residual.}
   \label{fig:wfuns:baselinefit}
  \end{center}
 \end{figure}



%%% Local Variables: 
%%% mode: latex 
%%% TeX-master: "main" 
%%% End:


%
% To start the document, use
%  \levela{...}
% For lover level, sections use
%  \levelb{...}
%  \levelc{...}
%
\levela{Sensor variables weighting functions}
 \label{sec:wfuns_sens}


%
% Document history, format:
%  \starthistory
%    date1 & text .... \\
%    date2 & text .... \\
%    ....
%  \stophistory
%
\starthistory
  000320 & Created and written by Patrick Eriksson.\\
\stophistory


%
% Symbol table, format:
%  \startsymbols
%    ... & \verb|...| & text ... \\
%    ... & \verb|...| & text ... \\
%    ....
%  \stopsymbols
%
%
\startsymbols
  -- & -- & -- \\
 \label{symtable:wfuns_sens}     
\stopsymbols



%
% Introduction
%
This section presents weighting functions for sensor variables, beside the
ones treated as measurement errors. The covered features are calibration,
pointing and frequency instability.




\levelb{Calibration weighting functions}
 \label{sec:wfuns_sens:cal}
 
 \levelc{Proportional calibration errors} 
 This section gives the WF for situations where a calibration
 uncertainty gives an error that is directly proportinal to the noise
 free spectrum.  Such a calibration uncertainty can be encountered for
 e.g. ground-based observations of altitudes above the tropoapuse,
 where a compensation of the tropospheric attenuation must be made, as an
 error of the assumed tropospheric opacity gives rise to a proportional 
 calibration error.

 A measurement with a proportional calibration uncertainty
 can be expressed as
 \begin{equation}
   \y = \Hd\left( (1+x_{cal})\Hs\iv+\merr' \right)
 \end{equation}
 See Equation \ref{eq:formalism:datared} for definition of the variables.
 The WF for this case is easily obtained
 \begin{equation}
   \K_{\xt} = \Hd\Hs\iv = \Hm\iv = \y - \merr
 \end{equation}
 that is, the WF is identical to the (noise free) spectrum given by the forward
 model. 

 
 \levelc{Calibration load temperatures} 
 The calibration of a Dicke switched radiometer is often performed by
 observing two loads with known intensity. The calibration formula is
 then (neglecting data reduction)
 \begin{equation}
   \y^i =  I_1^i + (I_2^i-I_1^i)\frac{V_{atm}^i-V_1^i}{V_2^i-V_1^i} 
  \label{sec:wfuns_sens:loadcal}
 \end{equation}
 where $\y^i$ is the calibrated value for channel $i$, $I_1$ and $I_2$
 are the assumed intensities of the two loads, $V_{atm}$, $V_1$ and
 $V_2$ are the voltage recorded when observing the atmosphere, load 1
 and load 2, respectively.
 
 The load temperature WFs are obatined by differenting Equation
 \ref{sec:wfuns_sens:loadcal}. For example, we have that \citep{eriksson:97a}
 \begin{equation}
   \frac{\partial \y^i}{\partial I_1^i} = 1 - 
     \frac{V_{atm}^i-V_1^i}{V_2^i-V_1^i} = \frac{I_2^i-\y^i}{I_2^i-I_1^i} 
 \end{equation}
 The WF for load temperature 1 is then
 \begin{equation}
   \K_{\xt} = \Hd \mat{a}
 \end{equation}
 where the elements of the vector $\mat{a}$ are
 \begin{equation}
   \mat{a}^i = \frac{I_2^i-\y^i}{I_2^i-I_1^i}
 \end{equation}
 The corresponding expression for load 2 is
 \begin{equation}
   \mat{a}^i = \frac{\y^i-I_1^i}{I_2^i-I_1^i}
 \end{equation}
 Hence, these WFs are easily calculated if the spectrum (before data
 reduction) is at hand.



%\levelb{Pointing weighting functions}
% \label{sec:wfuns_sens:point}
% 
% No suitable analytical expression for the pointing WFs has been
% found.  The pointing is a sensor variable and the first choice for a
% perturbatation calculation would be to recalculate the sensor
% transfer matrix for a slightly changed pointing pattern. However, to
% set-up the sensor matrix can be a time consuming task, and Equation
% \ref{eq:wfuns:Hpert} was instead selected for the calculation of the
% pointing WFs.
%
% Think and check with practical calculations deciding calculation approach!!
% 
% \begin{eqnarray}
%   \int_{0}^{2\pi} r_a(\view) I(\view) \dd \view \nonumber
% \end{eqnarray}
%
% \begin{eqnarray}
%   \int_{0}^{2\pi} r_a(\view+\Delta \view) I(\view) \dd \view \approx
%   \int_{0}^{2\pi} r_a(\view) I(\view-\Delta \view) \dd \view \nonumber
% \end{eqnarray}
%
%
%\levelb{Frequency weighting functions}
% \label{sec:wfuns_sens:freq}




%%% Local Variables: 
%%% mode: latex 
%%% TeX-master: "main" 
%%% End:


%
% To start the document, use
%  \levela{...}
% For lover level, sections use
%  \levelb{...}
%  \levelc{...}
%
\levela{The art of developing ARTS}
 \label{sec:development}

%
% Document history, format:
%  \starthistory
%    date1 & text .... \\
%    date2 & text .... \\
%    ....
%  \stophistory
%
\starthistory
  020425 & Stefan Buehler: Put this part back in the AUG. Updated.\\
  011005 & Stefan Buehler: Fixed TeX warnings, updated. \\
  000728 & Stefan Buehler: Added stuff about build system and howto cut a release. \\
  000615 & Created by Stefan Buehler. For now, this is basically the
  former content of the file \verb|notes.txt|. \\
\stophistory

%
% Symbol table, format:
%  \startsymbols
%    ... & \verb|...| & text ... \\
%    ... & \verb|...| & text ... \\
%    ....
%  \stopsymbols
%
%

%
% Introduction
%
The aim of this section is to describe how the program is organized
and to give detailed instructions how to make extensions. That means,
it is addressed to the ARTS developers, not the users. If you only
want to use ARTS, you should not need to read it. \textbf{But if you
  want to make changes or additions, you should definitely read this
  carefully, since it can safe you a lot of work to understand how
  things are organized.}

\levelb{Organization}
%====================
\label{sec:development:org}
 
ARTS is written in C++ with the help of the GNU development tools
(Autoconf, Automake, etc.). It is organized in a similar manner as
most GNU packages. The top-level ARTS directory is either called
\verb|arts| or \verb|arts-x.y|, where x.y is the release number.
It contains various sub-directories, notably \verb|doc| for
documentation, \verb|src| for the C++ source code, \verb|ami| for the
MATLAB interface, and \verb|aii| for the IDL interface. The document
that you are reading right now, the ARTS User Guide, is located in
\verb|doc/uguide|.

There are two different versions of the ARTS package: The developers
version and the end-user version. Both contain the complete source
code, the only difference is that the developers version also includes
the CVS housekeeping data. If you want to join in the ARTS development
(which we of course encourage you to do), you should write an email to
the authors to obtain access to the developers version, which makes it
easier to merge your changes with the `official' ARTS program.
Furthermore, for serious development work you need a computer running
Unix, the GNU development tools, LaTeX, and the Doxygen program.  All
this is freely and easily available on the Internet, and, what is
more, all these tools are included in the standard linux
distributions like Suse and Redhat.

The end-user version contains everything that you need in order to
compile and install ARTS in a fairly automatic manner. The only
thing you should need is an ANSI-C++ compiler and the standard Unix
\verb|make| utility. Please see files \verb|arts/README| and
\verb|arts/INSTALL| for installation instructions. We are developing
with the GNU C++ compiler, no other compilers have been tried so
far.

\levelb{The ARTS build system}
%============================

As mentioned above, GNU tools are used to construct the ARTS
build system. A good introduction to the GNU build system can be found in:
\begin{quote}
  \footnotesize
  \verb|http://www.amath.washington.edu/~lf/tutorials/autoconf/|
\end{quote}
Using these tools makes a lot of things very easy, but also some
things slightly more complicated.

The most important thing to keep in mind is that an ARTS release
is not just a copy of the ARTS development tree. Instead there is a
special make target `dist' that you can use to cut a release. How this
is done in detail is described in Section \ref{sec:release}. Mostly,
the GNU tools are smart enough to figure out automatically what should
go into the release. However, this can be controlled by editing the
\verb|Makefile.am| files which can be found in almost all directories.

The support for documentation other than info and man pages is not
very good in the GNU system, so we had to use some tricks to make sure
that the Doxygen automatic documentation and the User Guide work as they
should. 

%\levelc{Configure options}
%%============================

% FIXME: Oliver, please put in a part here about the different
% ARTS specific configure options. (Particularly maintainer-mode.)

% Here are some interesting options for \verb|configure|:
%
%\begin{description}
%\item[--disable-warnings:] 
%  Compile without \verb|-Wall| on g++ compilers (by default warnings are on).
%\item[--disable-assert:] Include \verb|#define NDEBUG 1| in
%  \verb|config.h|.  The central switch to turn off all debugging
%  features (index range checking for vectors, the trace facility,
%  assertions,...) \textbf{Not yet implemented.}
%
%\end{description}


\levelc{Adding Directories or files}
%==================================

If you add directories or just files, you have to make sure that they
also go into the distribution. In some cases (e.g., program source
code files) this is done automatically. But if you add any other kind
of file, for example a data or a documentation file, you have to edit
the \verb|Makefile.am| file in that directory to make sure that your
stuff goes into the distribution. It is a good idea to always check
the release in order to see if the things you added are really there.

\levelb{Conventions}
%===================
\label{sec:development:conv}

Here are some general rules for ARTS programming:

\levelc{} Never use \verb|float| or \verb|double| explicitly, use the
type \verb|Numeric| instead.  This is set by \verb|configure| (to
\verb|double| by default). Thus, it is possible to compile the program
for \verb|float| by simply running configure with a different option.
%
% FIXME: Oliver, please say how.
%
In the same way, use Index for all integers. It can take on positive
or negative values.

\levelc{} Use \verb|Vector| and \verb|Matrix| for mathematical vectors
and matrices (with elements of type \verb|Numeric|). Use
\verb|Array<something>| to create an array of \verb|something|s. Commonly
used Arrays have been predefined, they have names like
\verb|ArrayOfString|, \verb|ArrayOfMatrix|, and so forth.

\levelc{Terminology}
Calculations are carried out in the so called workspace (WS), on
workspace variables (WSVs). A WSV is for example the variable
containing the absorption coefficients. The WSVs are manipulated by 
workspace methods (WSMs). The WSMs to use are specified in the
controlfile in the same order in which they will be
executed. 

\levelc{Global variables}
   Are not visible by default. To use them you have to declare them
   like this:
   \begin{quote}
   \verb|extern const Numeric PI;|
   \end{quote}
   which will make the global constant PI=3.14... available. Other important globals are:

   \begin{quote}
   \begin{tabular}{ll}
   \verb|full_name|&         Full name of the program, including version.\\
   \verb|parameters|&        All command line parameters.\\
   \verb|basename|&          Used to construct output file names.\\
   \verb|out_path|&          Output path.\\
   \verb|messages|&          Controls the verbosity level.\\
   \verb|wsv_data|&          WSV lookup data.\\
   \verb|wsv_group_names|&   Lookup table for the names of \emph{types} of WSVs.\\
   \verb|WsvMap|&            The map associated with \verb|wsv_data|. \\
   \verb|md_data|&           WSM lookup data.\\
   \verb|MdMap|&             The map associated with \verb|md_data|. \\
   \verb|workspace|&         The workspace itself.\\
   \verb|species_data|&      Lookup information for spectroscopic species.\\
   \verb|SpeciesMap|&        The map associated with \verb|species_data|.
   \end{tabular}
   \end{quote}
   The only exception from this rule are the output streams \verb|out0| to
   \verb|out3|, which are visible by default.

\levelc{Files}
Always use the \verb|open_output_file| and \verb|open_input_file|
functions to open files. This switches on exceptions, so that any
error occurring later on with this file will result in an
exception. (Currently not really implemented in the GNU compiler,
but please use it anyway.)

\levelc{Version numbers} 
The package version number is set in file \verb|configure.in| in the
top level ARTS directory. Always increase this when you do a CVS
commit, even for small changes. in such cases increase the last digit
by one. If you make a new distribution, increase the middle digit by
one and omit the last digit. If you make a bug-fix distribution, you
can add the last digit to indicate this. 

\levelc{Header files} 
The global header file \verb|arts.h| \emph{must} be included by every
file. Apart from that you have to see yourself what header files you
need. If you use functions from the C or C++ standard library, you
have to also include the appropriate header file.

\levelc{Documentation}
Doxygen is used to generate automatic source code documentation. See
\begin{quote}
  \verb|http://www.stack.nl/~dimitri/doxygen/|
\end{quote}
for information. There is a complete User manual there. At the moment
we only generate the output as HTML, although latex, man-page, and rtf
format is also possible. The HTML version is particularly useful for
source code browsing, since it includes the complete source code! You
should add Doxygen headers to the following:

\begin{enumerate}
\item Files
\item Classes (Including all private and public members)
\item Functions
\item Global Variables
\end{enumerate}

The documentation headers are comment blocks that look like the
examples below. They should be put above the \emph{definition} of a
function, i.e., in the \verb|.cc| file.  Some functions are defined in
the \verb|.h| file (e.g., inline member functions). In that case the
comment can be put in the \verb|.h| file.

There is an Emacs package (Doxymacs) that makes the insertion of
documentation headers particularly easy. You can find documentation of
this on the Doxymacs webpage: \verb|http://doxymacs.sourceforge.net/|.
To use it for ARTS (provided you have it), put the following in your
Emacs initialization file:

\begin{verbatim}
(require 'doxymacs)

(setq doxymacs-doxygen-style "Qt")

(defun my-doxymacs-font-lock-hook ()
  (if (or (eq major-mode 'c-mode) (eq major-mode 'c++-mode))
      (progn
        (doxymacs-font-lock)
        (doxymacs-mode))))

(add-hook 'font-lock-mode-hook 'my-doxymacs-font-lock-hook)

(setq doxymacs-doxygen-root "../doc/doxygen/html/")
(setq doxymacs-doxygen-tags "../doc/doxygen/arts.tag")
\end{verbatim}

The only really important lines are the first two, where the second
line is the one selecting the style of documentation. The next block
just turns on syntax highlighting for the Doxygen headers, which looks
nice. The last two lines are needed if you want to use the tag lookup
features (see Doxymacs documentation if you want to find out what this
is).  The package allows you to automatically insert headers. The
standard key-bindings are:
\begin{quote}
\begin{tabularx}{.8\hsize}{@{}lX}
\texttt{C-c d ?} & look up documentation for the symbol under the point.\\
\texttt{C-c d r} & rescan your Doxygen tags file.\\
\texttt{C-c d f} & insert a Doxygen comment for the next function.\\
\texttt{C-c d i} & insert a Doxygen comment for the current file.\\
\texttt{C-c d ;} & insert a Doxygen comment for a member variable on the current line (like M-;).\\
\texttt{C-c d m} & insert a blank multi-line Doxygen comment.\\
\texttt{C-c d s} & insert a blank single-line Doxygen comment.\\
\texttt{C-c d @} & insert grouping comments around the current region.\\
\end{tabularx}
\end{quote}
You can call the macros also by name, e.g., \verb|doxymacs-insert-file-comment|.

\leveld{File comment:}

Generated by \verb|doxymacs-insert-file-comment|.

\begin{verbatim}
/*!
\file   dummy.cc
\author Stefan Buehler <sbuehler@uni-bremen.de>
\date   Thu Apr 25 15:58:50 2002

\brief  A dummy file.

 This file has no purpose at all,
 it just servers as an example... 
*/
\end{verbatim}

\leveld{Function comment:}

Generated by \verb|doxymacs-insert-function-comment|.
If arguments are modified by the function you should
add `Output:' after the \verb|\param| command, just like for the
parameter \verb|a| in the example below. If a parameter is both input
and output, you should say `Output and Input:'. The documentation for
each parameter should start with a capital letter and end with a
period, like in the example below.

Author and date tags are not inserted by default, since they would be
overkill if you have many small functions. However, you should include
them for important functions. 

\begin{verbatim}
//! A dummy function.
/*! 
 This function has no purpose at all,
 it just serves as an example... 

\param  a Output: This parameter is modified by the
          function.
\param  b This is the other parameter.         
\return   Dummy value computed from a and b.         
*/
int dummy(int& a, int b);
\end{verbatim}

\leveld{Generic multi-line comment:}

Generated by \verb|doxymacs-insert-blank-multiline-comment|.

\begin{verbatim}
//! A dummy comment.
/*! 
 Some more elaborate description about this variable, 
 class, or whatever. 
*/
\end{verbatim}

\leveld{Generic single-line comment:}

Generated by \verb|doxymacs-insert-blank-singleline-comment|.

\begin{verbatim}
//! Short comment here.
\end{verbatim}


\levelb{Extending ARTS}
%======================
 \label{sec:development:extending}

\levelc{How to add a workspace variable}
%---------------------------------------

You should read Section{sec:agendas:wsvs} to understand what workspace
variables are. Here is just the practical description how a new
variable can be added.

\begin{enumerate}
\item Create a record entry in file \verb|workspace.cc|. (Just add
  another one of the \verb|wsv_data.push_back| blocks.) Take the
  already existing entries as templates. The ARTS concept works best
  if WSVs are only of a rather limited number of different types, so
  that generic WSMs can be used extensively, for example for IO.
      
  The name must be \emph{exactly} like you use it in the source code,
  because this is used to generate interface functions.
  
  Make sure that the documentation string you give explains the
  variable and its purpose well. \textbf{In particular, state the
    dimensions (in the case of matrices) and the units!} This string
  is used for the online documentation. Please take some time to write
  it carefully. Use the template at the beginning of function
  \verb|define_wsv_data()| in file \verb|workspace.cc| as a
  guideline. 

\item That's it!
\end{enumerate}


\levelc{How to add a workspace variable group}
%--------------------------------------------

You should read Section{sec:agendas:wsvs} to understand what workspace
variable groups are. Here is just the practical description how a new
group can be added.

\begin{enumerate}
\item Add a \verb|wsv_group_names.push_back("your_type")| function to
  the function \verb|define_wsv_group_names()| in \verb|groups.cc|. The
  name must be \emph{exactly} like you use it in the source code,
  because this is used to generate interface functions.
\item That's it! (But as stated above, use this feature wisely)
\end{enumerate}



\levelc{How to add a workspace method}
%-------------------------------------

You should read Section{sec:agendas:wsms} to understand what workspace
methods are. Here is just the practical description how a new
method can be added.

\begin{enumerate}
\item Create an entry in the function \verb|define_md_data| in file
  \verb|methods.cc|.  (Make a copy of an existing entry (one of the
  \verb|md_data.push_back(...)| blocks) and edit it to fit your new
  method.) Don't forget the documentation string! Please refer to the
  example at the beginning of the file to see how to format it.
\item Run:
  \verb|make|.
\item Look in \verb|auto_md.h|. There is a new function prototype
  \begin{quote}
    \verb|void <YourNewMethod>(...)|
  \end{quote}
\item Add your function to one of the \verb|.cc| files which contain method
  functions. Such files must have names starting with \verb|m_|. (See
  separate HowTo if you want to create a new source file.) The header
  of your function must be compatible with the prototype in \verb|md.h|.
\item Check that everything looks nice by running 
  \begin{quote}
    \verb|arts -d YourNewMethod|
  \end{quote}
  If necessary, change the documentation string.

\item Thats it!
\end{enumerate}


\levelc{How to add a source code file}
%---------------------------------------
\begin{enumerate}
\item Create your file. Names of files containing workspace methods should
  start with \verb|m_|.
\item You have to register your file in the file \verb|src/Makefile.am|.
  This file states which source files are needed for arts. Should be
  self-explanatory where you have to add your file. The above goes for
  source (\verb|.cc|) and header (\verb|.h|) files likewise.
\item Then go to the top level arts directory and run: \verb|reconf|.
\item Go to \verb|src| and run: \verb|cvs add <my_file>| to make your
  file known to CVS.
\end{enumerate}


\levelc{How to add an example file}
%---------------------------------------
\begin{enumerate}
\item Create your own example file. The filename should end with
  \verb|_example.arts.in|.
\item If your example uses files from the arts-data package, replace
  the path to the data package (e.g. \verb|/pool/lookup2/arts-data|)
  with \verb|@ac_arts_data@|. Configure will replace this with the
  correct path.
\item Add your file to the variable \verb|arts_examples| in the file\newline
  \verb|doc/examples/Makefile.am|.
\item Add your file to the AC\_OUTPUT list near the end of configure.in.
\item The next time when you call \verb|make| the \verb|.arts.in| file will
  be automatically converted to \verb|.arts|.
\end{enumerate}


\levelb{CVS issues}
%======================
 \label{sec:development:cvs}

The arts project is controlled by CVS. This section describes some
basic CVS commands. For more information see the extensive CVS
documentation or our own CVS Howto on:
\begin{quote}
  \verb|http://www.sat.uni-bremen.de/docs/|
\end{quote}




\levelc{How to check out arts}
%-----------------------------
\begin{enumerate}
\item Go to a temporary directory.
\item Run: \verb|cvs co -P arts|.
\end{enumerate}


\levelc{How to update (if you already have a copy)}
%--------------------------------------------------
\begin{enumerate}
\item Go to the top ARTS directory (called simply \verb|arts|).
\item Run: \verb|cvs update -P|
   
  \textbf{IMPORTANT!} Always update, before you start to make changes
  to the program, especially after a longer pause. If you edit an
  outdated copy, it will be a lot more work to bring your changes into
  the current copy of the program.
\end{enumerate}


\levelc{How to commit your changes}
%---------------------------------------
\begin{enumerate}
\item You should make sure that the program compiles and runs without
  obvious errors before you commit.
\item If you have created a new source file, make it known to CVS by
  running the command \verb|cvs add <my_file>| in the directory where
  the file resides.
  
  In general, when you run \verb|cvs update|, it will warn you about
  any files it doesn't know by marking them with a \verb|?|. Files
  that are created during the compilation process, but should not be
  part of the package are listed in the \verb|.cvsignore| files in
  each directory.
\item Have you added the documentation for your new features?
\item Increase the subversion number in file \verb|configure.in| in
  the top level ARTS directory.
\item Open the file \verb|ChangeLog| in the top level ARTS directory
  with your favorite editor.
  
  With Emacs, you can very easily add an entry by typing either
  \begin{quote}
    \verb|M-x add-change-log-entry|
  \end{quote}
  or \verb|C-x 4 a|.
  
  Specify the new version number and describe your changes.

  \textbf{These keystrokes work also while you are editing some other
    file in Emacs. Thus it is best to write your ChangeLog entry
    already while you work on a file}. Whenever you make a change to a
  file, there should be a ChangeLog Entry!
\item Make sure that you have saved all your files. Go to the top
  level ARTS directory and run: \verb|cvs commit|.
\item This will pop up an editor. Use the mouse to cut and paste the
  Change-Log message also to this editor window. Safe the file and exit
  the editor. If you made changes in different directories, another
  editor will pop up, already containing your message. Save again and
  exit. Do this until no more editors come up. (Note: This works well
  if you set
  \begin{quote}
    \verb|export EDITOR=xedit|
  \end{quote}
  in you shell startup file.
 
  With smart editors there can be problems, because they might
  refuse to safe your file if you haven't made changes to it. With
  xedit you just have to push the save button twice to override.
\item You have to give your version of the program a symbolic name, so
  that it can be retrieved later on if necessary. Do this by running:
  \verb|cvs tag arts-x-y-z| where x,y,z must be replace by the version
  numbers. You have to use dashes to separate the numbers, a point
  (\verb|.|) will not work.
\item Tell the other developers about it. The best way to do this is
  to send an email to \verb|arts-dev@sat.physik.uni-bremen.de|.
\end{enumerate}


\levelc{How to cut a release}
%----------------------------
\label{sec:release}
\begin{enumerate}
\item Change the release number in the file \verb|configure.in| in the
  top-level ARTS directory. (The line that you have to change is the
  one with \verb|AM_INIT_AUTOMAKE|.) Omit the subversion number (last digit).
\item Commit your changes (see other howto). 
\item In the top-level ARTS directory, run \verb|reconf|.
\item In the top-level ARTS directory, run \verb|make distcheck|. This
  will not only cut the release, but also immediately try to build
  it, to see if it works. Unless you are on a very fast machine, this
  may take a while. Maybe you should go and have a cup of coffee.
\item If all goes well, you can find the release inside the top-level
  ARTS directory as a file \verb|arts-x.y.tar.gz|, where x.y is the
  release number.
\item Check the release carefully by trying to build and install the
  program. 
\end{enumerate}


\levelc{How to move your arts working directory}
%----------------------------------------------
\textbf{Never try to move CVS directories!} Instead:
\begin{enumerate}
\item Commit your changes.
\item Go \emph{above} the top level ARTS directory.
\item Run: \verb|cvs release -d arts|.
  
  This will ask for confirmation, and if you say \verb|y| delete your
  working copy of arts.
\item Go to the directory where you want to have your ARTS copy in the
  future.
\item Check out a new copy (see other howto above).
\end{enumerate}

\levelb{Debugging (use of assert)}
%================================
 \label{sec:development:assert}

This section is taken more or less literally from the GNU tools manual
of Eleftherios Gkioulekas:
\begin{quote}
{\footnotesize
\verb|http://www.amath.washington.edu/~lf/tutorials/autoconf/|}
\end{quote}

The idea behind assert is simple. Suppose that at a certain point in
your code, you expect two variables to be equal.  If this expectation
is a precondition that must be satisfied in order for the subsequent
code to execute correctly, you must assert it with a statement like
this:
\begin{quote}
\verb|assert(var1 == var2);|
\end{quote}

In general assert takes as argument a boolean expression. If the
boolean expression is true, execution continues. Otherwise the
\verb|abort| system call is invoked and the program execution is
stopped. If a bug prevents the precondition from being true, then you
can trace the bug at the point where the precondition breaks down
instead of further down in execution or not at all.  The \verb|assert| call
is implemented as a C preprocessor macro, so it can be enabled or
disabled at will. One way to enable assertions is to include
\verb|assert.h|.
\begin{quote}
  \verb|#include <assert.h>|
\end{quote}
Then it's possible to disable them by defining the `NDEBUG' macro.

%
% FIXME: Oliver, please update
%
%In ARTS, assertions are turned on and off with the global NDEBUG
%preprocessor macro, which can be set or unset in file
%\verb|arts.h|. In the future there will be also a configure option to
%achieve this (FIXME: Update this).

During debugging and testing it is a good idea to leave assertions
enabled. However, for production runs it's best to disable them. If
your program crashes at an assertion, then the first thing you should
do is to find out where the error happens. To do this, run the program
under the \verb|gdb| debugger. First invoke the debugger:
\begin{quote}
\verb|gdb|
\end{quote}
Then load the executable and set a breakpoint at the \verb|exit|
system call:
\begin{quote}
  \verb|(gdb) file arts|\\
  \verb|(gdb) break exit| (or \verb|break __assert_fail|)
\end{quote}
Now run the program: 
\begin{quote}
  \verb|(gdb) run|
\end{quote}

Instead of crashing, under the debugger the program will be paused
when the \verb|exit| system call is invoked, and you will get back the
debugger prompt. Now type:
\begin{quote}
  \verb|(gdb) where| 
\end{quote}  
to see where the crash happened. You can use the \verb|print| command to
look at the contents of variables and you can use the \verb|up| and \verb|down|
commands to navigate the stack. For more information, see the GDB
documentation or type \verb|help| at the prompt of gdb.

For ARTS, the assertion failures mostly happen inside the Matrix /
Vector package (usually because you triggered a range check error,
i.e., you tried to read or write beyond array bounds). In this case the
\verb|up| command of GDB is particularly useful. If you give this a
couple of times you will finally end up in the part of your code that
caused the error.

Recommendation: In Emacs there is a special GDB mode. With this you
can very conveniently step through your code.




%%% Local Variables: 
%%% mode: latex
%%% TeX-master: "uguide"
%%% End: 



%
% ===   Bibliography
%
\bibliography{references}

%
% ===   Appendices
%
%\begin{appendix}
%  \include{????}
%\end{appendix}

%===   End of report   =====================================================
\end{document}